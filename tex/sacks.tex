We now consider two closely related notions of forcing,
namely Sacks forcing and Silver forcing.
In addition to applying the results of Section~\ref{generic}
to these two examples, we show that generic reals
for these two notions of forcing can be made to be well-behaved in certain ways.

The \textit{conditions for Sacks forcing} are perfect subtrees of $2^{<\omega}$.
In other words, the conditions are subtrees $T\subseteq 2^{<\omega}$
such that for all $\tau\in T$ there is a $\sigma\supseteq\tau$ such that
$\sigma$ is a splitting node.
We say that $\sigma$ is a splitting node if
$\sigma\cat{0},\sigma\cat{1}\in T$.
Notice that Sacks forcing is persistent.

The \textit{conditions for Silver forcing} are usually thought of as
partial functions $f:\NN\to\{0,1\}$ whose domain is coinfinite.
Silver conditions can also be thought of as certain subtrees of $2^{<\omega}$.
Given a partial function $f:\NN\to\{0,1\}$ whose domain is coinfinite,
consider the tree $T$ defined by
\begin{itemize}
\item $\seq{}\in T$,
\item if $\tau\in T$ and $|\tau|\notin\dom(f)$ then $\tau$ splits in $T$,
\item if $\tau\in T$ and $|\tau|\in\dom(f)$ then $\tau$ has exactly one extension
	in $T$, namely $\tau\cat{f(|\tau|)}$.
\end{itemize}
Notice that the branches through $T$ are exactly the total extensions of $f$.
We can therefore define Silver conditions to be all trees $T\subseteq \bin$
with infinitely many nodes that split and such that for every $n\in\NN$,
one of the following holds:
\begin{itemize}
\item if $|\tau|=n$ then $\tau$ splits in $T$, or
\item if $|\tau|=n$ then $\tau\cat{0}\in T$ and $\tau\cat{1}\notin T$, or
\item if $|\tau|=n$ then $\tau\cat{0}\notin T$ and $\tau\cat{1}\in T$.
\end{itemize}
In other words, Silver conditions are the Sacks conditions such that
on every level, either every node splits, every node branches only to the left,
or every node branches only to the right.

Silver forcing and Sacks forcing are very similar.
In fact, everything that we will prove about Sacks
forcing will be proved about Silver forcing with a nearly identical proof.

\begin{prop}[\ACAo]\label{P:Sk:MCP}
Sacks forcing satisfies $\MCP$.
\end{prop}
\begin{proof}
Fix a sequence of monotone colorings $c_i:2^{<\omega}\rightarrow \{0,1\}$
of a condition $T$.
We will construct an extension $T'\leq T$ in stages,
so that the $n$-th splitting level of $T'$ is defined at stage $n$.
In fact, $T'$ will be defined so that for all $\sigma$ such that $|\sigma|\geq n$,
$c_n(\sigma)$ will be determined at the $n$-th splitting level of $T'$.
Therefore $T'$, along with its splitting levels as a set of layers,
will satisfy the conclusion of \MCP.

At stage $0$ we look for a $\sigma$ such that $c_0(\sigma)=1$.
If such a $\sigma$ exists, we place it (and every node below it) into $T'$.
Otherwise, we place $\langle\rangle$ into $T'$.

At stage $n+1$ we look at each of the $2^n$ nodes defined at stage $n$.
List these nodes as $\tau_1,\ldots,\tau_{2^n}$.
We look for a $\sigma\supseteq\tau_1\cat{0}$ such that $c_0(\sigma)=1$.
If such a $\sigma$ exists, we place it (and every node below it) into $T'$.
Otherwise, we place $\tau_1\cat{0}$ into $T'$.
We now repeat this process for $\tau_1\cat{1}$, and also for
$\tau_i\cat{j}$ for each $2\leq i\leq 2^n$ and $j<2$.
\end{proof}

\begin{lem}\label{L:Sk:realForc}
For every $\BSigma^0_2$-generic filter over a model of \RCAo\
there exists a generic real for Sacks forcing.
\end{lem}

\begin{proof}
By Lemma~\ref{L:Gen:realForc} it
suffices to show that if $T$ is a condition, then
$$\mathcal{D}_T=\{S:S\cap T\text{ is finite }\lor\ S\leq T\}$$
is open dense.

Suppose that $T'\cap T$ contains no perfect subtree.
If $T'\cap T$ is empty, then we're done.
Otherwise we can choose a $\sigma\in T'\cap T$ such that no
$\tau\supseteq\sigma$ splits in $T'\cap T$.
We can therefore find a $\tau\supseteq\sigma$ such that
$\tau\in T'\setminus T$.
Then $T'_\tau\leq T'$ and $T'_\tau\cap T$ is finite,
so $T'_\tau\in\mathcal{D}_T$.
\end{proof}

\begin{thm}
Let $\MM$ be a model of \RCAo\ (\ACAo)
and suppose that $G$ is a generic real for $\Forc$
corresponding to a $\BSigma^1_2$-generic filter $\mathcal{G}$ for Sacks forcing.

Then $\MM[G]$ is a model of \RCAo\ (\ACAo).
\end{thm}
\begin{proof}
Follows from Propositions \ref{T:presRCA} and Proposition~\ref{T:presACA}.
\end{proof}

\begin{prop}[\ACAo]\label{P:Sl:MCP}
Silver forcing satisfies $\MCP$.
\end{prop}
\begin{proof}
The proof is nearly identical to that of Proposition~\ref{P:Sk:MCP}.
We explain briefly how to modify the proof of Proposition~\ref{P:Sk:MCP}.

The proofs of of Proposition~\ref{P:Sk:MCP} used a stagewise
construction where at stage $n+1$ the $n$-th splitting
level was extended to the $(n+1)$-th splitting level, meeting some requirements.
Since Silver conditions are a bit more restrictive than Sacks conditions,
care is needed when extending nodes from one level to a higher level.

Suppose that $\tau_1,\tau_2,\ldots,\tau_k$ are the nodes of level $n$
of a Silver condition $T$, and that all these nodes split.
Just as was done in Sacks forcing,
we find an appropriate splitting node $\sigma_1\supseteq\tau_1$.
We then extend each of the other nodes $\tau_i$, for $2\leq i\leq k$
to match $\sigma_1$: let $\tau_i'(x)=\sigma_1(x)$ for each $|\tau_1|\leq k<|\sigma_1|$.
Note that $\sigma_i\in T$ for all $2\leq i\leq k$ since
$\sigma_1\in T$ and $T$ is a Silver condition.
It is necessary that we extend all the nodes $\sigma_i$
like this so that we end up with a Silver condition.
At this point, of the nodes $\sigma_1, \tau_2',\tau_3',\ldots,\tau_k'$,
only $\sigma_1$ meets whatever requirement we are concerned with.
We now find an extension $\sigma_2\supseteq\tau_2'$ that meets the relevant requirement.
We then extend each of $\sigma_1,\tau_3',\tau_4'$ so that the extensions match $\sigma_2$.
Continuing in this way, and eventually finding an extension $\rho_i$ of each $\tau_i$,
for $1\leq i\leq k$, such that $\rho_i$ meet the requirement and the set of all $\rho_i$
is a valid $(n+1)$-splitting level for a Silver condition.

With this adjustment, the proof of Proposition~\ref{P:Sk:MCP} suffices to prove
the same proposition for Silver forcing.
\end{proof}

\begin{lem}\label{L:Sl:realForc}
For every $\BSigma^0_2$-generic filter over a model of \RCAo\
there exists a generic real for Silver forcing.
\end{lem}
\begin{proof}
The proof is similar to that of Lemma~\ref{L:Sk:realForc}.
\end{proof}

\begin{thm}
Let $\MM$ be a model of \RCAo\ (\ACAo)
and suppose that $G$ is a generic real for $\Forc$
corresponding to a $\BSigma^1_2$-generic filter $\mathcal{G}$ for Sacks forcing.

Then $\MM[G]$ is a model of \RCAo\ (\ACAo).
\end{thm}
\begin{proof}
Follows from Propositions \ref{T:presRCA} and Proposition~\ref{T:presACA}.
\end{proof}

We now show that we can add a cone avoiding, or generalized low-2,
Sacks generic real.

\begin{prop}\label{P:Sk:CAgeneric}
Let $\MM$ be a model of $\ACAo$ and $A$ be a noncomputable set in $\MM$.
There is a Sacks condition $T$ such that no branch through $T$ computes $A$.
\end{prop}

\begin{proof}
We construct $T$ in stages so that $T$ is defined up
to the $n$-th splitting level at stage $n$.
Moreover, at stage $n$ we will guarantee
that $\Phi^f_n\neq\chi_A$ for all $f\in[T]$.

At stage 0 we look for a $\sigma\in\bin$
and an $x$ such that $\Phi_0^{\sigma}(x)\downarrow\neq\chi_{A}(x)$.
If such $\sigma$ and $x$ exist, we place $\sigma$ into $T$.
We have then ensured that $\Phi_0^f\neq\chi_A$ for all
$f\in[\bin]$ which extend $\sigma$.
Otherwise, if no such $\sigma$ and $x$ exist,
we place $\seq{}$ into $T$ and claim that
if $f\in [\bin]$ then $\Phi_0^f$ is \textit{not} total
(and hence distinct from $\chi_A$).
Suppose, for the sake of contradiction, that $\Phi_0^f$ is total.
We can then compute $\chi_A(x)$ by searching for
a $\sigma$ such that $\Phi_0^\sigma(x)\downarrow$.
Such a $\sigma$ exists since $\Phi_0^f$ is total,
and $\Phi_0^\sigma(x)=\Phi_0^f(x)=\chi_A(x)$ since,
by assumption, there is no $\sigma$ such that
$\Phi_0^\sigma(x)\neq\chi_A(x)$.
This is a contradiction since $A$ was assumed to be noncomputable.

At stage $(n+1)$ we begin by looking at $\rho\cat0$ and $\rho\cat1$
for each node $\rho$ defined during stage $n$.
List these nodes by $\tau_1,\tau_2,\ldots,\tau_{2^{n+1}}$.
We now look for a node $\sigma\supseteq\tau_1$ and an $x$
such that $\Phi_{n+1}^\sigma(x)\downarrow\neq\chi_A(x)$.
If such $\sigma$ and $x$ exist, we place $\sigma$ into $T$.
Otherwise we place $\tau_1$ into $T$.
By the same argument as in the previous paragraph,
$\Phi_{n+1}^f\neq\chi_A$ when ever $f$ is a path
through $2^{<\omega}$ which extends $\tau_1$.
We now repeat this procedure for $\tau_2,\ldots,\tau_{2^{n+1}}$.
\end{proof}

\begin{cor}
Let $\MM$ be a model of \ACAo.
Suppose that $A$ is a noncomputable set in $\MM$
and that $\mathcal{G}$ is a generic filter for Sacks forcing.
Then there is a generic real $G$ corresponding to $\mathcal{G}$
such that $A\not\leq_{\text{T}}G$.
\end{cor}

\begin{prop}\label{P:Sk:GL2generic}
Let $\MM$ be a model of $\ACAo$.
There is a Sacks condition $T$ such that for every branch $G\in[T]$,
$G''\leq G\oplus0''$.  In other words, $T$ only contains GL$_2$ generic reals.
\end{prop}

\begin{proof}
We construct $T$ in stages so that $T$ is defined up to the $n$-th splitting level at stage $n$.
At the beginning of stage $n+1$, each of the $2^n$ nodes defined in stage $n$
will be marked as either ``needing help with $m$" or ``not needing help with $m$" for each $m\leq n$.

At stage 0 we look for a node $\sigma$ and a $k$
such that for all $\tau\supseteq\sigma$, $\Phi_0^\tau(k)\uparrow$.
Note that we can ask such questions with a $0''$ oracle.
If such $\sigma$ and $k$ exist, we let $\sigma'$ and $\sigma''$
be any two incomparable nodes above $\sigma$.
We place $\sigma'$ and $\sigma''$
into $T$ and mark them as ``not needing help with 0".
Notice that if $G:\NN\to\{0,1\}$ extends $\sigma$
then $\Phi_0^f$ is not total.

If no such $\sigma$ and $k$ exist, then we know
that for every $\tau\supseteq\sigma$ and every $k$
there is a $\rho\supseteq\tau$ such that $\Phi_0^\rho(k)\downarrow$.
We place $\langle 0\rangle$ and $\langle 1\rangle$
into $T$ and mark them as ``needing help with 0".

At stage $n+1$ we begin by considering all the nodes
$\sigma_1',\ldots,\sigma_{2^{n}}'$ that were defined in the previous stage.
For each $\sigma_i'$ we find a $\sigma_i\supseteq\sigma_i'$ such
that for all $m\leq n$, if $\sigma_i'$ is marked as
``needing help with $m$", then $\Phi_m^{\sigma_i}(k)\downarrow$
for each $k\leq n$.
This is possible because $\sigma_i'$ could only have been
marked as ``needing help with $m$" if such an extension
is always possible.

For each $1\leq i\leq 2^n$ we do the following.
We look for a node $\sigma\supseteq\sigma_i$ and a $k$
such that for all $\tau\supseteq\sigma$, $\Phi_{n+1}^\tau(k)\uparrow$.
Note that we can ask such questions with a $0''$ oracle.
If such $\sigma$ and $k$ exist, we let $\sigma'$ and $\sigma''$
be any two incomparable nodes above $\sigma$.
We place $\sigma'$ and $\sigma''$
into $T$ and mark them as ``not needing help with $n+1$".
For $m\leq n$, we mark $\sigma'$ and $\sigma''$
as ``needing help with $m$" if and only if $\sigma_i$
was marked as ``needing help with $m$".
Notice that if $G:\NN\to\{0,1\}$ extends $\sigma$
then $\Phi_0^f$ is not total.

If no such $\sigma$ and $k$ exists, then we know
that for every $\tau\supseteq\sigma_i$ and every $k$
there is a $\rho\supseteq\tau$ such that $\Phi_0^\rho(k)\downarrow$.
We let $\sigma'$ and $\sigma''$
be any two incomparable nodes above $\sigma_i$, and we
place them into $T$ and mark them as ``needing help with $n+1$".
For $m\leq n$, we mark $\sigma'$ and $\sigma''$
as ``needing help with $m$" if and only if $\sigma_i$
was marked as ``needing help with $m$".

This ends the construction.
We now show that if $G\in[T]$, then $G''\leq G\oplus 0''$.
It suffices to show that we can use $G$ and $0''$
to compute the set of indices $e$ such that
$\Phi_e^G$ is total.
To determine if $\Phi^G_e$ is total, we do the following.
Suppose that $\sigma$ is the topmost node at
stage $e$ that meets $G$.
If $\sigma$ was marked as ``not needing help with $e$"
then $\Phi_e^G$ is not total.
In fact, if $k$ is the number that witnesses that
$\sigma$ was marked as ``not needing help with $e$",
then $\Phi_e^G(k)\uparrow$.
If, on the other hand, $\sigma$ was marked as
``needing help with $e$", then $\Phi_e^G(m)$
converges by the $m$-th splitting level of $T$.
In other words, if $\sigma$ is the topmost node at
stage $m$ that meets $G$, then $\Phi_e^G(m)\downarrow$.
\end{proof}

\begin{cor}
Let $\MM$ be a model of \ACAo.
Suppose $\mathcal{G}$ is a generic filter for Sacks forcing.
Then there is a generic real $G$ corresponding to $\mathcal{G}$
such that $G$ is GL$_2$.
\end{cor}

We now show that Sacks forcing does not add unbounded reals.

\begin{prop}[\RCAo]\label{P:Sk:StrBnding}
Let $T$ be a Sacks condition and $F$ be a $T$-local name.
There is a condition $T'\leq T$ such that for all $n$,
if $\tau\in T'$ is above the $n$-th splitting level of $T'$
then $F_\tau(x)$ is defined by stage $|\tau|$ for all $x\leq n$.
\end{prop}

\begin{proof}
We assume, for ease of notation,
that $F$ is a 1-ary name.
We will construct $T'$ in stages.

We begin with stage 0.
Since $F$ is $T$-local, there is a $\tau\in T$ such that $F^\tau(0)$ is defined.
Let $\sigma$ be a splitting node above $\tau$.
We place $\sigma\cat{0}$ and $\sigma\cat{1}$ into $T'$.

At stage $n+1$ we begin with the the $2^{n+1}$-many
nodes that were defined at stage $n$.
Let $\rho$ be one such node.
Since $F$ is $T_\rho$-local, there is a $\tau\in T_\rho$ such that $F^\tau(n+1)$ is defined.
Let $\sigma$ be a splitting node above $\tau$.
We place $\sigma\cat{0}$ and $\sigma\cat{1}$ into $T'$.

Finally, we close $T'$ downward so that it is in fact a tree.
This ends the construction, and it is easy to see that $T'$
satisfies the proposition.
\end{proof}

\begin{cor}[\RCAo]\label{P:Sk:bndedNames}
Let $T$ be a Sacks condition and $F$ be a $T$-local name.
There is an extension $T'\leq T$ and a function $B$
such that $$T' \Vdash (\forall\tup{v})[F(\tup{v}) \leq \check{B}(\tup{v})].$$
\end{cor}
\begin{proof}
Let $T'\leq T$ be as in the conclusion of Proposition~\ref{P:Sk:StrBnding}.
Let \newline $B(x)=\max\set{F^\tau(x)}{\tau\text{ is on the }n\text{-th splitting level of }T'}.$
\end{proof}

\begin{cor}
Let $\MM$ be a model of \RCAo\
and suppose that $G$ is a generic real for $\Forc$
corresponding to a generic filter $\mathcal{G}$ for Sacks forcing.

For every function $f:\NN\to\NN$ in $\MM[G]$
there is a function $b$ in $\MM$ such that
$f(x)\leq b(x)$ for all $x$.
\end{cor}

The results which we have now proved about Sacks forcing
also hold for Silver forcing.
In the same way that we modified the proof of Proposition~\ref{P:Sk:MCP}
to work for Silver forcing, namely by being careful about extending splitting levels,
we can modify the proofs of our recent results about Sacks forcing to hold for Silver forcing.
We now state, without proof, the corresponding results for Silver forcing.

\begin{prop}\label{P:Sl:CAgeneric}
Let $\MM$ be a model of $\ACAo$ and $A$ be a noncomputable set in $\MM$.
There is a Silver condition $T$ such that no branch through $T$ computes $A$.
\end{prop}

\begin{cor}
Let $\MM$ be a model of \ACAo.
Suppose that $A$ is a noncomputable set in $\MM$
and that $\mathcal{G}$ is a generic filter for Silver forcing.
Then there is a generic real $G$ corresponding to $\mathcal{G}$
such that $A\not\leq_{\text{T}}G$.
\end{cor}

\begin{prop}\label{P:Sl:GL2generic}
Let $\MM$ be a model of $\ACAo$.
There is a Silver condition $T$ such that for every branch $G\in[T]$,
$G''\leq G\oplus0''$.  In other words, $T$ only contains GL$_2$ generic reals.
\end{prop}

\begin{cor}
Let $\MM$ be a model of \ACAo.
Suppose $\mathcal{G}$ is a generic filter for Silver forcing.
Then there is a generic real $G$ corresponding to $\mathcal{G}$
such that $G$ is GL$_2$.
\end{cor}

\begin{prop}[\RCAo]\label{P:Sl:StrBnding}
Let $T$ be a Silver condition and $F$ be a $T$-local name.
There is a condition $T'\leq T$ such that for all $n$,
if $\tau\in T'$ is above the $n$-th splitting level of $T'$
then $F_\tau(x)$ is defined by stage $|\tau|$ for all $x\leq n$.
\end{prop}

\begin{cor}
Let $\MM$ be a model of \RCAo\
and suppose that $G$ is a generic real for $\Forc$
corresponding to a generic filter $\mathcal{G}$ for Silver forcing.

For every function $f:\NN\to\NN$ in $\MM[G]$
there is a function $b$ in $\MM$ such that
$f(x)\leq b(x)$ for all $x$.
\end{cor}
