The main result of this section is Proposition~\ref{P:witnessPi2}.
Though somewhat technical, Proposition~\ref{P:witnessPi2} is the key
to showing that if $\MM$ is a model of $\RCAo$ and $\Forc$ is a
persistent notion of forcing, then the generic extension of $\MM$ is also a model of $\RCAo$.

As mentioned earlier, the propositions in this section are
relatively straightforward generalizations of the propositions in \cite{varMathias}.
Dorais proved the statements in this section for a specific notion of forcing,
though the proofs generalize in a straightforward manner.

\begin{definition}\label{D:admitsNormal}
We say that a notion of forcing \textit{admits normal form}\index{normal form}
if the following statement holds:

Given a bounded formula $\varphi(\tup{v})$ of the forcing language,
there is a name $U_\varphi(\tup{v})$ such that for every condition $T$,
if $\varphi(\tup{v})$ is $T$-local, then so is $U_\varphi(\tup{v})$ and
$$T\Vdash\forall\tup{v}\left(\varphi(\tup{v})\leftrightarrow U_\varphi(\tup{v})=0\right).$$
We say that $U_\varphi$ is the normal form name for $\varphi$.
\end{definition}

The key to showing that a notion of forcing admits normal form
is showing that locality is closed under composition and primitive recursion.

\begin{definition}\label{D:comp&pr}
Let $F_0$ be an $\ell$-ary name and $F_1,\ldots,F_\ell$ be $k$-ary names.
The \textit{composition} $H=F_0\circ(F_1,\ldots,F_\ell)$ is defined by
%$$(\tau,\tup{x},z)\in H\Leftrightarrow \exists\tup{y} [(\tau,\tup{x},y_1)\in F_1\land\ldots\land(\tau,\tup{x},y_\ell)\in F_\ell\land(\tau,\tup{y},z)\in F_0].$$
$$H^\tau(\tup{x})=z\Leftrightarrow \exists\tup{y}
[F_1^\tau(\tup{x})=y_1\ \land\ \ldots\ \land\ F_\ell^\tau(\tup{x})=y_\ell\ \land\ F_0^\tau(\tup{y})=z)].$$

Let $F_0$ be a $(k-1)$-ary name and $F_1$ be a $(k+1)$-ary name.
We use $F_0$ and $F_1$ to define a $k$-ary name $H$ by \textit{primitive recursion} by letting
$H^\tau(\tup{x},y)=z$ if and only if there is a finite sequence
$\langle z_0,z_1,\ldots,z_{y}\rangle$ such that $z_{y}=z$,
$F_0^\tau(\tup{x})=z_0$, and $F_1^\tau(\tup{x},i,z_i)=z_{i+1}$ for all $i<y$.
\end{definition}

\begin{prop}[$\RCAo$]\label{P:comp&pr}
Let $\Forc$ be a persistent notion of forcing and $T\in\Forc$.
The names defined by composition and primitive recursion using $T$-local names
are themselves $T$-local.
\end{prop}
\begin{proof}
Since $\Forc$ is persistent, we use the alternate definition
of locality as given by Proposition~\ref{P:PersistentLocalNames}.

First we handle composition.
Let $F_0$ be an $\ell$-ary $T$-local name, $F_1,\ldots,F_\ell$ be $k$-ary $T$-local names,
and $H=F_0\circ(F_1,\ldots,F_\ell)$.
We show that $H$ is $T$-local.
In other words, we let $T'\leq T$ and show that $\dom(F)\cap[T']$ is nonempty.

Fix $\tup{x}\in\NN^k$.  First we find a $\tau\in T'$ so that $H^\tau(\tup{x})$ is  defined.
Since $F_1$ is $T$-local, there is a $\tau_1\in T'$ and a $z_1$ such that $F_1^{\tau_1}(\tup{x})=z_1$.
Since $F_2$ is $T$-local, there is a $\tau_2\in T'_{\tau_1}$ and a $z_2$ such that
$\tau_1\subseteq\tau_2$ and $F_2^{\tau_2}(\tup{x})=z_2$.
Repeating this process, we eventually get a $\tau\in T'$ such that
$H^\tau(\tup{x})$ is defined.

We can now repeat this process, relative to $T'_\tau$, to define $H$ on a new input.
In other words, we choose some $\tup{x}'$ distinct from $\tup{x}$ and then
find a $\sigma\in T'_\tau$ such that $\tau\subseteq\sigma$ and
$H^\sigma(\tup{x}')$ is defined.
By repeating this for every value in $\NN^k$, we construct $A\in[T']\cap\dom(H)$.
Note that $\Ind{\BSigma^0_1}$ suffices to show that $A$ is well-defined.

Now we handle primitive recursion.
Let $F_0$ be an $(k-1)$-ary $T$-local name, $F_1$ be a $(k+1)$-ary $T$-local name,
and $H$ be the $k$-ary name defined by primitive recursion with $F_0$ and $F_1$.
We now show that $H$ is $T$-local.
In other words, we let $T'\leq T$ and show that $\dom(F)\cap[T']$ is nonempty.

The construction of an element in $\dom(H)\cap[T']$ is similar
to the construction above for the composition of functions.
We continually extend elements $\tau\in T'$ using the locality of $F_0$ and $F_1$
so that $H$ is defined on more and more inputs.
\end{proof}

A consequence of Lemma~\ref{P:comp&pr} is that
our definitions of composition and primitive recursion correspond
to the usual definitions once the names have been evaluated.
In other words, if $A$ is in the domain of the names
$F_0,F_1,\ldots,F_\ell$, then $A$ is also in the domain of $H$
and $H^A=F_0^A\circ(F_1^A,\ldots,F_\ell^A)$,
and similarly for primitive recursion.

Proposition~\ref{P:comp&pr} is the key property
of persistent notions of forcing that we need
to prove nearly everything that remains in this section.
For this reason we make the following definition.

\begin{definition}\label{D:almostP}
A notion of forcing $\Forc$ is \textit{almost persistent}\index{almost persistent}
if it satisfies the following two properties:
\begin{itemize}
\item For every condition $T\in\Forc$, the names defined by composition and primitive
		recursion using T-local names are themselves T-local.
\item For every $n\in\NN$ and condition $T$ there is a $\tau\in T$ such that $|\tau|\geq n$
		and $T_\tau$ contains a condition.
\item For every condition $T$, $T$-local $k$-ary name $F$, and $\tup{x}\in\NN^k$, there is a $\tau\in T$
		such that $T_\tau\in\Forc$ and $F^\tau(\tup{x})$ is defined by stage $|\tau|$.
\end{itemize}
\end{definition}

The second property above ensures that the atomic forcing relation is not trivially satisfied.
The third property will be used later to show that the evaluation of a name
(with a particular locality requirement) at a generic real defines a total $k$-ary function.

Note that if $\Forc$ is a persistent notion of forcing,
then $\Forc$ is also almost persistent by Proposition~\ref{P:comp&pr}
(the second and third properties are immediate).

\begin{prop}[$\RCAo$]\label{P:PR:normalForm}\index{normal form}
Almost persistent notions of forcings admit normal form.
\end{prop}

\begin{proof}
Note that if $F$ and $F'$ are $T$-local names, then the names for
$F+F'$, $F\dotminus~F'$, $|F-~F'|$, and $\displaystyle \sum_{w\leq F(\tup{v})}F'(\tup{v},w)$
are also $T$-local names since they are defined by
composition and primitive recursion.

We define $U_\varphi$ inductively on the complexity $\varphi$:
\begin{itemize}
\item $U_{F(\tup{v})=F'(\tup{v}')}(\tup{v})=|F(\tup{v})-F'(\tup{v}')|$
\item $U_{\neg\varphi}(\tup{v})=1\dotminus U_\varphi(\tup{v})$
\item $U_{\varphi\land\psi}(\tup{v})=U_\varphi(\tup{v})+U_\psi(\tup{v})$
\item if $\varphi(\tup{v})=(\forall w\leq F(\tup{v}))\psi(\tup{v},w)$ then
		$U_{\varphi}(\tup{v})=\sum_{w\leq F(\tup{v})} U_{\psi(w)}(\tup{v},w)$
\end{itemize}
It is easy to check that these names work as advertised.
\end{proof}

One consequence of Proposition~\ref{P:PR:normalForm} is that the $\BPi^0_1$
forcing relation for a persistent notion of forcing is well behaved.
The proposition below, which captures this fact,
will be helpful later on when determining how much genericity is required
of our filters.

\begin{prop}\label{P:Pi1isPi1}
Let $\Forc$ be a persistent notion of forcing.
Let $\varphi(\tup{v})$ be a $\BPi^0_1$ formula of the forcing language.
There is a $\BPi^0_1$ formula $\theta(T,\tup{v})$ such that
for every condition $T$ such that $\varphi$ is $T$-local,
and every $\tup{x}$,
$$T\Vdash\varphi(\tup{x})\ \ \ \text{if and only if}\ \ \theta(T,\tup{x}).$$
Note that we are not making any claims about the complexity of
the statement `$\varphi(\tup{x})$ is $T$-local'.
\end{prop}
\begin{proof}
Let $\varphi(\tup{v})=\forall w\psi(w,\tup{v})$,
where $\psi(w,\tup{v})$ is a bounded formula.
Let $U=U_{\psi(w,\tup{v})}$ be the normal form name, as in Proposition~\ref{P:PR:normalForm}.
Then
$$T\Vdash \forall\tup{v}\forall w[\psi(\tup{v},w)\leftrightarrow U(\tup{v},w)=0].$$
We therefore let $\theta(T,\tup{v})$ be the statement saying that for all
$\tau\in T$ and $w$, if $U^\tau(\tup{v},w)$ is defined by stage
$n=|\tau|$ then $U^\tau(\tup{v},w)=0$.
\end{proof}

Note that it is necessary to assume that $\Forc$ is persistent,
and not just almost persistent, since it is vital that $T_\tau$
is a condition for every $\tau\in T$.
We now prove a version of Proposition~\ref{P:Pi1isPi1}
for almost persistent notions of forcing in the case when
$\Forc$ has an arithmetic definition.

\begin{prop}\label{P:Pi1isArith}
Let $\Forc$ be an almost persistent notion of forcing.
Suppose further that the conditions of $\Forc$ are defined
by an arithmetic formula $\alpha(T)$.
Let $\varphi(\tup{v})$ be a $\BPi^0_1$ formula of the forcing language.
There is an arithmetic formula $\theta(T,\tup{v})$ such that
for every condition $T$ such that $\varphi$ is $T$-local,
and every string of numbers $\tup{x}$, we have that
$$T\Vdash\varphi(\tup{x})\ \ \ \text{if and only if}\ \ \theta(T,\tup{x}).$$
Note that we are not making any claims about the complexity of
the statement `$\varphi(\tup{x})$ is $T$-local'.
\end{prop}
\begin{proof}
The proof is similar to that of Proposition~\ref{P:Pi1isPi1}.

Let $\varphi(\tup{v})=\forall w\psi(w,\tup{v})$,
where $\psi(w,\tup{v})$ is a bounded formula.
Let $U=U_{\psi(w,\tup{v})}$ be the normal form name, as in Proposition~\ref{P:PR:normalForm}.
Then
$$T\Vdash \forall\tup{v}\forall w[\psi(\tup{v},w)\leftrightarrow U(\tup{v},w)=0].$$
We therefore let $\theta(T,\tup{v})$ be the statement saying that for all
$\tau\in T$ and $w$, if $U^\tau(\tup{v},w)$ is defined by stage
$n=|\tau|$ and $\alpha(U^\tau)$ holds, then $U^\tau(\tup{v},w)=0$.
\end{proof}

We now begin to build some definition-dense machinery that will be key
in everything else that we do.
Let $\theta(\tup{v})$ be a formula of the forcing language and
let $W$ be a unary name.
We will construct a new formula $\theta_S(W;\tup{v})$ which
is essentially a $\BPi^0_1$ formula asserting that $W$ witnesses the truth of $\theta$
by coding the appropriate Skolem functions.
Dually, we will construct a formula $\theta_H(W;\tup{v})$ which
is essentially a $\BSigma^0_1$ formula asserting that $W$ does not witnesses
the the falsity of $\theta$.
We call $\theta_S$ the Skolemization\index{Skolemization} of $\theta$ and
we call $\theta_H$ the Herbrandization\index{Herbrandization} of $\theta$.

To avoid confusion between parameters and the variable used for the
unary names $W$, we use the $\lambda$-notation.
In other words, if $W$ is a $k+1$-ary name and $\tup{x}\in\NN^k$,
then $\lambda tW(\tup{x},t)$ denotes the unary name which is a function of $t$.

The definition of $\theta_S$\index{$\theta_S$} and $\theta_H$\index{$\theta_H$}
proceeds by induction on the complexity of $\theta$.

$$
\begin{array}{c|c|c}

\theta(\tup{v}) & \theta_S(W;\tup{v}) & \theta_H(W;\tup{v})\\
\hline
\text{atomic } & \theta(W;\tup{v}) & \theta(W;\tup{v})\\

\neg\phi(\tup{v}) & \neg\phi_H(W;\tup{v}) & \neg\phi_S(W;\tup{v})\\

\phi(\tup{v})\land\psi(\tup{v}) & \phi_S(\lambda tW(2t);\tup{v})\land\psi_S(\lambda tW(2t+1);\tup{v})
	&\phi_H(W;\tup{v})\land\psi_H(W;\tup{v})\\

\forall w\phi(\tup{v},w) & \forall w\phi_S(\lambda tW(\langle w,t\rangle);\tup{v},w)
	& \phi_H(\lambda tW(t+1);\tup{v},W(0))\\

\end{array}
$$

Here are some examples:
$$
\begin{array}{c|c|c}

\theta(\tup{v}) & \theta_S(W;\tup{v}) & \theta_H(W;\tup{v})\\
\hline
\forall v F(v)=0 & \forall v F(v)=0 & F(W(0))=0 \\
\forall v F(v)\neq 0 & \forall v F(v)\neq 0 & F(W(0))\neq 0 \\
\exists v F(v)=0 & \neg\neg F(W(0))=0 & \exists v F(v)=0 \\
\forall w\exists v F(w,v)=0 & \forall w \neg\neg F\big(w,W(\seq{w,0})\big)=0  &
	\exists v F\big(W(0),v\big)=0\\
\exists w\forall v F(v)=0 & \neg\neg\forall vF(W(0),v)=0 &
	\exists w F\big(w,W(\seq{w,0})\big)=0\\
\end{array}
$$

We now use Skolemizations and Herbrandizations with the forcing relation.
The intuitive idea behind Skolem and Herbrand forcing is that we would like $\theta(\tup{v})$
to be equivalent to $\exists W\theta_S(W;\tup{v})$ and $\forall W\theta_H(W;\tup{v})$.

\begin{definition}
Let $\theta$ be a $T$-local formula.
The \textit{Skolem forcing relation}\index{Skolem forcing relation} is defined by
$T\Vdash_S\theta$\index{$\Vdash_S$} if and only if $T\Vdash\theta_S(W)$ for some $T$-local unary name $W$.
The \textit{Herbrand forcing relation}\index{Herbrand forcing relation} is defined by
$T\Vdash_H\theta$\index{$\Vdash_H$} if and only if $T\Vdash\theta_H(W)$ for all $T$-local unary names $W$.
\end{definition}

Above we said that $\theta_S(W;\tup{v})$ was essentially a $\BPi^0_1$ formula
asserting that $W$ witnesses the truth of $\theta$.
We now make this precise.
We will assume that our notion of forcing is persistent
so that the names used in the construction of $\theta_S$ and $\theta_H$
are guaranteed to be local.

\begin{prop}[\RCAo]\label{P:skolemisPi1}
Let $\Forc$ be an almost persistent notion of forcing and $\theta(\tup{v})$
be a formula of the forcing language.
There is a bounded formula $\psi(W;u,\tup{v})$ of the forcing language such that
for all conditions $T$, names $W(\tup{v},t)$, and $\tup{x}$,
if $\theta(\tup{x})$ and $\lambda tW(\tup{x},t)$ are $T$-local, then
so is $\psi(\lambda tW(\tup{x},t);u,\tup{x})$ and
$$T\Vdash\theta_S(\lambda tW(\tup{x},t);\tup{x})\ \ \text{if and only if}\ \ T\Vdash\forall u\psi(\lambda tW(\tup{x},t);u,\tup{x})$$
\end{prop}
\begin{proof}
Notice that the only quantifiers that occur in $\theta_S$ are universal within the scope
of an even number of negations.
Let $u$ be a variable symbol that does not occur in $\theta_S$, and let
$\psi$ be the bounded formula obtained from $\theta_S(\lambda tW(\tup{x},t);\tup{x})$
by bounding all universal quantifiers with $u$.
Notice that if $\theta(\tup{x})$ and $\lambda tW(\tup{x},t)$ are $T$-local
then $\theta_S(\lambda tW(\tup{x},t);\tup{x})$ and $\psi(\lambda tW(\tup{x},t);u,\tup{x})$
are also $T$-local.  Moreover
$$T\Vdash\theta_S(\lambda tW(\tup{x},t);\tup{x})\ \ \text{if and only if}\ \ T\Vdash\forall u\psi(\lambda tW(\tup{x},t);u,\tup{x}).$$
\end{proof}

When the notion of forcing is persistent, we can say that $\theta_S(W;\tup{v})$
is essentially a $\BPi^0_1$ formula in an even stronger sense.

\begin{cor}[\RCAo]\label{C:skolemisActualPi1}
Let $\Forc$ be a persistent notion of forcing and $\theta(\tup{v})$
be a formula of the forcing language.
There is a $\BPi^0_1$ formula $\tilde{\theta}(T,W,\tup{v})$ such that
for all conditions $T$, names $W(\tup{v},t)$, and sequences of numbers $\tup{x}$,
if $\theta(\tup{x})$ and $\lambda tW(\tup{x},t)$ are $T$-local, then
$$T\Vdash\theta_S(\lambda tW(\tup{x},t);\tup{x})\ \ \text{if and only if}\ \ \tilde{\theta}(T,W,\tup{x}).$$
\end{cor}
\begin{proof}
Let $\psi(W;u,\tup{v})$ be as in the conclusion of Proposition~\ref{P:skolemisPi1}.
The corollary then follows by applying Proposition~\ref{P:Pi1isPi1} to $\forall u\psi(\lambda tW(\tup{x},t);u,\tup{v})$.
\end{proof}

Another corollary of Proposition~\ref{P:skolemisPi1}
is that the Skolem forcing relation is $\BSigma^1_2$ in certain cases.
This corollary will be helpful later on when
examining how much genericity is needed of our filters.

\begin{cor}[\RCAo]\label{C:skolemisPi11}
Let $\Forc$ be a notion of forcing which is either persistent
or is almost persistent and has an arithmetic definition.
Let $\theta(\tup{v})$ be a formula of the forcing language.
There is a $\BSigma^1_2$ formula $\tilde{\theta}(T,\tup{v})$ such that
for all conditions $T$ and sequences of numbers $\tup{x}$,
if $\theta(\tup{x})$ is $T$-local, then
$$T\Vdash_S\theta(\tup{x})\ \ \text{if and only if}\ \ \tilde{\theta}(T,\tup{x}).$$
\end{cor}
\begin{proof}
Let $\psi(W;u,\tup{v})$ be as in the conclusion of Proposition~\ref{P:skolemisPi1}.
In the case where $\Forc$ is persistent, let
$\hat{\psi}(T,W,\tup{v})$ be as in conclusion of Proposition~\ref{P:Pi1isPi1}
corresponding to $\forall u\psi(W;u,\tup{v})$.
In the case where $\Forc$ has an arithmetic definition, let
$\hat{\psi}(T,W,\tup{v})$ be as in conclusion of Proposition~\ref{P:Pi1isArith}
corresponding to $\forall u\psi(W;u,\tup{v})$.
Finally, let $\tilde{\theta}(T,\tup{v})$ be the statement saying that
there exists a Skolem name $W$ such that
$W$ is $T$-local and $\hat{\psi}(T,W,\tup{v})$ holds.
\end{proof}

\begin{prop}[\RCAo]\label{P:easySkolemEquiv}
Let $\Forc$ be an almost persistent notion of forcing,
let $T\in\Forc$, and let $\theta$ be a $T$-local sentence of the forcing language.
Then
$$T\Vdash_S\theta\ \Rightarrow\ T\Vdash\theta \Rightarrow\ T\Vdash_H\theta.$$
\end{prop}
\begin{proof}
We proceed by induction on the complexity of $\theta$.
The assumption that $\Forc$ is persistent ensures that
the names involved in the definition of $\theta_S$ and $\theta_H$ are $T$-local.

\begin{itemize}

\item $\theta$ is atomic:

Since $\theta_S(W)=\theta_H(W)=\theta$ the statement follows trivially.

\item $\theta=\neg\phi$:

If $T\Vdash_S\theta$ then $T\Vdash \neg\phi_H(W)$ for some unary name $W$.
Therefore there is no $T'\leq T$ such that $T'\Vdash\phi_H(W)$.
By the induction hypothesis, if $T'\Vdash\phi$ then $T'\Vdash\phi_H(W)$,
so we can conclude that there is no $T'\leq T$ such that
$T'\Vdash\phi$, and so $T\Vdash\theta$.

Now suppose that $T\Vdash\theta$.
Then there is no $T'\leq T$ such that $T'\Vdash\phi$.
By the induction hypothesis, if there is a unary name $W$ such that
$T'\Vdash\phi_S(W)$, then $T'\Vdash\theta$.
Therefore there is no $T'\leq T$ and no $W$ such that $T'\Vdash\phi_S(W)$.
Thus $T\Vdash_H\theta$.

\item Suppose that $\theta=\phi\land\psi$.

Assume that $T\Vdash \phi_S(\lambda tW(2t);\tup{v})\land\psi_S(\lambda tW(2t+1);\tup{v})$
for some unary name $W$.
Breaking this down, $T\Vdash\phi_S(\lambda tW(2t);\tup{v})$ and $T\Vdash\psi_S(\lambda tW(2t+1);\tup{v})$.
By the induction hypothesis $T\Vdash\phi$ and $T\Vdash\psi$, so $T\Vdash\phi\land\psi$.

Now suppose that $T\Vdash\theta$.
Then $T\Vdash\phi$ and $T\Vdash\psi$, so by the induction hypothesis
$T\Vdash\phi_H(W)$ and $T\Vdash\psi_H(W)$ for all unary names $W$.
Therefore $T\Vdash\phi_H(W)\land\psi_H(W)$ for all unary names $W$,
and so $T\Vdash_H\theta$.

\item Suppose that $\theta=\forall w\phi(w)$.

Assume that $T\Vdash \forall w\phi_S(\lambda tW(\langle w,t\rangle);\tup{v},w)$
for some unary name $W$.
Breaking this down, $T\Vdash\phi_S(\lambda tW(\langle x,t\rangle);\tup{v},w)$ for all $x$.
By the induction hypothesis $T\Vdash\phi(x)$ for all $x$.
Thus $T\Vdash\theta$.

Now suppose that $T\Vdash\theta$.
Then $T\Vdash\phi(x)$ for all $x$.
By the induction hypothesis $T\Vdash\phi_H(W,x)$ for all $x$ and all unary names $W$.
Thus $T\Vdash\theta_H(\lambda tW(t+1),W(0))$,
and so $T\Vdash_H\theta$.
\end{itemize}
\end{proof}

\begin{prop}[\RCAo]\label{P:witnessBnded}
Let $\Forc$ be an almost persistent notion of forcing.
Let $\theta(\tup{v})$ be a bounded formula of the forcing language.
There are names $W^\theta_S(\tup{v},t)$ and $W^\theta_H(\tup{v},t)$
such that for all $T\in\Forc$ and all $\tup{x}$, if $\theta(\tup{x})$ is $T$-local,
then then so are $\lambda tW^\theta_S(\tup{x},t)$ and $\lambda tW^\theta_H(\tup{x},t)$, and
\begin{align*}
T\Vdash\theta_S(\lambda tW_S^\theta(\tup{x},t);\tup{x})
	\Leftrightarrow  T\Vdash\theta(\tup{x})
	\Leftrightarrow  T\Vdash\theta_H(\lambda tW_H^\theta(\tup{x},t);\tup{x}).
\end{align*}
\end{prop}

\begin{proof}
Notice that the left to right implications follow
from Proposition~\ref{P:easySkolemEquiv}.
For if $T\Vdash\theta_S(\lambda tW_S^\theta(\tup{x},t);\tup{x})$,
then $T\Vdash_S\theta(\tup{x})$ and so $T\Vdash\theta(\tup{x}$).
And if $T\Vdash\theta(\tup{x})$, then $T\Vdash_H\theta(\tup{x})$
and so $T\Vdash\theta_H(\lambda tW_H^\theta(\tup{x},t);\tup{x})$.

We proceed by induction on the complexity of $\theta$
to define $W^\theta_S$ and $W^\theta_H$ and show that
$$T\Vdash\theta_H(\lambda tW_H^\theta(\tup{x},t);\tup{x})
	\Rightarrow   T\Vdash\theta(\tup{x})
	\Rightarrow  T\Vdash\theta_S(\lambda tW_S^\theta(\tup{x},t);\tup{x}).$$
The construction of $W^\theta_S$ and $W^\theta_H$ will employ only
obviously effective methods together with normal form names as in Definition~\ref{D:admitsNormal}.
Therefore the assumption that $\Forc$ is almost persistent guarantees
that $W^\theta_S$ and $W^\theta_H$ are $T$-local.

\begin{itemize}
\item $\theta(\tup{v})$ is atomic:

Let $W^\theta_S(\tup{v},t)=W^\theta_H(\tup{v},t)=0$.

The statement of the proposition follows trivially
since $\theta_S(W;\tup{x})=\theta_H(W;\tup{x})=\theta(\tup{x})$.

\item $\theta(\tup{v})=\neg\phi(\tup{v})$:

Let $W^\theta_S(\tup{v},t)=W^\phi_H(\tup{v},t)$ and
$W^\theta_H(\tup{v},t)=W^\phi_S(\tup{v},t)$.

Suppose that $T\Vdash\theta_H(\lambda tW_H^\theta(\tup{x},t);\tup{x})$.
Then there is no $T'\leq T$ such that
$T'\Vdash\phi_S(\lambda tW_H^\theta(\tup{x},t);\tup{x})$.
Thus there is no $T'\leq T$ such that
$T'\Vdash\phi_S(\lambda tW_S^\phi(\tup{x},t);\tup{x})$.
By the induction hypothesis, if $T'\Vdash\phi(\tup{x})$, then
$T'\Vdash\phi_S(\lambda tW_S^\phi(\tup{x},t);\tup{x})$.
Therefore we conclude that $T\Vdash\theta(\tup{x})$.

Suppose that $T\Vdash\theta(\tup{x})$.
Therefore there is no $T'\leq T$ such that $T'\Vdash\phi(\tup{x})$.
By the induction hypothesis, if
$T'\Vdash\phi_H(\lambda tW_H^\phi(\tup{x},t);\tup{x})$
then $T'\Vdash\phi(\tup{x})$.
Therefore $T\Vdash\neg\phi_H(\lambda tW_H^\phi(\tup{x},t);\tup{x})$,
and since $W^\phi_H=W^\theta_S$, $T\Vdash\theta_S(\lambda tW_S^\theta(\tup{x},t);\tup{x})$.

\item Suppose that $\theta(\tup{v})=\phi(\tup{v})\land\psi(\tup{v})$.

Let $W^\theta_S(\tup{v},2t)=W^\phi_S(\tup{v},t)$ and
$W^\theta_S(\tup{v},2t+1)=W^\psi_S(\tup{v},t)$.
Let $W^\theta_H(\tup{v},t)=W^\phi_H(\tup{v},t)$ if $U_\phi(\tup{v})=y$ for some $y\neq 0$,
and let $W^\theta_H(\tup{v},t)=W^\psi_H(\tup{v},t)$ if $U_\phi(\tup{v})=0$
(where $U_\phi$ is the normal form name, as in Proposition~\ref{P:PR:normalForm}).

Suppose that $T\Vdash\theta_H(\lambda tW_H^\theta(\tup{x},t);\tup{x})$.
Therefore $T\Vdash\phi_H(\lambda tW_H^\theta(\tup{x},t);\tup{x})$ and
$T\Vdash\psi_H(\lambda tW_H^\theta(\tup{x},t);\tup{x})$.
There are two cases.
In the first case we assume that $T\Vdash\phi(\tup{x})$.
In this case $T\Vdash U_\phi(\tup{x})=0$, so
$W^\theta_H(\tup{x},t)=W^\psi_H(\tup{x},t)$.
Therefore $T\Vdash\psi_H(\lambda tW_H^\psi(\tup{x},t);\tup{x})$,
and so by the induction hypothesis $T\Vdash\psi(\tup{x})$.
Thus $T\Vdash\theta(\tup{x})$.
In the other case we assume that $T\not\Vdash\phi(\tup{x})$.
Therefore there is some $T'\leq T$ such that $T'\Vdash\neg\phi(\tup{x})$,
and so $T'\Vdash U_\phi(\tup{x})\neq 0$.
Then $W^\theta_H(\tup{x},t)=W^\phi_H(\tup{x},t)$,
and $T'\Vdash\phi_H(\lambda tW_H^\phi(\tup{x},t);\tup{x})$.
By the induction hypothesis, $T'\Vdash\phi(\tup{x})$, a contradiction.

Now suppose that $T\Vdash\theta(\tup{x})$.
Therefore $T\Vdash\phi(\tup{x})$ and $T\Vdash\psi(\tup{x})$.
By the induction hypothesis, $T\Vdash\phi_S(\lambda tW_S^\phi(\tup{x},t);\tup{x})$
and $T\Vdash\psi_S(\lambda tW_S^\psi(\tup{x},t);\tup{x})$.
It follows that $T\Vdash\theta_S(\lambda tW_S^\theta(\tup{x},t);\tup{x})$.

\item Suppose that $\theta(\tup{v})=\forall w\leq F(\tup{v})\phi(\tup{v},w)$.

Let $W^\theta_S(\tup{v},t)$ be $W^\phi_S(\tup{v},\first{t},\lfloor(\second{t})/2\rfloor)$
if $\first{t}\leq F(\tup{v})$, and $0$ otherwise.
Let $W^\theta_H(\tup{v},0)=\sum_{w\leq F(\tup{v})}X(\tup{v},w)$,
where $X(\tup{v},w)=1\dotminus \sum_{u\leq w}U_\phi(\tup{v},u)$.
In other words, $W^\theta_H(\tup{v},0)$ is the first $w\leq F(\tup{v})$
such that $\phi(\tup{v},w)$ fails, if such a $w$ exists.
Otherwise $W^\theta_H(\tup{v},0)=F(\tup{v})+1$.
We then define $W^\theta_H(\tup{v},t)$ inductively
by letting $W^\theta_H(\tup{v},t+1)$ be $W^\phi_H(\tup{v},W^\theta_H(\tup{v},0),t)$
if $W^\theta_H(\tup{v},0)\leq F(\tup{v})$, and $0$ otherwise.

Suppose that
$$T\Vdash\theta_H(\lambda tW_H^\theta(\tup{x},t);\tup{x}).$$
Removing the shorthand,
$$\theta(\tup{v})=\forall w\neg[w\leq F(\tup{v})\land\neg\phi(\tup{v},w)].$$
Let $\psi(\tup{v},w)=\neg[w\leq F(\tup{v})\land\neg\phi(\tup{v},w)]$
so that $\theta(\tup{v})=\forall w\psi(\tup{v},w)$.
Then we have that $\theta_H(W;\tup{v})=\psi_H(\lambda tW(t+1);\tup{v},W(\tup{v},0))$
and so
$$\theta_H(\lambda tW_H^\theta(\tup{x},t);\tup{x})
	=\psi_H(\lambda tW_H^\psi(\tup{x},W^\theta_H(\tup{x},0),t);\tup{x},W^\theta_H(\tup{x},0)).$$
By the induction hypothesis $T\Vdash\psi(\tup{x},W^\theta_H(\tup{x},0))$.
Therefore there is no $T'\leq T$ such that
$T'\Vdash W^\theta_H(\tup{x},0)\leq F(\tup{x})$ and $T'\Vdash\neg\phi(\tup{x},W^\theta_H(\tup{x},0))$.
In other words, for all $T'\leq T$, if $T'\Vdash W^\theta_H(\tup{x},0)\leq F(\tup{x})$
then there is a $T''\leq T'$ such that $T''\Vdash\phi(\tup{x},W^\theta_H(\tup{x},0))$.
Therefore no such $T'$ exists, since otherwise we would have a condition $T''\leq T$
such that $T'\Vdash W^\theta_H(\tup{x},0)\leq F(\tup{x})$ and
$T''\Vdash\phi(\tup{x},W^\theta_H(\tup{x},0))$, contradicting how $W^\theta_H(\tup{x},0)$ was defined.
Therefore there is no $T'\leq T$ such that $T'\Vdash W^\theta_H(\tup{x},0)\leq F(\tup{x})$.
Notice that if there was a $w$ and a $T'\leq T$ such that
$T'\Vdash[w\leq F(\tup{x})\land\neg\phi(\tup{x},w)]$, then $T'\Vdash W^\theta_H(\tup{x},0)\leq F(\tup{x})$.
Therefore for all $w$ there is no $T'\leq T$ such that $T'\Vdash[w\leq F(\tup{x})\land\neg\phi(\tup{x},w)]$.
We conclude that $T\Vdash\theta(\tup{x})$.

Now suppose that $T\Vdash\theta(\tup{x})$.
Again we write $\theta(\tup{v})=\forall w\psi(\tup{v},w)=\neg[w\leq F(\tup{v})\land\neg\phi(\tup{v},w)]$.
In order to show that $T\Vdash\theta_S(W^\theta_S(\tup{x},t);\tup{x})$,
it suffices to show the following: for all $w$ and all $T'\leq T$,
if $T'\Vdash w\leq F(\tup{x})$ then there is a $T''\leq T'$ such that
$T''\Vdash\phi_S(\lambda tW^\theta_S(\tup{x},\langle w,t\rangle),\tup{x},w)$.
(This is just a matter of definition unpacking, albeit a good bit of it.)
Therefore we fix $w$ and $T'\leq T$ and find the appropriate $T''\leq T'$.
By assumption, if $T'\Vdash w\leq F(\tup{x})$,
then there is a $T''\leq T$ such that $T''\Vdash\phi(\tup{x},w)$.
By the induction hypothesis, $T''\Vdash\phi_S(\lambda tW^\phi_S(\tup{x},w,t);\tup{x},w)$.
It therefore remains to show that
$T''\Vdash\lambda tW^\phi_S(\tup{x},w,t)=\lambda tW^\theta_S(\tup{x},\langle w,2t+1\rangle)$
whenever $T''\Vdash w\leq F(\tup{x})$.
If $w\leq F(\tup{x})$ then
$$W^\theta_S(\tup{x},\langle w,2t+1\rangle)
	=W^\phi_S(\tup{x},w,\lfloor(2t+1)/2\rfloor)
	=W^\phi_S(\tup{x},w,t).$$
\end{itemize}
\end{proof}

We now prove the existence of $\BPi^0_1$ Skolem names.

\begin{prop}
Let $\Forc$ be an almost persistent notion of forcing.
Let $\theta(\tup{v})$ be a $\BPi^0_1$ formula of the forcing language.
There is a name $W^\theta_S(\tup{v},t)$
such that for all $T\in\Forc$ and all $\tup{x}$, if $\theta(\tup{x})$ is $T$-local,
then so is $W^\theta_S(\tup{v},t)$, and
\begin{align*}
T\Vdash\theta(\tup{x})
	\Leftrightarrow  T\Vdash\theta_S(\lambda tW_S^\theta(\tup{x},t);\tup{x}).
\end{align*}
\end{prop}
\begin{proof}
Note that the backwards direction follows from Proposition~\ref{P:easySkolemEquiv}.

Let $\theta(\tup{v})=\forall w\phi(\tup{v},w)$ where $\phi(\tup{v},w)$ is bounded.
We define $W^\theta_S(\tup{v},t)$ by $W^\theta_S(\tup{v},t)=W^\phi_S(\tup{v},\first{t},\second{t})$,
where $W^\phi_S$ is defined as in Proposition~\ref{P:witnessBnded}.

Assuming that $T\Vdash\theta(\tup{x})$, we show that $T\Vdash\theta_S(W^\theta_S(\tup{x},t);\tup{x})$.
In other words, we must show that $T\Vdash\phi_S(W^\theta_S(\tup{x},\langle w,t\rangle);\tup{x},w)$
for all $w$.
By assumption $T\Vdash\phi(\tup{x},w)$ for all $w$, so by
Proposition~\ref{P:witnessBnded} $T\Vdash\phi_S(\lambda tW^\phi_S(\tup{x},w,t);\tup{x},w)$ for all $w$.
Therefore the proposition follows from the fact that
$W^\theta_S(\tup{x},\langle w,t\rangle)=W^\phi_S(\tup{x},w,t)$.
\end{proof}

We now prove the existence of $\BSigma^0_1$ Skolem names.

\begin{prop}[\RCAo]\label{P:witnessSigma1}
Let $\Forc$ be an almost persistent notion of forcing.
Let $\theta(\tup{v})$ be a $\BSigma^0_1$ formula of the forcing language.
There is a name $W^\theta_S(\tup{v},t)$
such that for all $T\in\Forc$ and all $\tup{x}$, if $\theta(\tup{x})$ is $T$-local
and $T\Vdash\theta(\tup{x})$, then $W^\theta_S(\tup{v},t)$ is also $T$-local
and there is a $T'\leq T$ such that $T'\Vdash\theta_S(\lambda tW_S^\theta(\tup{x},t);\tup{x})$.

If $\Forc$ is persistent, then we may assume that $T'=T$.
\end{prop}

\begin{proof}
Let $\theta(\tup{v})=\exists w\varphi(\tup{v},w)$ where $\varphi(\tup{v},w)$ is bounded.
Define a name $X(\tup{v},w)=1\dotminus\sum_{u\leq w}U_{\neg\varphi}(\tup{v},u)$,
where $U_{\neg\varphi}$ is the normal form name, as in Proposition~\ref{P:PR:normalForm}.
Note that $X(\tup{v},w)$ can only switch from 1 to 0 once as $w$ increases.
We now let $W^\theta_S(\tup{v},0)=y$ if $y=\sum_{w\leq y}X(\tup{v},w)$.
In other words $W^\theta_S(\tup{v},0)$ is the least $w$ such that
$\varphi(\tup{v},w)$ holds, if such any such $w$ exists
(and is undefined otherwise).
Finally, we let $W^\theta_S(\tup{v},t+1)=W^\varphi_S(\tup{v},W^\theta_S(\tup{v},0),t)$.

Now we show that $W^\theta_S(\tup{x},t)$ is $T$-local.
Let $T'\leq T$.
There is a condition $S\leq T'$ and a $w$ such that $S\Vdash\varphi(\tup{x},w)$.
Since $\Forc$ is persistent, there is a $\sigma\in S$
such that $S_\sigma\in\Forc$ and
$(1\dotminus U_{\neg\varphi})^\sigma(\tup{x},w)=0$.
Therefore $W^\theta_S(\tup{x},0)$ is defined on everything in $[S_{\sigma}]$.
Since locality is closed under primitive recursion (Proposition~\ref{P:comp&pr}),
$W^\theta_S(\tup{x},t)$ is $S_\sigma$-local, so
$[S_\sigma]\cap\dom(W^\theta_S(\tup{x},t))\neq\emptyset$.
Therefore $W^\theta_S(\tup{x},t)$ is $T$-local.

Now assume that $T\Vdash\theta(\tup{x})$ and show that
$T\Vdash\theta_S(W^\theta_S(\tup{x},t);\tup{x})$.
Let $W_0=W^\theta_S(\tup{x},0)$.
Notice that
$$\begin{array}{lll}
\theta_S(W^\theta_S(\tup{x},t);\tup{x}) & = & \neg\neg\varphi_S(W^\theta_S(\tup{x},t+1);\tup{x},W_S^\theta(\tup{x},0))\\
& = & \neg\neg\varphi_S(W^\varphi_S(\tup{x},W_0,t);\tup{x},W_0).
\end{array}$$
By assumption, for all $T'\leq T$ there is a $w$ and a $T''\leq T'$
such that $T''\Vdash\varphi(\tup{x},w)$.
Therefore $T''\Vdash U_{\neg\varphi}(\tup{x},w)=y$ for some $y>0$,
and so $W^\theta_S(\tup{x},0)$ is defined on everything in $[T'']\cap\dom(U_\varphi)$.
Notice that $T''\Vdash\varphi(\tup{x},W_0)$.
By Proposition~\ref{P:witnessBnded} we have that
$T''\Vdash\varphi_S(W^\varphi_S(\tup{x},W_0,t);\tup{x},W_0)$.
This implies that $T''\Vdash\neg\neg\varphi_S(W^\varphi_S(\tup{x},W_0,t);\tup{x},W_0)$.
Note that $W^\theta_S(\tup{x},t+1)=W^\varphi_S(\tup{x},W_0,t)$.
Therefore we have shown that for all $T'\leq T$ there is a $T''\leq T$
such that $T''\Vdash\theta_S(W^\theta_S(\tup{x},t);\tup{x})$.
In other words, $T\Vdash\neg\neg\theta_S(W^\theta_S(\tup{x},t);\tup{x})$.
If $\Forc$ is persistent, then by Proposition~\ref{P:2neg} (the double negation rule),
$T\Vdash\theta_S(W^\theta_S(\tup{x},t);\tup{x})$.
Otherwise, by unraveling the definition of negation, there is a $T'\leq T$
such that $T'\Vdash\theta_S(W^\theta_S(\tup{x},t);\tup{x})$.
\end{proof}

We now improve Proposition~\ref{P:witnessSigma1}
by extending it to $\BPi^0_2$ formulas.

\begin{prop}\label{P:witnessPi2}
Let $\Forc$ be an almost persistent notion of forcing.
Let $\theta(\tup{v})$ be a $\BPi^0_2$ formula of the forcing language.
There is a name $W^\theta_S(\tup{v},t)$
such that for all $T\in\Forc$ and all $\tup{x}$, if $\theta(\tup{x})$ is $T$-local
and $T\Vdash\theta(\tup{x})$, then $W^\theta_S(\tup{v},t)$ is also $T$-local
and there is a $T'\leq T$ such that
$T'\Vdash\theta_S(\lambda tW_S^\theta(\tup{x},t);\tup{x})$.

If $\Forc$ is persistent, then we may assume that $T'=T$.
\end{prop}
\begin{proof}
Let $\theta(\tup{v})=\forall w\varphi(\tup{v},w)$ where $\varphi(\tup{v},w)$ is $\BSigma^0_1$.
We define $W^\theta_S(\tup{x},t)$ by
$W^\theta_S(\tup{x},t)=W^\varphi_S(\tup{v},\first{t},\second{t})$,
where $W^\varphi_S$ is defined as in Proposition~\ref{P:witnessSigma1}.
The proposition then holds for the $T'\leq T$ given by Proposition~\ref{P:witnessSigma1}.
\end{proof}

\begin{cor}\label{C:Pi1unif}
Let $\Forc$ be an almost persistent notion of forcing.
Let $\theta(\tup{v},w)$ be a $\BSigma^0_1$ formula of the forcing language
such that $T\Vdash\forall\tup{v}\exists w\theta(\tup{v},w)$.
There is a $T$-local name $W(\tup{v})$
and a $T'\leq T$ such that
$T'\Vdash\forall\tup{v}\theta(\tup{v},W(\tup{v}))$.

If $\Forc$ is persistent, then we may assume that $T'=T$.
\end{cor}
