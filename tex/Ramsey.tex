Ramsey's theorem \index{Ramsey's theorem} is often thought of as
a generalization of the pigeonhole principle.
Let $\NN$ denote the set of natural numbers.
Given a number $n$, let $[\NN]^n$ denote the
set of subsets of $\NN$ of size $n$.
The infinite version of Ramsey's theorem \cite{Ramsey}
(which we will refer to as Ramsey's theorem)
says that for every $n,k\geq 1$, and every map
$c$ from $[\NN]^n$ to the finite set $\{0,1,\ldots,k-1\}$,
there is an infinite set $H\subseteq\NN$ such that
$c$ is constant when restricted to $[H]^n$.
The map $c$ is referred to as a \textit{coloring} of $[\NN]^n$,
the finite set $\{0,1,\ldots,k-1\}$ is referred to as the set of
\textit{colors} of $c$, and the set $H$ is referred to as a
\textit{monochromatic} or \textit{homogeneous} set for $c$.
Sometimes we will fix a particular number $n$ and
refer to Ramsey's theorem for $n$-tuples (or pairs, etc).

Ramsey's Theorem was of interest in mathematical logic
even before Reverse Mathematics was around.
In 1971 Specker proved that there is a computable coloring
$c:[\NN]^2\to\{0,1\}$ that has no computable monochromatic set \cite{Specker:1,Specker:2}.
This corresponds to the fact in Reverse Mathematics that
Ramsey's theorem for pairs cannot be proved in \RCAo.
Many other interesting results about Ramsey's theorem
were proved by Jockusch in 1972 \cite{Jockusch:Ramsey}.
The results of Jockusch were later used by Simpson
to show that for any $n\geq 3$, Ramsey's theorem
for $n$-tuples is equivalent, over \RCAo, to \ACAo.
Hirst showed that Ramsey's theorem for singletons ($n=1$)
is equivalent to the bounding principle $\Bnd{\BSigma^0_2}$ \cite{Hirst:thesis}.
A definition of $\Bnd{\BSigma^0_2}$ will be given in Section~\ref{ElemIndec}.
Surprisingly, characterizing Ramsey's theorem for pairs remains an active
area of research \cite{SeetapunSlaman,CJS,HirschfeldtShore,DzhafarovHirst,DzhafarovJockusch}.
Much of what is known about Ramsey's theorem for
pairs was proved using the ideas of set theoretic forcing.

In Chapters~\ref{Posets} and \ref{Ords} we consider two different
generalizations of Ramsey's theorem.
In Chapter~\ref{Forcing} we consider a general framework
for making sense of forcing in Reverse Mathematics and prove
some conservation results.
