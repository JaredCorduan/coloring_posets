Reverse Mathematics is a framework for asking and answering
questions of interest in the foundations of mathematics.
To quote Stephen Simpson from his encyclopedic reference book
on the subject, Reverse Mathematics is primarily motivated
by the following question:
\begin{quote}
Which set existence axioms are needed to
prove the theorems of ordinary, non-set-theoretic mathematics?
(page 2 of \cite{SOSOA}.)
\end{quote}
The words `ordinary' and `non-set-theoretic' are
not being used in a precise sense here,
though they are nonetheless very important.

Insofar as anything in mathematics can be seen as having
a definite beginning, the Reverse Mathematics program began
in 1975 with with Harvey Friedman's
``Some systems of second order arithmetic and their use"
\cite{Friedman:SomeSystems}.
Many of the results in Reverse Mathematics were foreshadowed
by results in Recursion Theory, and the two fields still
enjoy a close relationship today.
See Rogers \cite{Rogers} for the Recursion Theory notation
and definitions implicitly used in anything that follows.

Reverse Mathematics exists inside second order arithmetic.
The language of second order arithmetic is a two-sorted language
which has variable symbols intended to range over the natural numbers,
and set variables which are intended to range over \textit{sets} of natural numbers.
The language also includes the constant symbols 0 and 1, the function symbols
$+$ and $\cdot$, the predicates $=$, $<$, and $\in$, and the usual logical
symbols (including quantifiers for both number and set variables).
Full second order arithmetic is the theory whose axioms are those of
Robinson Arithmetic together with comprehension and induction axioms for all formulas.

Full second order arithmetic suffices to prove the majority
of the theorems in non-set-theoretic countable mathematics.
We use the term `countable mathematics' in a very broad sense
here, for though continuous real-valued functions are
third-order objects, we can make use of the fact that
the rationals are dense in the reals to state
theorems involving continuous real-valued functions in
the language of second order arithmetic.
Similarly, a lot of mathematics can be phrased
in the language of second order arithmetic.

In order to make precise and meaningful the motivating question of Reverse Mathematics,
namely which set existence axioms are needed to
prove the theorems of ordinary, non-set-theoretic mathematics,
an important subsystem of second order arithmetic,
namely \RCAo, is used as a baseline.
Intuitively, \RCAo\ embodies computable/recursive mathematics.
In fact, the acronym stands for ``Recursive Comprehension Axiom"
(and the subscript indicates that induction is limited).
The axioms of \RCAo\ consist of Robinson Arithmetic together with
comprehension axioms for the $\BDelta^0_1$ definable sets
and induction axioms for $\BSigma^0_1$ formulas.
Many theorems are not provable in \RCAo,
such as the statement that every countable vector space
over the rationals has a basis.
Taking this example, we can ask what axioms must be added
to \RCAo\ in order to prove the statement about vector spaces.
A \textit{very} unsatisfying answer is that we could
add the statement in question to \RCAo.
A satisfying answer is that adding all instances of arithmetic
comprehension suffices to prove this theorem \cite{SOSOA}.
The subsystem just mentioned, whose axioms are those of \RCAo\ together
with arithmetic comprehension, is also an important subsystem and is
called \ACAo\ (for arithmetic comprehension).
In what way could we say that \ACAo\ is the weakest subsystem that
proves that every countable vector space over the rationals has a basis?
It turns out that the axioms of \RCAo, together with the statement
that every countable vector space over the rationals has a basis,
suffice to prove all instances of arithmetic comprehension.
In other words, \RCAo\ proves the equivalence of \ACAo\ and
``every countable vector space over the rationals has a basis".

The fascinating thing about Reverse Mathematics is that the large
majority of the theorems of ordinary, non-set-theoretic mathematics turn out to
either be provable in \RCAo\ or be equivalent, over \RCAo, to one of
four natural subsystems of second order arithmetic.
To see many of these equivalences, and
for a more thorough introduction to Reverse Mathematics, see Simpson's
\textit{Subsystems of Second Order Arithmetic} \cite{SOSOA}.
