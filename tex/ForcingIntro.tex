Forcing is a powerful technique from set theory that
has been used in Reverse Mathematics to build models of
subsystems of second order arithmetic.
In particular, variants of Mathias forcing have been
used to build models of $\RCAo+RT^2_2$ which \textit{are not}
models of $\ACAo$, therefore demonstrating that $\RCAo+RT^2_2$
is a strictly weaker subsystem than $\ACAo$ \cite{CJS}.
When forcing techniques have been used in Reverse Mathematics,
however, they have typically been used in a very recursion-theoretic fashion.
There are good reasons for this, since many of the definitions
used in set-theoretic forcing may not make sense when restricted to $\RCAo$.
Moreover, the Truth Lemma from set theory can fail over models of $\RCAo$,
due to restricted comprehension axioms.
We will have more to say about the truth lemma in Section~\ref{generic}.

In his paper ``A variant of Mathias forcing that
preserves \ACAo", Dorais guides the reader through a
interesting forcing construction \cite{varMathias}.
Much of the forcing construction has been tailor-made to work
in Reverse Mathematics, though the spirit is
still very much in line with set-theoretic forcing.
In what follows, we use the framework that Dorais
built in \cite{varMathias} to examine other notions of forcing.

In Section~\ref{ForcingBasics} we give the basic definitions,
following Dorais \cite{varMathias}.
In Section~\ref{witRCAo} we examine a collection of forcings
that preserve \RCAo.
The propositions in Section~\ref{witRCAo} are relatively straightforward generalizations
of the propositions in \cite{varMathias}.
In Section~\ref{witACAo} we examine a collection of notions of forcings
that preserve \ACAo.
In Section~\ref{generic} we examine the generic extension.
This section is also comprised of relatively straightforward generalizations
of the propositions in \cite{varMathias}.
In Section~\ref{forcingExamples} we look at examples.
