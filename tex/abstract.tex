We study two themes from Reverse Mathematics.
The first theme involves a generalization of the infinite version of
Ramsey's theorem to arbitrary partial orderings.
We say that a partial ordering $\Po$ has the $(n,k)$-Ramsey property,
and write $RT^n_k(\Po)$,
if for every $k$-coloring of the $n$-element chains of $\Po$
there is a homogeneous copy of $\Po$.

When $\Po$ is either a linear ordering or a tree, and $n\geq 3$, the statement
\newline $(\forall k\geq 1 )RT^n_k(\Po)$ is well understood
from the point of view of Reverse Mathematics \cite{CHM}\cite{CGM}.
We investigate $RT^n_k(\Po)$ for some partial orderings which are not trees.
We show that if $\Po$ is either the binary tree with multiplicities
or an amenable partial ordering, and if $n\geq 3$,
then the statement $(\forall k\geq 1 )RT^n_k(\Po)$
is equivalent to \ACAo\ over \RCAo.
We also classify which suborderings of
the binary tree with multiplicities have the Ramsey property.
Finally, we study the $(1,k)$-Ramsey property for the finite (ordinal) powers of $\omega$.
For these orderings it makes sense to consider a first-order definition
of ``an isomorphic copy of $\omega^n$'' and the corresponding version of $\forall k RT^1_k(\omega^n)$,
which we denote by $\EIndec^n$.
We place a lower bound on the complexity of $\EIndec^{n+1}$ by showing that it
is provable in $\RCAo+\Bnd{\Pi^0_n}$.
Jointly with Dorais, we show that $\RCAo+\Ind{\Sigma^0_{n+1}}$ proves $\EIndec^{n}$
and also that $RT^1_2(\omega^3)$ is equivalent to $\ACAo$ over $\RCAo$.

The second theme of our study involves set theoretic forcing over
models of \RCAo\ and \ACAo.
Our primary focus is on notions of forcing whose conditions are
subtrees of $\OO$ which are ordered by inclusion and have a simple property
that we call ``persistence".
In his paper ``A variant of Mathias forcing that
preserves \ACAo", Dorais guides the reader through an
interesting forcing construction \cite{varMathias}.
We use Dorais' framework and show that persistent notions of forcing
over models of \ACAo\ which satisfy a particular coloring property
give rise to generic extensions which also model \ACAo.
We also show that a slightly less restrictive property than persistence
suffices to guarantee that generic extensions of models of \RCAo\
are themselves models of \RCAo.
Lastly, we work through several examples:
Harrington, random, Sacks, Silver, and Miller forcing.
