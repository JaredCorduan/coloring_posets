The \textit{conditions for Miller forcing} are superperfect subtrees of $\omega^{<\omega}$.
In other words, perfect subtrees of $\omega^{<\omega}$ such that
every splitting node is infinitely splitting.
Notice that Miller forcing is persistent.

\begin{prop}[\ACAo]\label{P:M:MCP}
Miller forcing satisfies \MCP.
\end{prop}
\begin{proof}
Given a Miller condition $S$ and a $\sigma\in S$,
we let $Sp(S,\sigma)$ be the set of all immediate successors of $\tau$,
where $\tau$ is the first splitting node above $\sigma$.

Let $T$ be a Miller condition and $\langle c_i:i\in\NN\rangle$ be an
infinite sequence of monotone colorings $c_i:T\rightarrow \{0,1\}$.
We will construct $T'\leq T$ and a set of layers $X_0,X_1,\ldots$ for $T'$ in stages.

At stage 0 we do the following.
For each $\tau\in Sp(T,\langle\rangle)$,
if there is a $\sigma\in T_\tau$ such that $c_0(\sigma)=1$,
then we place $\sigma$ into $X_0$.
Otherwise, if $c_0(\sigma)=0$ for all $\sigma\in T_\tau$,
then we place $\tau\in X_0$.
This completely defines $X_0$.
We now define a new condition $T^0$ to be the set of all
nodes comparable to some element of $X_0$.

At stage $(n+1)$ we consider all
$\tau\in \bigcup_{\rho\in X_n}Sp(T^n,\rho)$.
If there is a $\sigma\in T^n_\tau$ such that $c_0(\sigma)=1$,
then we place $\sigma$ into $X_{n+1}$.
Otherwise, if $c_0(\sigma)=0$ for all $\sigma\in T^n_\tau$,
then we place $\tau\in X_{n+1}$.
This completely defines $X_{n+1}$.
We now define a new condition $T^{n+1}$ to be the set of all
nodes comparable to some element of $X_{n+1}$.

Finally, we let $T'=\bigcap_n T^n$.
Notice that $T'$ is a Miller condition, $T'\leq T$,
and $X_0,X_1,\ldots$ are a set of layers for $T'$
that satisfy the conclusion of \MCP.
\end{proof}

\begin{lem}\label{L:M:realForc}
For every $\BSigma^0_2$-generic filter over a model of \RCAo\
there exists a generic real for Miller forcing.
\end{lem}

\begin{proof}
By Lemma~\ref{L:Gen:realForc} it
suffices to show that if $T$ is a condition, then
$$\mathcal{D}_T=\{S:S\cap T\text{ is finite }\lor\ S\leq T\}$$
is open dense.

Suppose that $T'\cap T$ contains no superperfect subtree.
If $T'\cap T$ is empty, then we're done.
Otherwise we can choose a $\sigma\in T'\cap T$ such that no
$\tau\supseteq\sigma$ splits infinitely in $T'\cap T$.
We can therefore find a $\tau\supseteq\sigma$ such that
$\tau\in T'\setminus T$.
Then $T'_\tau\leq T'$ and $T'_\tau\cap T$ is finite,
so $T'_\tau\in\mathcal{D}_T$.
\end{proof}

\begin{thm}
Let $\MM$ be a model of \RCAo\ (\ACAo)
and suppose that $G$ is a generic real for $\Forc$
corresponding to a $\BSigma^1_2$-generic filter $\mathcal{G}$ for Miller forcing.

Then $\MM[G]$ is a model of \RCAo\ (\ACAo).
\end{thm}
\begin{proof}
Follows from Propositions \ref{T:presRCA} and Proposition~\ref{T:presACA}.
\end{proof}
