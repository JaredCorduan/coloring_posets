In this section we examine persistent notions of forcing which satisfy
a property we call \MCP.
The main result of this section is Proposition~\ref{P:PacaSigma2SkolemNames}.
Though somewhat technical, Proposition~\ref{P:PacaSigma2SkolemNames} is the key
to showing that if $\MM$ is a model of $\ACAo$ and $\Forc$ is a
persistent notion of forcing satisfying \MCP,
then the generic extension of $\MM$ is also a model of $\ACAo$.

Before we can define \MCP\ we must define what a set of layers for a condition $T$ is.
Two important examples to keep in mind are the levels of a tree
and the splitting levels of a tree.

\begin{definition}
Given a condition $T$, a \textit{set of layers} for $T$ is a
sequence $\seq{X_i:i\in\NN}$ of subsets of $T$ such that
\begin{itemize}
\item for each extension $T'\leq T$ and $i\in\NN$, $X_i\cap T'$ is a maximal antichain of $T'$,
\item if $\tau\in X_{i+1}$ then $\tau$ is a proper extension of some $\sigma\in X_i$.
\end{itemize}
Given a set of layers for $T$ and a $\tau\in T$,
we say that $\tau$ is above layer $n$ if $\tau\supseteq\sigma$
for some $\sigma\in X_n$.
\end{definition}

\begin{definition}
Given a condition $T$, we say that a coloring $c:T\to\{0,1\}$
is \textit{monotone} if $c(\tau)=1$ implies that $c(\sigma)=1$ for
all $\sigma\supseteq\tau$.
\end{definition}

\begin{definition}
\MCP\ (for Monotone coloring principle\index{Monotone coloring principle}\index{$\MCP$})
is the following statement:
For any condition $T$ and any infinite sequence $\langle c_i:i\in\NN\rangle$ of monotone
colorings $c_i:T\rightarrow \{0,1\}$,
there is a condition $T'\leq T$ and a set of layers for $T'$ such that
for every $i$ there is a $k$ such that if $\tau\in T'$ is above layer $k$
then $c_i$ is constant on $T'_\tau$.
\end{definition}

We now prove the existence of $\BSigma^0_2$ Skolem names.

\begin{prop}[\ACAo]\label{P:PacaSigma2SkolemNames}
Let $\Forc$ be a persistent notion of forcing which satisfies \MCP.
Then $\Forc$ has $\BSigma^0_2$ Skolem names.
In other words, for every $T$-local $\BSigma^0_2$ formula $\theta(\tup{v})$ of the
forcing relation there is an extension $T'\leq T$ and a name $W(\tup{v},t)$
such that for all extensions $S\leq T'$ and all $\tup{x}$,
if $S\Vdash\theta(\tup{x})$ then $\lambda tW(\tup{x},t)$
is $S$-local and $S\Vdash\theta_S(\lambda tW(\tup{x},t);\tup{x})$.
\end{prop}

\begin{proof}
Let $\theta(\tup{v})=\exists y\forall z\phi(\tup{v},y,z)$,
where $\phi$ is bounded.
Let $U=U_\phi$ be the normal form name from Proposition~\ref{P:PR:normalForm}.
Consider the sequence of monotone colorings
$c_{\seq{\tup{x},y}}:T\to\{0,1\}$ defined by
$$c_{\seq{\tup{x},y}}(\tau)=\begin{cases}
0&\text{if }(\forall z\leq|\tau|)U(\tup{x},y,z)\not>0,\\
1&\text{otherwise.}
\end{cases}$$
By \MCP\ there is a condition $T'\leq T$ and a set of layers
$X_0,X_1,\ldots$ for $T'$ such that
for every $\seq{\tup{x},y}$ there is a $k$ such that if $\tau\in T'$ is above layer $k$
then $c_{\seq{\tup{x},y}}$ is constant on $T'_\tau$.

We now define $W(\tup{x},t)$ in stages.
At stage $n=\langle \tup{x},y\rangle$ we consider the nodes in $X_n$.
If $W^\tau(\tup{x},0)$ is undefined for $\tau\in X_n$,
and $c_n(\sigma)=0$ for all $\sigma\in T'_\tau$,
then we let $W^\sigma(\tup{x},0)=y$ for every $\sigma\in T'_\tau$,
and we let $W(\tup{x},t+1)=W^\psi_S(\tup{x},W(\tup{x},0),t)$,
where $\psi(\tup{x},w)=\forall a\phi(\tup{x},w,a)$ and
$W^\psi_S$ is defined as in Proposition~\ref{P:witnessPi2}.

We now show that if $S\leq T'$ and $S\Vdash\theta(\tup{x})$,
then $\lambda tW(\tup{x},t)$ is $S$-local.
Since $\Forc$ is persistent, it suffices to show that if $S'\leq S$
then there is a $\sigma\in S'$ such that $W^\sigma(\tup{x},0)$ is defined.
Suppose that no such $\sigma$ exists.
For each $i\in\NN$ let $Y_i=X_i\cap S$.
Since $\seq{X_i:i\in\NN}$ is a set of layers for $T'$,
then $\seq{Y_i:i\in\NN}$ is a set of layers for $S$.
Therefore for every layer $n$ such that $n=\seq{\tup{x},y}$
for some $y$, we have that $c_n(\tau)=1$ for all $\tau\in Y_n$.
Then $S'\Vdash\forall y\exists z(1\dotminus U(\tup{x},y,z)=0)$,
contradicting that $S'\Vdash\theta(\tup{x})$.

Finally, we show that if $S\leq T'$ and $S\Vdash\theta(\tup{x})$,
then $S\Vdash\theta_S(\lambda tW(\tup{x},t);\tup{x})$.
We know that there is a $S'\leq S$ and a $y$ such that
$S'\Vdash\psi(\tup{x},y)$.
Therefore $W^\sigma(\tup{x},0)$ is defined for all $\sigma\in S'$
above layer $\seq{\tup{x},y}$.
Moreover $S'\Vdash\psi(\tup{x},W(\tup{x},0))$.
So by Proposition~\ref{P:witnessPi2},
$S'\Vdash\psi_S(\lambda tW^\psi_S(\tup{x},W(\tup{x},0),t);\tup{x},W(\tup{x},0))$.
Unraveling definitions, we see that $S\Vdash\theta_S(\lambda tW^\theta_S(\tup{x},t);\tup{x})$.
\end{proof}

We now improve Proposition~\ref{P:PacaSigma2SkolemNames}
by extending it to $\BPi^0_3$ formulas.

\begin{prop}[\ACAo]\label{P:PacaPi3SkolemNames}
Let $\Forc$ be a persistent notion of forcing which satisfies \MCP.
Then $\Forc$ has $\BPi^0_3$ Skolem names.
In other words, for every $T$-local $\BPi^0_3$ formula $\theta(v)$ of the
forcing relation there is an extension $T'\leq T$ and a name $W(v,t)$
such that for all extensions $S\leq T'$ and all $x$,
if $S\Vdash\theta(x)$ then $\lambda tW(x,t)$
is $S$-local and $S\Vdash\theta_S(\lambda tW(x,t);x)$.
\end{prop}
\begin{proof}
Let $\theta(\tup{v})=\forall w\phi(\tup{v},w)$ where $\phi(\tup{v},w)$ is $\BSigma^0_2$.
We define $W^\theta_S(\tup{x},t)$ by
$W^\theta_S(\tup{x},t)=W^\phi_S(\tup{v},\first{t},\second{t})$,
where $W^\phi_S$ is defined as in Proposition~\ref{P:PacaSigma2SkolemNames}.
\end{proof}

\begin{cor}\label{C:Pi2unif}
Let $\Forc$ be a persistent notion of forcing which satisfies \MCP.
Let $\theta(\tup{v},w)$ be a $\BSigma^0_2$ formula of the forcing relation
such that $T\Vdash\forall\tup{v}\exists w\theta(\tup{v},w)$.
There is a $T$-local name $W(\tup{v})$ such that
$T\Vdash\forall\tup{v}\theta(\tup{v},W(\tup{v}))$.
\end{cor}
