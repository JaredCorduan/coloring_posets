The \textit{conditions of Harrington forcing} are infinite subtrees of $2^{<\omega}$.

This notion of forcing has been much studied in Reverse Mathematics \cite{SOSOA}.
We choose Harrington forcing as a first example, though nothing
proved here about Harrington forcing is really new.

Note that Harrington forcing is not persistent.
We now show that it is, however, almost persistent.

\begin{lem}[\RCAo]\label{L:H:locallem}
Let $T$ be an infinite subtree of $2^{<\omega}$,
$F$ be a $k$-ary $T$-local name, and $\tup{x}\in\NN^k$.
The tree
$$T_{F(\tup{x})}=\set{\tau\in T}{F^\tau(\tup{x})\text{ is undefined at stage }|\tau|}$$
is finite.
(Recall that we consider $F^\tau(\tup{x})$ undefined at stage $n$ if
there are no $w,y\leq n$ and $\sigma\subseteq\tau$
such that $w$ witnesses that $F^\sigma(\tup{x})=y$.)
\end{lem}
\begin{proof}
Suppose for the sake of contradiction that $T_{F(\tup{x})}$ is infinite.
Then $T_{F(\tup{x})}$ is an extension of $T$.
Therefore, since $F$ is $T$-local, there is a $\tau\in T_{F(\tup{x})}$
such that $F^\tau(\tup{x})$ is defined by stage $|\tau|$.
Moreover, there is an extension $\sigma\supseteq\tau$, such that
$\sigma\in T_{F(\tup{x})}$ and $F^\tau(\tup{x})$ is defined at stage $|\sigma|$,
contradicting the definition of $T_{F(\tup{x})}$.
\end{proof}

Recall that composition and primitive recursion for names
was defined in Definition~\ref{D:comp&pr}.

\begin{prop}[$\RCAo$]\label{P:H:comp&pr}
For any condition $T$,
the names defined by composition and primitive recursion
using $T$-local names are themselves $T$-local.
\end{prop}
\begin{proof}
First we consider composition.
Let $F_0$ be an $\ell$-ary $T$-local name, $F_1,\ldots,F_\ell$ be $k$-ary $T$-local names,
and $H=F_0\circ(F_1,\ldots,F_\ell)$.
We show that $H$ is $T$-local.
In other words, we let $T'\leq T$ and $\tup{x}\in\NN^k$
and show that there is a $\tau\in T'$ such that $H^\tau(\tup{x})$ is defined.

By Lemma~\ref{L:H:locallem}, given a $T$-local name $F$ we can define an increasing function
$B_F(\tup{x})$ such that $F^\tau(\tup{x})$ is defined for all $\tau$
on level $B_F(\tup{x})$ of $T'$.
Let $m_0=\max\{B_{F_1}(\tup{x}),\ldots,B_{F_\ell}(\tup{x})\}$, and
let $R$ be the set of all $\tup{y}\in\NN^\ell$ such that
for some $\tau$ on level $m_0$ of $T'$, $F_i^\tau(\tup{x})=y_i$ for each $1\leq i\leq \ell$.
Let $m_1=\max\set{B_{F_0}(\tup{y})}{\tup{y}\in R}$.
Then $H^\tau(\tup{x})$ is defined for every $\tau$ on level $m_1$ of $T'$.

Now we consider primitive recursion.
Let $F_0$ be an $(k-1)$-ary $T$-local name, $F_1$ be a $(k+1)$-ary $T$-local name,
and $H$ be the $k$-ary name defined by primitive recursion with $F_0$ and $F_1$.
We now show that $H$ is $T$-local.
In other words, we let $T'\leq T$, $\tup{x}\in\NN^{k-1}$, and $y\in\NN$
and we show that there is a $\tau\in T'$ such that $H^\tau(\tup{x},y)$ is defined.

Let $m_0=B_{F_0(\tup{x})}$ and let $R_0$ be the set of all
values $F_0^\tau(\tup{x})$ such that $\tau$ is on level $m_0$ of $T'$.
Let $m_1=\max\set{B_{F_1}(\tup{x},0,z_0)}{z_0\in R_0}$ and let $R_1$ be the set of all
values $F_1^\tau(\tup{x},0,z_0)$ such that $\tau$ is on level $m_0$ of $T'$ and $z_0\in R_0$.
Continuing in this way, we let $m_{i+1}=\max\set{B_{F_1}(\tup{x},i,z_i)}{z_i\in R_i}$
and let $R_{i+1}$ be the set of all
values $F_1^\tau(\tup{x},i,z_i)$ such that $\tau$ is on level $m_0$ of $T'$ and $z_i\in R_i$.
Then $H^\tau(\tup{x},y)$ is defined for all $\tau$ on level $m_y$ of $T'$.
\end{proof}

\begin{prop}[\RCAo]\label{P:H:almost persistent2}
For every $n\in\NN$ and condition $T$ there is a $\tau\in T$ such that
$|\tau|\geq n$ and $T_\tau$ is a condition.
\end{prop}

Note that this proposition is slightly stronger
than the second requirement for being
an almost persistent notion of forcing.

\begin{proof}
For each $m\geq n$ let $L_m$ be the set of elements on level $n$ of $T$
which have an extension on level $m$.
In other words,
$$L_m=\set{\tau\in T}{|\tau|=n\text{ and }(\exists\sigma\in T)(\tau\subseteq\sigma \land |\sigma|=m)}.$$
Notice that $\seq{L_m}_{m\in\NN}$ is a nonincreasing sequence of finite sets.
By $\Ind{\BSigma^0_1}$ there is an $m\geq n$
such that $L_{m'}=L_{m''}$ for all $m',m''\geq m$.
Therefore given any $\sigma\in L_m$,
$T_\sigma$ is infinite.
\end{proof}

\begin{prop}[\RCAo]\label{P:H:almost persistent3}
Let $T$ be an infinite subtree of $2^{<\omega}$,
$F$ be a $k$-ary $T$-local name, and $\tup{x}\in\NN^k$.
There is a $\tau\in T$ such that $F^\tau(x)$ is defined
and $T_\tau$ is infinite.
\end{prop}
\begin{proof}
By Lemma~\ref{L:H:locallem} there is an $n\in\NN$ such that
$F^\tau(\tup{x})$ is defined for all $\tau\in T$ such that $|\tau|=n$.
For each $m\geq n$ let $L_m$ be the set of elements on level $n$ of $T$
which have an extension on level $m$.
In other words,
$$L_m=\set{\tau\in T}{|\tau|=n\text{ and }(\exists\sigma\in T)(\tau\subseteq\sigma \land |\sigma|=m)}.$$
Notice that $\seq{L_m}_{m\in\NN}$ is a nonincreasing sequence of finite sets.
By $\Ind{\BSigma^0_1}$ there is an $m\geq n$
such that $L_{m'}=L_{m''}$ for all $m',m''\geq m$.
Therefore given any $\sigma\in L_m$,
$T_\sigma$ is infinite and $F^\sigma(\tup{x})$ is defined.
\end{proof}

\begin{cor}[\RCAo]\label{C:H:almost persistent}
Harrington forcing is almost persistent.
\end{cor}
\begin{proof}
Follows immediately from Propositions~\ref{P:H:comp&pr}, \ref{P:H:almost persistent2}, and \ref{P:H:almost persistent3}.
\end{proof}

\begin{lem}[\RCAo]\label{L:H:realForc}
For every $\BSigma^0_2$-generic filter over a model of \RCAo\
there exists a generic real for Harrington forcing.
\end{lem}
\begin{proof}
Since $\mathcal{D}_T=\{S:S\cap T\text{ is finite }\lor\ S\leq T\}$ is clearly
open dense for every condition $T$, the lemma follows from Lemma~\ref{L:Gen:realForc}.
\end{proof}

\begin{thm}
Let $\MM$ be a model of \RCAo\
and suppose that $G$ is a generic real for $\Forc$
corresponding to a $\BSigma^1_2$-generic filter $\mathcal{G}$ for Harrington forcing.

Then $\MM[G]$ is a model of $\RCAo$.
\end{thm}
\begin{proof}
Follows from Corollary~\ref{C:H:almost persistent}, Lemma~\ref{L:H:realForc}, and Proposition~\ref{T:presRCA}.
\end{proof}

We finish this example by showing that Harrington
forcing does not add unbounded reals.

\begin{thm}[\RCAo]\label{P:H:bndedNames}
If $F$ is a $T$-local name, then there is a function $B$
such that $$T \Vdash (\forall\tup{v})[F(\tup{v}) \leq \check{B}(\tup{v})].$$
\end{thm}
\begin{proof}
By Lemma~\ref{L:H:locallem} we can define an increasing function
$L(\tup{x})$ such that $F^\tau(\tup{x})$ is defined for all $\tau$
on level $L(\tup{x})$ of $T$.
Let $B(\tup{x})=\max\set{F^\tau(\tup{x})}{|\tau|=L(\tup{x})}$.
Then $B$ satisfies the conclusion of the lemma.
\end{proof}

\begin{cor}
Let $\MM$ be a model of \RCAo\
and suppose that $G$ is a generic real for $\Forc$
corresponding to a generic filter $\mathcal{G}$ for Harrington forcing.

For every function $f:\NN\to\NN$ in $\MM[G]$
there is a function $b$ in $\MM$ such that
$f(x)\leq b(x)$ for all $x$.
\end{cor}
