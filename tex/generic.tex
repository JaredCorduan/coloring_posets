In this section we put the main results of the last two sections to use.
Theorem~\ref{T:presRCA} below says that for almost persistent notions of forcing,
the generic extension of a model of \RCAo\ is itself a model of \RCAo.
Similarly, Theorem~\ref{T:presACA} says that for persistent notions of forcing satisfying \MCP,
the generic extension of a model of \ACAo\ is itself a model of \ACAo.

The key to proving these two theorems is to prove special cases of the Truth Lemma.
Taken together, Proposition~\ref{P:witnessPi2} above and Proposition~\ref{P:presRCAkey} below
tell us that the Truth Lemma holds for $\BPi^0_2$ sentences.
In other words, and stated imprecisely, the $\BPi^0_2$ sentences which hold in
the generic extension are precisely those which are forced.
Similarly, Proposition~\ref{P:PacaPi3SkolemNames} above and Proposition~\ref{P:presACAkey} below
tell us that the Truth Lemma holds for $\BPi^0_3$ sentences.

Let $\MM$ be a model of second-order arithmetic and let $\Forc$ be a notion of forcing.
We say that a collection $\mathcal{D}\subseteq\Forc$ is \textit{dense}\index{dense} if
for every condition $T$ there is an extension $T'\leq T$ such that $T'\in\mathcal{D}$.
Note that $\mathcal{D}$, just like $\Forc$, is a third-order object and
therefore does not belong to $\MM$.
(The elements of $\mathcal{D}$ do, however, belong to $\MM$.)
We say that $\mathcal{D}$ is \textit{open}\index{open} if $T'\in\mathcal{D}$
whenever $T'\leq T\in\mathcal{D}$.
A collection $\mathcal{G}\subseteq\Forc$ is called a \textit{generic filter}\index{generic filter} for $\Forc$ if
\begin{itemize}
\item $\mathcal{G}$ is nonempty,
\item $\mathcal{G}$ is closed upward, meaning that if
		$T,S\in\Forc$, $T\leq S$, and $T\in\mathcal{G}$, then $S\in\mathcal{G}$,
\item any two elements of $\mathcal{G}$ are compatible in $\mathcal{G}$, meaning that if $T,T'\in\mathcal{G}$
		then there is an $S\in\mathcal{G}$ such that $S\leq T$ and $S\leq T'$,
\item and $\mathcal{G}$ meets every open dense collection, meaning that if $\mathcal{D}$ is open dense
		then $\mathcal{G}\cap\mathcal{D}\neq\emptyset$.
\end{itemize}
The first three requirements say that $\mathcal{G}$ is a \textit{filter}.
The last requirement says that $\mathcal{G}$ is \textit{generic}.
We will never actually use full genericity since only
some open dense collections need to be met.
Given a collection of formulas $\Gamma$,
we say that a filter $\mathcal{G}$ is \textit{$\Gamma$-generic}
if it meets every open dense collection which is defined by a formula in $\Gamma$.
In other words, $\mathcal{G}$ meets every open dense collection
$\mathcal{D}$ such that $T\in\mathcal{D}$ if and only if $T\in\Forc$ and $\phi(T)$
holds for some $\phi\in\Gamma$.

\begin{definition}
Let $\MM$ be a model and $\Forc$ be a notion of forcing.
Suppose that $\mathcal{G}$ is a generic filter for $\Forc$.
Let
$$G=\bigcup\Big\{\mathsf{stem}(T):T\in\mathcal{G}\Big\}.$$
We say that $G$ is a \textit{generic real}\index{generic real} for $\Forc$ if
$$T\in\mathcal{G}\ \ \Leftrightarrow\ \ G\in[T].$$
Note that in contrast with Definition~\ref{D:treestuff},
since we are now working outside of a model $\MM$,
we use $[T]$ to denote the set of all branches of $[T]$,
not just those branches in $\MM$.

If $G$ is a generic real for $\Forc$,
we say that a name $F$ is $G$\textit{-local}\index{$G$-local}\index{locality} if $F$ is $T$-local
for some $T\in\mathcal{G}$.
We say that a formula $\theta$ of the forcing language is $G$\textit{-local}
if every name occurring in $F$ is $G$-local.
\end{definition}

Later, in Section~\ref{forcingExamples} (the section on examples),
the following lemma will be useful for proving the existence of a generic real.

\begin{lem}\label{L:Gen:realForc}
Let $\Forc$ be an almost persistent notion of forcing such that
$$\mathcal{D}_T=\{S:S\cap T\text{ is finite }\lor\ S\leq T\}$$
is open dense for every $T\in\Forc$.
Let $\mathcal{G}$ be a $\BSigma^0_2$-generic filter for $\Forc$.
Then $G$ is a generic real for $\Forc$.
\end{lem}

\begin{proof}
For every $n\in\NN$, since $\Forc$ is almost persistent,
$$\mathcal{D}_n=\{S:|\mathsf{stem}(S)|\geq n\}$$
is open dense.

Let $T\in\mathcal{G}$.
For each $n$ there is a $S_n'\in\mathcal{G}\cap\mathcal{D}_n$.
Since $\mathcal{G}$ is a filter, there is a condition $S_n\leq S_n',T$.
Therefore $G\in[T]$.

Suppose now that $G\in[T]$.
Since $\mathcal{D}_T$ is open dense and $\BSigma^0_2$-definable,
there is a condition $S\in\mathcal{G}$ such that
either $S\cap T$ is finite or $S\leq T$.
But if $S\cap T$ is finite then $G\in[S]\setminus[T]$, a contradiction.
Therefore $S\leq T$ and $S\in\mathcal{G}$, so $T\in\mathcal{G}$ since
$\mathcal{G}$ is a filter.
\end{proof}

\begin{prop}\label{P:genericGivesTotFncs}
Let $\Forc$ be an almost persistent forcing.
Let $\mathcal{G}$ be a $\BSigma^0_1$-generic filter and
suppose that $G$ is a generic real for $\Forc$ corresponding to $\mathcal{G}$.
If $F$ is a $G$-local name then
$$F^G(\tup{x})=y\ \Leftrightarrow\ \exists nF^{G\uhr n}(\tup{x})=y$$
defines a total $k$-ary function.
\end{prop}
\begin{proof}
Let $F$ be $T$-local for some $T\in\mathcal{G}$.
Fix $\tup{x}\in\NN^k$ and $T'\leq T$.
Consider the collection $\mathcal{D}$ of all $S\in\Forc$
such that either $S\not\leq T$ or
$F^\tau(\tup{x})$ is defined by stage $|\tau|$,
where $\tau=\mathsf{stem}(S)$.
Note that $\mathcal{D}$ is $\BSigma^0_1$-definable.

Let $S\leq T$.
Since $F$ is $S$-local and $\Forc$ is almost persistent,
there is a $\tau\in S$ such that $F^\tau(\tup{x})$ is defined by stage $|\tau|$
and $S_\tau\in\Forc$.
Therefore $\mathcal{D}$ is open dense.
Therefore there is a $T'\in\mathcal{G}$ such that
$T'\leq T$ and $F^\tau(\tup{x})$ is defined by stage $|\tau|$,
where $\tau=\mathsf{stem}(S)$.
Therefore $F^G(\tup{x})$ defines a total $k$-ary function.
\end{proof}

We now use Proposition~\ref{P:genericGivesTotFncs} to define the generic extension.

\begin{definition}
Let $\MM$ be a model of \RCAo, $\Forc$ an almost persistent notion of forcing,
and $G$ a generic real for $\Forc$
corresponding to a $\BDelta^0_2$-generic filter $\mathcal{G}$.
We define the \textit{generic extension}\index{generic extension} of $\MM$,
denoted $\MM[G]$\index{$\MM[G]$}, to be the extension of $\MM$
whose sets are
$$\{\mathsf{zeros}(F^G):F\text{ is a }G\text{-local name}\},$$
where
$$\mathsf{zeros}(F^G)=\set{x}{F^G(x)=0},$$
and whose first-order part is the same as $\MM$.
\end{definition}

\begin{definition}
Let $\MM$ be a model  of \RCAo, $\Forc$ an almost persistent notion of forcing, and $G$ a generic real for $\Forc$
corresponding to a $\BDelta^0_2$-generic filter $\mathcal{G}$.
Given a formula $\phi$ of the forcing language,
we let $\phi^G$\index{$\phi^G$} be the formula obtained by replacing
all names $F$ in $\phi$ by $F^G$.

Note that $\phi^G$ is not a formula of the forcing language,
but rather a formula of the language of second-order arithmetic
augmented with constant symbols for elements of $\MM[G]$.
\end{definition}

\begin{prop}\label{P:Pi1ForcedIsTrue}
Let $\MM$ be a model of \RCAo, $\Forc$ an almost persistent notion of forcing,
and suppose that $G$ is a generic real for $\Forc$
corresponding to a $\BDelta^0_2$-generic filter $\mathcal{G}$.
Let $T\in\mathcal{G}$ and let $\phi$ be a $T$-local, $\BPi^0_1$-sentence
of the forcing language such that $T\Vdash\phi$.
Then $\MM[G]\vDash\phi^G$.
\end{prop}
\begin{proof}
Because of the existence of normal form names,
as given by Proposition~\ref{P:PR:normalForm},
we can assume that
$\phi=\forall \tup{v}(U(\tup{v})=0)$, where $U$ is $T$-local.
Since $T\Vdash\phi$, $U^G(\tup{v})=0$ for all $\tup{v}$.
Therefore $\MM[G]\vDash\forall\tup{v}(U^G(\tup{v})=0)$.
\end{proof}

\begin{cor}\label{C:Sforcedistrue}
Let $\MM$ be a model of \RCAo, $\Forc$ an almost persistent notion of forcing,
and suppose that $G$ is a generic real for $\Forc$
corresponding to a $\BDelta^0_2$-generic filter $\mathcal{G}$.
Let $\phi$ be a $T$-local sentence of the forcing language
for some $T\in\mathcal{G}$ ($T$ witnesses that $\phi$ is $G$-local).
If $T\Vdash_S\phi$ then $\MM[G]\vDash\phi^G$.
\end{cor}
\begin{proof}
Follows from Propositions \ref{P:skolemisPi1} and \ref{P:Pi1ForcedIsTrue}.
\end{proof}

We now begin to prove special cases of the Truth Lemma.

\begin{prop}\label{P:presRCAkey}
Let $\MM$ be a model of \RCAo, $\Forc$ an almost persistent notion of forcing,
and suppose that $G$ is a generic real for $\Forc$
corresponding to a generic filter $\mathcal{G}$.
Let $\phi$ be a $G$-local, $\BPi^0_2$-sentence of the forcing language.
Then $\MM[G]\vDash\phi^G$ if and only if there is a $T\in\mathcal{G}$
such that $T\Vdash_S\phi$.

Additionally, if $\Forc$ is either persistent or is almost persistent
and has an arithmetic definition, then $\mathcal{G}$ needs only be
$\BSigma^1_2$-generic.
\end{prop}

\begin{proof}
Let $T$ be a condition witnessing that $\phi$ is $G$-local
(so that $T\in\mathcal{G}$ and $F$ is $T$-local for every name $F$ in $\phi$).
Let $T'\leq T$ be any extension of $T'$.
By the definition of forcing the negation of a sentence,
there is a $T''\leq T'$ such that either $T''\Vdash\phi$ or $T''\Vdash\neg\phi$.

By Proposition~\ref{P:witnessPi2}, if $T''\Vdash\phi$ then
there is a $T'''\leq T''$ such that $T'''\Vdash_S\phi$.
On the other hand, suppose that $T''\Vdash\neg\phi$.
Write $\neg\phi=\exists v \theta(v,w)$, where $\theta(v)$ is $\BPi^0_1$.
Then there is an $S\leq T''$ and an $x$ such that $S\Vdash\theta(x)$.
By Proposition~\ref{P:witnessPi2} there is an $S'\leq S$ such that
$S'\Vdash_S\theta(x)$.

Therefore it is dense below $T$ that either
$T'\Vdash_S\phi$ or $T'\Vdash_S\theta(x)$ for some $x$.
Since $G$ is a generic filter there is an $S\in\mathcal{G}$ such that
$S\leq T$ and either $S\Vdash_S\phi$ or $S\Vdash_S\theta(x)$ for some $x$.
If $S\Vdash_S\phi$, then $\MM[G]\vDash\phi^G$ follows from Corollary~\ref{C:Sforcedistrue}.
If $S\Vdash_S\theta(x)$ for some $x$,
then $\MM[G]\vDash\theta(x)^G$ follows from Corollary~\ref{C:Sforcedistrue},
and so $\MM[G]\nvDash\phi^G$.

Finally, notice that we need only assume that
$\mathcal{G}$ is a $\BSigma^1_2$-generic filter by Corollary~\ref{C:skolemisPi11}.
\end{proof}

\begin{thm}\label{T:presRCA}
Let $\MM$ be a model of \RCAo, $\Forc$ an almost persistent notion of forcing,
and suppose that $G$ is a generic real for $\Forc$
corresponding to a generic filter $\mathcal{G}$.
Then $\MM[G]$ is also a model of $\RCAo$.

Additionally, if $\Forc$ is either persistent or is almost persistent
and has an arithmetic definition, then $\mathcal{G}$ needs only be
$\BSigma^1_2$-generic.
\end{thm}
\begin{proof}
It suffices to show that $\MM[G]$ satisfies
the following uniformization axiom:
\begin{quote}
For every $f:\NN^{k+1}\to\NN$ such that $\forall\tup{w}\exists xf(x,\tup{w})=0$
then there is a $g:\NN^k\to\NN$ such that $\forall\tup{w}f(g(\tup{w}),\tup{w})=0$.
\end{quote}
To see that it suffices to prove this uniformization axiom, see \cite{varMathias}.
Let $H$ be a $G$-local $(k+1)$-ary name such that
$\MM[G]\vDash\forall\tup{w}\exists xH^G(\tup{w},x)=0$.
By Proposition~\ref{P:presRCAkey} there is a $T\in\mathcal{G}$
such that $T\Vdash_S \forall\tup{w}\exists xH(\tup{w},x)=0$.
For any $T'\leq T$, by Proposition~\ref{P:easySkolemEquiv} and Corollary~\ref{C:Pi1unif}
there is a $T'$ local name $W$ and a $T''\leq T'$ and  such that
$T''\Vdash \forall\tup{w}H(\tup{w},W(\tup{w}))=0$.
Therefore it is dense below $T$ that
$S\Vdash \forall\tup{w}H(\tup{w},W(\tup{w}))=0$ for some $S$-local name $W$,
and so there is an $S\in\mathcal{G}$ and an $S$-local name $W$ such that
$S\Vdash \forall\tup{w}H(\tup{w},W(\tup{w}))=0$.
Therefore $\MM[G]\vDash \forall\tup{w}H^G(W^G(\tup{w}),\tup{w})=0$ by Proposition~\ref{P:Pi1ForcedIsTrue}.
\end{proof}

\begin{prop}\label{P:presACAkey}
Let $\MM$ be a model of \ACAo, $\Forc$ a persistent notion of forcing satisfying \MCP,
and suppose that $G$ is a generic real for $\Forc$
corresponding to a $\BSigma^1_2$-generic filter $\mathcal{G}$.
Let $\phi$ be a $G$-local, $\BPi^0_3$-formula of the forcing language.
Then $\MM[G]\vDash\phi^G$ if and only if there is a $T\in\mathcal{G}$
such that $T\Vdash_S\phi$.
\end{prop}
\begin{proof}
Let $T$ be a condition witnessing that $\phi$ is $G$-local
(so that $T\in\mathcal{G}$ and $F$ is $T$-local for every name $F$ in $\phi$).
Let $T'\leq T$ be any extension of $T'$.
By the definition of forcing the negation of a sentence,
there is a $T''\leq T'$ such that either $T''\Vdash\phi$ or $T''\Vdash\neg\phi$.

By Proposition~\ref{P:PacaPi3SkolemNames}, if $T''\Vdash\phi$ then $T''\Vdash_S\phi$.
On the other hand, suppose that $T''\Vdash\neg\phi$.
Write $\neg\phi=\exists v \theta(v,w)$, where $\theta(v)$ is $\BPi^0_1$.
Then there is an $S\leq T''$ and an $x$ such that $S\Vdash\theta(x)$.
By Proposition~\ref{P:witnessPi2} $S\Vdash_S\theta(x)$.

Therefore it is dense below $T$ that either
$S\Vdash_S\phi$ or $S\Vdash_S\theta(x)$ for some $x$.
Since $G$ is a generic filter there is an $S'\in\mathcal{G}$ such that
$S'\leq T$ and either $S'\Vdash_S\phi$ or $S'\Vdash_S\theta(x)$ for some $x$.
If $S'\Vdash_S\phi$, then $\MM[G]\vDash\phi^G$ follows from Corollary~\ref{C:Sforcedistrue}.
If $S'\Vdash_S\theta(x)$ for some $x$,
then $\MM[G]\vDash\theta(x)^G$ follows from Corollary~\ref{C:Sforcedistrue},
and so $\MM[G]\nvDash\phi^G$.

Finally, notice that we need only assume that
$\mathcal{G}$ is a $\BSigma^1_2$-generic filter by Corollary~\ref{C:skolemisPi11}.
\end{proof}

\begin{thm}\label{T:presACA}
Let $\MM$ be a model of \ACAo, $\Forc$ a persistent notion of forcing satisfying \MCP,
and suppose that $G$ is a generic real for $\Forc$
corresponding to a $\BSigma^1_2$-generic filter $\mathcal{G}$.
Then $\MM[G]$ is also a model of $\ACAo$.
\end{thm}
\begin{proof}
It suffices to show that $\MM[G]$ satisfies
the following minimization axiom:
\begin{quote}
For every $f:\NN^{k+1}\to\NN$ there is a $g:\NN^k\to\NN$ such that
$$\forall x\forall\tup{w}[f(x,\tup{w})\geq f(g(\tup{w}),\tup{w})].$$
\end{quote}
To see that it suffices to prove this minimization axiom, see \cite{varMathias}.
Let $H$ be a $G$-local $(k+1)$-ary name.
We use $H(\tup{w},y)\leq H(\tup{w},z)$ as a shorthand for $H(\tup{w},y)\dotminus H(\tup{w},z)=0$.
Since $\MM[G]\vDash \Ind{\BSigma^0_1}$,
$\MM[G]\vDash\forall\tup{w}\exists y\forall z[H^G(\tup{w},y)\leq H^G(\tup{w},z)]$.
By Proposition~\ref{P:presACAkey} there is a $T\in\mathcal{G}$
such that $T\Vdash_S\forall\tup{w}\exists y\forall z[H(\tup{w},y)\leq H(\tup{w},z)]$.
For any $T'\leq T$, by Proposition~\ref{P:easySkolemEquiv} and Corollary~\ref{C:Pi2unif}
there is a $T'$-local name $W$ such that $T'\Vdash\forall\tup{w}\forall z[H(\tup{w},W(\tup{w}))\leq H(\tup{w},z)]$.
Therefore it is dense below $T$ that
$S\Vdash\forall\tup{w}\forall z[H(\tup{w},W(\tup{w}))\leq H(\tup{w},z)]$ for some $S$-local name $W$,
and so there is an $S\in\mathcal{G}$ such that
$S\Vdash\forall\tup{w}\forall z[H(\tup{w},W(\tup{w}))\leq H(\tup{w},z)]$ for some $S$-local name $W$.
Therefore $\MM[G]\vDash\forall\tup{w}\forall z[H^G(\tup{w},W^G(\tup{w}))\leq H^G(\tup{w},z)]$
by Proposition~\ref{P:Pi1ForcedIsTrue}.
\end{proof}
