The conditions of random forcing are the closed subsets of
Cantor space with positive Lebesgue measure.
We now make this precise.

Given a tree $T\subseteq\bin$, let
$$\mu_n(T) = |\{\tau\in T:|\tau|=n\}|\ /\ 2^n.$$
In other words, $\mu_n(T)$ is the number of nodes on level $n$
of $T$, divided by $2^n$.

We say that a tree $T\subseteq 2^{<\omega}$ has \textit{positive measure} if
$$\text{there is an }\epsilon>0\text{ such that } \mu_n(T)>\epsilon \text{ for all }n.$$
If $T$ does not have positive measure, we say that $T$ has \textit{measure zero}.
The \textit{conditions for random forcing} are the trees $T\subseteq 2^{<\omega}$ with positive measure.

Note that for any tree $T\subseteq 2^{<\omega}$,
$\mu_n(T)$ is monotonically decreasing.
From this it follows that if $T$ has measure zero then
$\displaystyle \lim_{n\to\infty} \mu_n(T)=0$.

The first thing to notice about random forcing is that
it fails to be persistent.
In other words, it could be the case that $T_\tau$ does not have positive measure
for some condition $T$ and $\tau\in T$.
We now show that it is, however, almost persistent.

\begin{lem}[\RCAo]\label{L:R:locallem}
Let $T$ be an subtree of $2^{<\omega}$ with positive measure,
$F$ be a $k$-ary $T$-local name, and $\tup{x}\in\NN^k$.
The tree
$$T_{F(\tup{x})}=\set{\tau\in T}{F^\tau(\tup{x})\text{ is undefined at stage }|\tau|}$$
has measure zero.
(Recall that we consider $F^\tau(\tup{x})$ undefined at stage $n$ if
there are no $w,y\leq n$ and $\sigma\subseteq\tau$
such that $w$ witnesses that $F^\sigma(\tup{x})=y$.)
\end{lem}
\begin{proof}
Suppose for the sake of contradiction that $T_{F(\tup{x})}$ has positive measure.
Then $T_{F(\tup{x})}$ is an extension of $T$.
Therefore, since $F$ is $T$-local, there is a $\tau\in T_{F(\tup{x})}$
such that $F^\tau(\tup{x})$ is defined by stage $|\tau|$,
contradicting the definition of $T_{F(\tup{x})}$.
\end{proof}

Recall that composition and primitive recursion for names
was defined in Definition~\ref{D:comp&pr}.

\begin{prop}[$\RCAo$]\label{P:R:comp&pr}
For any condition $T$,
the names defined by composition and primitive recursion
using $T$-local names are themselves $T$-local.
\end{prop}
\begin{proof}
First we consider composition.
Let $F_0$ be an $\ell$-ary $T$-local name, $F_1,\ldots,F_\ell$ be $k$-ary $T$-local names,
and $H=F_0\circ(F_1,\ldots,F_\ell)$.
We show that $H$ is $T$-local.
In other words, we let $T'\leq T$ and $\tup{x}\in\NN^k$
and show that there is a $\tau\in T'$ such that $H^\tau(\tup{x})$ is defined.

By Lemma~\ref{L:H:locallem}, given a $T$-local name $F$ we can define a function
$B_F(\tup{x};\epsilon)=b$ so that $F^\tau(\tup{x})$ is defined by stage $b$ for all
but $(\epsilon\cdot 2^b)$-many $\tau$ on level $b$ of $T'$.
Fix $\epsilon>0$.
Let $m_0=\max\{B_{F_1}(\tup{x};\epsilon/2\ell),\ldots,B_{F_\ell}(\tup{x};\epsilon/2\ell)\}$, and
let $R$ be the set of all $\tup{y}\in\NN^\ell$ such that
for some $\tau$ on level $m_0$ of $T'$, $F_i^\tau(\tup{x})=y_i\leq m_0$ for each $1\leq i\leq \ell$.
Notice that there are at most $(\frac{\epsilon}{2}\cdot 2^{m_0})$-many elements $\tau$ on level
$m_0$ of $T'$ such that $\seq{F_1^\tau(\tup{x}),\ldots,F_\ell^\tau(\tup{x})}$ is not defined.
Let $m_1=\max\set{B_{F_0}\left(\tup{y};\frac{\epsilon}{2|R|}\right)}{\tup{y}\in R}$.
Then $H^\tau(\tup{x})$ is defined for all
but $(\epsilon\cdot 2^{m_1})$-many $\tau$ on level $m_1$ of $T'$.

Now we consider primitive recursion.
Let $F_0$ be an $(k-1)$-ary $T$-local name, $F_1$ be a $(k+1)$-ary $T$-local name,
and $H$ be the $k$-ary name defined by primitive recursion with $F_0$ and $F_1$.
We now show that $H$ is $T$-local.
In other words, we let $T'\leq T$, $\tup{x}\in\NN^{k-1}$, and $y\in\NN$
and we show that there is a $\tau\in T'$ such that $H^\tau(\tup{x},y)$ is defined.

Fix $\epsilon>0$.
Let $m_0=B_{F_0}\left(\tup{x};\frac{\epsilon}{y+1}\right)$ and let $R_0$ be the set of all
$z_0$ such that $F_0^\tau(\tup{x})=z_0\leq m_0$ for some $\tau$ is on level $m_0$ of $T'$.
Notice that there are at most $\left(\frac{\epsilon}{y+1}\cdot 2^{m_0}\right)$-many elements $\tau$ on level
$m_0$ of $T'$ such that $F_0^\tau(\tup{x})$ is not defined.
Let $m_1=\max\set{B_{F_1}\left(\tup{x},0,z_0;\frac{\epsilon}{(y+1)|R_0|}\right)}{z_0\in R_0}$
and let $R_1$ be the set of all $z_1$ such that $F_1^\tau(\tup{x},0,z_0)=z_1\leq m_1$
for some $\tau$ is on level $m_1$ of $T'$ and $z_0\in R_0$.
Notice that there are at most $\left(\frac{2\epsilon}{y+1}\cdot 2^{m_1}\right)$-many elements $\tau$ on level
$m_1$ of $T'$ such that $H^\tau(\tup{x},1)=F_1(\tup{x},0,F_0^\tau(\tup{x}))$ is not defined.
Continuing in this way, we let
$m_{i+1}=\max\set{B_{F_1}\left(\tup{x},i,z_i;\frac{\epsilon}{(y+1)|R_i|}\right)}{z_i\in R_i}$
and let $R_{i+1}$ be the set of all $z_{i+1}$ such that $F_1^\tau(\tup{x},i,z_i)=z_{i+1}\leq m_{i+1}$
for some $\tau$ is on level $m_{i+1}$ of $T'$ and $z_i\in R_i$.
Notice that there are at most $\left(\frac{(i+2)\epsilon}{y+1}\cdot 2^{m_{i+1}}\right)$-many elements $\tau$ on level
$m_{i+1}$ of $T'$ such that $H^\tau(\tup{x},i+1)$ is not defined.
Then $H^\tau(\tup{x},y)$ is defined for all
but $(\epsilon\cdot 2^{m_y})$-many $\tau$ on level $m_y$ of $T'$.
\end{proof}

\begin{prop}[\RCAo]\label{P:R:almostPersistent2}
For every $n\in\NN$ and condition $T$ there is a $\tau\in T$ such that
$|\tau|\geq n$ and $T_\tau$ is a condition.
\end{prop}

Note that this proposition is slightly stronger
than the second requirement for being
an almost persistent notion of forcing.

\begin{proof}
Suppose, for the sake of contradiction, that $T_\tau$ has measure zero
for all $\tau\in T$ such that $|\tau|\geq n$.
Fix $\epsilon>0$.
Let $\tau_1,\ldots,\tau_k$ be the elements on level $n$ of $T$.
Since $T_{\tau_i}$ has measure zero for each $1\leq i\leq k$,
there is an $m\geq n$ such that the number of nodes on level $m$ of $T_{\tau_i}$,
divided by $2^m$, is less than $\epsilon/k$.
Therefore the total number of nodes of $T$ on level $m$, divided by $2^m$,
is less than $k(\epsilon/k)=\epsilon$,
contradicting that $T$ has positive measure.
\end{proof}

\begin{prop}[\RCAo]\label{P:R:almostPersistent3}
Let $T$ be an infinite subtree of $2^{<\omega}$,
$F$ be a $k$-ary $T$-local name, and $\tup{x}\in\NN^k$.
There is a $\tau\in T$ such that $F^\tau(\tup{x})$ is defined
and $T_\tau$ has positive measure.
\end{prop}

\begin{proof}
Suppose, for the sake of contradiction, that $T_\tau$ has measure zero
for all $\tau\in T$ such that $F^\tau(\tup{x})$ is defined.
Fix $\epsilon>0$.
By Lemma~\ref{L:R:locallem} $T_{F(\tup{x})}$ has measure zero,
so there is a level $N$ of $T$ such that the number of nodes $\tau$ on level
$N$ such that $F^\tau(\tup{x})$ is undefined at stage $|\tau|$,
divided by $2^N$, is less than $\epsilon/2$.

Let $\tau_1,\ldots,\tau_k$ be the elements on level $N$ of $T$
such that $F^{\tau_i}(\tup{x})$ is defined at stage $|\tau_i|$.
By assumption $T_{\tau_i}$ has measure zero for each $1\leq i\leq k$.
Therefore there is an $M\geq N$ such that for each $1\leq i\leq k$,
the number of nodes on level $M$ of $T_{\tau_i}$, divided by $2^M$,
is less than $\epsilon/2k$.

Therefore the total number of nodes of $T$ on level $M$, divided by $2^M$,
is less than $\epsilon/2+k(\epsilon/2k)=\epsilon$, contradicting that
$T$ has positive measure.
\end{proof}

\begin{cor}[\RCAo]\label{C:R:almost persistent}
Random forcing is almost persistent.
\end{cor}
\begin{proof}
Follows immediately from Propositions~\ref{P:R:comp&pr}, \ref{P:R:almostPersistent2}, and \ref{P:R:almostPersistent3}.
\end{proof}

\begin{lem}\label{L:R:realForc}
For every $\BSigma^0_2$-generic filter over a model of \RCAo\
there exists a generic real for random forcing.
\end{lem}

\begin{proof}
By Lemma~\ref{L:Gen:realForc} it
suffices to show that if $T$ is a condition, then
$$\mathcal{D}_T=\{S:S\cap T\text{ is finite }\lor\ S\leq T\}$$
is open dense.

Suppose that $T'$ is a condition and that
$T'\cap T$ has measure zero.
Since $T'$ has positive measure
there is a $\tau\in T'\setminus(T'\cap T)$ such that $T'_\tau$ also has positive measure.
Then $T'_\tau\leq T'$ and $T'_\tau\cap T$ is finite,
so $T'_\tau\in\mathcal{D}_T$.
\end{proof}

\begin{thm}
Let $\MM$ be a model of \RCAo\
and suppose that $G$ is a generic real for $\Forc$
corresponding to a $\BSigma^1_2$-generic filter $\mathcal{G}$ for random forcing.

Then $\MM[G]$ is a model of $\RCAo$.
\end{thm}
\begin{proof}
Follows from Corollary~\ref{C:R:almost persistent}, Lemma~\ref{L:R:realForc}, and Proposition~\ref{T:presRCA}.
\end{proof}

Over models of $\ACAo$ we can actually assume that
random forcing is persistent.
More precisely, the next propositions shows that for every condition $T$
there is an extension $T\leq T'$ with the same measure which obeys the persistent criterion.

\begin{prop}[\ACAo]\label{P:R:makePersistent}
Let $T$ have measure at least $\epsilon$.
Let $$T'=\set{\tau\in T}{T_\tau\text{ has positive measure}}.$$
Then $T'$ has measure at least $\epsilon$.
\end{prop}
\begin{proof}
Suppose that $\mu_n(T')<\epsilon$ for some $n$.
Let $\tau_0,\ldots,\tau_k$ be the nodes on level $n$ of $T\setminus T'$.
Since $T_{\tau_i}$ has measure zero for all $i\leq k$,
for any $\delta>0$ there is a level $m_\delta$ such that
$\mu_{m_\delta}(\bigcup_{i\leq k}T_{\tau_i})<\delta$.
Therefore there is a level $m$ such that
$$\mu_m(T)=\mu_m(T')+\mu_{m}\left(\bigcup_{i\leq k}T_{\tau_i}\right)<\epsilon.$$
But this contradicts that $T$ has measure at least $\epsilon$.
\end{proof}

We now proceed to show that random forcing satisfies $\MCP$.
We begin with some lemmas.

\begin{lem}[\RCAo]\label{L:R:subConditionPartition}
Let $T$ have measure at least $\epsilon>0$ and
let $S$ be a subtree of $T$, possibly with measure zero.
Let $$R(n)=\bigcup_{\tau\in X_n}T_\tau$$
where $X_n$ is the set of all $\tau\in (T\setminus S)$ such that $|\tau|=n$.
Then for all $\delta>0$ there is an $n$ such that the
the measure of $S\cup R(n)$ is at least $\epsilon-\delta$.
\end{lem}
\begin{proof}
Note that the elements of $T\setminus(S\cup R(n))$ are precisely
the elements $\tau\in T\setminus S$ such that $\tau\uhr n\in S$.

Suppose that $S$ has measure at least $\delta_1$ and has
measure no greater than $\delta_2$ for some
$0\leq \delta_1<\delta_2\leq \epsilon$.
We will show that there is an $n$ such that
$S\cup R(n)$ has measure at least $\delta_1+(\epsilon-\delta_2)$.
The lemma then follows by choosing $\delta_1$ and $\delta_2$
to be as close as is necessary.

Since $T$ has measure at least $\epsilon$,
and since $S$ has measure no greater than $\delta_2$,
there is a level $n$ such that for every $m\geq n$
there are at least $\lfloor(\epsilon-\delta_2)\cdot 2^m\rfloor$-many
elements of $(T\setminus S)$ on level $m$.
Moreover, since $S$ has measure at least $\delta_1$,
there are at least $\lfloor\delta_1\cdot 2^m\rfloor$-many
elements of $S$ on level $m$ for every $m\geq n$.
Therefore $S\cup R(n)$ has measure at least $\delta_1+(\epsilon-\delta_2)$.
\end{proof}

\begin{lem}[\RCAo]\label{L:R:preMCP}
Suppose that $c:T\to\{0,1\}$ is a monotone coloring
of a tree $T$ which has measure at least $\epsilon$.
For any $\delta>0$ there is a level $n$ and values
$y_0,y_1,\ldots,y_k\in\{0,1\}$ such that
$$\mu\left(\bigcup_{i=0}^k\set{\sigma\in T_{\tau_i}}{c(\sigma)=y_i}\right)>\epsilon-\delta,$$
where $\tau_0,\tau_1,\ldots,\tau_k$ are the nodes of $T$ on level $n$.
\end{lem}
\begin{proof}
Let $S=\set{\tau\in T}{c(\tau)=0}$.
Let $n$ be the level guaranteed by Lemma~\ref{L:R:subConditionPartition}.
Since $c$ is a monotone coloring, if $c(\tau)=1$ then
$\set{\sigma\in T_\tau}{c(\sigma)=1}=T_\tau$.
The lemma then follows by letting $y_i=0$ if and only $\tau_i\in S$.
\end{proof}


\begin{prop}[\ACAo]\label{P:R:MCP}
Random forcing satisfies $\MCP$.

In fact, for any condition $T$ and any infinite sequence $\langle c_i:i\in\NN\rangle$ of monotone
colorings $c_i:T\rightarrow \{0,1\}$, there is a condition $T'\leq T$
whose measure is arbitrarily close to the measure of $T$ such that
for every $i$ there is a $k$ such that if $\tau\in T'$ is above level $k$
then $c_i$ is constant on $T'_\tau$.
\end{prop}

\begin{proof}
Fix $k\in\NN$.
We will construct $T'\leq T$ so that if $T$ has measure
at least $\epsilon$, then $T'$ has measure at least $\epsilon-1/k$.
We define $T'$ and $H$ in stages.

At stage $0$ we find a level $m_0$ and values $y_0,y_1,\ldots,y_\ell\in\{0,1\}$ such that
$$\mu\left(\bigcup_{i=0}^\ell\set{\sigma\in T_{\tau_i}}{c_0(\sigma)=y_i}\right)>\epsilon-1/(k+1),$$
where $\tau_0,\tau_1,\ldots,\tau_\ell$ are the nodes of $T$ on level $m_0$.
Such nodes and numbers exist by Lemma~\ref{L:R:preMCP}.
We then let
$$T^0=\bigcup_{i=0}^\ell\set{\sigma\in T_{\tau_i}}{c_0(\sigma)=y_i}.$$

At stage $n+1$ we find a level $m_{n+1}>m_n$ and values $y_0,y_1,\ldots,y_\ell\in\{0,1\}$ such that
$$\mu\left(\bigcup_{i=0}^\ell\set{\sigma\in T^n_{\tau_i}}{c_0(\sigma)=y_i}\right)>\epsilon-1/(k+1)^{n+2},$$
where $\tau_0,\tau_1,\ldots,\tau_\ell$ are the nodes of $T$ on level $m_n$.
Such nodes and numbers exist by Lemma~\ref{L:R:preMCP}.
We then let
$$T^{n+1}=\bigcup_{i=0}^\ell\set{\sigma\in T^n_{\tau_i}}{c_0(\sigma)=y_i}.$$

Finally, we let $T'=\bigcap T^n$.
Notice that $T'\leq T$, $T'$ has measure at least
$\epsilon-\sum_{i=0}^\infty1/(k+1)^{i+1}=\epsilon-1/k$,
and $T'$ thus satisfies the conclusion of $\MCP$.
\end{proof}

\begin{thm}
Let $\MM$ be a model of \ACAo\
and suppose that $G$ is a generic real for $\Forc$
corresponding to a $\BSigma^1_2$-generic filter $\mathcal{G}$ for random forcing.

Then $\MM[G]$ is a model of $\ACAo$.
\end{thm}
\begin{proof}
Follows from Proposition~\ref{P:R:makePersistent}, Lemma~\ref{L:R:realForc}, and Proposition~\ref{T:presACA}.
\end{proof}

We finish this example by showing that random
forcing does not add unbounded reals.

\begin{thm}[\RCAo]\label{P:R:bndedNames}
Let $T$ have measure at least $\epsilon$,
$F$ be a $T$-local name, and $0<\delta<\epsilon$.
There is an extension $T'\leq T$ with measure at least $\epsilon-\delta$
and a function $B$
such that $$T' \Vdash (\forall\tup{v})[F(\tup{v}) \leq \check{B}(\tup{v})].$$
\end{thm}
\begin{proof}
The proof is very similar to the proof of Proposition~\ref{P:R:MCP}.
Fix $k\in\NN$.
We will construct $T'\leq T$ so that if $T$ has measure
at least $\epsilon$, then $T'$ has measure at least $\epsilon-1/k$.
We define $T'$ and $H$ in stages.

For ease of notation we assume that $F$ is a $1$-ary name.
By Lemma~\ref{L:R:locallem}, for each $x\in\NN$ the tree
$T_{F(x)}=\set{\tau\in T}{F^\tau(x)\text{ is undefined at stage }|\tau|}$
has measure zero.

We begin at stage 0.
Let $S_0=T_{F(0)}$.
By Lemma~\ref{L:R:subConditionPartition} there is a level $m_0$ such that
$$\mu\left(\bigcup_{\tau\in C_0}T_\tau\right)> \epsilon-1/(k+1),$$
where $C_0$ is the set of nodes on level $m_0$ of $T$ such that
$F^\tau(0)$ is defined at stage $|\tau|$.
Let $\displaystyle T^0=\left(\bigcup_{\tau\in C_0}T_\tau\right)$ and
$B(0)=\max\set{F^\tau(0)}{\tau\in C_0}$.

At stage $n+1$ we do the following.
Let
$$S_{n+1}=T^n_{F(n+1)}=\set{\tau\in T^n}{F^\tau(n+1)\text{ is undefined at stage }|\tau|}.$$
By Lemma~\ref{L:R:subConditionPartition} there is a level $m_{n+1}>m_n$ such that
$$\mu\left(\bigcup_{\tau\in C_{n+1}}T_\tau\right)> \epsilon-1/(k+1)^{n+2},$$
where $C_{n+1}$ is the set of nodes on level $m_{n+1}$ of $T^n$ such that
$F^\tau(n+1)$ is defined at stage $|\tau|$.
Let $\displaystyle T^{n+1}=\left(\bigcup_{\tau\in C_{n+1}}T_\tau\right)$ and
$B(n+1)=\max\set{F^\tau(n+1)}{\tau\in C_{n+1}}$.

Finally, we let $T'=\bigcap T^n$.
Note that the intersection $\bigcap T^n$ is well defined in \RCAo\
since $\displaystyle \tau\in \bigcap_{n\in\NN} T^n$ if and only if
$\displaystyle \tau\in \bigcap_{n\leq k} T^n$ for some $k$ such that $|\tau|\leq m_k$.
Notice also that $T'\leq T$, $T'$ has measure at least
$\displaystyle \epsilon-\sum_{i=0}^\infty1/(k+1)^{i+1}=\epsilon-1/k$,
and that $F^\tau(x)\leq B(x)$ for every $x\in\NN$ and
every $\tau\in T'$ such that $F^\tau(x)$ is defined.
\end{proof}

\begin{cor}
Let $\MM$ be a model of \RCAo\
and suppose that $G$ is a generic real for $\Forc$
corresponding to a generic filter $\mathcal{G}$ for random forcing.

For every function $f:\NN\to\NN$ in $\MM[G]$
there is a function $b$ in $\MM$ such that
$f(x)\leq b(x)$ for all $x$.
\end{cor}
