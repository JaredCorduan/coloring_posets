A notion of forcing is usually defined as a partial ordering.
Many of the notions of forcing that have been useful in set theory,
particularly those that add generic reals,
can be thought of as collections of subtrees of $\OO$, ordered by inclusion.
We restrict ourselves to such notions of forcing.

\begin{definition}
We write $\OO$\index{$\OO$} for the set of all finite, increasing sequences from $\NN$.
A subset $T\subseteq\OO$ is a \textit{tree}
if it is downward closed, meaning that if
$\sigma\subseteq\tau$ and $\tau\in T$, then $\sigma\in T$.

Given a model $\MM$ of second order arithmetic,
a \textit{notion of forcing}\index{notion of forcing} $\Forc$ is a collection of trees in $\MM$.
The elements of $\Forc$ are called \textit{conditions}\index{condition} and are ordered by inclusion.
If $T$ and $T'$ are conditions and and $T'\subseteq T$, we say that $T'$ \textit{extends} $T$
and we write $T'\leq T$.
When there is a possibility that either $T$ or $T'$ is not a condition,
we write $T'\subseteq T$, thus reserving the notation $T'\leq T$ only
for the case when both $T'$ and $T$ are conditions.
\end{definition}

Note that a notion of forcing is a third-order object,
and is therefore never an element of the model in question.
The conditions, however, are elements of the model.

The following notation will be useful in dealing with conditions.
\begin{definition}\label{D:treestuff}
Let $T\subseteq\OO$ be a tree.
A \textit{path}\index{path} through $T$ is a function $f:\NN\to\NN$
such that $f[n]=\seq{f(0),f(1),\ldots,f(n-1)}\in T$ for all $n\in\NN$.
The collection of all paths through $T$ is denoted by $[T]$\index{$[T]$}.

Let $\tau\in T$.
We use $T_\tau$\index{$T_\tau$} to denote the tree consisting of all nodes
$\sigma\in T$ such that either $\sigma\supseteq\tau$ or $\sigma\subseteq\tau$.
\end{definition}

Given a model $\MM$ and a notion of forcing $\Forc$,
we define a formal language, called the forcing language, inside $\MM$.
We start by describing the \textit{symbols of the forcing language}.
The forcing language contains \textit{constant symbols} $0,1,2,\ldots$ for
the natural numbers, and symbols for \textit{number variables},
which we will usually denote by lowercase letters late in the alphabet,
such as $v$ and $w$.
The \textit{logical symbols} of the forcing language will consist of
=, $\land$, $\neg$, and $\forall v$ (where $v$ is a number variable).
There are also symbols that behave like function parameters.
More specifically, there is a
\textit{symbol $F$ for each $k$-ary name} in $\MM$, which we will now define.
\begin{definition}\label{D:names}\index{name}
A \textit{$k$-ary name} is a $\BSigma^0_1$ set $F\subseteq \OO\times\NN^{k}\times\NN$ such that
\begin{itemize}
\item If $(\tau,\tup{x},y)\in F$ and $\tau\subseteq\tau'$ then $(\tau',\tup{x},y)\in F$
\item If $(\tau,\tup{x},y),(\tau,\tup{x},y')\in F$ then $y=y'$.
\end{itemize}
To each subset $A\subseteq\NN$ we associate an element \rune$\in[\OO]$
by letting \rune$(n)$ be the $n$-th least element of $A$.
Given $\tau\in\OO$, we write $\tau\leq A$ to mean that $\tau$ is an initial segment of \rune.

The \textit{domain of $F$} is the class
$$\dom(F)=\{A\subseteq\NN:\forall\tup{x}\exists\tau,y(\tau\leq A\land(\tau,\tup{x},y)\in F)\}.$$
Given $A\in\dom(F)$, the \textit{evaluation}\index{evaluation of a name}
$$F^A(\tup{x})=y\ \Leftrightarrow \ \exists\tau(\tau\leq A\land(\tau,\tup{x},y)\in F).$$
is a total $k$-ary function.
\end{definition}

Note that all ground model functions $G$ (meaning that $G\in\MM$)
have \textit{canonical names}\index{canonical names} $\check{G}$ defined by
$$(\tau,\tup{x},y)\in\check{G}\Leftrightarrow y=G(\tup{x}).$$
$\check{G}$ is a name for $G$ in the sense that $\check{G}^A=G$ for all sets $A$.
We usually write $F^\tau(x)=y$\index{$F^\tau(x)$} in place of $(\tau,\tup{x},y)\in F$.
In light of the fact that $F$ is a $\BSigma^0_1$ set,
we will say that $F^\tau(\tup{x})$ is \textit{defined by stage $n$}
if there are $w,y\leq n$ and a $\sigma\subseteq\tau$
such that $w$ witnesses that $F^\sigma(\tup{x})=y$.
We use the abbreviation $F^\tau(\tup{x})\neq y$ to mean that
either $F^\tau(\tup{x})$ is not defined at stage $n=|\tau|$,
or it is defined at stage $n=|\tau|$ and is distinct from $y$.

We now define what it means for a name to be local for a condition.

\begin{definition}\label{D:localNames}
Let $T$ be a condition.
A $k$-ary name $F$ is $T$\textit{-local}\index{$T$-local}\index{locality}
if for every $\tup{x}\in\NN^k$ and extension $T'\leq T$,
there is a $\tau\in T'$ and a $y$ such that
$F^\tau(\tup{x})$ is defined by stage $|\tau|$ and $F^\tau(x)=y$.
\end{definition}

Notice that if $G$ is a ground model function then its canonical name
$\check{G}$ is a $T$-local name for every condition $T$.

For many of the notions of forcing that we will be concerned with,
there is a convenient definition of locality which is equivalent
to the definition just given.
We explain this in detail with the following definition and proposition.

\begin{definition}
A notion of forcing $\Forc$ is \textit{persistent}\index{persistent}
if it satisfies the following two properties:
\begin{itemize}
\item $T_\tau\in\Forc$ whenever $T\in\Forc$ and $\tau\in T$,
\item for every $n\in\NN$ and condition $T$ there is a
		$\tau\in T$ such that $|\tau|\geq n$.
\end{itemize}
\end{definition}
The first property above is the one that really
characterizes persistent notions of forcing.
The second property ensures that the
atomic forcing relation is not trivially satisfied.

\begin{prop}[\RCAo]\label{P:PersistentLocalNames}
Let $\Forc$ be a persistent notion of forcing, $T\in\Forc$,
and $F$ be a $k$-ary name.
Then $F$ is $T$-local if and only if $[T']\cap\dom(F)$ is nonempty
for every extension $T'\leq T$.
\end{prop}

\begin{proof}
For the purposes of the proof only, we say that a $k$-ary name $F$ is
$T$-local$_2$ if $[T']\cap\dom(F)$ is nonempty for every extension $T'\leq T$.
It follows immediately that if $F$ is $T$-local$_2$ then $F$ is also $T$-local.
Therefore we assume that $F$ is $T$-local and show that $F$ is also $T$-local$_2$.

Let $T'\leq T$ be given.  We will construct an $A\in[T']\cap\dom(F)$ in stages.
For notational convenience we assume that $F$ is a $1$-ary name.
We begin at stage $0$ by finding a $\tau_0\in T'$ and a $y_0\in\NN$ such that $F^{\tau_0}(0)=y_0$.
Such $\tau_0$ and $y_0$ are guaranteed to exist since $F$ is $T$-local.
Let $T^1=T'_{\tau_0}$.
We now proceed to stage $1$ where we find a $\tau_1\in T^1$
and a $y_1\in\NN$ such that $F^{\tau_1}(1)=y_1$.
Such $\tau_1$ and $y_1$ are guaranteed to exist since $F$ is $T$-local.
Let $T^2=T^1_{\tau_1}$.
Continuing in this way, at stage $n+1$ we find a $\tau_{n+1}\in T^n$
and a $y_{n+1}\in\NN$ such that $F^{\tau_{n+1}}(n+1)=y_{n+1}$.
Finally, we let $\displaystyle A=\bigcup_{n\in\NN}\tau_n$.
Then $\Ind{\BSigma^0_1}$ suffices to show that $A\in[T']\cap\dom(F)$.
\end{proof}

Whenever we have a persistent notion of forcing,
we will use the alternate definition of locality as given
by Proposition~\ref{P:PersistentLocalNames}.
We now define the formulas of the forcing language and the forcing relation.

\begin{definition}\label{D:forcingLang}\index{forcing language}
The \textit{formulas of the forcing language} are the smallest family which is closed under the following rules:
\begin{itemize}
\item Let $F$ be a $k$-ary  name, $F'$ be a $k'$-ary name, and let
		$\tup{w}=w_1,\ldots,w_k$, $\tup{w}'=w_1',\ldots,w_{k'}'$, where each $w_i$ and $w_i'$
		is either a variable symbol or a constant symbol.
		Then $F(\tup{w})=F'(\tup{w}')$ is a formula.
\item If $\varphi$ is a formula, then so is $\neg\varphi$.
\item If $\varphi$ and $\psi$ are formulas, then so is $\varphi\land\psi$.
\item If $\varphi$ is a formula and $x$ is a variable, then $\forall x\varphi$ is a formula.
\end{itemize}
\end{definition}

We now define locality for formulas.

\begin{definition}\label{D:localFormulas}\index{locality}
Let $T$ be a condition.
A formula $\varphi$ of the forcing language is a $T$\textit{-local formula}
of the forcing language if every name that occurs in $\varphi$ is $T$-local.

We say that $\varphi$ is a $T$\textit{-local sentence} of the forcing language
if $\varphi$ is obtained from a $T$-local formula by replacing all the
free variables with constant symbols.
\end{definition}

\begin{definition}\label{D:forcingRel}\index{forcing relation}
The \textit{forcing relation} $T\Vdash\varphi$ is defined by induction on complexity of the
$T$-local sentences of the forcing language.
Assume that all names that occur in the sentences below are $T$-local.
\begin{itemize}
\item $T\Vdash F(\tup{x})=F'(\tup{x}')$ if $y_1=y_2$ whenever
	$\tau\in T$, $(\tau,\tup{x},y_1)\in F$, and $(\tau,\tup{x}',y_2)\in F'$.
\item $T\Vdash \varphi\land\psi$ if $T\Vdash \varphi$ and $T\Vdash \psi$.
\item $T\Vdash \forall v \varphi(v)$ if $T\Vdash \varphi(x)$ for all $x\in\NN$.
\item $T\Vdash\neg\varphi$ if there is no $T'\leq T$ such that $T'\Vdash\varphi$.
\end{itemize}
\end{definition}

The first thing to notice about the forcing relation is that if
$T\Vdash\varphi$ and $T'\leq T$, then $T'\Vdash\varphi$.
(This is an easy proof by induction on the complexity of $\varphi$.)

Notice also that from the definition of forcing the negation
of a sentence it immediately follows that
if $T\in\Forc$ and $\varphi$ is a $T$-local sentence,
then either $T\Vdash\neg\phi$ or $T'\Vdash\phi$ for some $T'\leq T$.

We have defined the forcing language in terms of
negation, conjunction, and universal quantification.
We consider disjunction, implication, and
existential quantification as abbreviations in the usual way.
It is worthwhile to unravel the abbreviations and
see what they mean in the context of the forcing language.

First consider the statement that $T\Vdash\varphi\lor\psi$.
This statement is shorthand for $T\Vdash\neg(\neg\varphi\land\neg\psi)$.
Unpacking the definitions, we see that this means that
there is no $T'\leq T$ such that $T'\Vdash\neg\varphi$ and
$T'\Vdash\neg\psi$.
Therefore for every $T'\leq T$, there is a $T''\leq T'$
such that either $T''\Vdash\varphi$ or $T''\Vdash\psi$.

Now consider the statement that $T\Vdash\varphi\rightarrow \psi$.
This statement is shorthand for $T\Vdash\neg(\varphi\land\neg\psi)$.
Unpacking the definitions, we see that this means that
there is no $T'\leq T$ such that $T'\Vdash\varphi$ and $T'\Vdash\neg\psi$.
Therefore every $T'\leq T$, if $T'\Vdash\varphi$ then
there is some $T''\Vdash T'$ such that $T''\Vdash\psi$.

Now consider the statement that $T\Vdash\exists x\varphi(x)$.
This statement is shorthand for $T\Vdash\neg\forall x\neg(\varphi(x))$.
Unpacking the definitions, we see that this means that
there is no $T'\leq T$ such that $T'\Vdash\neg\varphi(x)$ for all $x$.
Therefore for every $T'\leq T$ there is an $x$ and a $T''\leq T$
such that $T''\Vdash\varphi(x)$.

Double negation is not a shorthand,
though it is still helpful to unravel its definition.
If $T\Vdash\neg\neg\varphi$, then there is no $T'\leq T$
such that $T'\Vdash\neg\varphi$.
Therefore for all $T'\leq T$, there is a $T''\leq T'$
such that $T''\Vdash\varphi$.

In order to prove the usual rule about double negation,
we assume that the notion of forcing in question is persistent.

\begin{prop}[\RCAo]\label{P:2neg}
Let $\Forc$ be a persistent notion of forcing.
Let $T\in\Forc$ and $\varphi$ be a $T$-local sentence.
$T\Vdash\neg\neg\varphi$ if and only if $T\Vdash\varphi$.
\end{prop}
\begin{proof}
First we show that if $T\nVdash\varphi$ then there is a condition
$T'\leq T$ such that $T'\Vdash\neg\varphi$.
The proof is by induction on the complexity of $\varphi$,
though only the atomic case merits description.
Suppose that $T\nVdash F_1(\tup{x})=F_2(\tup{x})$.
Then there is a $\tau\in T$ and numbers $\tup{x}$, $y_1$, and $y_2$
such that $F_1^\tau(\tup{x})=y_1$, $F_2^\tau(\tup{x})=y_2$, and $y_1\neq y_2$.
Therefore $T_\tau\Vdash F_1(\tup{x}) \neq F_2(\tup{x})$.

We now show that $T\Vdash\neg\neg\varphi$ if and only if $T\Vdash\varphi$.
The backwards direction follows from the fact that
if $T\Vdash \varphi$, then $T'\Vdash \varphi$ for every $T'\leq T$.
So suppose that $T\nVdash\varphi$.
By the first paragraph we know that there is a
$T'\leq T$ such that $T'\Vdash\neg\varphi$.
Thus $T'\Vdash\neg\neg\neg\varphi$, and so $T'\nVdash\neg\neg\varphi$.
Therefore $T\nVdash\neg\neg\varphi$.
\end{proof}
