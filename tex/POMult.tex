We now turn our attention to another collection of
tree-like partial orderings with the $(n,k)$-Ramsey properties.
We begin by fattening the binary tree.

\begin{definition}\label{D:withMultis}
The \textit{binary tree with multiplicities}, \index{binary tree with multiplicities}\index{$\binm$}
denoted $\binm$, is the partial ordering whose elements are pairs $(\sigma,x)$,
where $\sigma\in\bin$, $x\in\NN$, and $x\leq|\sigma|$.
We order $\binm$ by essentially ignoring the second coordinate.
We let $(\sigma,x)<(\tau,y)$ if and only if $\sigma\subsetneq\tau$.

Similarly, \textit{omega with multiplicities}, denoted $\omegam$, \index{omega with multiplicities}\index{$\omegam$}
is the partial ordering whose elements are pairs $(a,x)\in\NN^2$
such that $x\leq a$.
We order $\omegam$ by ignoring the second coordinate.
\end{definition}

We will now show that $\binm$ has the $(n,k)$-Ramsey property
for all $n\geq 1$, $k\geq 2$.
The proofs involved will be adaptations of the proofs used
by Chubb, Hirst, and McNichol \cite{CHM} to show that $\bin$ has the
$(n,k)$-Ramsey properties.
We will also characterize the partial orderings contained in $\binm$
that have the $(n,k)$-Ramsey properties.
In terms of Reverse Mathematics,
we consider the strength of $RT^n(\binm)$ and
also the strength of the theorem which characterizes
the suborderings of $\binm$ having Ramsey properties.
The main results of this section are Theorem~\ref{T:binm&ACA} and Theorem~\ref{T:charRamInBinm}.
Theorem~\ref{T:binm&ACA} says that $RT^n(\binm)$ is equivalent
to $\ACAo$ for any fixed $n\geq 3$.
Theorem~\ref{T:charRamInBinm} characterizes the suborderings of $\binm$
with the Ramsey property and states that the characterization
is equivalent to $\ACAo$.

We can think of $RT^1(\Po)$ as a pigeonhole principle\index{pigeonhole principle} for $\Po$.
In fact, $RT^1(\omega)$ is the usual (infinite) pigeonhole principle.
We now show that adding multiplicities to $\omega$ and to $\bin$
does not add any complexity over \RCAo\ to the corresponding
pigeonhole principles.

\begin{prop}[\RCAo]\label{P:bin&binm1}
For all $k\geq 2$, $RT^1_k(\omegam)\leftrightarrow RT^1_k(\omega)$.
\end{prop}
\begin{proof}
Suppose $RT^1_k(\omegam)$ holds and $c:\NN\to\{0,1,\ldots,k-1\}$.
Let $\tilde{c}:\omegam\to\{0,1,\ldots,k-1\}$ be the coloring
defined by $\tilde{c}(a,x)=c(a)$.
Then any infinite homogeneous copy of $\omegam$ for $\tilde{c}$ induces an
infinite homogeneous set for $c$ by projecting onto the first coordinate.

On the other hand, suppose that $RT^1_k(\omega)$ holds and
let $c:\omegam\to\{0,1,\ldots,k-1\}$.
Let $\tilde{c}:\NN\to\{0,1,\ldots,k-1\}$ where
$\tilde{c}(a)$ is the color most used by $c$
on the set $\set{(a,x)}{x\leq a}$
(it does not matter how ties are settled, provided it is effective).
If $H\subseteq\NN$ is an infinite homogeneous set for $\tilde{c}$,
say in color $i$, then for each $a\in H$ there are at least
$\lceil \frac{a}{k}\rceil$-many numbers $x$ such that $c(a,x)=i$.
Therefore an infinite homogeneous set for $\tilde{c}$ computes
an infinite homogeneous set for $c$.
\end{proof}


\begin{prop}[\RCAo]
For all $k\geq 2$, $RT^1_k(\binm)\leftrightarrow RT^1_k(\bin)$.
\end{prop}
\begin{proof}
Suppose that $RT^1_k(\binm)$ holds and $c:\bin\to\{0,1,\ldots,k-1\}$.
We define a coloring $\tilde{c}:\binm\to\{0,1,\ldots,k-1\}$
by $\tilde{c}(\sigma,x)=c(\sigma)$.
Then any homogeneous copy of $\binm$ for $\tilde{c}$ induces a homogeneous
copy of $\bin$ for $c$ by projecting onto the first coordinate.

On the other hand, suppose that $RT^1_k(\bin)$ holds and
let $c:\binm\to\{0,1,\ldots,k-1\}$.
Let $\tilde{c}:\bin\to\{0,1,\ldots,k-1\}$ where
$\tilde{c}(\sigma)$ is the color most used by $c$
on the set $\set{(\sigma,x)}{x\leq |\sigma|}$.
Similarly to Proposition~\ref{P:bin&binm1},
any homogeneous copy of $\bin$ for $\tilde{c}$ computes
a homogeneous copy of $\binm$ for $c$.
\end{proof}

\begin{cor}[$\RCAo+\Ind{\BSigma^0_2}$]
$RT^1(\binm)$ holds.
\end{cor}
\begin{proof}
Chubb, Hirst, and McNichol proved that that $RT^1(\bin)$
holds in $\RCAo+\Ind{\BSigma^0_2}$ \cite{CHM}.
\end{proof}

We now show that $\binm$ has the $(n,k)$-Ramsey property for $n\geq 2$.
We begin with a few definitions and a helpful lemma.

Let $c:\binm\to\{\red,\blue\}$.
Given $\sigma\in\bin$, the \textit{standard red copy of $\binm$ above $\sigma$}, \index{standard red (blue) copy of $\binm$}
if it exists, is the isomorphism of $\binm$ into
$\Po=\set{(\tau,x)\in\binm}{c(\tau,x)=\red\ \land\ \tau\supseteq\sigma}$
obtained from the following stage-wise computable procedure.
At stage $0$ we search for a pair $(\tau,x)$ such that $c(\tau,x)=\red$ and
$\tau\supseteq\sigma$.
We map $(\langle\rangle,0)$ to the first such pair that is found,
which we call $(\tau_{\langle\rangle},x_{\langle\rangle})$.
At stage $1$ we search above $(\tau_{\langle\rangle},x_{\langle\rangle})$ for two incomparable nodes
$\tau_{\langle 0\rangle}$ and $\tau_{\langle 1\rangle}$ above $\tau_{\langle \rangle}$
and numbers $x^0_{\langle 0\rangle}$, $x^1_{\langle 0\rangle}$,
$x^0_{\langle 1\rangle}$, and $x^1_{\langle 1\rangle}$ such that
$c(\tau_{\langle i\rangle},x^j_{\langle i\rangle})=\red$ for each $i,j<2$.
We then map $(\langle i\rangle,j)$ to
$(\tau_{\langle i\rangle},x^j_{\langle i\rangle})$.
We continue in this way to define the entire isomorphism.
At the beginning of stage $(n+1)$ we will have already defined the
isomorphism up to level $n$.
During stage $(n+1)$ we do the following for every $\rho\in\bin$
such that $|\rho|=n$.
Search for 2 incomparable nodes
$\tau_{\rho\cat 0}$ and $\tau_{\rho\cat 1}$ above $\tau_\rho$ and numbers
$x^0_{\rho\cat 0}$, $x^1_{\rho\cat 0}$, $\ldots$, $x^n_{\rho\cat 0}$,
$x^0_{\rho\cat 1}$, $x^1_{\rho\cat 1}$, $\ldots$, $x^n_{\rho\cat 1}$
such that
$c(\tau_{\langle i\rangle},x^j_{\langle i\rangle})=\red$ for each $i<2$ and $j\leq n$.
We then map $(\langle i\rangle,j)$ to
$(\tau_{\langle i\rangle},x^j_{\langle i\rangle})$.

The \textit{standard blue copy of $\binm$ above $\sigma$},
if it exists, is defined similarly.

\begin{lem}[\RCAo]\label{L:stndCpyBinm2}
Let $c:\binm\to\{\red,\blue\}$.
For each $(\sigma,m)\in\binm$, either
the standard red copy of $\binm$ above $\sigma$ exists,
or there is a $\tau\supseteq\sigma$ such that
the standard blue copy of $\binm$ above $\tau$ exists.
\end{lem}
\begin{proof}
Suppose that
$$(\exists \tau\supseteq\sigma)(\exists m\in\NN)(\forall \rho\supseteq\tau)\big(|\set{(\rho,x)}{c(\rho,x)=\red}|\leq m\big).\ \ \ \ \ (*)$$
From $(*)$ it follows that the standard blue copy of $\binm$ above $\tau$ exists.
If $(*)$ fails, however, then for every $\tau\supseteq\sigma$ and every $m$ there is a
$\rho\supseteq\tau$ such that
$$|\set{x\leq|\rho|}{c(\rho,x)=\red}|\geq m.$$
From this it follows that the standard red copy of $\binm$ above $\sigma$ exists.
\end{proof}

We now prove a more general version of Lemma~\ref{L:stndCpyBinm2}.
Given a coloring $c:\binm\to\{0,1,\ldots,k-1\}$, a $\sigma\in\bin$, and an $\ell<k$,
we define the standard copy of $\binm$ in color $\ell$ above $\sigma$,
if it exists, in the same way that we defined the standard red copy of
$\binm$ above $\sigma$ when dealing with colorings of the form $c:\binm\to\{\red,\blue\}$.

\begin{lem}[$\RCAo+\Ind{\BSigma^0_2}$]\label{L:stndCpyBinmK}
Let $c:\binm\to\{0,1,\ldots,k-1\}$.
There is a $\tau\supseteq\sigma$ and a $j<k$
such that the standard copy of $\binm$ in color $j$ above $\sigma$ exists.
\end{lem}
\begin{proof}
Consider the finite set $C$ consisting of all $j<k$ such that
$$(\exists\tau\supseteq\sigma)(\exists m)(\forall\rho\supseteq\tau)\Big[|\set{x\leq|\rho|}{c(\rho,x)<j}|\leq m\Big].$$
Note that $C$ exists by bounded $\BSigma^0_2$ comprehension, which is equivalent to $\Ind{\BSigma^0_2}$.
Also note that $C$ is nonempty since $0\in C$.
Let $j=\max\{C\}$.

Let $\tau$ and $m$ be such that $|\set{x\leq|\rho|}{c(\rho,x)<j}|\leq m$
for all $\rho\supseteq\tau$.
If $j=k-1$, then we have that
$|\set{x\leq|\rho|}{c(\rho,x)=k-1}|\geq (|\rho|-m)$
for all $\rho\supseteq\tau$.
Therefore the standard copy of $\binm$ in color $k-1$ above $\sigma$ exists.

We now suppose, for the sake of contradiction, that $j<k-1$ and that the
standard copy of $\binm$ in color $j$ above $\sigma$ does not exist.
Then there is a $\rho\supseteq\tau$ and an $m'$
such that for all $\rho'\supseteq\rho$, the set $\set{x\leq|\rho'|}{c(\rho',x)=j}$
has at most $m'$ many elements.
But then $|\set{x\leq|\rho'|}{c(\rho',x)<j+1}|\leq (m+m')$ for all $\rho'\supseteq\rho$
contradicting the maximality of $j$ in $C$.
\end{proof}

\begin{prop}[\ACAo]\label{P:binmRamsey2}
$RT^2(\binm)$ holds.
\end{prop}
\begin{proof}
Let $c:[\binm]^2\to\{0,1,\ldots k-1\}$.
We will define a coloring $\tilde{c}:\binm\to\{0,1,\ldots,k-1\}$ in stages.
In the construction, we will also make use of a family of (partially defined) colorings
of the singletons of $\binm$. Given an element $\alpha=(\sigma,x)\in\binm$,
let $c_\alpha$ be the coloring defined by
$$c_\alpha(\tau,y)=c\big((\sigma,x),(\tau,y)\Big).$$
Note that $c_\alpha$ is only defined for $(\tau,y)>(\sigma,x)$.

Our construction will be computable relative to the following
arithmetic function $\mathcal{S}$.
The function $\mathcal{S}$ takes as input the following:
an index $e$ for a computable embedding $\Phi_e$ of $\binm$ into itself,
an index $e'$ for a computable $k$-coloring of the image of $\Phi_e$,
and a node $\sigma$ in the range of $\Phi_e$.
Given valid $e$, $e'$, and $\sigma$, the function $\mathcal{S}$
outputs a pair $(\tau,j)$ such that $\tau\supseteq\sigma$
and the standard copy of $\binm$ in color $j$ inside the image of $\Phi_e$
above $\tau$ exists.
By Lemma~\ref{L:stndCpyBinmK} $\mathcal{S}(e,e',\tau)$ is defined
whenever $e$, $e,$, and $\tau$ are valid inputs.
Since the statement that the standard copy of $\binm$ in color $j$ above $\sigma$ exists
is arithmetic, it follows that $\mathcal{S}$ is arithmetic.

At stage 0 we begin by letting $\sigma_{\seq{}}=\seq{}$,
$x_{\seq{}}^0=0$, and $\alpha=(\sigma_{\seq{}},x^0_{\seq{}})$.
Let $e$ be an index for the identity map on $\binm$,
let $e'$ be an index for $c_{\alpha}$, and let $\mathcal{S}(e,e',\sigma_{\seq{}})=(\tau,j)$.
Let $e_{\seq{}}$ be an index for the standard copy of $\binm$ in color $j$
above $\tau$, and let $B_{\seq{}}$ be the image of $\Phi_{e_{\seq{}}}$.
Finally, let $\tilde{c}(\seq{},0)=j$.

At stage 1 we find incomparable nodes $\sigma_{\langle 0\rangle}$, $\sigma_{\langle 1\rangle}$
and numbers $x^0_{\langle 0\rangle}$, $x^1_{\langle 0\rangle}$, $x^0_{\langle 1\rangle}$
and $x^1_{\langle 1\rangle}$ such that
$(\sigma_{\langle i\rangle},x^j_{\langle i\rangle})\in B_{\langle \rangle}$
for each $i,j<2$.
Let $\alpha^j_{\seq{i}}=(\sigma_{\langle i\rangle},x^j_{\seq{i}})$.
We are guaranteed to find the elements $\alpha^j_{\seq{i}}$ since $B_{\langle \rangle}$
is isomorphic to $\binm$.
Let $e'$ be an index for the coloring of $B_{\seq{}}$ induced by $c_{\alpha^0_{\seq{0}}}$,
and let $\mathcal{S}(e_{\seq{}},e',\Phi_{e_{\seq{}}}(\seq{0}))=(\tau,j)$.
Therefore the standard copy of $\binm$ in color $j$ inside $B_{\seq{}}$ above
$\tau\supseteq\Phi_{e_{\seq{}}}(\seq{0})$
(with respect to the coloring that $c_{\alpha^0_{\seq{0}}}$ induces on $B_{\seq{}}$) exists.
We let $B_{\langle 0\rangle}'\subseteq B_{\seq{}}$ be this standard copy.
We also let $\tilde{c}(\seq{0},0)=j$.
Besides finding the nodes $\alpha^j_{\seq{i}}$,
what we have accomplished so far during stage 1 is to
find a computable copy $B_{\seq{0}}'\subseteq B_{\seq{}}$ of $\binm$
such that $c_{\alpha^0_{\seq{0}}}$ is constant on $B_{\seq{0}}'$.
We can mimic what we have just done to compute
(still relative to the function $\mathcal{S}$)
a copy $B_{\seq{0}}\subseteq B_{\seq{0}}'$ of $\binm$ such that
$c_{\alpha^1_{\seq{0}}}$ is constant on $B_{\seq{0}}$.
Using this same method we also compute (using $\mathcal{S}$)
a copy $B_{\seq{1}}\subseteq B_{\seq{}}$ of $\binm$ above $\Phi_{e_{\seq{}}}(\seq{1})$
such that $c_{\alpha^i_{\seq{0}}}$ is constant on $B_{\seq{0}}$ for each $i<2$.
Consistent with our definition of $\tilde{c}(\seq{0},0)$,
we let $\tilde{c}(\seq{i},j)$ be the color that
$c_{\alpha^j_{\seq{i}}}$ makes constantly on $B_{\seq{i}}$.

At stage $n+1$ we essentially mimic what we did at stage 1 above each node
$\sigma_{\tau}$, where $|\tau|=n$, which was defined at stage $n$.
Pick one such $\tau$ and let $\sigma=\sigma_\tau$ and
$\tau_i=\tau\cat{\seq{i}}$ for each $i<2$.
Note that $B_{\tau}$ was also defined at stage $n$.
We now find incomparable nodes $\sigma_{\tau_0}$, $\sigma_{\tau_1}$
and numbers $x^j_{\tau_i}$ such that
$(\sigma_{\tau_i},x^j_{\tau_i})\in B_{\tau}$
for each $j\leq (n+1)$ and $i<2$.
Let $\alpha^j_{\tau_i}=(\sigma_{\tau_i},x^j_{\tau_i})$.
We are guaranteed to find the elements $\alpha^j_{\tau_i}$ since $B_{\tau}$
is isomorphic to $\binm$.
Just as in stage 1, using the function $\mathcal{S}$
to guide the construction,
we find indices for computable copies
$B_{\tau_0},B_{\tau_1}\subseteq B_{\tau}$
of $\binm$ such that $c_{\alpha^j_{\tau_i}}$ is constant on
$B_{\tau_i}$ for each $i<2$ and $j\leq{n+1}$.
We also define $\tilde{c}(\tau_i,j)$ to be the color
that $c_{\alpha^j_{\tau_i}}$ is on $B_{\tau_i}$.

This ends the construction.
We now have a coloring $\tilde{c}:\binm\to\{0,1,\ldots,k-1\}$
and an embedding $\gamma$ of $\binm$ into itself,
defined by $\gamma(\sigma,j)=\alpha^j_\sigma$,
such that if $\tau,\rho\supset\sigma$ then
$$\tilde{c}(\sigma,x)=c(\alpha_\sigma^x,\alpha_\tau^y)=c(\alpha_\sigma^x,\alpha_\rho^z)$$
for all $x\leq|\sigma|$, $y\leq|\tau|$, and $z\leq|\rho|$.
Since $RT^1(\binm)$ holds, there is an isomorphic copy
$B$ of $\binm$ that is homogeneous for $\tilde{c}$.
Then the image of $B$ under $\gamma$ is a homogeneous
copy of $\binm$ for $c$.
\end{proof}

\begin{prop}[\ACAo; $m\geq 2$]\label{P:binmRamseyInduct}
If $RT^{m}(\binm)$ holds then so does $RT^{m+1}(\binm)$.
\end{prop}
\begin{proof}
The following proof will be similar to that of Proposition~\ref{P:binmRamsey2}.

Let $c:[\binm]^{m+1}\to\{0,1,\ldots k-1\}$.
We will define a coloring $\tilde{c}:[\binm]^m\to\{0,1,\ldots,k-1\}$ in stages.
In the construction, we will also make use of a family of colorings of the singletons of $\binm$.
Suppose $F\subset[\binm]^m$ is a finite collection of $m$-chains of $\binm$
such that there is a $\tau\in\bin$ with the property that if
$\seq{\alpha_1,\ldots,\alpha_m}\in F$ then $\alpha_m=(\tau,x)$ for some $x$.
We can use $F$ to define a partial coloring $c_F$ of the singletons of $\binm$
with $k^{|F|}$ many colors by
$$c_F(\beta)=\Big\langle c(\alpha_{1},\ldots,\alpha_{m},\beta):\seq{\alpha_{1},\ldots,\alpha_{m},\beta}\in F \Big\rangle.$$
Note that $c_F$ is only defined for $\beta=(\sigma,x)\in\binm$ such that $\tau\subsetneq\sigma$.

Our construction will be computable relative to the following
arithmetic function $\mathcal{S}$.
The function $\mathcal{S}$ takes as input the following:
an index $e$ for a computable embedding $\Phi_e$ of $\binm$ into itself,
an index $e'$ for a computable $k$-coloring of the image of $\Phi_e$,
and a node $\sigma$ in the range of $\Phi_e$.
Given valid $e$, $e'$, and $\sigma$, the function $\mathcal{S}$
outputs a pair $(\tau,j)$ such that $\tau\supseteq\sigma$
and the standard copy of $\binm$ in color $j$ inside the image of $\Phi_e$
above $\tau$ exists.
By Lemma~\ref{L:stndCpyBinmK} $\mathcal{S}(e,e',\tau)$ is defined
whenever $e$, $e'$, and $\tau$ are valid inputs.
Since the statement that the standard copy of $\binm$ in color $j$ above $\sigma$ exists
is arithmetic, it follows that $\mathcal{S}$ is arithmetic.

At stage 0 we begin by letting $\sigma_{\tau}=\tau$,
$x_{\tau}^i=i$, and $\alpha^i_{\tau}=(\sigma_{\tau},x_{\tau}^i)$
for each $|\tau|\leq m$ and $i\leq|\tau|$.
In other words, the map $(\tau,i)\mapsto\alpha^i_\tau$
is the identity map for elements on or below level $m$ of $\binm$.
For each $\tau$ such that $|\tau|=m$ we do the following.
Let $F$ be the set of $m$-chains $\seq{\alpha_1,\ldots,\alpha_m}$ such that
$\alpha_m=(\tau,x)$ for some $x$.
Let $e$ be an index for the identity map on $\binm$,
let $e'$ be an index for $c_{F}$, and let $\mathcal{S}(e,e',\tau)=(\rho,j)$.
Let $B_{\tau}$ be the standard copy of $\binm$ in color $j$ above $\rho$.
Finally, for each $m$-chain $\seq{\alpha_1,\ldots,\alpha_m}\in F$
let $\tilde{c}(\seq{\alpha_1,\ldots,\alpha_m})$ be the color that
$c_F$ makes constantly on $B_\tau$.
Stage 0 is complete when the procedure just described has been
completed for each $\tau$ such that $|\tau|=m$.

At stage $n+1$ we to the following for each $\tau$
such that $|\tau|=n+m-1$.
The node $\sigma_\tau\in\bin$, as well as $B_\tau$,
was defined in stage $n$
(in fact $\sigma_\rho$ has been defined for all
$|\rho|<n+m$ by the beginning of stage $n+1$).
Let $\tau_i=\tau\cat{\seq{i}}$ for each $i<2$.
We now find incomparable nodes $\sigma_{\tau_0}$, $\sigma_{\tau_1}$
and numbers $x^j_{\tau_i}$ such that
$(\sigma_{\tau_i},x^j_{\tau_i})\in B_{\tau}$
for each $j\leq (n+m)$ and $i<2$.
First we focus on $\sigma_{\tau_0}$.
Let $F$ be the set of all $m$-chains
$\seq{(\sigma_{\rho_1},x^{j_1}_{\rho_1}),\ldots,(\sigma_{\rho_{m-1}},x^{j_{m-1}}_{\rho_{m-1}}),(\sigma_{\tau_0},x)}$
where $x\leq|\tau|$ and the elements $(\sigma_{\rho_\ell},x^{j_\ell}_{\rho_\ell})$
range over all those defined in earlier stages.
The partial coloring $c_F$ induces a coloring on the elements
of $B_\tau$ above the image of $\tau_0$ in $B_\tau$.
We then use $\mathcal{S}$ to get a pair $(\rho,j)$
such that the standard copy of $\binm$ in color $j$ above $\rho$
in $B_\tau$ exists.
Let $B_{\tau_0}$ be the standard copy of $\binm$ in color $j$ above $\rho$ in $B_\tau$.
Finally, for each $m$-chain $\seq{\alpha_1,\ldots,\alpha_m}\in F$
let $\tilde{c}(\seq{\alpha_1,\ldots,\alpha_m})$ be the color that
$c_F$ makes constantly on $B_{\tau_0}$.
Similarly we define $B_{\tau_1}\subseteq B_\tau$ and
$\tilde{c}(\seq{\alpha_1,\ldots,\alpha_m})$ for $\alpha_m=(\tau_1,x)$.
Stage $n+1$ is complete when the procedure just described has been
completed for each $\tau$ such that $|\tau|=n+m-1$.

This ends the construction.
We now have a coloring $\tilde{c}:[\binm]^m\to\{0,1,\ldots,k-1\}$
and an embedding $\gamma$ of $\binm$ into itself,
defined by $\gamma(\tau,j)=(\sigma_\tau,x_\tau^j)$,
such that for any $m$-chain $\seq{\alpha_i}_{i=1}^m\in[\binm]^m$,
if $\beta_1,\beta_2\in\binm$ and $\beta_1,\beta_2\geq \alpha_m$, then
$$\tilde{c}(\alpha_1,\ldots,\alpha_m)=c\Big(\gamma(\alpha_1),\ldots,\gamma(\alpha_m),\gamma(\beta_1)\Big)=c\Big(\gamma(\alpha_1),\ldots,\gamma(\alpha_m),\gamma(\beta_2)\Big).$$
Since $RT^m(\binm)$ holds, there is an isomorphic copy
$\Po$ of $\binm$ that is homogeneous for $\tilde{c}$.
Then the image of $B$ under $\gamma$ is a homogeneous
copy of $\binm$ for $c$.
\end{proof}

\begin{cor}[\ACAo; $m\geq 1$]\label{C:binmRamsey}
$RT^m(\binm)$ holds.
\end{cor}

\begin{thm}\label{T:binm&ACA}
Fix $m\geq 3$.  $RT^m(\binm)$ and $\ACAo$ are equivalent over $\RCAo$.
\end{thm}
\begin{proof}
By Theorem~5 of Chubb, Hirst, McNichol \cite{CHM},
in order to show that $RT^m(\binm)$ implies $\ACAo$ over $\RCAo$
it suffices to show that $RT^m(\binm)$ implies $RT^m(\bin)$ over $\RCAo$.
Let $c:[\bin]^m\to\{0,1,\ldots k-1\}$ be a coloring.
Then we can define a coloring $\tilde{c}:[\binm]^m\to\{0,1,\ldots k-1\}$
by $\tilde{c}(\sigma,x)=c(\sigma)$.
Notice then that by projection onto the first coordinate,
any homogeneous copy $H$ of $\binm$ for $\tilde{c}$
computes a homogeneous copy $H'$ of $\bin$ for $c$.
\end{proof}

Note that the proofs of Propositions \ref{P:binmRamsey2} and \ref{P:binmRamseyInduct}
can easily be adapted for $\omegam$
($\omegam$ was defined in Definition~\ref{D:withMultis}).
In particular, the following proposition holds.
\begin{prop}\label{P:omegamRamsey}
Fix $m\geq 3$.  $RT^m(\omegam)$ and $\ACAo$ are equivalent over $\RCAo$.
\end{prop}

We now characterize the partial orderings contained in $\binm$
that have the $(n,k)$-Ramsey property.

\begin{definition}
We say that two partial orderings $\Po,\Qo$ are \textit{biembeddable} if
there exists embeddings $f:\Po\to\Qo$ and $g:\Qo\to\Po$.
\end{definition}

The characterization of the partial orderings contained in $\binm$
that have the $(n,k)$-Ramsey property will make use of the following easy lemma.

\begin{lem}[\RCAo]\label{L:biemebedRam}
Suppose that $\Po$, $\Qo$ are biembeddable partial orderings.
If $RT^n_k(\Po)$ holds then so does $RT^n_k(\Qo)$.
\end{lem}

\begin{proof}
Let $c:[\Qo]^n\to\{0,1,\ldots,k-1\}$ be a coloring.
Let $f:\Po\to\Qo$ and $g:\Qo\to\Po$ be embeddings.
Then $c\circ f$ defines a coloring of $\Po$.
Since $RT^n_k(\Po)$ holds there is a homogeneous set $H$ for $c\circ f$
and an isomorphism $\gamma:\Po\to H$ which preserves $\leq_\Po$.
Then the range of $f\circ \gamma\circ g$ is homogeneous for $c$ and
isomorphic to $\Qo$.
\end{proof}

\begin{definition}
Given a partial ordering $(\Po,\leq_\Po)$,
we say that $(\Qo,\leq_\Qo)$ is a \textit{subordering} of $\Po$ if
$\Qo\subseteq\Po$ and $\leq_\Qo$ is $\leq_\Po$ restricted to $\Qo$.
\end{definition}

\begin{prop}[\RCAo]\label{P:charRamBinm}
Let $n\geq 2$ and suppose that $\Po$ is a subordering of $\binm$
with a least element such that $RT^n(\Po)$ holds.
Suppose also that $\Po$ has elements on all levels.
Then $\Po$ is biembeddable with one of the following four partial orderings:
$\omega$, $\omegam$, $\bin$, or $\binm$.
\end{prop}

The proof of Proposition~\ref{P:charRamBinm} makes use of two results
from Corduan, Groszek, and Mileti \cite{CGM}.
\begin{prop}[Proposition 2.4 of Corduan-Groszek-Mileti \cite{CGM}, \RCAo]\label{P:CGM1}
Let $\Po$ be a countable partial ordering with a least element such that
$RT^2_2(\Po)$ holds and such that no pair of $\leq_\Po$-incomparable elements
have a $\leq_\Po$-upper bound.
Then $\Po$ is isomorphic to a downward-closed subtree of $\omega^{<\omega}$
(where $\omega^{<\omega}$ is the lexicographic ordering on all finite sequences in $\NN$).
\end{prop}

\begin{prop}[Lemma 2.7 of Corduan-Groszek-Mileti \cite{CGM}, \RCAo]\label{P:CGM2}
Let $n\geq 1$ and $k\geq 2$.
Let $T$ be a downward-closed subtree of $\omega^{<\omega}$
which is not linearly ordered.
Suppose also that $T$ has elements on all levels.
If $RT^n_k(T)$ holds then $T$ is biembeddable with $\bin$.
\end{prop}


\begin{proof}[Proof of Proposition~\ref{P:charRamBinm}]
Since $RT^n(\Po)$ holds, $RT^2(\Po)$ also holds.

Let $T\subseteq\bin$ be the projection of $\Po$ onto the first coordinate.
In other words, $T=\set{\sigma}{(\exists x\leq|\sigma|)[(\sigma,x)\in\Po]\ }$.
We now claim that $RT^2(T)$ holds.
Let $c':[T]^2\to\{0,1,\ldots,k-1\}$ be a coloring of $T$.
Then we can define another coloring $c'':[\Po]^2\to\{0,1,\ldots\}$
by $c''((\sigma,x),(\tau,y))=c'(\sigma,\tau)$.
Notice then that any set $H\subseteq\Po$ which is isomorphic
to $\Po$ and monochromatic for $c''$ computes a set $H'\subseteq\bin$
which is isomorphic to $T$ and monochromatic for $c'$.
Thus $RT^2(T)$ holds.
Therefore by Propositions~\ref{P:CGM1} and \ref{P:CGM2},
$T$ is either biembeddable with $\omega$ or $\bin$.

Given $\sigma\in\bin$, let $S(\sigma)=|\set{x}{(\sigma,x)\in\Po}|$.
Consider the coloring $c:[\Po]^2\to\{\red,\blue\}$ defined by
coloring $(\sigma,x)<(\tau,y)$ red if and only if $S(\sigma)<S(\tau)$.

There are now four cases.
In the first case, $T$ is biembeddable with $\bin$ and
there is a $\red$-homogeneous copy of $\Po$ for the coloring $c$.
It's then easy to see that there is an embedding of $\binm$ into $\Po$.

In the second case, $T$ is biembeddable with $\bin$ and
there is a $\blue$-homogeneous copy of $\Po$,
which we will name $\Po'$.
Let $T'$ be the projection of $\Po'$ onto the first coordinate,
let $g:\bin\to T'$ be an embedding, and let $m=S(g(\seq{}))$.
Notice that $S(\sigma)\leq m$ for every $\sigma\in T'$
since $\Po'$ is $\blue$-homogeneous.
We can then color the singletons of $\Po'$ with $m$-many
colors so that $c(\sigma,x)\neq c(\sigma,y)$ for all $(\sigma,x),(\sigma,y)\in\Po'$.
Since $RT^1(\Po)$ holds, we conclude that $S(\sigma)=1$
for all $\sigma\in T'$, and thus $\Po$ is isomorphic to $\bin$.

In the third [fourth] case we assume that $T$ is biembeddable with
$\omega$ and that there is a $\red$-homogeneous [$\blue$-homogeneous]
copy of $\Po$ for $c$.
It is then easy to see that $\Po$ is biembeddable with $\omegam$ [$\omega$].
\end{proof}

\begin{thm}\label{T:charRamInBinm}
Let $n\geq 3$.
The following statement is equivalent to \ACAo\ over \RCAo:
Let $\Po$ be a subordering of $\binm$ which has elements on all levels
and which has a least element.
Then $RT^n(\Po)$ holds if and only if $\Po$ is biembeddable with
$\omega$, $\omegam$, $\bin$, or $\binm$.
\end{thm}

\begin{proof}
First note that the statements $RT^n(\omega)$, $RT^n(\omegam)$, $RT^n(\bin)$, and $RT^n(\binm)$
are all individually equivalent to \ACAo\ over \RCAo\
(By \cite{SOSOA}, Proposition~\ref{P:omegamRamsey},
\cite{CHM}, and Corollary~\ref{C:binmRamsey} respectively).
Therefore the corollary holds by Proposition~\ref{P:charRamBinm} and Lemma~\ref{L:biemebedRam}.
\end{proof}
