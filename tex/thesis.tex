\documentclass{dcthesis}[]

\usepackage{amsthm,amssymb}
\usepackage{MnSymbol}
\usepackage[all]{xy}
\usepackage{hyperref}
\usepackage{makeidx}
\usepackage{allrunes}

\makeindex

\renewcommand{\labelenumi}{(\alph{enumi})}
\renewcommand{\labelenumii}{(\roman{enumii})}
\renewcommand{\chaptermark}[1]{\markboth{{\bf #1}}{{\bf #1}}}
\renewcommand{\sectionmark}[1]{\markright{{\bf \thesection\ #1}}}

\newtheorem{prop}{Proposition}[chapter]
\newtheorem{thm}[prop]{Theorem}
\newtheorem{lem}[prop]{Lemma}
\newtheorem{cor}[prop]{Corollary}
\theoremstyle{definition}
\newtheorem{definition}[prop]{Definition}


%Macros
\newcommand{\RCAo}{\ensuremath{\mathsf{RCA}_0}}
\newcommand{\WKLo}{\ensuremath{\mathsf{WKL}_0}}
\newcommand{\ACAo}{\ensuremath{\mathsf{ACA}_0}}
\newcommand{\Bnd}[1]{\ensuremath{\mathsf{B}{#1}}}
\newcommand{\Ind}[1]{\ensuremath{\mathsf{I}{#1}}}
\newcommand{\Reg}[1]{\ensuremath{\mathsf{R}{#1}}}
\newcommand{\BSigma}{\mathbf{\Sigma}}
\newcommand{\BDelta}{\mathbf{\Delta}}
\newcommand{\BPi}{\mathbf{\Pi}}
\newcommand{\EIndec}{\mathsf{Elem{\mbox{-}}Indec}}
\newcommand{\SEIndec}{\mathsf{Stab{\mbox{-}}Elem{\mbox{-}}Indec}}

\newcommand{\NN}{\mathbb{N}}
\newcommand{\bin}{2^{<\omega}}
\newcommand{\binm}{2^{<\omega}_{\mathsf{m}}}
\newcommand{\omegam}{\omega_\mathsf{m}}
\newcommand{\OO}{\omega^{<\omega}}
\newcommand{\Po}{\mathbb{P}}
\newcommand{\Qo}{\mathbb{Q}}
\newcommand{\MM}{\mathcal{M}}
\newcommand{\Forc}{\mathbb{F}}

\newcommand{\set}[2]{\left\{#1\ :\ #2\right\}}
\newcommand{\seq}[1]{\left\langle #1\right\rangle}
\newcommand{\tup}[1]{\overline{#1}}
\newcommand{\first}[1]{\mathsf{1st}(#1)}
\newcommand{\second}[1]{\mathsf{2nd}(#1)}

\newcommand{\cat}{{}^\frown}
\newcommand{\lex}{<_{\mathrm{lex}}}
\newcommand{\uhr}{{\upharpoonright}}
\newcommand{\LB}{\mathcal{L}}
\newcommand{\Pred}{\mathsf{Pred}}
\newcommand{\red}{\mathsf{red}}
\newcommand{\blue}{\mathsf{blue}}
\newcommand{\dom}{\mathsf{dom}}
\newcommand{\MCP}{\ensuremath{\mathsf{MCP}}}

% MYSTIC RUNE
\newcommand{\rune}{\textara{Y}}
%\newcommand{\rune}{$\tilde{A}$} % boring substitution

\title{Coloring Posets and Reverse Mathematics}
\author{Jared Ralph Corduan}
\date{January 21, 2011}
\field{Mathematics}
\degree{Doctor of Philosophy}
\committee{Marcia Groszek}{Jeff Hirst}{Rebecca Weber}{Peter Winkler}

\begin{document}

\frontmatter

\maketitle

\chapter*{Abstract}
\addcontentsline{toc}{section}{Abstract}
	We study two themes from Reverse Mathematics.
The first theme involves a generalization of the infinite version of
Ramsey's theorem to arbitrary partial orderings.
We say that a partial ordering $\Po$ has the $(n,k)$-Ramsey property,
and write $RT^n_k(\Po)$,
if for every $k$-coloring of the $n$-element chains of $\Po$
there is a homogeneous copy of $\Po$.

When $\Po$ is either a linear ordering or a tree, and $n\geq 3$, the statement
\newline $(\forall k\geq 1 )RT^n_k(\Po)$ is well understood
from the point of view of Reverse Mathematics \cite{CHM}\cite{CGM}.
We investigate $RT^n_k(\Po)$ for some partial orderings which are not trees.
We show that if $\Po$ is either the binary tree with multiplicities
or an amenable partial ordering, and if $n\geq 3$,
then the statement $(\forall k\geq 1 )RT^n_k(\Po)$
is equivalent to \ACAo\ over \RCAo.
We also classify which suborderings of
the binary tree with multiplicities have the Ramsey property.
Finally, we study the $(1,k)$-Ramsey property for the finite (ordinal) powers of $\omega$.
For these orderings it makes sense to consider a first-order definition
of ``an isomorphic copy of $\omega^n$'' and the corresponding version of $\forall k RT^1_k(\omega^n)$,
which we denote by $\EIndec^n$.
We place a lower bound on the complexity of $\EIndec^{n+1}$ by showing that it
is provable in $\RCAo+\Bnd{\Pi^0_n}$.
Jointly with Dorais, we show that $\RCAo+\Ind{\Sigma^0_{n+1}}$ proves $\EIndec^{n}$
and also that $RT^1_2(\omega^3)$ is equivalent to $\ACAo$ over $\RCAo$.

The second theme of our study involves set theoretic forcing over
models of \RCAo\ and \ACAo.
Our primary focus is on notions of forcing whose conditions are
subtrees of $\OO$ which are ordered by inclusion and have a simple property
that we call ``persistence".
In his paper ``A variant of Mathias forcing that
preserves \ACAo", Dorais guides the reader through an
interesting forcing construction \cite{varMathias}.
We use Dorais' framework and show that persistent notions of forcing
over models of \ACAo\ which satisfy a particular coloring property
give rise to generic extensions which also model \ACAo.
We also show that a slightly less restrictive property than persistence
suffices to guarantee that generic extensions of models of \RCAo\
are themselves models of \RCAo.
Lastly, we work through several examples:
Harrington, random, Sacks, Silver, and Miller forcing.


\chapter*{Acknowledgments}
\addcontentsline{toc}{section}{Acknowledgments}
	I wish to specially thank my advisor, Marcia Groszek, for all her guidance and unending patience.
I have learned much from her thoughtfulness, carefulness, and passion.

I also wish to specially thank Fran\c{c}ois Dorais for the enormous amount of
time that he has spent with me.
I have had many incredibly inspiring and awesome conversations
with him, each of which has taught me so much.

I am grateful for the mathematics department at Dartmouth.
My classmates Jon, Giulio, Geoff, and Matt were all instrumental in my education,
as was the logic group: Marcia, Rebecca, Pete, Joe, Brooke, John, Fran\c{c}ois, and Rachel.

I am also grateful for the many other wonderful mathematicians that I have met along the way.
I learned much during my first year of graduate school from Luke and Liz.
I am thankful for the logic community in general, and especially to Jeff Hirst and Reed Soloman.

I wish to express my gratitude to Tom Dechand for introducing me to
G\"{o}del's Incompleteness Theorem and for getting me excited about mathematics in general.

Finally, I would like to thank my friends and family for all the blessings that they have given me.
I would like to especially thank my grandparents, Frank and Florence, for all their support.
My final thanks are revserved for Erin, the best wife-beans in the world.


\tableofcontents

\mainmatter

\chapter{Introduction}\label{Intro}
\section{Reverse Mathematics}\label{IntroRevMath}
	Reverse Mathematics is a framework for asking and answering
questions of interest in the foundations of mathematics.
To quote Stephen Simpson from his encyclopedic reference book
on the subject, Reverse Mathematics is primarily motivated
by the following question:
\begin{quote}
Which set existence axioms are needed to
prove the theorems of ordinary, non-set-theoretic mathematics?
(page 2 of \cite{SOSOA}.)
\end{quote}
The words `ordinary' and `non-set-theoretic' are
not being used in a precise sense here,
though they are nonetheless very important.

Insofar as anything in mathematics can be seen as having
a definite beginning, the Reverse Mathematics program began
in 1975 with with Harvey Friedman's
``Some systems of second order arithmetic and their use"
\cite{Friedman:SomeSystems}.
Many of the results in Reverse Mathematics were foreshadowed
by results in Recursion Theory, and the two fields still
enjoy a close relationship today.
See Rogers \cite{Rogers} for the Recursion Theory notation
and definitions implicitly used in anything that follows.

Reverse Mathematics exists inside second order arithmetic.
The language of second order arithmetic is a two-sorted language
which has variable symbols intended to range over the natural numbers,
and set variables which are intended to range over \textit{sets} of natural numbers.
The language also includes the constant symbols 0 and 1, the function symbols
$+$ and $\cdot$, the predicates $=$, $<$, and $\in$, and the usual logical
symbols (including quantifiers for both number and set variables).
Full second order arithmetic is the theory whose axioms are those of
Robinson Arithmetic together with comprehension and induction axioms for all formulas.

Full second order arithmetic suffices to prove the majority
of the theorems in non-set-theoretic countable mathematics.
We use the term `countable mathematics' in a very broad sense
here, for though continuous real-valued functions are
third-order objects, we can make use of the fact that
the rationals are dense in the reals to state
theorems involving continuous real-valued functions in
the language of second order arithmetic.
Similarly, a lot of mathematics can be phrased
in the language of second order arithmetic.

In order to make precise and meaningful the motivating question of Reverse Mathematics,
namely which set existence axioms are needed to
prove the theorems of ordinary, non-set-theoretic mathematics,
an important subsystem of second order arithmetic,
namely \RCAo, is used as a baseline.
Intuitively, \RCAo\ embodies computable/recursive mathematics.
In fact, the acronym stands for ``Recursive Comprehension Axiom"
(and the subscript indicates that induction is limited).
The axioms of \RCAo\ consist of Robinson Arithmetic together with
comprehension axioms for the $\BDelta^0_1$ definable sets
and induction axioms for $\BSigma^0_1$ formulas.
Many theorems are not provable in \RCAo,
such as the statement that every countable vector space
over the rationals has a basis.
Taking this example, we can ask what axioms must be added
to \RCAo\ in order to prove the statement about vector spaces.
A \textit{very} unsatisfying answer is that we could
add the statement in question to \RCAo.
A satisfying answer is that adding all instances of arithmetic
comprehension suffices to prove this theorem \cite{SOSOA}.
The subsystem just mentioned, whose axioms are those of \RCAo\ together
with arithmetic comprehension, is also an important subsystem and is
called \ACAo\ (for arithmetic comprehension).
In what way could we say that \ACAo\ is the weakest subsystem that
proves that every countable vector space over the rationals has a basis?
It turns out that the axioms of \RCAo, together with the statement
that every countable vector space over the rationals has a basis,
suffice to prove all instances of arithmetic comprehension.
In other words, \RCAo\ proves the equivalence of \ACAo\ and
``every countable vector space over the rationals has a basis".

The fascinating thing about Reverse Mathematics is that the large
majority of the theorems of ordinary, non-set-theoretic mathematics turn out to
either be provable in \RCAo\ or be equivalent, over \RCAo, to one of
four natural subsystems of second order arithmetic.
To see many of these equivalences, and
for a more thorough introduction to Reverse Mathematics, see Simpson's
\textit{Subsystems of Second Order Arithmetic} \cite{SOSOA}.

\section{Ramsey's Theorem}\label{IntroRam}
	Ramsey's theorem \index{Ramsey's theorem} is often thought of as
a generalization of the pigeonhole principle.
Let $\NN$ denote the set of natural numbers.
Given a number $n$, let $[\NN]^n$ denote the
set of subsets of $\NN$ of size $n$.
The infinite version of Ramsey's theorem \cite{Ramsey}
(which we will refer to as Ramsey's theorem)
says that for every $n,k\geq 1$, and every map
$c$ from $[\NN]^n$ to the finite set $\{0,1,\ldots,k-1\}$,
there is an infinite set $H\subseteq\NN$ such that
$c$ is constant when restricted to $[H]^n$.
The map $c$ is referred to as a \textit{coloring} of $[\NN]^n$,
the finite set $\{0,1,\ldots,k-1\}$ is referred to as the set of
\textit{colors} of $c$, and the set $H$ is referred to as a
\textit{monochromatic} or \textit{homogeneous} set for $c$.
Sometimes we will fix a particular number $n$ and
refer to Ramsey's theorem for $n$-tuples (or pairs, etc).

Ramsey's Theorem was of interest in mathematical logic
even before Reverse Mathematics was around.
In 1971 Specker proved that there is a computable coloring
$c:[\NN]^2\to\{0,1\}$ that has no computable monochromatic set \cite{Specker:1,Specker:2}.
This corresponds to the fact in Reverse Mathematics that
Ramsey's theorem for pairs cannot be proved in \RCAo.
Many other interesting results about Ramsey's theorem
were proved by Jockusch in 1972 \cite{Jockusch:Ramsey}.
The results of Jockusch were later used by Simpson
to show that for any $n\geq 3$, Ramsey's theorem
for $n$-tuples is equivalent, over \RCAo, to \ACAo.
Hirst showed that Ramsey's theorem for singletons ($n=1$)
is equivalent to the bounding principle $\Bnd{\BSigma^0_2}$ \cite{Hirst:thesis}.
A definition of $\Bnd{\BSigma^0_2}$ will be given in Section~\ref{ElemIndec}.
Surprisingly, characterizing Ramsey's theorem for pairs remains an active
area of research \cite{SeetapunSlaman,CJS,HirschfeldtShore,DzhafarovHirst,DzhafarovJockusch}.
Much of what is known about Ramsey's theorem for
pairs was proved using the ideas of set theoretic forcing.

In Chapters~\ref{Posets} and \ref{Ords} we consider two different
generalizations of Ramsey's theorem.
In Chapter~\ref{Forcing} we consider a general framework
for making sense of forcing in Reverse Mathematics and prove
some conservation results.


\chapter{Coloring Posets}\label{Posets}
	Let us rephrase Ramsey's theorem \index{Ramsey's theorem} in a way that
anticipates a particular generalization.
Let $\omega$ be the usual ordering of the natural numbers.
A chain of length $n$, or $n$-chain for short, in $\omega$ is a sequence
$\langle a_1,a_2,\ldots,a_n\rangle$ of natural numbers
such that $a_i<a_{i+1}$ for each $1\leq i<n$.
Let $[\omega]^n$ denote the set of $n$-chains of $\omega$.
Ramsey's theorem for $n$-tuples says that for any
coloring of $[\omega]^n$ with finitely many colors,
there is a subset of natural numbers $H$ such that
$H$ is isomorphic (with respect to the ordering it
inherits from $\omega$) to $\omega$ and such that
$c$ is constant on the set of $n$-chains of $H$.
The idea of the following generalization of Ramsey's
theorem is to replace $\omega$ with another partial ordering.
Recall that a partial ordering is a pair $(\Po,\leq_{\Po})$,
where $\Po$ is a set and $\leq_{\Po}$ is a binary relation on $\Po$
which is reflexive, antisymmetric, and transitive.
Often times we will conflate $\Po$ and $\leq_{\Po}$.
We write $a_1<_{\Po}a_2$ to mean that $a_1\leq_{\Po}a_2$ and $a_1\neq a_2$.

\begin{definition}
Let $(\Po,\leq_{\Po})$ be a partial ordering and fix $n\in\NN$.
The set of $n$\textit{-chains} of $\Po$ is the set
$$[\Po]^n=\set{\seq{a_1,a_2\ldots,a_n}\in\Po^n}{a_1<_{\Po}a_2<_{\Po}\ldots <_{\Po}a_n}.$$
A $k$\textit{-coloring} of $[\Po]^n$ is a map
$$c:[\Po]^n\to\{0,1,\ldots,k-1\}.$$
A subset $H\subseteq\Po$ is \textit{homogeneous} for $c$ if there
is a $j<k$ such that $c(\tup{a})=j$ for all $\tup{a}\in[H]^n$.
We say that a set $H\subseteq\Po$ is a \textit{homogeneous copy} of $\Po$
for $c$ if $H$ is homogeneous and the partial ordering
$(H,\leq_{\Po}\uhr H)$ is isomorphic to $\Po$.
\end{definition}

\begin{definition}\label{D:RamseyPOsets} \index{$RT^n(\Po)$}
We say that a partial ordering $\Po$ has the $(n,k)$\textit{-Ramsey property}
if for every $k$-coloring of the $n$-chains of $[\Po]^n$
there is a homogeneous copy of $\Po$ for $c$.
We let $RT^n_k(\Po)$ denote the statement that $\Po$ has
the $(n,k)$-Ramsey property, and we let $RT^n(\Po)$
denote the statement that $\Po$ has the $(n,k)$-Ramsey property for all $k\geq 1$.
\end{definition}

Note that we require only that the partial ordering $(H,\leq_{\Po}\uhr H)$
be isomorphic to $\Po$ as a partial ordering.
We do not require the isomorphism to preserve lower bounds, etc.

Using this new notation, Ramsey's theorem for $n$-tuples
is denoted $RT^n(\omega)$.
It turns out that for $n\geq 2$, $\omega$ is essentially
the only countable linear ordering with the $n$-Ramsey property.
To see this, let $\Po=(\NN,\leq_{\Po})$ be a countable linear
ordering such that $RT^2(\Po)$ holds and consider the 2-coloring
$c(a,b)=0$ if and only if $a\leq_\Po b\leftrightarrow a\leq b$,
where $\leq$ is the usual ordering on $\NN$.
Notice then that every infinite 0-homogeneous set for $c$
is isomorphic to $\omega$, and every infinite 1-homogeneous set for $c$
is isomorphic to $\omega^*$
($\omega^*$ is the ordering $\leq^*$ of $\NN$
where $a\leq^*b$ if and only if $a\geq b$).
Therefore if $\Po=(\NN,\leq_{\Po})$ is a countable linear ordering such that
$RT^2(\Po)$ holds, then $\Po$ is isomorphic to either $\omega$ or $\omega^*$.
In Chapter~\ref{Ords} we will have more to say about
coloring problems for other linear orderings.

It is worth mentioning why we consider $n$-chains and not merely
$n$-element subsets of $\Po$.
Consider the following coloring of the pairs of a partial ordering $\Po$ with two
colors: let $c(a,b)=0$ exactly when $a$ is comparable to $b$
(we say that $a$ is comparable to $b$ if either $a\leq b$ or $b\leq a$).
The existence of a $0$-homogeneous copy of $\Po$ for this coloring
implies that $\Po$ is a linear ordering, while
existence of a $1$-homogeneous copy of $\Po$ implies that $\Po$ is a
countable antichain.
Therefore defining the Ramsey property in terms of coloring
of $n$-element \textit{subsets} is too restrictive to be of interest.
One could also consider coloring other substructures besides $n$-chains,
but these are usually too restrictive as well.

Chubb, Hirst, and McNichol investigated the statement
$RT^n(\bin)$, where $\bin$ is the \textit{complete binary tree}
(the set of finite binary sequences ordered by inclusion) \cite{CHM}.
They proved that $RT^n(\bin)$ holds for all $n\geq 1$
and that $RT^n_k(\bin)$ behaves very similarly to
$RT^n_k(\omega)$.
In particular, for all $n\geq 3$, they showed that the
statement $RT^3(\bin)$ is equivalent to \ACAo,
and that $RT^1(\bin)$ is provable from
$\Ind{\BSigma^0_2}$ and implies $\Bnd{\BSigma^0_2}$.

Groszek, Mileti and I considered the statement $RT^n(T)$
for arbitrary trees \cite{CGM}.
A \textit{tree} is a downward closed subset of $\omega^{<\omega}$
(the set of finite sequences, ordered by inclusion), and
a tree is \textit{nontrivial}\index{nontrivial tree} if it is not linearly ordered and
has at least one element on every level.
Two partial orderings are \textit{biembeddable}\index{biembeddable} if each
partial ordering can be embedded in the other.
We showed that for all $n\geq 1$, $k\geq 2$, \RCAo\ proves the following:
for all nontrivial trees $T$, $RT^n_k(T)$ holds if and only if $T$ is
biembeddable with $\bin$ and $RT^n_k(\bin)$ holds.
By the results of Chubb, Hirst, and McNichol it then
follows that for all $n\geq 3$, \ACAo\ is equivalent, over \RCAo, to the
statement:
\begin{quote}
If $T$ is a nontrivial tree, then $RT^n(T)$
holds if and only if there is an embedding of $\bin$ into $T$.
\end{quote}
We also showed that $RT^1(\bin)$ is strictly stronger than $\Bnd{\BSigma^0_2}$
(in other words, the reverse implication does not hold).
Recall that $RT^1(\omega)$ is equivalent to $\Bnd{\BSigma^0_2}$,
and so $RT^1(\bin)$ is a stronger statement than $RT^1(\omega)$.
This is in contrast with the fact that $RT^n(\omega)$
is equivalent to $RT^n(\bin)$ for $n\geq 3$, the result of Chubb, Hirst, and McNichol
mentioned earlier.

In Section~\ref{POMult} we investigate a partial ordering which
is not a tree, but is very similar to $\bin$.
We call this partial ordering \textit{the binary tree with multiplicities}.
We classify all suborderings (modulo some nontriviality requirements)
of the binary tree with multiplicities that have the Ramsey property for $n\geq 2$.
In Section~\ref{POAmenable} we investigate another family of partial
orderings satisfying the $(n,k)$-Ramsey property for each $n\geq 1$ and $k\geq 2$.
In Sections \ref{ElemIndec} and \ref{Indec&Embed} we consider some linear orderings
with the 1-Ramsey property.
More specifically, we consider the finite (ordinal) powers of $\omega$.
In each of these sections the investigation includes
analyzing, with respect to Reverse Mathematics, the strength of the
important statements.

\section{Binary Tree with Multiplicities}\label{POMult}
	We now turn our attention to another collection of
tree-like partial orderings with the $(n,k)$-Ramsey properties.
We begin by fattening the binary tree.

\begin{definition}\label{D:withMultis}
The \textit{binary tree with multiplicities}, \index{binary tree with multiplicities}\index{$\binm$}
denoted $\binm$, is the partial ordering whose elements are pairs $(\sigma,x)$,
where $\sigma\in\bin$, $x\in\NN$, and $x\leq|\sigma|$.
We order $\binm$ by essentially ignoring the second coordinate.
We let $(\sigma,x)<(\tau,y)$ if and only if $\sigma\subsetneq\tau$.

Similarly, \textit{omega with multiplicities}, denoted $\omegam$, \index{omega with multiplicities}\index{$\omegam$}
is the partial ordering whose elements are pairs $(a,x)\in\NN^2$
such that $x\leq a$.
We order $\omegam$ by ignoring the second coordinate.
\end{definition}

We will now show that $\binm$ has the $(n,k)$-Ramsey property
for all $n\geq 1$, $k\geq 2$.
The proofs involved will be adaptations of the proofs used
by Chubb, Hirst, and McNichol \cite{CHM} to show that $\bin$ has the
$(n,k)$-Ramsey properties.
We will also characterize the partial orderings contained in $\binm$
that have the $(n,k)$-Ramsey properties.
In terms of Reverse Mathematics,
we consider the strength of $RT^n(\binm)$ and
also the strength of the theorem which characterizes
the suborderings of $\binm$ having Ramsey properties.
The main results of this section are Theorem~\ref{T:binm&ACA} and Theorem~\ref{T:charRamInBinm}.
Theorem~\ref{T:binm&ACA} says that $RT^n(\binm)$ is equivalent
to $\ACAo$ for any fixed $n\geq 3$.
Theorem~\ref{T:charRamInBinm} characterizes the suborderings of $\binm$
with the Ramsey property and states that the characterization
is equivalent to $\ACAo$.

We can think of $RT^1(\Po)$ as a pigeonhole principle\index{pigeonhole principle} for $\Po$.
In fact, $RT^1(\omega)$ is the usual (infinite) pigeonhole principle.
We now show that adding multiplicities to $\omega$ and to $\bin$
does not add any complexity over \RCAo\ to the corresponding
pigeonhole principles.

\begin{prop}[\RCAo]\label{P:bin&binm1}
For all $k\geq 2$, $RT^1_k(\omegam)\leftrightarrow RT^1_k(\omega)$.
\end{prop}
\begin{proof}
Suppose $RT^1_k(\omegam)$ holds and $c:\NN\to\{0,1,\ldots,k-1\}$.
Let $\tilde{c}:\omegam\to\{0,1,\ldots,k-1\}$ be the coloring
defined by $\tilde{c}(a,x)=c(a)$.
Then any infinite homogeneous copy of $\omegam$ for $\tilde{c}$ induces an
infinite homogeneous set for $c$ by projecting onto the first coordinate.

On the other hand, suppose that $RT^1_k(\omega)$ holds and
let $c:\omegam\to\{0,1,\ldots,k-1\}$.
Let $\tilde{c}:\NN\to\{0,1,\ldots,k-1\}$ where
$\tilde{c}(a)$ is the color most used by $c$
on the set $\set{(a,x)}{x\leq a}$
(it does not matter how ties are settled, provided it is effective).
If $H\subseteq\NN$ is an infinite homogeneous set for $\tilde{c}$,
say in color $i$, then for each $a\in H$ there are at least
$\lceil \frac{a}{k}\rceil$-many numbers $x$ such that $c(a,x)=i$.
Therefore an infinite homogeneous set for $\tilde{c}$ computes
an infinite homogeneous set for $c$.
\end{proof}


\begin{prop}[\RCAo]
For all $k\geq 2$, $RT^1_k(\binm)\leftrightarrow RT^1_k(\bin)$.
\end{prop}
\begin{proof}
Suppose that $RT^1_k(\binm)$ holds and $c:\bin\to\{0,1,\ldots,k-1\}$.
We define a coloring $\tilde{c}:\binm\to\{0,1,\ldots,k-1\}$
by $\tilde{c}(\sigma,x)=c(\sigma)$.
Then any homogeneous copy of $\binm$ for $\tilde{c}$ induces a homogeneous
copy of $\bin$ for $c$ by projecting onto the first coordinate.

On the other hand, suppose that $RT^1_k(\bin)$ holds and
let $c:\binm\to\{0,1,\ldots,k-1\}$.
Let $\tilde{c}:\bin\to\{0,1,\ldots,k-1\}$ where
$\tilde{c}(\sigma)$ is the color most used by $c$
on the set $\set{(\sigma,x)}{x\leq |\sigma|}$.
Similarly to Proposition~\ref{P:bin&binm1},
any homogeneous copy of $\bin$ for $\tilde{c}$ computes
a homogeneous copy of $\binm$ for $c$.
\end{proof}

\begin{cor}[$\RCAo+\Ind{\BSigma^0_2}$]
$RT^1(\binm)$ holds.
\end{cor}
\begin{proof}
Chubb, Hirst, and McNichol proved that that $RT^1(\bin)$
holds in $\RCAo+\Ind{\BSigma^0_2}$ \cite{CHM}.
\end{proof}

We now show that $\binm$ has the $(n,k)$-Ramsey property for $n\geq 2$.
We begin with a few definitions and a helpful lemma.

Let $c:\binm\to\{\red,\blue\}$.
Given $\sigma\in\bin$, the \textit{standard red copy of $\binm$ above $\sigma$}, \index{standard red (blue) copy of $\binm$}
if it exists, is the isomorphism of $\binm$ into
$\Po=\set{(\tau,x)\in\binm}{c(\tau,x)=\red\ \land\ \tau\supseteq\sigma}$
obtained from the following stage-wise computable procedure.
At stage $0$ we search for a pair $(\tau,x)$ such that $c(\tau,x)=\red$ and
$\tau\supseteq\sigma$.
We map $(\langle\rangle,0)$ to the first such pair that is found,
which we call $(\tau_{\langle\rangle},x_{\langle\rangle})$.
At stage $1$ we search above $(\tau_{\langle\rangle},x_{\langle\rangle})$ for two incomparable nodes
$\tau_{\langle 0\rangle}$ and $\tau_{\langle 1\rangle}$ above $\tau_{\langle \rangle}$
and numbers $x^0_{\langle 0\rangle}$, $x^1_{\langle 0\rangle}$,
$x^0_{\langle 1\rangle}$, and $x^1_{\langle 1\rangle}$ such that
$c(\tau_{\langle i\rangle},x^j_{\langle i\rangle})=\red$ for each $i,j<2$.
We then map $(\langle i\rangle,j)$ to
$(\tau_{\langle i\rangle},x^j_{\langle i\rangle})$.
We continue in this way to define the entire isomorphism.
At the beginning of stage $(n+1)$ we will have already defined the
isomorphism up to level $n$.
During stage $(n+1)$ we do the following for every $\rho\in\bin$
such that $|\rho|=n$.
Search for 2 incomparable nodes
$\tau_{\rho\cat 0}$ and $\tau_{\rho\cat 1}$ above $\tau_\rho$ and numbers
$x^0_{\rho\cat 0}$, $x^1_{\rho\cat 0}$, $\ldots$, $x^n_{\rho\cat 0}$,
$x^0_{\rho\cat 1}$, $x^1_{\rho\cat 1}$, $\ldots$, $x^n_{\rho\cat 1}$
such that
$c(\tau_{\langle i\rangle},x^j_{\langle i\rangle})=\red$ for each $i<2$ and $j\leq n$.
We then map $(\langle i\rangle,j)$ to
$(\tau_{\langle i\rangle},x^j_{\langle i\rangle})$.

The \textit{standard blue copy of $\binm$ above $\sigma$},
if it exists, is defined similarly.

\begin{lem}[\RCAo]\label{L:stndCpyBinm2}
Let $c:\binm\to\{\red,\blue\}$.
For each $(\sigma,m)\in\binm$, either
the standard red copy of $\binm$ above $\sigma$ exists,
or there is a $\tau\supseteq\sigma$ such that
the standard blue copy of $\binm$ above $\tau$ exists.
\end{lem}
\begin{proof}
Suppose that
$$(\exists \tau\supseteq\sigma)(\exists m\in\NN)(\forall \rho\supseteq\tau)\big(|\set{(\rho,x)}{c(\rho,x)=\red}|\leq m\big).\ \ \ \ \ (*)$$
From $(*)$ it follows that the standard blue copy of $\binm$ above $\tau$ exists.
If $(*)$ fails, however, then for every $\tau\supseteq\sigma$ and every $m$ there is a
$\rho\supseteq\tau$ such that
$$|\set{x\leq|\rho|}{c(\rho,x)=\red}|\geq m.$$
From this it follows that the standard red copy of $\binm$ above $\sigma$ exists.
\end{proof}

We now prove a more general version of Lemma~\ref{L:stndCpyBinm2}.
Given a coloring $c:\binm\to\{0,1,\ldots,k-1\}$, a $\sigma\in\bin$, and an $\ell<k$,
we define the standard copy of $\binm$ in color $\ell$ above $\sigma$,
if it exists, in the same way that we defined the standard red copy of
$\binm$ above $\sigma$ when dealing with colorings of the form $c:\binm\to\{\red,\blue\}$.

\begin{lem}[$\RCAo+\Ind{\BSigma^0_2}$]\label{L:stndCpyBinmK}
Let $c:\binm\to\{0,1,\ldots,k-1\}$.
There is a $\tau\supseteq\sigma$ and a $j<k$
such that the standard copy of $\binm$ in color $j$ above $\sigma$ exists.
\end{lem}
\begin{proof}
Consider the finite set $C$ consisting of all $j<k$ such that
$$(\exists\tau\supseteq\sigma)(\exists m)(\forall\rho\supseteq\tau)\Big[|\set{x\leq|\rho|}{c(\rho,x)<j}|\leq m\Big].$$
Note that $C$ exists by bounded $\BSigma^0_2$ comprehension, which is equivalent to $\Ind{\BSigma^0_2}$.
Also note that $C$ is nonempty since $0\in C$.
Let $j=\max\{C\}$.

Let $\tau$ and $m$ be such that $|\set{x\leq|\rho|}{c(\rho,x)<j}|\leq m$
for all $\rho\supseteq\tau$.
If $j=k-1$, then we have that
$|\set{x\leq|\rho|}{c(\rho,x)=k-1}|\geq (|\rho|-m)$
for all $\rho\supseteq\tau$.
Therefore the standard copy of $\binm$ in color $k-1$ above $\sigma$ exists.

We now suppose, for the sake of contradiction, that $j<k-1$ and that the
standard copy of $\binm$ in color $j$ above $\sigma$ does not exist.
Then there is a $\rho\supseteq\tau$ and an $m'$
such that for all $\rho'\supseteq\rho$, the set $\set{x\leq|\rho'|}{c(\rho',x)=j}$
has at most $m'$ many elements.
But then $|\set{x\leq|\rho'|}{c(\rho',x)<j+1}|\leq (m+m')$ for all $\rho'\supseteq\rho$
contradicting the maximality of $j$ in $C$.
\end{proof}

\begin{prop}[\ACAo]\label{P:binmRamsey2}
$RT^2(\binm)$ holds.
\end{prop}
\begin{proof}
Let $c:[\binm]^2\to\{0,1,\ldots k-1\}$.
We will define a coloring $\tilde{c}:\binm\to\{0,1,\ldots,k-1\}$ in stages.
In the construction, we will also make use of a family of (partially defined) colorings
of the singletons of $\binm$. Given an element $\alpha=(\sigma,x)\in\binm$,
let $c_\alpha$ be the coloring defined by
$$c_\alpha(\tau,y)=c\big((\sigma,x),(\tau,y)\Big).$$
Note that $c_\alpha$ is only defined for $(\tau,y)>(\sigma,x)$.

Our construction will be computable relative to the following
arithmetic function $\mathcal{S}$.
The function $\mathcal{S}$ takes as input the following:
an index $e$ for a computable embedding $\Phi_e$ of $\binm$ into itself,
an index $e'$ for a computable $k$-coloring of the image of $\Phi_e$,
and a node $\sigma$ in the range of $\Phi_e$.
Given valid $e$, $e'$, and $\sigma$, the function $\mathcal{S}$
outputs a pair $(\tau,j)$ such that $\tau\supseteq\sigma$
and the standard copy of $\binm$ in color $j$ inside the image of $\Phi_e$
above $\tau$ exists.
By Lemma~\ref{L:stndCpyBinmK} $\mathcal{S}(e,e',\tau)$ is defined
whenever $e$, $e,$, and $\tau$ are valid inputs.
Since the statement that the standard copy of $\binm$ in color $j$ above $\sigma$ exists
is arithmetic, it follows that $\mathcal{S}$ is arithmetic.

At stage 0 we begin by letting $\sigma_{\seq{}}=\seq{}$,
$x_{\seq{}}^0=0$, and $\alpha=(\sigma_{\seq{}},x^0_{\seq{}})$.
Let $e$ be an index for the identity map on $\binm$,
let $e'$ be an index for $c_{\alpha}$, and let $\mathcal{S}(e,e',\sigma_{\seq{}})=(\tau,j)$.
Let $e_{\seq{}}$ be an index for the standard copy of $\binm$ in color $j$
above $\tau$, and let $B_{\seq{}}$ be the image of $\Phi_{e_{\seq{}}}$.
Finally, let $\tilde{c}(\seq{},0)=j$.

At stage 1 we find incomparable nodes $\sigma_{\langle 0\rangle}$, $\sigma_{\langle 1\rangle}$
and numbers $x^0_{\langle 0\rangle}$, $x^1_{\langle 0\rangle}$, $x^0_{\langle 1\rangle}$
and $x^1_{\langle 1\rangle}$ such that
$(\sigma_{\langle i\rangle},x^j_{\langle i\rangle})\in B_{\langle \rangle}$
for each $i,j<2$.
Let $\alpha^j_{\seq{i}}=(\sigma_{\langle i\rangle},x^j_{\seq{i}})$.
We are guaranteed to find the elements $\alpha^j_{\seq{i}}$ since $B_{\langle \rangle}$
is isomorphic to $\binm$.
Let $e'$ be an index for the coloring of $B_{\seq{}}$ induced by $c_{\alpha^0_{\seq{0}}}$,
and let $\mathcal{S}(e_{\seq{}},e',\Phi_{e_{\seq{}}}(\seq{0}))=(\tau,j)$.
Therefore the standard copy of $\binm$ in color $j$ inside $B_{\seq{}}$ above
$\tau\supseteq\Phi_{e_{\seq{}}}(\seq{0})$
(with respect to the coloring that $c_{\alpha^0_{\seq{0}}}$ induces on $B_{\seq{}}$) exists.
We let $B_{\langle 0\rangle}'\subseteq B_{\seq{}}$ be this standard copy.
We also let $\tilde{c}(\seq{0},0)=j$.
Besides finding the nodes $\alpha^j_{\seq{i}}$,
what we have accomplished so far during stage 1 is to
find a computable copy $B_{\seq{0}}'\subseteq B_{\seq{}}$ of $\binm$
such that $c_{\alpha^0_{\seq{0}}}$ is constant on $B_{\seq{0}}'$.
We can mimic what we have just done to compute
(still relative to the function $\mathcal{S}$)
a copy $B_{\seq{0}}\subseteq B_{\seq{0}}'$ of $\binm$ such that
$c_{\alpha^1_{\seq{0}}}$ is constant on $B_{\seq{0}}$.
Using this same method we also compute (using $\mathcal{S}$)
a copy $B_{\seq{1}}\subseteq B_{\seq{}}$ of $\binm$ above $\Phi_{e_{\seq{}}}(\seq{1})$
such that $c_{\alpha^i_{\seq{0}}}$ is constant on $B_{\seq{0}}$ for each $i<2$.
Consistent with our definition of $\tilde{c}(\seq{0},0)$,
we let $\tilde{c}(\seq{i},j)$ be the color that
$c_{\alpha^j_{\seq{i}}}$ makes constantly on $B_{\seq{i}}$.

At stage $n+1$ we essentially mimic what we did at stage 1 above each node
$\sigma_{\tau}$, where $|\tau|=n$, which was defined at stage $n$.
Pick one such $\tau$ and let $\sigma=\sigma_\tau$ and
$\tau_i=\tau\cat{\seq{i}}$ for each $i<2$.
Note that $B_{\tau}$ was also defined at stage $n$.
We now find incomparable nodes $\sigma_{\tau_0}$, $\sigma_{\tau_1}$
and numbers $x^j_{\tau_i}$ such that
$(\sigma_{\tau_i},x^j_{\tau_i})\in B_{\tau}$
for each $j\leq (n+1)$ and $i<2$.
Let $\alpha^j_{\tau_i}=(\sigma_{\tau_i},x^j_{\tau_i})$.
We are guaranteed to find the elements $\alpha^j_{\tau_i}$ since $B_{\tau}$
is isomorphic to $\binm$.
Just as in stage 1, using the function $\mathcal{S}$
to guide the construction,
we find indices for computable copies
$B_{\tau_0},B_{\tau_1}\subseteq B_{\tau}$
of $\binm$ such that $c_{\alpha^j_{\tau_i}}$ is constant on
$B_{\tau_i}$ for each $i<2$ and $j\leq{n+1}$.
We also define $\tilde{c}(\tau_i,j)$ to be the color
that $c_{\alpha^j_{\tau_i}}$ is on $B_{\tau_i}$.

This ends the construction.
We now have a coloring $\tilde{c}:\binm\to\{0,1,\ldots,k-1\}$
and an embedding $\gamma$ of $\binm$ into itself,
defined by $\gamma(\sigma,j)=\alpha^j_\sigma$,
such that if $\tau,\rho\supset\sigma$ then
$$\tilde{c}(\sigma,x)=c(\alpha_\sigma^x,\alpha_\tau^y)=c(\alpha_\sigma^x,\alpha_\rho^z)$$
for all $x\leq|\sigma|$, $y\leq|\tau|$, and $z\leq|\rho|$.
Since $RT^1(\binm)$ holds, there is an isomorphic copy
$B$ of $\binm$ that is homogeneous for $\tilde{c}$.
Then the image of $B$ under $\gamma$ is a homogeneous
copy of $\binm$ for $c$.
\end{proof}

\begin{prop}[\ACAo; $m\geq 2$]\label{P:binmRamseyInduct}
If $RT^{m}(\binm)$ holds then so does $RT^{m+1}(\binm)$.
\end{prop}
\begin{proof}
The following proof will be similar to that of Proposition~\ref{P:binmRamsey2}.

Let $c:[\binm]^{m+1}\to\{0,1,\ldots k-1\}$.
We will define a coloring $\tilde{c}:[\binm]^m\to\{0,1,\ldots,k-1\}$ in stages.
In the construction, we will also make use of a family of colorings of the singletons of $\binm$.
Suppose $F\subset[\binm]^m$ is a finite collection of $m$-chains of $\binm$
such that there is a $\tau\in\bin$ with the property that if
$\seq{\alpha_1,\ldots,\alpha_m}\in F$ then $\alpha_m=(\tau,x)$ for some $x$.
We can use $F$ to define a partial coloring $c_F$ of the singletons of $\binm$
with $k^{|F|}$ many colors by
$$c_F(\beta)=\Big\langle c(\alpha_{1},\ldots,\alpha_{m},\beta):\seq{\alpha_{1},\ldots,\alpha_{m},\beta}\in F \Big\rangle.$$
Note that $c_F$ is only defined for $\beta=(\sigma,x)\in\binm$ such that $\tau\subsetneq\sigma$.

Our construction will be computable relative to the following
arithmetic function $\mathcal{S}$.
The function $\mathcal{S}$ takes as input the following:
an index $e$ for a computable embedding $\Phi_e$ of $\binm$ into itself,
an index $e'$ for a computable $k$-coloring of the image of $\Phi_e$,
and a node $\sigma$ in the range of $\Phi_e$.
Given valid $e$, $e'$, and $\sigma$, the function $\mathcal{S}$
outputs a pair $(\tau,j)$ such that $\tau\supseteq\sigma$
and the standard copy of $\binm$ in color $j$ inside the image of $\Phi_e$
above $\tau$ exists.
By Lemma~\ref{L:stndCpyBinmK} $\mathcal{S}(e,e',\tau)$ is defined
whenever $e$, $e'$, and $\tau$ are valid inputs.
Since the statement that the standard copy of $\binm$ in color $j$ above $\sigma$ exists
is arithmetic, it follows that $\mathcal{S}$ is arithmetic.

At stage 0 we begin by letting $\sigma_{\tau}=\tau$,
$x_{\tau}^i=i$, and $\alpha^i_{\tau}=(\sigma_{\tau},x_{\tau}^i)$
for each $|\tau|\leq m$ and $i\leq|\tau|$.
In other words, the map $(\tau,i)\mapsto\alpha^i_\tau$
is the identity map for elements on or below level $m$ of $\binm$.
For each $\tau$ such that $|\tau|=m$ we do the following.
Let $F$ be the set of $m$-chains $\seq{\alpha_1,\ldots,\alpha_m}$ such that
$\alpha_m=(\tau,x)$ for some $x$.
Let $e$ be an index for the identity map on $\binm$,
let $e'$ be an index for $c_{F}$, and let $\mathcal{S}(e,e',\tau)=(\rho,j)$.
Let $B_{\tau}$ be the standard copy of $\binm$ in color $j$ above $\rho$.
Finally, for each $m$-chain $\seq{\alpha_1,\ldots,\alpha_m}\in F$
let $\tilde{c}(\seq{\alpha_1,\ldots,\alpha_m})$ be the color that
$c_F$ makes constantly on $B_\tau$.
Stage 0 is complete when the procedure just described has been
completed for each $\tau$ such that $|\tau|=m$.

At stage $n+1$ we to the following for each $\tau$
such that $|\tau|=n+m-1$.
The node $\sigma_\tau\in\bin$, as well as $B_\tau$,
was defined in stage $n$
(in fact $\sigma_\rho$ has been defined for all
$|\rho|<n+m$ by the beginning of stage $n+1$).
Let $\tau_i=\tau\cat{\seq{i}}$ for each $i<2$.
We now find incomparable nodes $\sigma_{\tau_0}$, $\sigma_{\tau_1}$
and numbers $x^j_{\tau_i}$ such that
$(\sigma_{\tau_i},x^j_{\tau_i})\in B_{\tau}$
for each $j\leq (n+m)$ and $i<2$.
First we focus on $\sigma_{\tau_0}$.
Let $F$ be the set of all $m$-chains
$\seq{(\sigma_{\rho_1},x^{j_1}_{\rho_1}),\ldots,(\sigma_{\rho_{m-1}},x^{j_{m-1}}_{\rho_{m-1}}),(\sigma_{\tau_0},x)}$
where $x\leq|\tau|$ and the elements $(\sigma_{\rho_\ell},x^{j_\ell}_{\rho_\ell})$
range over all those defined in earlier stages.
The partial coloring $c_F$ induces a coloring on the elements
of $B_\tau$ above the image of $\tau_0$ in $B_\tau$.
We then use $\mathcal{S}$ to get a pair $(\rho,j)$
such that the standard copy of $\binm$ in color $j$ above $\rho$
in $B_\tau$ exists.
Let $B_{\tau_0}$ be the standard copy of $\binm$ in color $j$ above $\rho$ in $B_\tau$.
Finally, for each $m$-chain $\seq{\alpha_1,\ldots,\alpha_m}\in F$
let $\tilde{c}(\seq{\alpha_1,\ldots,\alpha_m})$ be the color that
$c_F$ makes constantly on $B_{\tau_0}$.
Similarly we define $B_{\tau_1}\subseteq B_\tau$ and
$\tilde{c}(\seq{\alpha_1,\ldots,\alpha_m})$ for $\alpha_m=(\tau_1,x)$.
Stage $n+1$ is complete when the procedure just described has been
completed for each $\tau$ such that $|\tau|=n+m-1$.

This ends the construction.
We now have a coloring $\tilde{c}:[\binm]^m\to\{0,1,\ldots,k-1\}$
and an embedding $\gamma$ of $\binm$ into itself,
defined by $\gamma(\tau,j)=(\sigma_\tau,x_\tau^j)$,
such that for any $m$-chain $\seq{\alpha_i}_{i=1}^m\in[\binm]^m$,
if $\beta_1,\beta_2\in\binm$ and $\beta_1,\beta_2\geq \alpha_m$, then
$$\tilde{c}(\alpha_1,\ldots,\alpha_m)=c\Big(\gamma(\alpha_1),\ldots,\gamma(\alpha_m),\gamma(\beta_1)\Big)=c\Big(\gamma(\alpha_1),\ldots,\gamma(\alpha_m),\gamma(\beta_2)\Big).$$
Since $RT^m(\binm)$ holds, there is an isomorphic copy
$\Po$ of $\binm$ that is homogeneous for $\tilde{c}$.
Then the image of $B$ under $\gamma$ is a homogeneous
copy of $\binm$ for $c$.
\end{proof}

\begin{cor}[\ACAo; $m\geq 1$]\label{C:binmRamsey}
$RT^m(\binm)$ holds.
\end{cor}

\begin{thm}\label{T:binm&ACA}
Fix $m\geq 3$.  $RT^m(\binm)$ and $\ACAo$ are equivalent over $\RCAo$.
\end{thm}
\begin{proof}
By Theorem~5 of Chubb, Hirst, McNichol \cite{CHM},
in order to show that $RT^m(\binm)$ implies $\ACAo$ over $\RCAo$
it suffices to show that $RT^m(\binm)$ implies $RT^m(\bin)$ over $\RCAo$.
Let $c:[\bin]^m\to\{0,1,\ldots k-1\}$ be a coloring.
Then we can define a coloring $\tilde{c}:[\binm]^m\to\{0,1,\ldots k-1\}$
by $\tilde{c}(\sigma,x)=c(\sigma)$.
Notice then that by projection onto the first coordinate,
any homogeneous copy $H$ of $\binm$ for $\tilde{c}$
computes a homogeneous copy $H'$ of $\bin$ for $c$.
\end{proof}

Note that the proofs of Propositions \ref{P:binmRamsey2} and \ref{P:binmRamseyInduct}
can easily be adapted for $\omegam$
($\omegam$ was defined in Definition~\ref{D:withMultis}).
In particular, the following proposition holds.
\begin{prop}\label{P:omegamRamsey}
Fix $m\geq 3$.  $RT^m(\omegam)$ and $\ACAo$ are equivalent over $\RCAo$.
\end{prop}

We now characterize the partial orderings contained in $\binm$
that have the $(n,k)$-Ramsey property.

\begin{definition}
We say that two partial orderings $\Po,\Qo$ are \textit{biembeddable} if
there exists embeddings $f:\Po\to\Qo$ and $g:\Qo\to\Po$.
\end{definition}

The characterization of the partial orderings contained in $\binm$
that have the $(n,k)$-Ramsey property will make use of the following easy lemma.

\begin{lem}[\RCAo]\label{L:biemebedRam}
Suppose that $\Po$, $\Qo$ are biembeddable partial orderings.
If $RT^n_k(\Po)$ holds then so does $RT^n_k(\Qo)$.
\end{lem}

\begin{proof}
Let $c:[\Qo]^n\to\{0,1,\ldots,k-1\}$ be a coloring.
Let $f:\Po\to\Qo$ and $g:\Qo\to\Po$ be embeddings.
Then $c\circ f$ defines a coloring of $\Po$.
Since $RT^n_k(\Po)$ holds there is a homogeneous set $H$ for $c\circ f$
and an isomorphism $\gamma:\Po\to H$ which preserves $\leq_\Po$.
Then the range of $f\circ \gamma\circ g$ is homogeneous for $c$ and
isomorphic to $\Qo$.
\end{proof}

\begin{definition}
Given a partial ordering $(\Po,\leq_\Po)$,
we say that $(\Qo,\leq_\Qo)$ is a \textit{subordering} of $\Po$ if
$\Qo\subseteq\Po$ and $\leq_\Qo$ is $\leq_\Po$ restricted to $\Qo$.
\end{definition}

\begin{prop}[\RCAo]\label{P:charRamBinm}
Let $n\geq 2$ and suppose that $\Po$ is a subordering of $\binm$
with a least element such that $RT^n(\Po)$ holds.
Suppose also that $\Po$ has elements on all levels.
Then $\Po$ is biembeddable with one of the following four partial orderings:
$\omega$, $\omegam$, $\bin$, or $\binm$.
\end{prop}

The proof of Proposition~\ref{P:charRamBinm} makes use of two results
from Corduan, Groszek, and Mileti \cite{CGM}.
\begin{prop}[Proposition 2.4 of Corduan-Groszek-Mileti \cite{CGM}, \RCAo]\label{P:CGM1}
Let $\Po$ be a countable partial ordering with a least element such that
$RT^2_2(\Po)$ holds and such that no pair of $\leq_\Po$-incomparable elements
have a $\leq_\Po$-upper bound.
Then $\Po$ is isomorphic to a downward-closed subtree of $\omega^{<\omega}$
(where $\omega^{<\omega}$ is the lexicographic ordering on all finite sequences in $\NN$).
\end{prop}

\begin{prop}[Lemma 2.7 of Corduan-Groszek-Mileti \cite{CGM}, \RCAo]\label{P:CGM2}
Let $n\geq 1$ and $k\geq 2$.
Let $T$ be a downward-closed subtree of $\omega^{<\omega}$
which is not linearly ordered.
Suppose also that $T$ has elements on all levels.
If $RT^n_k(T)$ holds then $T$ is biembeddable with $\bin$.
\end{prop}


\begin{proof}[Proof of Proposition~\ref{P:charRamBinm}]
Since $RT^n(\Po)$ holds, $RT^2(\Po)$ also holds.

Let $T\subseteq\bin$ be the projection of $\Po$ onto the first coordinate.
In other words, $T=\set{\sigma}{(\exists x\leq|\sigma|)[(\sigma,x)\in\Po]\ }$.
We now claim that $RT^2(T)$ holds.
Let $c':[T]^2\to\{0,1,\ldots,k-1\}$ be a coloring of $T$.
Then we can define another coloring $c'':[\Po]^2\to\{0,1,\ldots\}$
by $c''((\sigma,x),(\tau,y))=c'(\sigma,\tau)$.
Notice then that any set $H\subseteq\Po$ which is isomorphic
to $\Po$ and monochromatic for $c''$ computes a set $H'\subseteq\bin$
which is isomorphic to $T$ and monochromatic for $c'$.
Thus $RT^2(T)$ holds.
Therefore by Propositions~\ref{P:CGM1} and \ref{P:CGM2},
$T$ is either biembeddable with $\omega$ or $\bin$.

Given $\sigma\in\bin$, let $S(\sigma)=|\set{x}{(\sigma,x)\in\Po}|$.
Consider the coloring $c:[\Po]^2\to\{\red,\blue\}$ defined by
coloring $(\sigma,x)<(\tau,y)$ red if and only if $S(\sigma)<S(\tau)$.

There are now four cases.
In the first case, $T$ is biembeddable with $\bin$ and
there is a $\red$-homogeneous copy of $\Po$ for the coloring $c$.
It's then easy to see that there is an embedding of $\binm$ into $\Po$.

In the second case, $T$ is biembeddable with $\bin$ and
there is a $\blue$-homogeneous copy of $\Po$,
which we will name $\Po'$.
Let $T'$ be the projection of $\Po'$ onto the first coordinate,
let $g:\bin\to T'$ be an embedding, and let $m=S(g(\seq{}))$.
Notice that $S(\sigma)\leq m$ for every $\sigma\in T'$
since $\Po'$ is $\blue$-homogeneous.
We can then color the singletons of $\Po'$ with $m$-many
colors so that $c(\sigma,x)\neq c(\sigma,y)$ for all $(\sigma,x),(\sigma,y)\in\Po'$.
Since $RT^1(\Po)$ holds, we conclude that $S(\sigma)=1$
for all $\sigma\in T'$, and thus $\Po$ is isomorphic to $\bin$.

In the third [fourth] case we assume that $T$ is biembeddable with
$\omega$ and that there is a $\red$-homogeneous [$\blue$-homogeneous]
copy of $\Po$ for $c$.
It is then easy to see that $\Po$ is biembeddable with $\omegam$ [$\omega$].
\end{proof}

\begin{thm}\label{T:charRamInBinm}
Let $n\geq 3$.
The following statement is equivalent to \ACAo\ over \RCAo:
Let $\Po$ be a subordering of $\binm$ which has elements on all levels
and which has a least element.
Then $RT^n(\Po)$ holds if and only if $\Po$ is biembeddable with
$\omega$, $\omegam$, $\bin$, or $\binm$.
\end{thm}

\begin{proof}
First note that the statements $RT^n(\omega)$, $RT^n(\omegam)$, $RT^n(\bin)$, and $RT^n(\binm)$
are all individually equivalent to \ACAo\ over \RCAo\
(By \cite{SOSOA}, Proposition~\ref{P:omegamRamsey},
\cite{CHM}, and Corollary~\ref{C:binmRamsey} respectively).
Therefore the corollary holds by Proposition~\ref{P:charRamBinm} and Lemma~\ref{L:biemebedRam}.
\end{proof}

\section{Amenable Posets}\label{POAmenable}
	We now look at a family of partial orderings that have the $(n,k)$-Ramsey property
for all $n\geq 1,k\geq 2$.
We call the members of this family \textit{amenable} partial orderings,
and we define them in Definition~\ref{D:amenable}.
Groszek first proved that the amenable partial orderings
have the Ramsey properties \cite{GroszekAmenable}.
The main result of this section is Theorem~\ref{T:amenableIsRamsey},
which states that \ACAo\ suffices to prove that
the amenable partial orderings have the Ramsey properties.
The reversal of Theorem~\ref{T:amenableIsRamsey}, which is
stated as Corollary~\ref{C:amenable&ACA}, follows easily
from Theorem~\ref{T:amenableIsRamsey} and a theorem
of Chubb, Hirst, and McNichol \cite{CHM}.

Observe that if $RT^2_2(\Po)$ holds, then $\Po$ has a linearization of order type
$\omega$ or $\omega^*$ (the negative integers, with the usual ordering).
This follows by looking at the homogeneous set for the coloring $c:\Po\to 2$ defined by letting
$c(x,y)=0$ if $x\leq_{\Po}y \Rightarrow  x\leq y$, and letting $c(x,y)=1$ otherwise
(where $\leq$ is the usual ordering on $\NN$).
It's easy to see that $RT^n(\Po)$ holds if and only if $RT^n(\Po^*)$ holds,
where $\Po^*$ is the ordering defined by $x\leq_{\Po^*} y$ if and only if $y\leq_\Po x$.
We therefore make the simplifying assumption that our partial orderings have height $\omega$.

Sometimes we will be interested the function which, given any $n\in\NN$,
gives the set of all elements of $\Po$ on level $n$.
This function is not necessarily computable from $\Po$.
If $\Po$ has finite levels, then this function is, however,
computable from $\Po$ and any level bounding function $\LB_{\Po}$.
By a \textit{level bounding function}\index{level bounding function} we mean a
function $\LB_{\Po}:\NN\to\NN$ such that
the elements of $\Po$ on level $n$ are contained in $\{0,1,\ldots,\LB_{\Po}(n)\}$.

Given a partial ordering $\Po$ with height $\omega$,
we define $\Pred(x)$ \index{$\Pred(x)$} to be the set of all predecessors of $x$ and we define
$\Pred_n(x)$ \index{$\Pred_n(x)$} to be the set of all predecessors of $x$ on level $n$.
For $x,y\in\Po$, we define $x\equiv_p y$ \index{$\equiv_p $} if $\Pred(x)=\Pred(y)$
and denote the equivalence class of $x$ by $[x]_p$.

We define a second equivalence relation $\equiv$ \index{$\equiv $} to be the transitive closure
of the compatibility relation, where $x$ is \textit{compatible} to $y$
if $x$ and $y$ are on the same level and there is an
element $z$ such that $x,y\leq_\Po z$.
We denote the $\equiv$ equivalence class of $x$ by $[x]$.

Given a $\equiv$-equivalence class $a$ in level $n$, we define a finite bipartite
graph $G_{a}$\index{$G_{a}$} as follows.
One part of the graph, which we denote by $M(G_a)$\index{$M(G_a)$}, consists of the elements of $a$.
The second part, denoted $S(G_a)$\index{$S(G_a)$}, is the collection of all subsets $\Pred_n(y)$ such that
there is an $x\in a$ with $x\leq y$.
The edge relation is set membership.
The collection of all graphs $G_{a}$ in $\Po$ is denoted by $\mathcal{G}(\Po)$.

In what follows, we will want to compute
$\mathcal{G}(\Po)$ from $\Po$ and $\LB_\Po$.
It may not even be the case, however,
that the set of $\equiv$ classes of $\Po$
can be computed from $\Po$ and $\LB_\Po$.
We therefore assume that $\Po$ omits a certain configuration,
and then prove that we can compute $\mathcal{G}(\Po)$ from $\Po$ and $\LB_\Po$.

For $x,y\in\Po$, we let $x\perp y$ mean that $x\not\leq_\Po y$ and $y\not\leq_\Po x$,
and we let $x<_\Po y$ mean that $x\leq_\Po y$ and $x\neq_\Po y$.
\textbf{For ease of notation, for the rest of this section (\ref{POAmenable}) we
write $\leq$ and $<$ in place of $\leq_\Po$ and $<_\Po$ when the underlying
partial ordering $\Po$ is understood.}

\begin{definition}
A \textit{shed}\index{shed} is a four-tuple $(w,x,y,z)$ of $\Po$ such that:
$x\perp y$, $x<z$, $y<z$, $w<x$ and $w\not\leq y$.
\end{definition}
You can visualize a shed as follows:
$$\xymatrix{
  & z\ar@{-}[ld]\ar@{-}[rd] &   \\
x\ar@{-}[d] &   &  y \\
w &   &   \\
}$$
Note that $x$ and $y$ are \textit{not} necessarily on the same level.

\begin{lem}[\RCAo]
Let $\Po$ be a partial ordering with height $\omega$
that has finite levels and which omits sheds.
Then $\mathcal{G}(\Po)$ is computable from $\Po$ and $\LB_\Po$,
where $\LB_\Po$ is a level bounding function for $\Po$.
\end{lem}

\begin{proof}
First we show that if $w$ and $y$ are on the same level
and have a common successor, then they have a common immediate successor.
Suppose otherwise, and let $w,y< z$.
Since $z$ is not an immediate successor of $w$,
there is an immediate successor $x$ of $w$ such that $w< x< z$.
By assumption $y\not\leq x$ since $w$ and $y$ do not
have a common immediate successor.
We now have a contradiction, since $(w,x,y,z)$ is a shed.

It follows that the compatibility relation is computable
from $\Po$ and $\LB_\Po$, since to determine
if two elements $x$ and $y$ on level $\ell$ are compatible
we need only check if there is an element
$z\in\{0,1,\ldots,\LB_\Po(\ell+1)\}$ such that $x,y<z$.
Therefore there is a $(\Po\oplus\LB_\Po)$-computable function
which returns $M(G_{[x]})$ on input $x$.

Now we show that if $a$ is an $\equiv$-class on level $n$,
$w\in a$, and $w< z$, then there is an immediate successor
$z'$ of $w$ such that $\Pred_n(z')=\Pred_n(z)$.
Otherwise there is an immediate successor $x$ of $w$ below $z$
such that there is an element $y\in\Pred_n(z)\backslash\Pred_n(x)$.
We now have a contradiction, since $(w,x,y,z)$ is a shed.

It follows that there is a $(\Po\oplus\LB_\Po)$-computable function
which returns $S(G_{[x]})$ on input $x$.
\end{proof}

Another useful property of partial orderings
which omits sheds is given by the next lemma.

\begin{lem}[\RCAo]\label{L:shedAvoidingThinsEquivp}
If $\Po$ omits sheds, then $x\equiv_p y$ whenever $x\equiv y$.
\end{lem}
\begin{proof}
It suffices to show that for every $x$ and $y$ on the same level,
if there is a $z$ such that $x< z$ and $y< z$, then $x\equiv_p y$.
Suppose otherwise, and without loss of generality
let $w\in\Pred(x)\backslash\Pred(y)$.
We now have a contradiction, since $(w,x,y,z)$ is a shed.
\end{proof}

Another useful lemma is the following.

\begin{lem}[\RCAo]\label{L:shedAvoidinglevLem}
Suppose that $\Po$ omits sheds.
If $x\perp y$ and there is a $z$ such that $x,y<z$,
then $x$ and $y$ are on the same level.
\end{lem}
\begin{proof}
Suppose otherwise, and without loss of generality
let the level of $y$ be greater than the level of $x$.
Then there is an element $w$ on the same level as $y$
such that $x<w<z$.
We now have a contradiction, since $(x,w,y,z)$ is a shed.
\end{proof}

We make a quick detour into graph theory before making our final definitions.
We will be interested in (finite) bipartite graphs whose partitions
have distinguished pieces.
The bipartite graphs $G$ that we consider all have a piece $T_G$ labeled `top'
and a piece $B_G$ labeled `bottom', so that the disjoint union $G=T_G\cup B_G$
is the bipartition of $G$.
An \textit{embedding} of a bipartite graph $G$ into a bipartite graph $H$
is a map $f:G\to H$ with two properties:
1) if $x,y\in G$ then there is an edge between $x$ and $y$ if and only if
there is an edge between $f(x)$ and $f(y)$, and 2) $f$ respects the labels,
meaning that if $x\in T_G$ then $f(x)\in T_H$, and if $x\in B_G$ then $f(x)\in B_H$.
We write $G\hookrightarrow H$ to mean that there is an embedding of $G$ into $H$.

Earlier we defined a collection of graphs $\mathcal{G}(\Po)$ corresponding
to partial ordering $\Po$ with height $\omega$.
More specifically, given a $\equiv$-equivalence class $a$ in level $n$ of $\Po$,
we defined a finite bipartite graph $G_{a}=S(G_a)\cup M(G_a)$.
In what follows, we consider the pieces $S(G_a)$ to be labeled `top',
the pieces $M(G_a)$ to be labeled `bottom'.

\begin{definition}
Given finite bipartite graphs $H$ and $G$, we write $H\rightarrow (G)^e_k$\index{$H\rightarrow (G)^e_k$}
if for every coloring of the edges of $H$ in $k$ colors,
there is an embedding of $G$ into $H$ with monochromatic edges.

A collection $\mathcal{G}$ of bipartite graphs is \textit{edge-Ramsey}\index{edge-Ramsey} if
$$(\forall G\in\mathcal{G})(\forall k)(\exists H\in\mathcal{G})(H\rightarrow (G)^e_k),$$
and has the \textit{joint embedding property}\index{joint embedding property} if
$$(\forall G_1,G_2\in\mathcal{G})(\exists H\in\mathcal{G})(G_1\hookrightarrow H \ \& \ G_2\hookrightarrow H).$$
\end{definition}

\begin{definition}
Let $\Po$ and $\Qo$ be partial orderings with height $\omega$.
We say that $\mathbb{Q}$ is \textit{dense}\index{dense} in $\Po$ if for every $u\in\mathbb{Q}$ and $z\in\Po$
there is an $x\geq_\Po z$ such that $G_{[u]}\hookrightarrow G_{[x]}$.
\end{definition}

\begin{definition}\label{D:amenable}
We say that a partial ordering $\Po$ is \textit{amenable}\index{amenable} if the following hold:
$\Po$ has a least element, $\Po$ has height $\omega$, $\Po$ has finite levels,
$\Po$ is shed-omitting, $\Po$ is dense in itself,
$\mathcal{G}(\Po)$ is edge-Ramsey and has the joint embedding property,
and every element of $\Po$ has incompatible successors.
\end{definition}
Note that the last requirement of an amenable partial ordering,
that every element has incompatible successors, is satisfied
whenever $\Po$ is dense in itself and contains at least one element $\sigma$
such that $|S(G_{[\sigma]})|\geq 2$.

\begin{thm}[Groszek \cite{GroszekAmenable}, $n\geq 1,k\geq 2$]\label{T:MarciaAmenableIsRamsey}
Every amenable partial ordering is $(n,k)$-Ramsey.
\end{thm}

\begin{thm}[$n\geq 1,k\geq 2$]\label{T:amenableIsRamsey}
The statement of Theorem~\ref{T:MarciaAmenableIsRamsey} holds in \ACAo.
\end{thm}

The following corollary then follows from Theorem~\ref{T:amenableIsRamsey}.

\begin{cor}\label{C:amenable&ACA}
Fix $n\geq 3$ and $k\geq 2$.
The following statement is equivalent to \ACAo, over \RCAo:
If $\Po$ is an amenable partial ordering, then $RT^n_k(\Po)$ holds.
\end{cor}
\begin{proof}
The result follows from Theorem~\ref{T:binm&ACA} and the theorem
of Chubb, Hirst, and McNichol \cite{CHM} that $RT^n(\bin)$ implies \ACAo\ over \RCAo.
\end{proof}

We will prove this theorem with the help of some lemmas.
Theorem~\ref{T:amenableIsRamsey} will follow immediately from
Lemmas~\ref{L:denseAboveSigmaInColorI}, \ref{L:reduceHalfDimension}, and \ref{lem:useGraphInduced}.

The first lemma we consider will be useful not only for its application,
but also for its illustrative proof.

\begin{lem}\label{L:denseIsActuallyDense}
Let $\Po$ and $\Qo$ be computable, amenable partial orderings and
suppose that $\LB_\Po$ and $\LB_\Qo$ are computable level bounding functions for $\Po$ and $\Qo$.
If $\Qo$ is dense in $\Po$, then
given any $\sigma_0\in\Po$, there is a computable embedding $f:\Qo\to\Po$ that maps
the least element of $\Qo$, denoted $0_\Qo$, to $\sigma_0$.
Moreover, there is a computable function which,
given indices for $\Po$, $\Qo$, $\LB_\Po$ and $\LB_\Qo$,
returns indices for the range $\Qo'$ of $f$
and for a level bounding function $\LB_{\Qo'}$ for $\Qo'$.
\end{lem}

Note that for this lemma, it is not necessary to assume that $\mathcal{G}(\Po)$ and $\mathcal{G}(\Qo)$ are edge-Ramsey,
nor that $\mathcal{G}(\Po)$ and $\mathcal{G}(\Qo)$ have the joint embedding property.

\begin{proof}
We use the letters $\alpha$ and $\beta$ to denote elements of $\Qo$,
the letters $a$ and $b$ to denote $\equiv$-classes of $\Qo$,
and the letters $u$ and $v$ to denote $\equiv_p$-classes of $\Qo$.
We use the letters $\sigma$ and $\tau$ to denote elements of $\Po$,
the letters $s$ and $t$ to denote $\equiv$-classes of $\Po$,
and the letters $x$ and $y$ to denote $\equiv_p$-classes of $\Po$.

Let $\Qo_n$ denote the set of elements of $\Qo$ on level $n$ and $\Qo_{<n}$
denote the set of elements of $\Qo$ on levels $<n$.
Similarly define $\Po_{n}$ and $\Po_{<n}$.

We say that $(j,k)$ is
an \textit{extendible $n$-embedding} of $\Qo$ into $\Po$ if
$j$ is an embedding $j:\Qo_{<n}\rightarrow \Po$,
$k$ is a function $k:\Qo_n\rightarrow \Po_m$ for some $m$, and:
\begin{itemize}
\item if $\alpha\in\Qo_{<n}$, $\beta\in\Qo_{n}$, and $\alpha<\beta$ then $j(\alpha)< k(\beta)$,
\item if $\alpha,\beta\in\Qo_{n}$ and $\Pred(\alpha)=\Pred(\beta)$ then $k(\alpha)=k(\beta)$,
\item if $\alpha,\beta\in\Qo_{n}$ and $\Pred(\alpha)\neq \Pred(\beta)$ then $k(\alpha)\not\equiv k(\beta)$.
\end{itemize}
If $(j',k')$ is an extendible $(n+1)$-embedding and
$(j,k)$ is an extendible $(n+1)$-embedding,
we say that $(j',k')$ \textit{extends} $(j,k)$ if
$j\subseteq j'$ and $j'(\alpha)\geq k(\alpha)$ for $\alpha\in\Qo_n$.

We begin with the $0$-embedding $(j,k)$ where $j=\emptyset$ and $k(0_\Qo)=\sigma_0$.
We now describe a computable procedure to extend an extendible $n$-embedding to an extendible $(n+1)$-embedding.

Since $k(\alpha)$ is determined by its $\equiv_p$-class, we will abuse notation and
write $k(u)$ in place of $k(\alpha)$ when $u$ is a $\equiv_p$-class and $\alpha\in u$.
Since the $\equiv$-classes refine the $\equiv_p$-classes,
we will also write $k(a)$ when $a$ is a $\equiv$-class.

First we extend $j$ to $j'$ by mapping the $\equiv$-classes $a$
of $\Qo$ on level $n$ into $\Po$ above the respective $k(a)$.
Moreover, we will actually embed each graph $G_a$ above $k(a)$.
To make sure that we can then define $k'$,
we must make sure that if $\alpha,\beta\in\Qo_n$ and
$\alpha$ and $\beta$ are incompatible, then
$j'(\alpha)$ and $j'(\beta)$ are also incompatible.
Therefore if $a$ and $b$ are distinct $\equiv$-classes
that we map into $\equiv$-classes $s$ and $t$ respectively,
then we must make sure that every element of $s$
is incompatible with every element of $t$.
We say that elements $\sigma$ and $\tau$
are strongly incompatible if every element of $[\sigma]$
is incompatible with every element of $[\tau]$.

We claim that if $\sigma_1$ and $\sigma_2$
are incompatible, $\sigma_1<\tau_1$, and
$\sigma_2<\tau_2$, then $\tau_1$ and $\tau_2$
are strongly incompatible.
Suppose otherwise, and let $\tau_1',\in[\tau_1]$,
$\tau_2',\in[\tau_2]$, and $\tau_1',\tau_2'\leq\tau$.
Since $\tau_1'\equiv\tau_1$, then $\Pred(\tau_1)=\Pred(\tau_1')$,
so $\sigma_1<\tau_1$.  Similarly $\sigma_2<\tau_2$.
Notice that $\tau'_1\perp\tau'_2$ since
$\sigma_1$ and $\sigma_2$ are incompatible.
If $\tau'_1<\tau$ then $(\sigma_1,\tau_1',\sigma_2,\tau)$
is a shed.
Similarly if $\tau'_2<\tau$ then
$(\sigma_2,\tau_2',\sigma_1,\tau)$ is a shed.
Since $\Po$ is shed omitting,
we conclude that $\tau'_1=\tau=\tau'_2$,
which then contradicts that $\sigma_1$
and $\sigma_2$ are incompatible.

Therefore to define $j'$ it suffices to find,
for each $\equiv$-class $a$ on level $n$,
an element $\sigma_a\geq k(a)$ and
an embedding $h_{a}:G_a\hookrightarrow G_{[\tau_a]}$
for some $\tau_a>\sigma_a$,
such that the set of all $\sigma_a$'s is mutually incompatible.
For if we have such embeddings, then we
let $j'(\alpha)=h_{[\alpha]}(\alpha)$.
To see that $j'$ preserves
incompatibility, let $\alpha,\beta\in\Qo_n$ and
suppose that $\alpha\equiv_p\beta$
and $\alpha\not\equiv\beta$.
Since $\sigma_{[\alpha]}$ and $\sigma_{[\beta]}$
are incompatible, $\tau_{[\alpha]}$ and $\tau_{[\beta]}$
are strongly incompatible.
Therefore no element of $[\tau_{[\alpha]}]$ is compatible
with any element of $[\tau_{[\beta]}]$.
So in particular $j'(\alpha)$ and $j'(\beta)$
are incompatible.
From this it follows that $j'$ is one-one.
Notice also that $j'$ is order-preserving.

Now we show that we can find such $\sigma_a$, $\tau_a$,
and embeddings $h_{a}$ for each $\equiv$-class $a$ on level $n$.
First notice that if $a$ and $b$ are contained
in different $\equiv_p$-classes, then
$\sigma_a$ and $\sigma_b$ will be incompatible
provided $\sigma_a\geq k(a)$ and $\sigma_b\geq k(b)$,
since $k(a)$ and $k(b)$ are incompatible.
We therefore take a $\equiv_p$-class $u$
and show how to define $\sigma_a$, $\tau_a$,
and $h_a$ for all $\equiv$-classes $a$ contained in $u$.
Let $m$ be the number of $\equiv$-classes contained in $u$.
Since we assumed that every element in $\Po$ has
incompatible successors, there are $m$-many
incompatible successors of $k(u)$.
Assign to each $\equiv$-class $a$ one such element, denoted by $\sigma_a$.
Since $\Qo$ is dense in $\Po$, there is a $\tau_a> \sigma_a$ and an $h_a$
such that $h_a:G_{a}\hookrightarrow G_{[\tau_a]}$.

It remains only to show how to define $k'$.
Given a $\equiv_p$-class $u$ of $\Qo_{n+1}$,
let $\alpha\in u$ and
let $b$ be the $\equiv_p$-class on level
$n$ of $\Qo_n$ such that $\Pred_n(\alpha)\subseteq b$.
Let $m$ be the level of the element $\tau_b$
that was defined above.
Let $\tau\in\Po_{m+1}$ be such that
$h_b(\Pred_n(\alpha))=\Pred_{m+1}(\tau)$.
If we then let $\overline{k}(\alpha')=\tau$
for all $\alpha'\in u$,
we claim that $\overline{k}$ satisfies most of
the requirements that we need $k'$ to satisfy.

If $\alpha\in\Qo_n$, $\beta\in\Qo_{n+1}$,
and $\alpha<\beta$, we will show that $j'(\alpha)<\overline{k}(\beta)$.
Since $\alpha<\beta$, then $\alpha\in\Pred_n(\beta)$.
Therefore $h_{[\alpha]}(\alpha)\in\Pred_{m}(\overline{k}(\beta))$,
and so $j'(\alpha)<\overline{k}(\beta)$.
If $\alpha,\beta\in\Qo_{n+1}$ and $\Pred(\alpha)=\Pred(\beta)$ then
$\overline{k}(\alpha)=\overline{k}(\beta)$ since we defined $\overline{k}$ to be constant
on the $\equiv_p$-classes.
Finally, assume $\alpha_1,\alpha_2\in\Qo_{n+1}$ and
$\Pred(\alpha_1)\neq \Pred(\alpha_2)$.
We will show that $\overline{k}(\alpha_1)$ and $\overline{k}(\alpha_2)$ are incompatible.
There are two cases, namely whether or not
there is a $\equiv$-class $b$ in $\Qo_n$
such that $\Pred_n(\alpha_1),\Pred_n(\alpha_2)\subseteq b$.
Suppose there were such a $b$ and that there was a $\tau$ such that
$\overline{k}(\alpha_1),\overline{k}(\alpha_1)\leq\tau$.
Since $\Pred(\alpha_1)\neq \Pred(\alpha_2)$ and since
$h_b:G_b\hookrightarrow G_{[\tau_b]}$ is an embedding,
there is an elements $\sigma\in [\tau_b]$ such that,
without loss of generality, $\sigma<\overline{k}(\alpha_1)$
and $\sigma\not\leq \overline{k}(\alpha_2)$.
Then $\tau$ cannot be distinct from $\overline{k}(\alpha_1)$ and $\overline{k}(\alpha_1)$,
since otherwise $(\sigma,\overline{k}(\alpha_1),\overline{k}(\alpha_2),\tau)$ would be a shed.
But $\overline{k}(\alpha_1)$ and $\overline{k}(\alpha_2)$ are on the same level,
so then $\tau=\overline{k}(\alpha_1)=\overline{k}(\alpha_2)$,
contradicting that $h_b$ is an embedding.
In the second case, there is no
$\equiv$-class $b$ in $\Qo_n$
such that $\Pred_n(\alpha_1),\Pred_n(\alpha_2)\subseteq b$.
Therefore there are distinct $\equiv$-classes $b_1$ and $b_2$ in $\Qo_n$
such that $\Pred_n(\alpha_1)\subseteq b_1$ and $\Pred_n(\alpha_2)\subseteq b_2$.
Recall that we embedded $b_1$ and $b_2$ into
$[\tau_{b_1}]$ and $[\tau_{b_2}]$, where
$\tau_{b_1}$ and $\tau_{b_2}$ are strongly incompatible elements.
Since $\overline{k}(\alpha_1)\geq \tau_{b_1}'$ for some $\tau_{b_1}'\in[\tau_{b_1}]$,
and $\overline{k}(\alpha_2)\geq \tau_{b_2}'$ for some $\tau_{b_2}'\in[\tau_{b_2}]$,
we conclude that $\overline{k}(\alpha_1)$ and $\overline{k}(\alpha_2)$ are incompatible.

Finally, we define $k'$.
Let $m$ be the maximum level of range of $\overline{k}$.
For each $\equiv_p$-class $u$ of $\Qo_{n+1}$,
choose $\alpha\in u$ and let $k'(\alpha)$
be any element $\sigma$ of $\Po_{m}$ such that $\overline{k}(\alpha)\leq\sigma$.
Then $k'$ satisfies the same requirements that were
just proved for $\overline{k}$, and has the additional property
that its range is contained in a single level.
This ends the construction.

We have therefore described a computable procedure to build the
embedding $f$ level by level.  More precisely, if $\sigma$ is on level $n$,
we let $f(\sigma)=j(\sigma)$, where $(j,k)$ is the $(n+1)$-th extendible extension
of $(\emptyset,\{(0_\Qo,\sigma_0)\})$.
Moreover, given indices for $\Po$, $\Qo$, $\LB_\Po$ and $\LB_\Qo$,
we have described a procedure to compute
indices for the range $\Qo'$ of $f$
and for a level bounding function $\LB_{\Qo'}$ for $\Qo'$.
\end{proof}

We now define a special class of colorings.

\begin{definition}
A coloring of pairs $c:[\Po]^2\to k$ is called a \textit{graph induced coloring}\index{graph induced coloring}
if the color of $\langle\sigma,\tau\rangle$ is determined by $\sigma$ and $\Pred_n(\tau)$,
where $n$ is the level of $\sigma$.
\end{definition}

Graph induced colorings are notable because they induce a coloring of the graphs in $\mathcal{G}(\Po)$.
The induced coloring of graphs is given by coloring an edge $(\sigma,\Pred_n(\tau))$ of $G_a$
with the color $c(\sigma,\tau)$,
where $a$ is an $\equiv$-class on level $n$, $\sigma\in a$, and $\tau\geq\sigma$.
Notice that the induced graph colorings are computable from the original coloring and $\mathcal{G}(\Po)$.

\begin{lem}[RCA$_0$+I$\BSigma^0_2$]\label{L:denseAboveSigmaInColorI}
Suppose that $\Po$ is amenable, $\LB_\Po$ is a level bounding function for $\Po$,
and $c:[\Po]^2\to k$ is a graph induced coloring.
Then there there is a color $i\leq k$ and a $\sigma\in\Po$ such that for each $\equiv$-class $a$,
the set of $\tau$ such that there is a homogeneous embedding
$h:G_a\hookrightarrow G_{[\tau]}$ in color $i$ is dense in $\Po$ above $\sigma$.
\end{lem}

\begin{proof}
Given an $\equiv$-class $a$, a $\tau\in\Po$, and a color $i\leq k$,
let $\theta(a,\tau,i)$ be the statement that there is a color $i$
homomorphism $h:G_a\hookrightarrow G_{[\tau]}$.
Let $F$ be the set of all $i<k$ such that
$$\exists\sigma\exists a(\forall\tau\geq\sigma)(\forall j< i)\neg\theta(a,\tau,j).$$
Notice that $F$ exists by bounded $\BSigma^0_2$ comprehension, which is equivalent to I$\BSigma^0_2$.
Let $i$ be the maximal element of $F$, and let $\sigma_0$ and $a_0$ be such that
$(\forall\tau\geq\sigma_0)(\forall j<i)(\neg\theta(a_0,\tau,j))$.

We claim that for each $\equiv$-class $a$,
the set of $\tau$ such that there is a homogeneous embedding
$h:G_a\hookrightarrow G_{[\tau]}$ in color $i$ is dense in $\Po$ above $\sigma_0$.
Suppose, for the sake of contradiction, that there is a $\sigma_1\geq\sigma_0$
and an $\equiv$-class $a_1$ such that for no $\tau\geq\sigma_1$ is there a color
$i$ homogeneous copy of $G_{a_1}$.
Since $\mathcal{G}(\Po)$ has the joint embedding property, there is an $\equiv$-class $a_2$ such that
$G_{a_0}\hookrightarrow G_{a_2}$ and $G_{a_1}\hookrightarrow G_{a_2}$.
Therefore $(\forall\tau\geq\sigma_1)(\forall j<i+1)(\neg\theta(a_2,\tau,j))$,
contradicting the maximality of $i$ in $F$.
\end{proof}

\begin{lem}\label{L:GraphInducedColoringSolutions}
Let $\Po$ be a computable amenable partial ordering and $\LB_\Po$ be a computable level bounding function for $\Po$.
Let $c$ be a graph induced coloring of $\Po$ in $k$ colors.
Then there is a computable, monochromatic embedding $f$ of $\Po$ into itself such that
range $\Po'$ of $f$ is computable, as is a level bounding function $\LB_{\Po'}$ for $\Po'$.
\end{lem}

\begin{proof}
By Lemma~\ref{L:denseAboveSigmaInColorI}
there there is a color $i\leq k$ and a $\sigma\in\Po$ such that for each $\equiv$-class $a$,
the set of $\tau$ such that there is a homogeneous embedding
$h:G_a\hookrightarrow G_{[\tau]}$ in color $i$ is dense in $\Po$ above $\sigma$.

We now proceed, exactly as in Lemma~\ref{L:denseIsActuallyDense}, to build an embedding of
$\Po$ above $\sigma$, except that at the point in the construction where we choose $h_a$ and $\sigma_a$
such that $h_{a}:G_{a}\hookrightarrow G_{[\sigma_a]}$ for some $\sigma_a\geq k(a)$,
we ensure that $h_a$ is monochromatic in color $i$.
We are guaranteed such an $h_a$ and $\sigma_a$ by our choice of $i$ and $\sigma$ from Lemma~\ref{L:denseAboveSigmaInColorI}.
\end{proof}

\begin{lem}\label{L:nEquals1}
Let $\Po$ be a computable amenable partial ordering and $\LB_\Po$ be a computable level bounding function for $\Po$.
Let $c$ be a singleton coloring of $\Po$ in $k$ colors.
Then there is a computable, monochromatic embedding $f$ of $\Po$ into itself such that
range $\Po'$ of $f$ is computable, as is a level bounding function $\LB_{\Po'}$ for $\Po'$.
\end{lem}

\begin{proof}
The lemma follows from Lemma~\ref{L:GraphInducedColoringSolutions} by considering
the graph induced coloring $c'$ of $\Po$ defined by letting $c'(\langle\sigma,\tau\rangle)=c(\sigma)$.
\end{proof}

We make some final definitions for use in the last two lemmas.
An \textit{instruction} is either an element $\tau\in\Po$, or
a triple $(e,\sigma,i)$ where $e$ is an index for a computable,
graph induced coloring of $\Po$, and $\sigma$ and $i$ are
such that for each $\equiv$-class $a$,
the set of $\rho$ such that there is a homogeneous embedding
$h:G_a\hookrightarrow G_{[\rho]}$ in color $i$ is dense in $\Po$ above $\sigma$.

Let $\Po_{\langle\tau\rangle}$ be the copy of $\Po$ above $\tau$ as given
by Lemma~\ref{L:denseIsActuallyDense}, and
$\Po_{\langle(e,\sigma,i)\rangle}$ be the monochromatic copy of $\Po$
given by Lemma~\ref{L:GraphInducedColoringSolutions}.
Moreover, if $s$ is a sequence of instructions,
then we let $\Po_{s^\frown\langle\tau\rangle}$ be the copy of $\Po$ in $\Po_{s}$
above $\tau$ as given by Lemma~\ref{L:denseIsActuallyDense},
and $\Po_{s^\frown\langle(e,\sigma,i)\rangle}$ be the monochromatic copy of $\Po$ in $\Po_s$
given by Lemma~\ref{L:GraphInducedColoringSolutions}.
Assuming that $\Po$ is computable and that there is a computable
level bounding function $\LB_\Po$ for $\Po$,
Lemmas~\ref{L:denseIsActuallyDense} and \ref{L:GraphInducedColoringSolutions}
not only guarantee that $\Po_s$ is defined for every sequence of instructions $s$,
but also that each $\Po_s$ has a computable level bounding function
whose index can be uniformly computed from $s$.

\begin{lem}[ACA$_0$]\label{L:reduceHalfDimension}
Suppose that $\Po$ is amenable and that $c$ is a $(m+1)$-ary coloring of $\Po$ in $k$ colors.
Then there is an embedding $J$ of $\Po$ into itself whose image $\Qo=J(\Po)$
satisfies the following:
the color of each $(m+1)$-chain
$\langle \tau_1,\tau_2,\ldots,\tau_{m+1}\rangle$ in $\Qo$ depends only on
$\langle \tau_1,\tau_2,\ldots,\tau_m,\Pred_{lev(\tau_m)}(\tau_{m+1})\rangle$
(where $\Pred$ refers to the predecessor function in $\Qo$).
\end{lem}

\begin{proof}
Notice that in \ACAo\ there exists a level bounding function $\LB_\Po$ for $\Po$.
In \ACAo\ we also have the function $g$ which,
given an index for a computable (in $\Po\oplus\LB_\Po$), isomorphic copy $\Po'$ of $\Po$,
an index for a computable (in $\Po\oplus\LB_\Po$) level bounding function for $\Po'$,
and an index for a computable (in $\Po\oplus\LB_\Po$) graph induced coloring of $\Po'$,
returns a pair $(\sigma,i)$, where $\sigma$ and $i$ are as in Lemma~\ref{L:denseAboveSigmaInColorI}.

We now construct an embedding of $\Po$ into itself that satisfies the lemma
and is computable in $\Po\oplus\LB_\Po\oplus g$.
Our construction will be similar to that of Lemma~\ref{L:denseIsActuallyDense}.
We define a \textit{color-extendible} $n$-embedding to be a triple $(j,k,\ell)$, where
\begin{itemize}
\item $(j,k)$ is an extendible $n$-embedding,
\item $\ell$ is a function defined on the $\equiv_p$-classes on level $n$ of $\Po$
such that $\ell(u)$ is a sequence of instructions for a copy of $\Po$ above $k(u)$,
\item if $\overline{\tau}=\langle \tau,\ldots,\tau_m\rangle$ is an $m$-chain in the range of $j$
	and $\sigma_1,\sigma_2\in\Po_{\ell(u)}$ for some $\equiv_p$-class $u$ on level $n$,
	then $c(\overline{\tau}^\frown\langle\sigma_1\rangle)=c(\overline{\tau}^\frown\langle\sigma_2\rangle)$.
\end{itemize}

We say that a color-extendible $(n+1)$-embedding $(j',k',\ell')$ of $\Po$
extends the color-extendible $n$-embedding $(j,k,\ell)$ if
\begin{itemize}
\item $(j',k')$ extends $(j,k)$ as extendible embeddings,
\item if $\alpha\in\Po_n$ and $\beta$ is an immediate successor of $\alpha$,
	\begin{itemize}
	\item then $j'(\alpha),k'(\beta)\in \Po_{\ell([\alpha]_p)}$,
	\item $\ell'([\beta]_p)$ extends $\ell([\alpha]_p)$
	\end{itemize}
\end{itemize}

Let $(j,k,\ell)$ be a color-extendible $n$-embedding of $\Po$.
We describe an $X$-computable procedure which extends $(j,k,\ell)$ to a
color-extendible $(n+1)$-embedding $(j',k',\ell')$ of $\Po$.

The first thing we do is
extend the extendible $n$-embedding $(j,k)$ to $(j',k')$.
We proceed exactly as in the proof of Lemma~\ref{L:denseIsActuallyDense},
except that except that instead of merely choosing $j'$ and $k'$
above $k(\alpha)$, we choose $j'$ and $k'$ above $k(\alpha)$
inside $\Po_{\ell([\alpha]_p)}$

It remains only to show how to define $\ell'$.
Let $u$ be an $\equiv_p$-class on level $(n+1)$,
and $\alpha$ be any immediate predecessor of some $\beta\in u$.
Let $S$ be the set of all $m$-chains in the range of $j$ below $u$.
Let $\Po'$ the copy of $\Po$ above $k'(u)$ in $\Po_{\ell([\alpha]_p)}$ as given by Lemma~\ref{L:denseIsActuallyDense}.
Let $c'$ be the singleton coloring of $\Po'$ in $2^{|S|}$ colors
defined by
$$c'(\rho)=\left\langle c(\overline{\upsilon}^\frown\langle\rho\rangle)\ :\ \overline{\upsilon}\in S\right\rangle.$$
Note that there is a uniform (in $X$) procedure that takes
$\ell(u)$ and returns an index $e$ for $c'$.
We now use $g$ to obtain the $i$ and $\tau$ given by Lemma~\ref{L:denseAboveSigmaInColorI} for $c'$,
and we let $\ell'(u)=\ell([\alpha]_p)^\frown\langle k'(u),(e,\tau,i)\rangle$.

This ends the construction.
We therefore have sequence $\big((j_n,k_n,\ell_n)\big)_{n=1}^\infty$
such that for each $n$, $(j_n,k_n,\ell_n)$ is a color-extendible $n$-embedding
and $(j_{n+1},k_{n+1},\ell_{n+1})$ extends $(j_n,k_n,\ell_n)$.

We claim that $J=\bigcup j_n$ is an embedding which satisfies the lemma.
For suppose that $\langle\tau_1,\tau_2,\ldots,\tau_m\rangle$ is an $m$-chain
in the range of $J$ and that $\sigma_1,\sigma_2\geq\tau_m$ are also in the
range of $J$ and that $\sigma_1\equiv_p\sigma_2$.
Let $n$ be the level of $\tau_m$.
Since $\sigma_1\equiv_p\sigma_2$, there is a an $\equiv_p$-class
$u$ on level $n$ of $\Po$ such that $\sigma_1,\sigma_2\in\Po_{\ell_n(u)}$.
Therefore
$$c(\langle\tau_1,\tau_2,\ldots,\tau_m,\sigma_1\rangle)=c(\langle\tau_1,\tau_2,\ldots,\tau_m,\sigma_2\rangle).$$
\end{proof}

\begin{lem}[ACA$_0$]\label{lem:useGraphInduced}
Suppose that $\Po$ is amenable and that $c$ is a $(m+2)$-ary coloring of $\Po$ in $k$ colors
such that the color of each $\langle \tau_1,\tau_2,\ldots,\tau_{m+2}\rangle$ depends only on
$\langle \tau_1,\tau_2,\ldots,\tau_{m+1},\Pred_{\tau_{m+1}}(\tau_{m+2})\rangle$.
Then there is an embedding of $\Po$ into itself such that the color of each
$\langle \tau_1,\tau_2,\ldots,\tau_{m+2}\rangle$ depends only on
$\langle \tau_1,\tau_2,\ldots,\tau_{m},\Pred_{\tau_{m}}(\tau_{m+1})\rangle$.
\end{lem}

\begin{proof}
The proof of Lemma~\ref{lem:useGraphInduced} is nearly
identical to the proof of Lemma~\ref{L:reduceHalfDimension}.

We slightly change the definition of a \textit{color-extendible} $n$-embedding
by changing the requirement
\begin{itemize}
\item if $\overline{\tau}=\langle \tau,\ldots,\tau_m\rangle$ is an $m$-chain in the range of $j$
	and $\sigma_1,\sigma_2\in\Po_{\ell(u)}$ for some $\equiv_p$-class $u$ on level $n$,
	then $c(\overline{\tau}^\frown\langle\sigma_1\rangle)=c(\overline{x}^\frown\langle\sigma_2\rangle)$.
\end{itemize}
to
\begin{itemize}
\item if $\overline{\tau}=\langle \tau,\ldots,\tau_m\rangle$ is an $m$-chain in the range of $j$
	and $\sigma_1,\sigma_2,\sigma_3,\sigma_4\in\Po_{\ell(u)}$ for some $\equiv_p$-class $u$ on level $n$,
	then $c(\overline{\tau}^\frown\langle\sigma_1,\sigma_2\rangle)=c(\overline{\tau}^\frown\langle\sigma_3,\sigma_4\rangle)$.
\end{itemize}

Let $(j,k,\ell)$ be a color-extendible $n$-embedding of $\Po$.
We describe a $\Po\oplus\LB_\Po\oplus g$-computable procedure which extends $(j,k,\ell)$ to a
color-extendible $(n+1)$-embedding $(j',k',\ell')$ of $\Po$,
where $g$ is the same function as in the proof of Lemma~\ref{L:reduceHalfDimension}.

The first thing we do is
extend the extendible $n$-embedding $(j,k)$ to $(j',k')$.
We proceed exactly as in the proof of Lemma~\ref{L:denseIsActuallyDense},
except that except that instead of merely choosing $j'$ and $k'$
above $k(\alpha)$, we choose $j'$ and $k'$ above $k(\alpha)$
inside $\Po_{\ell([\alpha]_p)}$.

It remains only to show how to define $\ell'$.
Let $u$ be an $\equiv_p$-class on level $(n+1)$,
and $\alpha$ be any immediate predecessor of some $\beta\in u$.
Let $S$ be the set of all $m$-chains in the range of $j$ below $u$.
Let $\Po'$ the copy of $\Po$ above $k'(u)$ in $\Po_{\ell([\alpha]_p)}$ as given by Lemma~\ref{L:denseIsActuallyDense}.
Let $c'$ be the graph induced coloring of $\Po'$ in $2^{|S|}$ colors
defined by
$$c'(\rho_1,\rho_2)=\left\langle c(\overline{\upsilon}^\frown\langle\rho_1,\rho_2\rangle)\ :\ \overline{\upsilon}\in S\right\rangle.$$
Notice $c'$ is computable in $\Po\oplus\LB_\Po\oplus g$,
so there is an index $e$ for $c'$.
We now use $g$ to obtain the $i$ and $\tau$ given by Lemma~\ref{L:denseAboveSigmaInColorI} for $c'$,
and we let $\ell'(u)=\ell([\alpha]_p)^\frown\langle k'(u),(e,\tau,i)\rangle$.

This ends the construction.
We therefore have sequence $\big((j_n,k_n,\ell_n)\big)_{n=1}^\infty$
such that for each $n$, $(j_n,k_n,\ell_n)$ is a color-extendible $n$-embedding
and $(j_{n+1},k_{n+1},\ell_{n+1})$ extends $(j_n,k_n,\ell_n)$.

We claim that $J=\bigcup j_n$ is an embedding which satisfies the lemma.
For suppose that $\langle\tau_1,\tau_2,\ldots,\tau_m\rangle$ is an $m$-chain
in the range of $J$ and that $\sigma_1,\sigma_2,\sigma_3,\sigma_4\geq\tau_m$ are also in the
range of $J$ and that $\sigma_1\equiv_p\sigma_3$.
Let $n$ be the level of $\tau_m$.
Since $\sigma_1\equiv_p\sigma_3$, there is a an $\equiv_p$-class
$u$ on level $n$ of $\Po$ such that $\sigma_1,\sigma_3\in\Po_{\ell_n(u)}$.
Therefore
$$c(\langle\tau_1,\tau_2,\ldots,\tau_m,\sigma_1,\sigma_2\rangle)=c(\langle\tau_1,\tau_2,\ldots,\tau_m,\sigma_3,,\sigma_4\rangle).$$
\end{proof}

Note that Theorem~\ref{T:amenableIsRamsey} follows immediately from
Lemmas~\ref{L:denseAboveSigmaInColorI}, \ref{L:reduceHalfDimension}, and \ref{lem:useGraphInduced}.


\chapter{Coloring Ordinals}\label{Ords}
	In Chapter~\ref{Posets} we mentioned that $\omega$ and $\omega^*$
are the only two countable linear orderings
that have the $n$-Ramsey property for any $n\geq 2$.
The case when $n=1$, however, has many more examples.
It is well known that the well orderings with the 1-Ramsey
property are exactly the ordinal powers of $\omega$
(see, for instance, Section 6.8.1 of \cite{Fraisse:ThR}).

We now shift our attention to the finite powers of $\omega$.
For each $n\in\NN$, we will choose a particular representation of $\omega^n$.
In particular, we let $\omega^n$ be the lexicographic ordering of $\NN^n$.

\begin{definition}
For each $n\in\NN$, $\omega^n$\index{$\omega^n$} is the ordering $(\NN^n, \lex)$, \index{$\lex$}
where
$$\seq{x_0,\ldots,x_{n-1}}\lex\seq{y_0,\ldots,y_{n-1}}\ \ \Leftrightarrow\ \ (\exists i<n)[x_i<y_i\land(\forall j<i)x_j=y_j].$$
In other words, $\tup{x}\lex\tup{y}$ if and only if $\tup{x}$
is smaller than $\tup{y}$ on the first coordinate where they disagree.
\end{definition}

The statement $RT^1_k(\omega^n)$ then says that for every coloring
$c:\NN^n\to\{0,1\ldots,k-1\}$ there is a homogeneous set
$H\subseteq\NN^n$ such that $(H,\lex)$ is isomorphic to $\omega^n$.
When $(H,\lex)$ is isomorphic to $\omega^n$, it is often said
that $(H,\lex)$ has order type $\omega^n$.
There are also other equivalent ways to define that a set
$(H,\lex)$ has order type $\omega^n$, which give alternate
versions of $RT^1_k(\omega^n)$.
In Section~\ref{ElemIndec} we examine a first order definition
of having order type $\omega^n$ and the resulting indecomposability statement.
In Section~\ref{Indec&Embed} we examine a second order definition
of having order type $\omega^n$ and the resulting indecomposability statement.

\section{Elementary Indecomposability}\label{ElemIndec}
	In this section, we will define what it means for a set $A\subseteq\NN^n$
to have order type $\omega^n$ using a first order definition.
We write $(\exists^\infty x)\phi(x)$ as shorthand for the formula
$(\forall x)(\exists y)[y\geq x\land\phi(y)]$.
Similarly, we write $(\forall^\infty x)\phi(x)$ as shorthand for the formula
$(\exists x)(\forall y)[y\geq x\rightarrow \phi(y)]$.

\begin{definition}\label{D:ElemOmegaN}
Fix $n\in\NN$.  A set $A\subseteq\NN^n$ has \textit{order type} $\omega^n$ if
$$(\exists^\infty x_1)(\exists^\infty x_2)\ldots(\exists^\infty x_n)[\seq{x_1,x_2,\ldots,x_n}\in A].$$
\end{definition}

Definition~\ref{D:ElemOmegaN} inspires the following elementary version of $RT^1(\omega^n)$:

\begin{definition}\label{D:ElemIndec}
$\EIndec^n$\index{$\EIndec^n$} is the statement that for every $k$ and every coloring
$c:\NN^n\to\{0,1,\ldots,k-1\}$, there is a $d<k$ such that
$$(\exists^\infty x_1)(\exists^\infty x_2)\ldots(\exists^\infty x_n)[c(x_1,x_2,\ldots,x_n)=d].$$
\end{definition}

We will see that $\EIndec^n$ is related to the bounding
principle and to induction.
The main results of this section are Theorem~\ref{T:EIndec&Bnd},
and Theorem~\ref{T:EIndec&Ind}.
Note that Theorem~\ref{T:EIndec&Ind} is joint work with Fran\c{c}ois Dorais.

Given a class of formulas $\Gamma$, the bounding scheme\index{bounding scheme} for
this class, denoted by $\Bnd{\Gamma}$\index{$\Bnd{\BSigma^0_n}$}, is the collection of formulas
of the form
\begin{equation*}
  (\forall x<y)(\exists z)\phi(x,z) \rightarrow (\exists w)(\forall
  x<y)(\exists z<w)\phi(x,z),
\end{equation*}
where $\phi\in\Gamma$.
The statement $\EIndec^1$ says that for every
finite coloring $c:\NN\to\{0,\dots,k-1\}$ there is a color $d < k$
such that the set $A_d = \{x: c(x) = d\}$ is infinite.
This statement was proved to be equivalent to $\Bnd{\BSigma^0_2}$ by Hirst.  \cite{Hirst:thesis}
We now consider the relationship between $\EIndec^n$ and
the bounding principle for larger values of $n$.

We will make use of another principle which is equivalent to bounding,
namely the regularity principle\index{regularity principle}
of H{\'a}jek and Pudl{\'a}k  \cite{Hajek&Pudlak}.

Given a class of formulas $\Gamma$, the regularity scheme for
this class, denoted by $\Reg{\Gamma}$\index{$\Reg{\BSigma^0_n}$}, is the collection of formulas
of the form
\begin{equation*}
  (\exists^\infty x)(\exists y<u)\varphi(x,y) \rightarrow (\exists y<u)(\exists^\infty x)\varphi(x,y)
\end{equation*}
where $\varphi\in\Gamma$.
You can think of $\Reg{\Gamma}$ as a kind of infinite pigeonhole principle\index{pigeonhole principle}
for colorings in $\Gamma$.
H{\'a}jek and Pudl{\'a}k showed that $\Reg{\BSigma^0_n}$ is equivalent to $\Bnd{\BPi^0_n}$ \cite{Hajek&Pudlak}.
We will use this equivalence to prove the following proposition.

\begin{thm}\label{T:EIndec&Bnd}
Let $n\geq1$.  Then $\RCAo+\EIndec^{n+1}$ proves $\Bnd{\BPi^0_{n+1}}$.
\end{thm}

Theorem~\ref{T:EIndec&Bnd} follows immediately from
Proposition~\ref{P:SEIndec&Reg} below.
We will prove Proposition~\ref{P:SEIndec&Reg} with the help of some lemmas.

First we will show that we can change the assumptions of $\Reg{\BSigma^0_n}$
slightly without changing its strength.
Given a class of formulas $\Gamma$, the scheme denoted by $\Reg'{\Gamma}$
is the collection of formulas of the form
\begin{equation*}
  (\forall x)(\exists y<u)\varphi(x,y) \rightarrow (\exists y<u)(\exists^\infty x)\varphi(x,y)
\end{equation*}
where $\varphi\in\Gamma$.
Notice that $\Reg'{\BSigma^0_n}$ follows immediately from $\Reg{\BSigma^0_n}$.
We now show that $\Reg'{\BSigma^0_n}$ is not actually weaker than $\Reg{\BSigma^0_n}$ over \RCAo.

\begin{lem}\label{P:RisR'}
Fix $n\geq 1$.  Then $\RCAo+\Reg'{\BSigma^0_n}$ proves $\Reg{\BSigma^0_n}$.
\end{lem}
\begin{proof}
Suppose $(\exists^\infty x)(\exists y<u)\varphi(x,y)$ holds for some $\varphi\in\BSigma^0_n$.
Let
$$\theta(x,y)=(\exists z> x)\varphi(z,y).$$
Note that $\theta\in\BSigma^0_n$.

The statement $\exists^\infty x(\exists y<u)\varphi(x,y)$
is shorthand for $\forall x(\exists z>x)(\exists y<u)\varphi(z,y)$, which is equivalent to
$\forall x(\exists y<u)(\exists z>x)\varphi(z,y)=\forall x(\exists y<u)\theta(x,y)$.
Therefore by $\mathsf{R}'{\BSigma^0_n}$, the statement
$(\exists y<u)(\exists^\infty{x})\theta(x,y)$ holds.
In other words $(\exists y<u)(\exists^\infty{x})(\exists z>x)\varphi(z,y)$.
From this we conclude that $(\exists y<u)(\exists^\infty{x})\varphi(x,y)$.
\end{proof}

We will now use $\Reg'{\BSigma^0_n}$ to handle a certain class of colorings.
We say that a function $c:\NN^{m}\to\NN$ is \textit{weakly $n$-stable}\index{stability} (where $n<m$)
if for all $x_1,\dots,x_{m-n} \in \NN$ there is a $y \in \NN$ such that
\begin{equation*}
  (\forall^\infty z_1)\cdots(\forall^\infty z_n)[y = c(x_1,\dots,x_{m-n},z_1,\dots,z_n)].
\end{equation*}
This is very similar to saying that the iterated limit\index{iterated limit}
\begin{equation*}
  \lim_{z_1\to\infty} \cdots \lim_{z_n\to\infty} c(x_1,\dots,x_{m-n},z_1,\dots,z_n)
\end{equation*}
exists for all $x_1,\dots,x_{m-n} \in \NN$.
\textit{However, the typical definition of such limits
requires that intermediate limits all exist too,
which is not required by weak $n$-stability.}
We say that $c$ is \textit{strongly $n$-stable} if it is weakly $i$-stable for each $1
\leq i \leq n$; this guarantees the existence of all intermediate
limits and corresponds to the usual meaning of iterated limit.
Note that when $n = 1$ the two notions agree with each other
and with definition of \emph{stable} introduced by Cholak, Jockusch, and
Slaman~\cite{CJS}.

If $c:\NN^{m}\to\NN$ is strongly $n$-stable then the iterated limit
\begin{equation*}
  f(x_1,\dots,x_{m-n}) = \lim_{z_1\to\infty} \cdots \lim_{z_n\to\infty} c(x_1,\dots,x_{m-n},z_1,\dots,z_n)
\end{equation*}
defines a total $\BSigma^0_{n+1}$ map $f:\NN^{m-n}\to\NN$.
(More precisely, the graph of $f$ is $\BSigma^0_{n+1}$-definable).
Take for example a map $f$ defined by
$f(x)=\lim_{z_1}\lim_{z_2}\lim_{z_3}g(x,z_1,z_2,z_3)$.
Then $f(x)=y$ if and only if
$$(\exists w_1)(\forall z_1>w_1)(\forall w_2)(\exists z_2>w_2)(\exists w_3)(\forall z_3>w_3)[c(x,z_1,z_2,z_3)=y],$$
and so the graph of $f$ is $\BSigma^0_{4}$-definable.
The converse of this fact about maps defined by limits is due
to \v{S}vejdar~\cite{Svejdar} and is stated below
(more precisely, what is stated below is an iterated version of \v{S}vejdar's result).
Note that when working in \RCAo\ we cannot assume that $f$ exists.
For this reason we use the word `map' for such a function whose existence is uncertain.


\begin{lem}[Theorem 1 of \v{S}vejdar \cite{Svejdar}, $\RCAo + \Bnd{\BPi^0_{n-1}}$; $1 \leq n < \omega$]\label{L:ittLimitLem}
  Every total $\BSigma^0_{n+1}$-definable map $f:\NN\to\NN$ is
  representable in the form
  \begin{equation*}
    f(x) = \lim_{z_1\to\infty} \cdots \lim_{z_n\to\infty} c(x,z_1,\dots,z_n),
  \end{equation*}
  where $c:\NN^{n+1}\to\NN$ is a strongly $n$-stable function.
\end{lem}

Now consider a version of $\EIndec^n$ which only considers strongly $n$-stable colorings.
\begin{definition}
Fix $n\geq 2$.
We let $\SEIndec^{n}$ be the statement that for every $k$ and every
strongly $(n-1)$-stable coloring $c:\NN^{n}\to\{0,1,\ldots k-1\}$
there is a $d<k$ such that
$$(\exists^\infty x)(\exists^\infty z_1)\ldots(\exists^\infty z_{n-1})[f(x,z_1,\ldots,z_{n-1})=d].$$
\end{definition}

\begin{prop}\label{P:SEIndec&Reg}
Fix $n\geq 1$.  $\SEIndec^{n+1}$ is equivalent, over \RCAo, to $R{\BSigma^0_{n+1}}$.
\end{prop}
\begin{proof}
Let $c:\NN^{n+1}\to\{0,1\ldots, k-1\}$ be as in the statement of $\SEIndec^n$.
Let $f:\NN\to\{0,1\ldots, k-1\}$ be the map defined by the limit
$f(x) = \lim_{z_1} \cdots \lim_{z_n} c(x,z_1,\dots,z_n)$.
As we have seen before, the graph of $f$ is $\BSigma^0_{n+1}$.
By $\mathsf{R}{\BSigma^0_{n+1}}$ there is a $d<k$ such that $(\exists^\infty{x})f(x)=d$.
In other words, $(\exists^\infty{x})(\forall^\infty {z_1})\ldots(\forall^\infty {z_n})c(x,z_1,\ldots,z_n)=d$.
Therefore $R{\BSigma^0_{n+1}}$ implies $\SEIndec^{n}$.

To prove the other direction, by proposition \ref{P:RisR'} it suffices to prove
$\mathsf{R}'{\BSigma^0_{n+1}}$ from $\SEIndec^{n+1}$, which we will do by (external) induction on $n$.
The induction hypothesis allows us to assume that $\mathsf{R}\BSigma^0_{n}$ holds,
which is equivalent to $\Bnd{\BPi^0_{n}}$.

Let $\varphi(x,y,w)$ be $\BPi^0_{n}$ and suppose that
$(\forall x)(\exists y<k)(\exists w)\varphi(x,y,w)$.
Let $\seq{}:\NN^2\to\NN$ be a pairing function
and let $g(x)$ be the least number $a=\seq{y,w}$ such that $\varphi(x,y,w)$ holds.
In other words $g(x)=\seq{y,w}$ if and only if
$$\Big[\big(\forall\seq{y',w'}<\seq{y,w}\big)\neg\varphi(x,y',w')\Big]\land\varphi(x,y,w).$$
Notice that the graph of $g$ is $\BSigma^0_{n+1}-$definable
(the current description of $g$ is not technically a $\BSigma^0_{n+1}$
statement, but it can be put in normal form using $\Bnd{\BPi^0_n}$).
Notice also that $g$ is well-defined and total by $\Ind{\BPi^0_n}$
(which follows from $\Bnd{\BPi^0_n}$).
In the base case, where $n=0$, we are assuming $\Bnd{\BPi^0_1}$.
This is a safe assumption since $\SEIndec^1$ implies $RT^1(\omega)$,
which Hirst proved was equivalent to $\Bnd{\BPi^0_1}$ \cite{Hirst:thesis}.

Since  $g$ is a total $\BSigma^0_{n+1}$-definable function,
by Lemma~\ref{L:ittLimitLem} there is a map $c:\NN^{n+1}\to\NN$ such that
$$g(x) = \lim_{z_1\to\infty} \cdots \lim_{z_n\to\infty} c(x,z_1,\dots,z_n),$$
where $c:\NN^{n+1}\to\NN$ is a strongly $n$-stable function.
Let $c':\NN^{n+1}\to\NN$ be the strongly $n$-stable function
defined by $c'(x,z_1,\dots,z_n)=\min\{c(x,z_1,\dots,z_n),k\}$.
By $\SEIndec^n$ there is a $d\leq k$ such that
$$(\exists^\infty x)(\exists^\infty z_1)\ldots(\exists^\infty z_n)c(x,z_1,\ldots, z_n)=d.$$
Moreover, since $g$ is total $d\neq k$.

Notice that for each $x$ such that $(\exists^\infty z_1)\ldots(\exists^\infty z_n)f(x,z_1,\ldots, z_n)=d$,
we have that $g(x)=d$.
Therefore there are infinitely many $x$ such that $\exists w\phi(x,d,w)$ holds.
\end{proof}

We now consider an upper bound for the amount of
induction needed to prove $\EIndec^n$.

We will need the following result which is essentially due to Jockusch
and Stephan \cite{JockuschStephan}.
We now need notation to distinguish the set of natural numbers
from the first-order part of a model of second-order arithmetic.
We use $\NN$ to denote the natural numbers and $\omega$
to denote the first-order part of a model.

\begin{lem}[Theorem~2.1 of Jockusch-Stephan \cite{JockuschStephan}]\label{L:JockuschStephan}
Let $\mathcal{N}$ be a model of $\RCAo$.
Given a sequence of sets $A = \seq{A_n}_{n=0}^\infty$ such that
$A''\in\mathcal{N}$, there is an infinite set $X\in\mathcal{N}$ such that $(X \oplus A)''
\equiv_T A''$ and, for all $n$, either $X \subseteq^* A_n$ or
$X \subseteq^* \omega \setminus A_n$.
\end{lem}

\noindent
Since Theorem~2.1 of \cite{JockuschStephan} was not originally
proved in terms of models of \RCAo, we present a proof of this theorem
(which is nearly identical to that in \cite{JockuschStephan}).

\begin{proof}
We wish to effectively list all the sets $A_n$, together with $\omega$
and all sets resulting from finite applications of intersection and
complementation of the sets $A_n$.  We assume that
$\seq{B_n}_{n=0}^\infty$ is an enumeration of all these sets such that
$B_0 = \omega$ and such that an index $e$ for $B_e = B_n \cap B_m$ or
$B_e = B_n \setminus B_m$ can be effectively computed from $n$ and $m$.

Consider the following partial $A'$-computable function
  \begin{equation*}
    f(m,n) =
    \begin{cases}
      0 & \text{if } |B_n\cap B_m|<|B_n\setminus B_m|,\\
      1 & \text{if } |B_n\cap B_m|>|B_n\setminus B_m|,\\
      {\uparrow} & \text{otherwise.}
    \end{cases}
  \end{equation*}
By the Low Basis Theorem of Jockusch and Soare \cite{lowbasis},
there is a total function $g:\NN^2\to\{0,1\}$ such that:
\begin{itemize}
  \item if $|B_n\cap B_m|<|B_n\backslash B_m|$ then $g(n,m)=0$;
  \item if $|B_n\cap B_m|>|B_n\backslash B_m|$ then $g(n,m)=1$; and
  \item $(A'\oplus g)' \equiv_T A''$.
\end{itemize}
Moreover, by the Friedberg Jump Inversion Theorem \cite{jumpinversion},
there is a set $C$ such that $C' \equiv_T C \oplus A' \equiv_T g \oplus A'$.
We now have that $C'' \equiv_T (A'\oplus f)' \equiv_T A''$.

We will now use the set $C$ to construct a sequence of sets
$\seq{B_{e_n}}_{n=0}^\infty$ from $\seq{B_n}_{n=0}^\infty$.
Let $e_0=0$, so $B_{e_0}=\NN$.
We now define the indices $e_n$ inductively in such a way that
\begin{equation*}
  B_{e_{n+1}}=
  \begin{cases}
    B_{e_n}\backslash B_n & \text{if $g(e_n,n)=0$ and thus $|B_{e_n}\cap B_n|\leq|B_{e_n}\backslash B_n|$,} \\
    B_{e_n}\cap B_n & \text{if $g(e_n,n)=1$ and thus $|B_{e_n}\backslash B_n|\leq|B_{e_n}\cap B_n|$.} \\
  \end{cases}
\end{equation*}
Notice that all the sets $B_{e_n}$ are infinite.
Notice also that the indices $e_n$ can be effectively computed from $g$.
Therefore since $g \leq_T C'$ there is a uniformly $C$-recursive
approximation $e_{n,s}$ for each $e_n$.

We now take a diagonal intersection:
\begin{align*}
  x_0 &= 0 & x_{n+1} &= \min\set{x>x_n}{(\forall m\leq n)[x\in B_{e_{m,x}}]}.
\end{align*}
Let $X = \{x_n\}_{n=0}^\infty$.
Notice also that $X \subseteq^*B_{e_m}$ for all $m$ since
$x_{n+1} \in B_{e_{m,x_{n+1}}}$ for all $n\geq m$ and $\lim_n e_{m,x_{n+1}}=e_m$.
So since either $B_{e_{m+1}}\subseteq B_m$ or $B_{e_{m+1}} \subseteq \omega\setminus B_m$
for all $m$, $X$ has the property that either $X \subseteq^*A_m$ or
$X \subseteq^* \omega\setminus A_m$ for all $m$.
\end{proof}

Recall that Hirst proved that $\EIndec^1$ is equivalent to $\Bnd{\BSigma^0_2}$ over \RCAo\ \cite{Hirst:thesis},
and that $\Bnd{\BSigma^0_2}$ is provable in \RCAo+ $\Ind{\BSigma^0_2}$.

\begin{thm}[Corduan-Dorais]\label{T:EIndec&Ind}
Fix $n\geq 2$.
$\RCAo+\Ind{\BSigma^0_{n+1}}$ implies $\EIndec^n$.
\end{thm}

\begin{proof}
Let $\mathcal{N}$ be a model of $\RCAo + \Ind{\BSigma^0_{n+1}}$ and let
$c_0:\omega^n\to\{0,1,\dots,k-1\}$ be a coloring in $\mathcal{N}$.
Let $\MM$ be the model of $\RCAo$ whose second-order part consists of all
$\BDelta^0_{n+1}$-definable sets with parameters from $\mathcal{N}$,
and whose first order part is the same as $\mathcal{N}$.

Given $\overline{x}\in\omega^{n-1}$ and $i<k$, let
$A_{\overline{x},i}=\set{y\in\NN}{c_0(\overline{x},y)=i}$.
Let $A=\seq{A_n}_{n=0}^\infty$ effectively enumerate all such sets $A_{\overline{x},i}$.
Since $A'' \equiv_T c_0'' \in\MM$, by
Lemma~\ref{L:JockuschStephan} there is an infinite set $X_1\in\MM$ such
that $(c_0 \oplus X_1)'' \equiv_T c_0''$ and, for all $\overline{x}$ and $i$,
either $X_1\subseteq^* A_{\overline{x},i}$ or
$X_1\subseteq^*\omega\setminus A_{\overline{x},i}$.
We now define a new coloring $c_1:\omega^{n-1}\to\{0,1,\ldots, k-1\}$ by
$$c_1(z_1,z_2,\ldots,z_{n-1}) = \lim_{x \in X_1} c_0(z_1,z_2,\ldots,z_{n-1},x),$$
which is computable from $(c_0 \oplus X_1)'$.
Note also that $c_1'\leq_T c_0''$, and so $c_1\in\MM$.

If $n \geq 3$, we now repeat this process for the coloring $c_1$.
For this construction to work, use the fact that $c_1'' \leq_T (c_0
\oplus X_1)''' \equiv_T c_0''' \in\MM$ in order to apply
Lemma~\ref{L:JockuschStephan} as above.  We are left with an
infinite set $X_2\in\MM$ such that $(c_1 \oplus X_2)'' \equiv_T c_1''
\leq_T c_0'''$ and which defines a coloring
$$c_2(z_1,\ldots,z_{n-2}) = \lim_{x \in X_2} c_1(z_1,\ldots,z_{n-2},x),$$
which is computable in $(c_1 \oplus X_2)'$.
Notice that $c_2'\leq_T c_1''\leq_T c_0'''$, and so $c_2\in\MM$.

Continuing this process as necessary we end with a set $X_{n-1}\in\MM$
such that $(c_{n-2} \oplus X_{n-1})'' \equiv_T c_{n-2}'' \in \MM$ and
$$c_{n-1}(z_1) = \lim_{x\in X_{n-1}} c_{n-2}(z_1,x)$$
exists for all $z_1$.
Since $c_{n-1}' \leq_T c_{n-2}'' \leq_T c_0^{(n)} \in \MM$,
there is an $i$ such that there are infinitely many $z$ such that $c_1(z) = i$.
Unraveling the definition of all the colorings we see that
$$(\exists^\infty x_1)\ldots(\exists^\infty x_n)[c_0(x_1,\dots,x_n)=i]$$
holds in $\MM$.
Therefore the same holds in $\mathcal{N}$.
\end{proof}

\section{Indecomposability and Embeddings}\label{Indec&Embed}
	This section is joint work with Dorais.

We now consider $RT^1(\omega^n)$ and the
more natural definition of `order type $\omega^n$'.
As in Section~\ref{ElemIndec}, we choose a particular
representation of the elements of $\omega^n$, namely $\NN^n$,
which we order lexicographically.
In other words, if $\tup{x},\tup{y}\in\NN^n$, then
$\tup{x}\lex\tup{y}$ if $x_i<y_i$, where $i=\mathsf{min}\set{j<k}{x_j\neq y_j}$.
The following definition is then a special case of Definition~\ref{D:RamseyPOsets}.

\begin{definition}
$RT_k^1(\omega^n)$ is the following statement:

For every finite coloring $c: \NN^n \to \{0,\dots,k-1\}$,
there is a lexicographic embedding\index{lexicographic embedding}
$h: \NN^n \to \NN^n$ such that $c \circ h$ is constant.
\end{definition}

We will use $RT^1(\omega^n)$ to denote $(\forall k)RT_k^1(\omega^n)$.
Notice that $RT^1(\omega^1)$, just like $\EIndec^1$,
is a rephrasing of the infinite pigeonhole principle,
which Hirst proved equivalent to $\Bnd{\BSigma^0_2}$ over \RCAo\ \cite{Hirst:thesis}.
The main result of this section is that $RT^1_2(\omega^3)$ is equivalent to $\ACAo$
over \RCAo, which we will prove using three lemmas.

Given $h:\NN^n\to\NN^n$ and $1\leq i\leq n$,
we let $h_i:\NN^n\to\NN$ be projection of $h$
onto the $i$-th coordinate.

\begin{lem}[Corduan-Dorais; $\RCAo$; $n\geq 1$]\label{L:lexfst}
If $h:\NN^n\to\NN^n$ is a lexicographic embedding then
$$x_1 \leq h_1(x_1,x_2,\dots,x_n) < h_1(x_1+1,0,\dots,0)$$
for all $x_1,\dots,x_n \in \NN$.
\end{lem}

\begin{proof}
By (external) induction on $n$.
The case $n=1$ is trivial.
Suppose then that the result is true for $n$.
Let $h:\NN^{n+1}\to\NN^{n+1}$ be a lexicographic embedding.

We show that
$$h_1(x_1,x_2,\dots,x_{n+1}) < h_1(x_1+1,0,\dots,0)$$
for all $x_1,x_2,\dots,x_{n+1} \in \NN$.
The fact that $x_1 \leq h_1(x_1,x_2,\dots,x_{n+1})$ then follows by induction.
Suppose, for the sake of contradiction, that
$h_1(x_1,x_2,\dots,x_{n+1})=h_1(x_1+1,0,\dots,0)$.
Let $\tilde{h}:\NN^n\to\NN^n$ be the lexicographic embedding defined by
$$\tilde{h}_i(z_1,\dots,z_n) = h_{i+1}(x_1,x_2+1+z_1,z_2,\dots,z_n)$$
for each $1\leq i\leq n$.
By the induction hypothesis,
$$z_1 \leq \tilde{h}_1(z_1,0,\dots,0) = h_2(x_1,x_2+1+z_1,0,\dots,0).$$
Since $h$ is a lexicographic embedding and
$h_1(x_1,x_2,\dots,x_{n+1})=h_1(x_1+1,0,\dots,0)$,
then $h_2(x_1,x_2+y,0,\ldots,0)\leq h_2(x_1+1,0,\dots,0)$ for all $y>0$.
Therefore, for all $z_1 \in \NN$,
$$z_1 \leq \tilde{h}_1(z_1,0,\dots,0) = h_2(x_1,x_2+1+z_1,0,\dots,0)\leq h_2(x_1+1,0,\dots,0),$$
which is clearly impossible.
\end{proof}

\begin{lem}[Corduan-Dorais; $\RCAo$; $n\geq 1$]\label{L:lexlim}
If $h:\NN^n\to\NN^n$ is a lexicographic embedding and
$1 \leq j < i \leq n$, then
$$\lim_{x_i\to\infty} h_j(x_1,\dots,x_{i-1},x_i,0,\dots,0)$$
exists and is bounded above by $h_j(x_1,\dots,x_{i-1}+1,0,\dots,0)$.
\end{lem}

\begin{proof}
We proceed by induction on $j < i$.
By the induction hypothesis, find $\tilde{x}_i$ such that
$$h_k(x_1,\dots,x_{i-1},x_i,0,\dots,0) = h_k(x_1,\dots,x_{i-1},\tilde{x}_i,0,\dots,0)$$
for all $x_i \geq \tilde{x}_i$ and $1 \leq k < j$.
Note that we must then have
\begin{multline*}
  h_j(x_1,\dots,x_{i-1},x_i,0,\dots,0) \\
  \leq h_j(x_1,\dots,x_{i-1},x'_i,0,\dots,0) \\
  \leq h_j(x_1,\dots,x_{i-1}+1,0,0,\dots,0)
\end{multline*}
for all $x'_i \geq x_i \geq \tilde{x}_i$.
It follows immediately that
$$\lim_{x_i\to\infty} h_j(x_1,\dots,x_{i-1},x_i,0,\dots,0)$$
exists and is bounded above by $h_j(x_1,\dots,x_{i-1}+1,0,0,\dots,0)$.
\end{proof}

\begin{lem}[Corduan-Dorais; $\RCAo$; $n\geq 1$]\label{L:lexinf}
If $h:\NN^n\to\NN^n$ is a lexicographic embedding and
$1 \leq i \leq n$, then
$$\lim_{x_i\to\infty} h_i(x_1,\dots,x_{i-1},x_i,0,\dots,0) = \infty$$
for all $x_1,\dots,x_{i-1} \in \NN$.
\end{lem}

\begin{proof}
By Lemma~\ref{L:lexlim}, we can find $\tilde{x}_i$ such that
$$h_j(x_1,\dots,x_{i-1},x_i,0,\dots,0) = h_j(x_1,\dots,x_{i-1},\tilde{x}_i,0,\dots,0)$$
for all $x_i \geq \tilde{x}_i$ and all $1 \leq j < i$.
Note that the map $\tilde{h}:\NN^{n-i+1}\to\NN^{n-i+1}$ defined by
$$\tilde{h}_k(y_1,\dots,y_{n-i+1}) = h_{i+k-1}(x_1,\dots,x_{i-1},\tilde{x}_i+y_1,y_2,\dots,y_{n-i+1})$$
is then a lexicographic embedding and the result follows immediately
by applying Lemma~\ref{L:lexfst} to $\tilde{h}$.
\end{proof}

\begin{thm}[Corduan-Dorais]
  $RT^1_2(\omega^3)$ is equivalent to $\ACAo$ over $\RCAo$.
\end{thm}

\begin{proof}
Proving $RT^1_2(\omega^3)$ in \ACAo\ is straightforward.
Since $\Ind{\BSigma^0_4}$ holds in \ACAo, by Theorem~\ref{T:EIndec&Ind}
we may assume $\EIndec^3$.
Therefore for every coloring $c:\NN^3\to\{0,1\}$, there is a $d<2$ such that
$(\exists^\infty x_1)(\exists^\infty x_2)(\exists^\infty x_3)[c(x_1,x_2,x_3)=d].$
We can then use arithmetic comprehension to define a lexicographic embedding
into the set $\set{(x_1,x_2,x_3)\in\NN^3}{c(x_1,x_2,x_3)=d}$.

We now show how to compute the range of a function $f:\NN\to\NN$ using $RT^1_2(\omega^3)$.
For each $z$, let $f[z] = \{f(0),\dots,f(z)\}$.
Consider the coloring $c:\NN^3\to\{0,1\}$ defined by
$$c(x,y,z) =
\begin{cases}
0 & \text{when $(\forall w \leq x)(w \in f[y] \leftrightarrow w \in f[z])$,} \\
1 & \text{otherwise.}
\end{cases}$$
Suppose $h:\NN^3\to\NN^3$ is a lexicographic embedding
such that $c\circ h$ is constant.
First, note that $c \circ h$ must have constant value $0$.

For suppose instead that $c \circ h$ has constant value $1$.
By Lemma~\ref{L:lexlim} there is a $y_0$ such that
$h_1(0,y,0) = h_1(0,y_0,0)$ for all $y \geq y_0$.
Again by Lemma~\ref{L:lexlim} there is a $z_0$ such that
$h_2(0,y_0,z) = h_2(0,y_0,z_0)$ for all $z \geq z_0$.
By Lemma~\ref{L:lexinf} there is a $y_1>y_0$ such that
$h_2(0,y_0,z_0)<h_2(0,y_1,0)$.
Putting this together we see that
$h_2(0,y_0,z_0) < h_2(0,y_1,0) \leq h_3(0,y_0,z_0)$.
Similarly, we can find can find pairs $(y_1,z_1),\dots,(y_k,z_k)$,
where $k=h_1(0,y_0,0)+1$,
such that $h_3(0,y_{i-1},z_{i-1}) \leq h_2(0,y_i,z_i) < h_3(0,y_i,z_i)$
for each $1 \leq i \leq k$.
Since $c(h(0,y_i,z_i)) = 1$, we can find
$w_i \in f[h_3(0,y_i,z_i)] - f[h_2(0,y_i,z_i)]$ where
$w_i \leq h_1(0,y_i,z_i) = h_1(0,y_0,0)$.
Our choice of $y_i,z_i$ guarantees that the $w_i$ must be distinct,
which is impossible since $k > h_1(0,y_0,0)$.
Therefore $c \circ h$ has constant value $0$.

We now show how to compute the range of $f$ from $h$ and $c$.
To determine whether $x$ is in the range of $f$, use the following procedure:
\begin{itemize}
\item[] First find $y$ such that $h_1(x,y,0) = h_1(x,y+1,0)$.
		Answer yes if $x \in f[h_2(x,y+1,0)]$, and answer no otherwise.
\end{itemize}
This procedure will never return false positive answers,
so we suppose that $x = f(s)$ and check that the algorithm answers yes on input $x$.
The existence of a $y$ such that $h_1(x,y,0) = h_1(x,y+1,0)$
is guaranteed by Lemma~\ref{L:lexlim}.
Given such a $y$ we can then use Lemma~\ref{L:lexinf} to find $z$
such that $s \leq h_3(x,y,z)$.
Since
$$h_1(x,y,0) = h_1(x,y,z) = h_1(x,y+1,0),$$
we then have
$$h_2(x,y,0) \leq h_2(x,y,z) \leq h_2(x,y+1,0).$$
Since $c(h(x,y,z)) = 0$ and $x \leq h_1(x,y,z)$ by Lemma~\ref{L:lexfst},
we know that
$$x \in f[h_2(x,y,z)] \leftrightarrow x \in f[h_3(x,y,z)].$$
Since $s \leq h_3(x,y,z)$ we know that $x \in f[h_3(x,y,z)]$,
and since $h_2(x,y,z) \leq h_2(x,y+1,0)$ we conclude that $x \in f[h_2(x,y+1,0)]$.
\end{proof}


\chapter{Forcing in Reverse Mathematics}\label{Forcing}
	Forcing is a powerful technique from set theory that
has been used in Reverse Mathematics to build models of
subsystems of second order arithmetic.
In particular, variants of Mathias forcing have been
used to build models of $\RCAo+RT^2_2$ which \textit{are not}
models of $\ACAo$, therefore demonstrating that $\RCAo+RT^2_2$
is a strictly weaker subsystem than $\ACAo$ \cite{CJS}.
When forcing techniques have been used in Reverse Mathematics,
however, they have typically been used in a very recursion-theoretic fashion.
There are good reasons for this, since many of the definitions
used in set-theoretic forcing may not make sense when restricted to $\RCAo$.
Moreover, the Truth Lemma from set theory can fail over models of $\RCAo$,
due to restricted comprehension axioms.
We will have more to say about the truth lemma in Section~\ref{generic}.

In his paper ``A variant of Mathias forcing that
preserves \ACAo", Dorais guides the reader through a
interesting forcing construction \cite{varMathias}.
Much of the forcing construction has been tailor-made to work
in Reverse Mathematics, though the spirit is
still very much in line with set-theoretic forcing.
In what follows, we use the framework that Dorais
built in \cite{varMathias} to examine other notions of forcing.

In Section~\ref{ForcingBasics} we give the basic definitions,
following Dorais \cite{varMathias}.
In Section~\ref{witRCAo} we examine a collection of forcings
that preserve \RCAo.
The propositions in Section~\ref{witRCAo} are relatively straightforward generalizations
of the propositions in \cite{varMathias}.
In Section~\ref{witACAo} we examine a collection of notions of forcings
that preserve \ACAo.
In Section~\ref{generic} we examine the generic extension.
This section is also comprised of relatively straightforward generalizations
of the propositions in \cite{varMathias}.
In Section~\ref{forcingExamples} we look at examples.

\section{The Basics}\label{ForcingBasics}
	A notion of forcing is usually defined as a partial ordering.
Many of the notions of forcing that have been useful in set theory,
particularly those that add generic reals,
can be thought of as collections of subtrees of $\OO$, ordered by inclusion.
We restrict ourselves to such notions of forcing.

\begin{definition}
We write $\OO$\index{$\OO$} for the set of all finite, increasing sequences from $\NN$.
A subset $T\subseteq\OO$ is a \textit{tree}
if it is downward closed, meaning that if
$\sigma\subseteq\tau$ and $\tau\in T$, then $\sigma\in T$.

Given a model $\MM$ of second order arithmetic,
a \textit{notion of forcing}\index{notion of forcing} $\Forc$ is a collection of trees in $\MM$.
The elements of $\Forc$ are called \textit{conditions}\index{condition} and are ordered by inclusion.
If $T$ and $T'$ are conditions and and $T'\subseteq T$, we say that $T'$ \textit{extends} $T$
and we write $T'\leq T$.
When there is a possibility that either $T$ or $T'$ is not a condition,
we write $T'\subseteq T$, thus reserving the notation $T'\leq T$ only
for the case when both $T'$ and $T$ are conditions.
\end{definition}

Note that a notion of forcing is a third-order object,
and is therefore never an element of the model in question.
The conditions, however, are elements of the model.

The following notation will be useful in dealing with conditions.
\begin{definition}\label{D:treestuff}
Let $T\subseteq\OO$ be a tree.
A \textit{path}\index{path} through $T$ is a function $f:\NN\to\NN$
such that $f[n]=\seq{f(0),f(1),\ldots,f(n-1)}\in T$ for all $n\in\NN$.
The collection of all paths through $T$ is denoted by $[T]$\index{$[T]$}.

Let $\tau\in T$.
We use $T_\tau$\index{$T_\tau$} to denote the tree consisting of all nodes
$\sigma\in T$ such that either $\sigma\supseteq\tau$ or $\sigma\subseteq\tau$.
\end{definition}

Given a model $\MM$ and a notion of forcing $\Forc$,
we define a formal language, called the forcing language, inside $\MM$.
We start by describing the \textit{symbols of the forcing language}.
The forcing language contains \textit{constant symbols} $0,1,2,\ldots$ for
the natural numbers, and symbols for \textit{number variables},
which we will usually denote by lowercase letters late in the alphabet,
such as $v$ and $w$.
The \textit{logical symbols} of the forcing language will consist of
=, $\land$, $\neg$, and $\forall v$ (where $v$ is a number variable).
There are also symbols that behave like function parameters.
More specifically, there is a
\textit{symbol $F$ for each $k$-ary name} in $\MM$, which we will now define.
\begin{definition}\label{D:names}\index{name}
A \textit{$k$-ary name} is a $\BSigma^0_1$ set $F\subseteq \OO\times\NN^{k}\times\NN$ such that
\begin{itemize}
\item If $(\tau,\tup{x},y)\in F$ and $\tau\subseteq\tau'$ then $(\tau',\tup{x},y)\in F$
\item If $(\tau,\tup{x},y),(\tau,\tup{x},y')\in F$ then $y=y'$.
\end{itemize}
To each subset $A\subseteq\NN$ we associate an element \rune$\in[\OO]$
by letting \rune$(n)$ be the $n$-th least element of $A$.
Given $\tau\in\OO$, we write $\tau\leq A$ to mean that $\tau$ is an initial segment of \rune.

The \textit{domain of $F$} is the class
$$\dom(F)=\{A\subseteq\NN:\forall\tup{x}\exists\tau,y(\tau\leq A\land(\tau,\tup{x},y)\in F)\}.$$
Given $A\in\dom(F)$, the \textit{evaluation}\index{evaluation of a name}
$$F^A(\tup{x})=y\ \Leftrightarrow \ \exists\tau(\tau\leq A\land(\tau,\tup{x},y)\in F).$$
is a total $k$-ary function.
\end{definition}

Note that all ground model functions $G$ (meaning that $G\in\MM$)
have \textit{canonical names}\index{canonical names} $\check{G}$ defined by
$$(\tau,\tup{x},y)\in\check{G}\Leftrightarrow y=G(\tup{x}).$$
$\check{G}$ is a name for $G$ in the sense that $\check{G}^A=G$ for all sets $A$.
We usually write $F^\tau(x)=y$\index{$F^\tau(x)$} in place of $(\tau,\tup{x},y)\in F$.
In light of the fact that $F$ is a $\BSigma^0_1$ set,
we will say that $F^\tau(\tup{x})$ is \textit{defined by stage $n$}
if there are $w,y\leq n$ and a $\sigma\subseteq\tau$
such that $w$ witnesses that $F^\sigma(\tup{x})=y$.
We use the abbreviation $F^\tau(\tup{x})\neq y$ to mean that
either $F^\tau(\tup{x})$ is not defined at stage $n=|\tau|$,
or it is defined at stage $n=|\tau|$ and is distinct from $y$.

We now define what it means for a name to be local for a condition.

\begin{definition}\label{D:localNames}
Let $T$ be a condition.
A $k$-ary name $F$ is $T$\textit{-local}\index{$T$-local}\index{locality}
if for every $\tup{x}\in\NN^k$ and extension $T'\leq T$,
there is a $\tau\in T'$ and a $y$ such that
$F^\tau(\tup{x})$ is defined by stage $|\tau|$ and $F^\tau(x)=y$.
\end{definition}

Notice that if $G$ is a ground model function then its canonical name
$\check{G}$ is a $T$-local name for every condition $T$.

For many of the notions of forcing that we will be concerned with,
there is a convenient definition of locality which is equivalent
to the definition just given.
We explain this in detail with the following definition and proposition.

\begin{definition}
A notion of forcing $\Forc$ is \textit{persistent}\index{persistent}
if it satisfies the following two properties:
\begin{itemize}
\item $T_\tau\in\Forc$ whenever $T\in\Forc$ and $\tau\in T$,
\item for every $n\in\NN$ and condition $T$ there is a
		$\tau\in T$ such that $|\tau|\geq n$.
\end{itemize}
\end{definition}
The first property above is the one that really
characterizes persistent notions of forcing.
The second property ensures that the
atomic forcing relation is not trivially satisfied.

\begin{prop}[\RCAo]\label{P:PersistentLocalNames}
Let $\Forc$ be a persistent notion of forcing, $T\in\Forc$,
and $F$ be a $k$-ary name.
Then $F$ is $T$-local if and only if $[T']\cap\dom(F)$ is nonempty
for every extension $T'\leq T$.
\end{prop}

\begin{proof}
For the purposes of the proof only, we say that a $k$-ary name $F$ is
$T$-local$_2$ if $[T']\cap\dom(F)$ is nonempty for every extension $T'\leq T$.
It follows immediately that if $F$ is $T$-local$_2$ then $F$ is also $T$-local.
Therefore we assume that $F$ is $T$-local and show that $F$ is also $T$-local$_2$.

Let $T'\leq T$ be given.  We will construct an $A\in[T']\cap\dom(F)$ in stages.
For notational convenience we assume that $F$ is a $1$-ary name.
We begin at stage $0$ by finding a $\tau_0\in T'$ and a $y_0\in\NN$ such that $F^{\tau_0}(0)=y_0$.
Such $\tau_0$ and $y_0$ are guaranteed to exist since $F$ is $T$-local.
Let $T^1=T'_{\tau_0}$.
We now proceed to stage $1$ where we find a $\tau_1\in T^1$
and a $y_1\in\NN$ such that $F^{\tau_1}(1)=y_1$.
Such $\tau_1$ and $y_1$ are guaranteed to exist since $F$ is $T$-local.
Let $T^2=T^1_{\tau_1}$.
Continuing in this way, at stage $n+1$ we find a $\tau_{n+1}\in T^n$
and a $y_{n+1}\in\NN$ such that $F^{\tau_{n+1}}(n+1)=y_{n+1}$.
Finally, we let $\displaystyle A=\bigcup_{n\in\NN}\tau_n$.
Then $\Ind{\BSigma^0_1}$ suffices to show that $A\in[T']\cap\dom(F)$.
\end{proof}

Whenever we have a persistent notion of forcing,
we will use the alternate definition of locality as given
by Proposition~\ref{P:PersistentLocalNames}.
We now define the formulas of the forcing language and the forcing relation.

\begin{definition}\label{D:forcingLang}\index{forcing language}
The \textit{formulas of the forcing language} are the smallest family which is closed under the following rules:
\begin{itemize}
\item Let $F$ be a $k$-ary  name, $F'$ be a $k'$-ary name, and let
		$\tup{w}=w_1,\ldots,w_k$, $\tup{w}'=w_1',\ldots,w_{k'}'$, where each $w_i$ and $w_i'$
		is either a variable symbol or a constant symbol.
		Then $F(\tup{w})=F'(\tup{w}')$ is a formula.
\item If $\varphi$ is a formula, then so is $\neg\varphi$.
\item If $\varphi$ and $\psi$ are formulas, then so is $\varphi\land\psi$.
\item If $\varphi$ is a formula and $x$ is a variable, then $\forall x\varphi$ is a formula.
\end{itemize}
\end{definition}

We now define locality for formulas.

\begin{definition}\label{D:localFormulas}\index{locality}
Let $T$ be a condition.
A formula $\varphi$ of the forcing language is a $T$\textit{-local formula}
of the forcing language if every name that occurs in $\varphi$ is $T$-local.

We say that $\varphi$ is a $T$\textit{-local sentence} of the forcing language
if $\varphi$ is obtained from a $T$-local formula by replacing all the
free variables with constant symbols.
\end{definition}

\begin{definition}\label{D:forcingRel}\index{forcing relation}
The \textit{forcing relation} $T\Vdash\varphi$ is defined by induction on complexity of the
$T$-local sentences of the forcing language.
Assume that all names that occur in the sentences below are $T$-local.
\begin{itemize}
\item $T\Vdash F(\tup{x})=F'(\tup{x}')$ if $y_1=y_2$ whenever
	$\tau\in T$, $(\tau,\tup{x},y_1)\in F$, and $(\tau,\tup{x}',y_2)\in F'$.
\item $T\Vdash \varphi\land\psi$ if $T\Vdash \varphi$ and $T\Vdash \psi$.
\item $T\Vdash \forall v \varphi(v)$ if $T\Vdash \varphi(x)$ for all $x\in\NN$.
\item $T\Vdash\neg\varphi$ if there is no $T'\leq T$ such that $T'\Vdash\varphi$.
\end{itemize}
\end{definition}

The first thing to notice about the forcing relation is that if
$T\Vdash\varphi$ and $T'\leq T$, then $T'\Vdash\varphi$.
(This is an easy proof by induction on the complexity of $\varphi$.)

Notice also that from the definition of forcing the negation
of a sentence it immediately follows that
if $T\in\Forc$ and $\varphi$ is a $T$-local sentence,
then either $T\Vdash\neg\phi$ or $T'\Vdash\phi$ for some $T'\leq T$.

We have defined the forcing language in terms of
negation, conjunction, and universal quantification.
We consider disjunction, implication, and
existential quantification as abbreviations in the usual way.
It is worthwhile to unravel the abbreviations and
see what they mean in the context of the forcing language.

First consider the statement that $T\Vdash\varphi\lor\psi$.
This statement is shorthand for $T\Vdash\neg(\neg\varphi\land\neg\psi)$.
Unpacking the definitions, we see that this means that
there is no $T'\leq T$ such that $T'\Vdash\neg\varphi$ and
$T'\Vdash\neg\psi$.
Therefore for every $T'\leq T$, there is a $T''\leq T'$
such that either $T''\Vdash\varphi$ or $T''\Vdash\psi$.

Now consider the statement that $T\Vdash\varphi\rightarrow \psi$.
This statement is shorthand for $T\Vdash\neg(\varphi\land\neg\psi)$.
Unpacking the definitions, we see that this means that
there is no $T'\leq T$ such that $T'\Vdash\varphi$ and $T'\Vdash\neg\psi$.
Therefore every $T'\leq T$, if $T'\Vdash\varphi$ then
there is some $T''\Vdash T'$ such that $T''\Vdash\psi$.

Now consider the statement that $T\Vdash\exists x\varphi(x)$.
This statement is shorthand for $T\Vdash\neg\forall x\neg(\varphi(x))$.
Unpacking the definitions, we see that this means that
there is no $T'\leq T$ such that $T'\Vdash\neg\varphi(x)$ for all $x$.
Therefore for every $T'\leq T$ there is an $x$ and a $T''\leq T$
such that $T''\Vdash\varphi(x)$.

Double negation is not a shorthand,
though it is still helpful to unravel its definition.
If $T\Vdash\neg\neg\varphi$, then there is no $T'\leq T$
such that $T'\Vdash\neg\varphi$.
Therefore for all $T'\leq T$, there is a $T''\leq T'$
such that $T''\Vdash\varphi$.

In order to prove the usual rule about double negation,
we assume that the notion of forcing in question is persistent.

\begin{prop}[\RCAo]\label{P:2neg}
Let $\Forc$ be a persistent notion of forcing.
Let $T\in\Forc$ and $\varphi$ be a $T$-local sentence.
$T\Vdash\neg\neg\varphi$ if and only if $T\Vdash\varphi$.
\end{prop}
\begin{proof}
First we show that if $T\nVdash\varphi$ then there is a condition
$T'\leq T$ such that $T'\Vdash\neg\varphi$.
The proof is by induction on the complexity of $\varphi$,
though only the atomic case merits description.
Suppose that $T\nVdash F_1(\tup{x})=F_2(\tup{x})$.
Then there is a $\tau\in T$ and numbers $\tup{x}$, $y_1$, and $y_2$
such that $F_1^\tau(\tup{x})=y_1$, $F_2^\tau(\tup{x})=y_2$, and $y_1\neq y_2$.
Therefore $T_\tau\Vdash F_1(\tup{x}) \neq F_2(\tup{x})$.

We now show that $T\Vdash\neg\neg\varphi$ if and only if $T\Vdash\varphi$.
The backwards direction follows from the fact that
if $T\Vdash \varphi$, then $T'\Vdash \varphi$ for every $T'\leq T$.
So suppose that $T\nVdash\varphi$.
By the first paragraph we know that there is a
$T'\leq T$ such that $T'\Vdash\neg\varphi$.
Thus $T'\Vdash\neg\neg\neg\varphi$, and so $T'\nVdash\neg\neg\varphi$.
Therefore $T\nVdash\neg\neg\varphi$.
\end{proof}


% NOTE: care is needed with the labeling of the next two sections.
%	if the hyperref package is used, we use the first section declaration,
%	otherwise we use the second one

\section{Witnessing in \texorpdfstring{\RCAo}{RCA0}}\label{witRCAo}
%\section{Witnessing in \RCAo}\label{witRCAo}
	The main result of this section is Proposition~\ref{P:witnessPi2}.
Though somewhat technical, Proposition~\ref{P:witnessPi2} is the key
to showing that if $\MM$ is a model of $\RCAo$ and $\Forc$ is a
persistent notion of forcing, then the generic extension of $\MM$ is also a model of $\RCAo$.

As mentioned earlier, the propositions in this section are
relatively straightforward generalizations of the propositions in \cite{varMathias}.
Dorais proved the statements in this section for a specific notion of forcing,
though the proofs generalize in a straightforward manner.

\begin{definition}\label{D:admitsNormal}
We say that a notion of forcing \textit{admits normal form}\index{normal form}
if the following statement holds:

Given a bounded formula $\varphi(\tup{v})$ of the forcing language,
there is a name $U_\varphi(\tup{v})$ such that for every condition $T$,
if $\varphi(\tup{v})$ is $T$-local, then so is $U_\varphi(\tup{v})$ and
$$T\Vdash\forall\tup{v}\left(\varphi(\tup{v})\leftrightarrow U_\varphi(\tup{v})=0\right).$$
We say that $U_\varphi$ is the normal form name for $\varphi$.
\end{definition}

The key to showing that a notion of forcing admits normal form
is showing that locality is closed under composition and primitive recursion.

\begin{definition}\label{D:comp&pr}
Let $F_0$ be an $\ell$-ary name and $F_1,\ldots,F_\ell$ be $k$-ary names.
The \textit{composition} $H=F_0\circ(F_1,\ldots,F_\ell)$ is defined by
%$$(\tau,\tup{x},z)\in H\Leftrightarrow \exists\tup{y} [(\tau,\tup{x},y_1)\in F_1\land\ldots\land(\tau,\tup{x},y_\ell)\in F_\ell\land(\tau,\tup{y},z)\in F_0].$$
$$H^\tau(\tup{x})=z\Leftrightarrow \exists\tup{y}
[F_1^\tau(\tup{x})=y_1\ \land\ \ldots\ \land\ F_\ell^\tau(\tup{x})=y_\ell\ \land\ F_0^\tau(\tup{y})=z)].$$

Let $F_0$ be a $(k-1)$-ary name and $F_1$ be a $(k+1)$-ary name.
We use $F_0$ and $F_1$ to define a $k$-ary name $H$ by \textit{primitive recursion} by letting
$H^\tau(\tup{x},y)=z$ if and only if there is a finite sequence
$\langle z_0,z_1,\ldots,z_{y}\rangle$ such that $z_{y}=z$,
$F_0^\tau(\tup{x})=z_0$, and $F_1^\tau(\tup{x},i,z_i)=z_{i+1}$ for all $i<y$.
\end{definition}

\begin{prop}[$\RCAo$]\label{P:comp&pr}
Let $\Forc$ be a persistent notion of forcing and $T\in\Forc$.
The names defined by composition and primitive recursion using $T$-local names
are themselves $T$-local.
\end{prop}
\begin{proof}
Since $\Forc$ is persistent, we use the alternate definition
of locality as given by Proposition~\ref{P:PersistentLocalNames}.

First we handle composition.
Let $F_0$ be an $\ell$-ary $T$-local name, $F_1,\ldots,F_\ell$ be $k$-ary $T$-local names,
and $H=F_0\circ(F_1,\ldots,F_\ell)$.
We show that $H$ is $T$-local.
In other words, we let $T'\leq T$ and show that $\dom(F)\cap[T']$ is nonempty.

Fix $\tup{x}\in\NN^k$.  First we find a $\tau\in T'$ so that $H^\tau(\tup{x})$ is  defined.
Since $F_1$ is $T$-local, there is a $\tau_1\in T'$ and a $z_1$ such that $F_1^{\tau_1}(\tup{x})=z_1$.
Since $F_2$ is $T$-local, there is a $\tau_2\in T'_{\tau_1}$ and a $z_2$ such that
$\tau_1\subseteq\tau_2$ and $F_2^{\tau_2}(\tup{x})=z_2$.
Repeating this process, we eventually get a $\tau\in T'$ such that
$H^\tau(\tup{x})$ is defined.

We can now repeat this process, relative to $T'_\tau$, to define $H$ on a new input.
In other words, we choose some $\tup{x}'$ distinct from $\tup{x}$ and then
find a $\sigma\in T'_\tau$ such that $\tau\subseteq\sigma$ and
$H^\sigma(\tup{x}')$ is defined.
By repeating this for every value in $\NN^k$, we construct $A\in[T']\cap\dom(H)$.
Note that $\Ind{\BSigma^0_1}$ suffices to show that $A$ is well-defined.

Now we handle primitive recursion.
Let $F_0$ be an $(k-1)$-ary $T$-local name, $F_1$ be a $(k+1)$-ary $T$-local name,
and $H$ be the $k$-ary name defined by primitive recursion with $F_0$ and $F_1$.
We now show that $H$ is $T$-local.
In other words, we let $T'\leq T$ and show that $\dom(F)\cap[T']$ is nonempty.

The construction of an element in $\dom(H)\cap[T']$ is similar
to the construction above for the composition of functions.
We continually extend elements $\tau\in T'$ using the locality of $F_0$ and $F_1$
so that $H$ is defined on more and more inputs.
\end{proof}

A consequence of Lemma~\ref{P:comp&pr} is that
our definitions of composition and primitive recursion correspond
to the usual definitions once the names have been evaluated.
In other words, if $A$ is in the domain of the names
$F_0,F_1,\ldots,F_\ell$, then $A$ is also in the domain of $H$
and $H^A=F_0^A\circ(F_1^A,\ldots,F_\ell^A)$,
and similarly for primitive recursion.

Proposition~\ref{P:comp&pr} is the key property
of persistent notions of forcing that we need
to prove nearly everything that remains in this section.
For this reason we make the following definition.

\begin{definition}\label{D:almostP}
A notion of forcing $\Forc$ is \textit{almost persistent}\index{almost persistent}
if it satisfies the following two properties:
\begin{itemize}
\item For every condition $T\in\Forc$, the names defined by composition and primitive
		recursion using T-local names are themselves T-local.
\item For every $n\in\NN$ and condition $T$ there is a $\tau\in T$ such that $|\tau|\geq n$
		and $T_\tau$ contains a condition.
\item For every condition $T$, $T$-local $k$-ary name $F$, and $\tup{x}\in\NN^k$, there is a $\tau\in T$
		such that $T_\tau\in\Forc$ and $F^\tau(\tup{x})$ is defined by stage $|\tau|$.
\end{itemize}
\end{definition}

The second property above ensures that the atomic forcing relation is not trivially satisfied.
The third property will be used later to show that the evaluation of a name
(with a particular locality requirement) at a generic real defines a total $k$-ary function.

Note that if $\Forc$ is a persistent notion of forcing,
then $\Forc$ is also almost persistent by Proposition~\ref{P:comp&pr}
(the second and third properties are immediate).

\begin{prop}[$\RCAo$]\label{P:PR:normalForm}\index{normal form}
Almost persistent notions of forcings admit normal form.
\end{prop}

\begin{proof}
Note that if $F$ and $F'$ are $T$-local names, then the names for
$F+F'$, $F\dotminus~F'$, $|F-~F'|$, and $\displaystyle \sum_{w\leq F(\tup{v})}F'(\tup{v},w)$
are also $T$-local names since they are defined by
composition and primitive recursion.

We define $U_\varphi$ inductively on the complexity $\varphi$:
\begin{itemize}
\item $U_{F(\tup{v})=F'(\tup{v}')}(\tup{v})=|F(\tup{v})-F'(\tup{v}')|$
\item $U_{\neg\varphi}(\tup{v})=1\dotminus U_\varphi(\tup{v})$
\item $U_{\varphi\land\psi}(\tup{v})=U_\varphi(\tup{v})+U_\psi(\tup{v})$
\item if $\varphi(\tup{v})=(\forall w\leq F(\tup{v}))\psi(\tup{v},w)$ then
		$U_{\varphi}(\tup{v})=\sum_{w\leq F(\tup{v})} U_{\psi(w)}(\tup{v},w)$
\end{itemize}
It is easy to check that these names work as advertised.
\end{proof}

One consequence of Proposition~\ref{P:PR:normalForm} is that the $\BPi^0_1$
forcing relation for a persistent notion of forcing is well behaved.
The proposition below, which captures this fact,
will be helpful later on when determining how much genericity is required
of our filters.

\begin{prop}\label{P:Pi1isPi1}
Let $\Forc$ be a persistent notion of forcing.
Let $\varphi(\tup{v})$ be a $\BPi^0_1$ formula of the forcing language.
There is a $\BPi^0_1$ formula $\theta(T,\tup{v})$ such that
for every condition $T$ such that $\varphi$ is $T$-local,
and every $\tup{x}$,
$$T\Vdash\varphi(\tup{x})\ \ \ \text{if and only if}\ \ \theta(T,\tup{x}).$$
Note that we are not making any claims about the complexity of
the statement `$\varphi(\tup{x})$ is $T$-local'.
\end{prop}
\begin{proof}
Let $\varphi(\tup{v})=\forall w\psi(w,\tup{v})$,
where $\psi(w,\tup{v})$ is a bounded formula.
Let $U=U_{\psi(w,\tup{v})}$ be the normal form name, as in Proposition~\ref{P:PR:normalForm}.
Then
$$T\Vdash \forall\tup{v}\forall w[\psi(\tup{v},w)\leftrightarrow U(\tup{v},w)=0].$$
We therefore let $\theta(T,\tup{v})$ be the statement saying that for all
$\tau\in T$ and $w$, if $U^\tau(\tup{v},w)$ is defined by stage
$n=|\tau|$ then $U^\tau(\tup{v},w)=0$.
\end{proof}

Note that it is necessary to assume that $\Forc$ is persistent,
and not just almost persistent, since it is vital that $T_\tau$
is a condition for every $\tau\in T$.
We now prove a version of Proposition~\ref{P:Pi1isPi1}
for almost persistent notions of forcing in the case when
$\Forc$ has an arithmetic definition.

\begin{prop}\label{P:Pi1isArith}
Let $\Forc$ be an almost persistent notion of forcing.
Suppose further that the conditions of $\Forc$ are defined
by an arithmetic formula $\alpha(T)$.
Let $\varphi(\tup{v})$ be a $\BPi^0_1$ formula of the forcing language.
There is an arithmetic formula $\theta(T,\tup{v})$ such that
for every condition $T$ such that $\varphi$ is $T$-local,
and every string of numbers $\tup{x}$, we have that
$$T\Vdash\varphi(\tup{x})\ \ \ \text{if and only if}\ \ \theta(T,\tup{x}).$$
Note that we are not making any claims about the complexity of
the statement `$\varphi(\tup{x})$ is $T$-local'.
\end{prop}
\begin{proof}
The proof is similar to that of Proposition~\ref{P:Pi1isPi1}.

Let $\varphi(\tup{v})=\forall w\psi(w,\tup{v})$,
where $\psi(w,\tup{v})$ is a bounded formula.
Let $U=U_{\psi(w,\tup{v})}$ be the normal form name, as in Proposition~\ref{P:PR:normalForm}.
Then
$$T\Vdash \forall\tup{v}\forall w[\psi(\tup{v},w)\leftrightarrow U(\tup{v},w)=0].$$
We therefore let $\theta(T,\tup{v})$ be the statement saying that for all
$\tau\in T$ and $w$, if $U^\tau(\tup{v},w)$ is defined by stage
$n=|\tau|$ and $\alpha(U^\tau)$ holds, then $U^\tau(\tup{v},w)=0$.
\end{proof}

We now begin to build some definition-dense machinery that will be key
in everything else that we do.
Let $\theta(\tup{v})$ be a formula of the forcing language and
let $W$ be a unary name.
We will construct a new formula $\theta_S(W;\tup{v})$ which
is essentially a $\BPi^0_1$ formula asserting that $W$ witnesses the truth of $\theta$
by coding the appropriate Skolem functions.
Dually, we will construct a formula $\theta_H(W;\tup{v})$ which
is essentially a $\BSigma^0_1$ formula asserting that $W$ does not witnesses
the the falsity of $\theta$.
We call $\theta_S$ the Skolemization\index{Skolemization} of $\theta$ and
we call $\theta_H$ the Herbrandization\index{Herbrandization} of $\theta$.

To avoid confusion between parameters and the variable used for the
unary names $W$, we use the $\lambda$-notation.
In other words, if $W$ is a $k+1$-ary name and $\tup{x}\in\NN^k$,
then $\lambda tW(\tup{x},t)$ denotes the unary name which is a function of $t$.

The definition of $\theta_S$\index{$\theta_S$} and $\theta_H$\index{$\theta_H$}
proceeds by induction on the complexity of $\theta$.

$$
\begin{array}{c|c|c}

\theta(\tup{v}) & \theta_S(W;\tup{v}) & \theta_H(W;\tup{v})\\
\hline
\text{atomic } & \theta(W;\tup{v}) & \theta(W;\tup{v})\\

\neg\phi(\tup{v}) & \neg\phi_H(W;\tup{v}) & \neg\phi_S(W;\tup{v})\\

\phi(\tup{v})\land\psi(\tup{v}) & \phi_S(\lambda tW(2t);\tup{v})\land\psi_S(\lambda tW(2t+1);\tup{v})
	&\phi_H(W;\tup{v})\land\psi_H(W;\tup{v})\\

\forall w\phi(\tup{v},w) & \forall w\phi_S(\lambda tW(\langle w,t\rangle);\tup{v},w)
	& \phi_H(\lambda tW(t+1);\tup{v},W(0))\\

\end{array}
$$

Here are some examples:
$$
\begin{array}{c|c|c}

\theta(\tup{v}) & \theta_S(W;\tup{v}) & \theta_H(W;\tup{v})\\
\hline
\forall v F(v)=0 & \forall v F(v)=0 & F(W(0))=0 \\
\forall v F(v)\neq 0 & \forall v F(v)\neq 0 & F(W(0))\neq 0 \\
\exists v F(v)=0 & \neg\neg F(W(0))=0 & \exists v F(v)=0 \\
\forall w\exists v F(w,v)=0 & \forall w \neg\neg F\big(w,W(\seq{w,0})\big)=0  &
	\exists v F\big(W(0),v\big)=0\\
\exists w\forall v F(v)=0 & \neg\neg\forall vF(W(0),v)=0 &
	\exists w F\big(w,W(\seq{w,0})\big)=0\\
\end{array}
$$

We now use Skolemizations and Herbrandizations with the forcing relation.
The intuitive idea behind Skolem and Herbrand forcing is that we would like $\theta(\tup{v})$
to be equivalent to $\exists W\theta_S(W;\tup{v})$ and $\forall W\theta_H(W;\tup{v})$.

\begin{definition}
Let $\theta$ be a $T$-local formula.
The \textit{Skolem forcing relation}\index{Skolem forcing relation} is defined by
$T\Vdash_S\theta$\index{$\Vdash_S$} if and only if $T\Vdash\theta_S(W)$ for some $T$-local unary name $W$.
The \textit{Herbrand forcing relation}\index{Herbrand forcing relation} is defined by
$T\Vdash_H\theta$\index{$\Vdash_H$} if and only if $T\Vdash\theta_H(W)$ for all $T$-local unary names $W$.
\end{definition}

Above we said that $\theta_S(W;\tup{v})$ was essentially a $\BPi^0_1$ formula
asserting that $W$ witnesses the truth of $\theta$.
We now make this precise.
We will assume that our notion of forcing is persistent
so that the names used in the construction of $\theta_S$ and $\theta_H$
are guaranteed to be local.

\begin{prop}[\RCAo]\label{P:skolemisPi1}
Let $\Forc$ be an almost persistent notion of forcing and $\theta(\tup{v})$
be a formula of the forcing language.
There is a bounded formula $\psi(W;u,\tup{v})$ of the forcing language such that
for all conditions $T$, names $W(\tup{v},t)$, and $\tup{x}$,
if $\theta(\tup{x})$ and $\lambda tW(\tup{x},t)$ are $T$-local, then
so is $\psi(\lambda tW(\tup{x},t);u,\tup{x})$ and
$$T\Vdash\theta_S(\lambda tW(\tup{x},t);\tup{x})\ \ \text{if and only if}\ \ T\Vdash\forall u\psi(\lambda tW(\tup{x},t);u,\tup{x})$$
\end{prop}
\begin{proof}
Notice that the only quantifiers that occur in $\theta_S$ are universal within the scope
of an even number of negations.
Let $u$ be a variable symbol that does not occur in $\theta_S$, and let
$\psi$ be the bounded formula obtained from $\theta_S(\lambda tW(\tup{x},t);\tup{x})$
by bounding all universal quantifiers with $u$.
Notice that if $\theta(\tup{x})$ and $\lambda tW(\tup{x},t)$ are $T$-local
then $\theta_S(\lambda tW(\tup{x},t);\tup{x})$ and $\psi(\lambda tW(\tup{x},t);u,\tup{x})$
are also $T$-local.  Moreover
$$T\Vdash\theta_S(\lambda tW(\tup{x},t);\tup{x})\ \ \text{if and only if}\ \ T\Vdash\forall u\psi(\lambda tW(\tup{x},t);u,\tup{x}).$$
\end{proof}

When the notion of forcing is persistent, we can say that $\theta_S(W;\tup{v})$
is essentially a $\BPi^0_1$ formula in an even stronger sense.

\begin{cor}[\RCAo]\label{C:skolemisActualPi1}
Let $\Forc$ be a persistent notion of forcing and $\theta(\tup{v})$
be a formula of the forcing language.
There is a $\BPi^0_1$ formula $\tilde{\theta}(T,W,\tup{v})$ such that
for all conditions $T$, names $W(\tup{v},t)$, and sequences of numbers $\tup{x}$,
if $\theta(\tup{x})$ and $\lambda tW(\tup{x},t)$ are $T$-local, then
$$T\Vdash\theta_S(\lambda tW(\tup{x},t);\tup{x})\ \ \text{if and only if}\ \ \tilde{\theta}(T,W,\tup{x}).$$
\end{cor}
\begin{proof}
Let $\psi(W;u,\tup{v})$ be as in the conclusion of Proposition~\ref{P:skolemisPi1}.
The corollary then follows by applying Proposition~\ref{P:Pi1isPi1} to $\forall u\psi(\lambda tW(\tup{x},t);u,\tup{v})$.
\end{proof}

Another corollary of Proposition~\ref{P:skolemisPi1}
is that the Skolem forcing relation is $\BSigma^1_2$ in certain cases.
This corollary will be helpful later on when
examining how much genericity is needed of our filters.

\begin{cor}[\RCAo]\label{C:skolemisPi11}
Let $\Forc$ be a notion of forcing which is either persistent
or is almost persistent and has an arithmetic definition.
Let $\theta(\tup{v})$ be a formula of the forcing language.
There is a $\BSigma^1_2$ formula $\tilde{\theta}(T,\tup{v})$ such that
for all conditions $T$ and sequences of numbers $\tup{x}$,
if $\theta(\tup{x})$ is $T$-local, then
$$T\Vdash_S\theta(\tup{x})\ \ \text{if and only if}\ \ \tilde{\theta}(T,\tup{x}).$$
\end{cor}
\begin{proof}
Let $\psi(W;u,\tup{v})$ be as in the conclusion of Proposition~\ref{P:skolemisPi1}.
In the case where $\Forc$ is persistent, let
$\hat{\psi}(T,W,\tup{v})$ be as in conclusion of Proposition~\ref{P:Pi1isPi1}
corresponding to $\forall u\psi(W;u,\tup{v})$.
In the case where $\Forc$ has an arithmetic definition, let
$\hat{\psi}(T,W,\tup{v})$ be as in conclusion of Proposition~\ref{P:Pi1isArith}
corresponding to $\forall u\psi(W;u,\tup{v})$.
Finally, let $\tilde{\theta}(T,\tup{v})$ be the statement saying that
there exists a Skolem name $W$ such that
$W$ is $T$-local and $\hat{\psi}(T,W,\tup{v})$ holds.
\end{proof}

\begin{prop}[\RCAo]\label{P:easySkolemEquiv}
Let $\Forc$ be an almost persistent notion of forcing,
let $T\in\Forc$, and let $\theta$ be a $T$-local sentence of the forcing language.
Then
$$T\Vdash_S\theta\ \Rightarrow\ T\Vdash\theta \Rightarrow\ T\Vdash_H\theta.$$
\end{prop}
\begin{proof}
We proceed by induction on the complexity of $\theta$.
The assumption that $\Forc$ is persistent ensures that
the names involved in the definition of $\theta_S$ and $\theta_H$ are $T$-local.

\begin{itemize}

\item $\theta$ is atomic:

Since $\theta_S(W)=\theta_H(W)=\theta$ the statement follows trivially.

\item $\theta=\neg\phi$:

If $T\Vdash_S\theta$ then $T\Vdash \neg\phi_H(W)$ for some unary name $W$.
Therefore there is no $T'\leq T$ such that $T'\Vdash\phi_H(W)$.
By the induction hypothesis, if $T'\Vdash\phi$ then $T'\Vdash\phi_H(W)$,
so we can conclude that there is no $T'\leq T$ such that
$T'\Vdash\phi$, and so $T\Vdash\theta$.

Now suppose that $T\Vdash\theta$.
Then there is no $T'\leq T$ such that $T'\Vdash\phi$.
By the induction hypothesis, if there is a unary name $W$ such that
$T'\Vdash\phi_S(W)$, then $T'\Vdash\theta$.
Therefore there is no $T'\leq T$ and no $W$ such that $T'\Vdash\phi_S(W)$.
Thus $T\Vdash_H\theta$.

\item Suppose that $\theta=\phi\land\psi$.

Assume that $T\Vdash \phi_S(\lambda tW(2t);\tup{v})\land\psi_S(\lambda tW(2t+1);\tup{v})$
for some unary name $W$.
Breaking this down, $T\Vdash\phi_S(\lambda tW(2t);\tup{v})$ and $T\Vdash\psi_S(\lambda tW(2t+1);\tup{v})$.
By the induction hypothesis $T\Vdash\phi$ and $T\Vdash\psi$, so $T\Vdash\phi\land\psi$.

Now suppose that $T\Vdash\theta$.
Then $T\Vdash\phi$ and $T\Vdash\psi$, so by the induction hypothesis
$T\Vdash\phi_H(W)$ and $T\Vdash\psi_H(W)$ for all unary names $W$.
Therefore $T\Vdash\phi_H(W)\land\psi_H(W)$ for all unary names $W$,
and so $T\Vdash_H\theta$.

\item Suppose that $\theta=\forall w\phi(w)$.

Assume that $T\Vdash \forall w\phi_S(\lambda tW(\langle w,t\rangle);\tup{v},w)$
for some unary name $W$.
Breaking this down, $T\Vdash\phi_S(\lambda tW(\langle x,t\rangle);\tup{v},w)$ for all $x$.
By the induction hypothesis $T\Vdash\phi(x)$ for all $x$.
Thus $T\Vdash\theta$.

Now suppose that $T\Vdash\theta$.
Then $T\Vdash\phi(x)$ for all $x$.
By the induction hypothesis $T\Vdash\phi_H(W,x)$ for all $x$ and all unary names $W$.
Thus $T\Vdash\theta_H(\lambda tW(t+1),W(0))$,
and so $T\Vdash_H\theta$.
\end{itemize}
\end{proof}

\begin{prop}[\RCAo]\label{P:witnessBnded}
Let $\Forc$ be an almost persistent notion of forcing.
Let $\theta(\tup{v})$ be a bounded formula of the forcing language.
There are names $W^\theta_S(\tup{v},t)$ and $W^\theta_H(\tup{v},t)$
such that for all $T\in\Forc$ and all $\tup{x}$, if $\theta(\tup{x})$ is $T$-local,
then then so are $\lambda tW^\theta_S(\tup{x},t)$ and $\lambda tW^\theta_H(\tup{x},t)$, and
\begin{align*}
T\Vdash\theta_S(\lambda tW_S^\theta(\tup{x},t);\tup{x})
	\Leftrightarrow  T\Vdash\theta(\tup{x})
	\Leftrightarrow  T\Vdash\theta_H(\lambda tW_H^\theta(\tup{x},t);\tup{x}).
\end{align*}
\end{prop}

\begin{proof}
Notice that the left to right implications follow
from Proposition~\ref{P:easySkolemEquiv}.
For if $T\Vdash\theta_S(\lambda tW_S^\theta(\tup{x},t);\tup{x})$,
then $T\Vdash_S\theta(\tup{x})$ and so $T\Vdash\theta(\tup{x}$).
And if $T\Vdash\theta(\tup{x})$, then $T\Vdash_H\theta(\tup{x})$
and so $T\Vdash\theta_H(\lambda tW_H^\theta(\tup{x},t);\tup{x})$.

We proceed by induction on the complexity of $\theta$
to define $W^\theta_S$ and $W^\theta_H$ and show that
$$T\Vdash\theta_H(\lambda tW_H^\theta(\tup{x},t);\tup{x})
	\Rightarrow   T\Vdash\theta(\tup{x})
	\Rightarrow  T\Vdash\theta_S(\lambda tW_S^\theta(\tup{x},t);\tup{x}).$$
The construction of $W^\theta_S$ and $W^\theta_H$ will employ only
obviously effective methods together with normal form names as in Definition~\ref{D:admitsNormal}.
Therefore the assumption that $\Forc$ is almost persistent guarantees
that $W^\theta_S$ and $W^\theta_H$ are $T$-local.

\begin{itemize}
\item $\theta(\tup{v})$ is atomic:

Let $W^\theta_S(\tup{v},t)=W^\theta_H(\tup{v},t)=0$.

The statement of the proposition follows trivially
since $\theta_S(W;\tup{x})=\theta_H(W;\tup{x})=\theta(\tup{x})$.

\item $\theta(\tup{v})=\neg\phi(\tup{v})$:

Let $W^\theta_S(\tup{v},t)=W^\phi_H(\tup{v},t)$ and
$W^\theta_H(\tup{v},t)=W^\phi_S(\tup{v},t)$.

Suppose that $T\Vdash\theta_H(\lambda tW_H^\theta(\tup{x},t);\tup{x})$.
Then there is no $T'\leq T$ such that
$T'\Vdash\phi_S(\lambda tW_H^\theta(\tup{x},t);\tup{x})$.
Thus there is no $T'\leq T$ such that
$T'\Vdash\phi_S(\lambda tW_S^\phi(\tup{x},t);\tup{x})$.
By the induction hypothesis, if $T'\Vdash\phi(\tup{x})$, then
$T'\Vdash\phi_S(\lambda tW_S^\phi(\tup{x},t);\tup{x})$.
Therefore we conclude that $T\Vdash\theta(\tup{x})$.

Suppose that $T\Vdash\theta(\tup{x})$.
Therefore there is no $T'\leq T$ such that $T'\Vdash\phi(\tup{x})$.
By the induction hypothesis, if
$T'\Vdash\phi_H(\lambda tW_H^\phi(\tup{x},t);\tup{x})$
then $T'\Vdash\phi(\tup{x})$.
Therefore $T\Vdash\neg\phi_H(\lambda tW_H^\phi(\tup{x},t);\tup{x})$,
and since $W^\phi_H=W^\theta_S$, $T\Vdash\theta_S(\lambda tW_S^\theta(\tup{x},t);\tup{x})$.

\item Suppose that $\theta(\tup{v})=\phi(\tup{v})\land\psi(\tup{v})$.

Let $W^\theta_S(\tup{v},2t)=W^\phi_S(\tup{v},t)$ and
$W^\theta_S(\tup{v},2t+1)=W^\psi_S(\tup{v},t)$.
Let $W^\theta_H(\tup{v},t)=W^\phi_H(\tup{v},t)$ if $U_\phi(\tup{v})=y$ for some $y\neq 0$,
and let $W^\theta_H(\tup{v},t)=W^\psi_H(\tup{v},t)$ if $U_\phi(\tup{v})=0$
(where $U_\phi$ is the normal form name, as in Proposition~\ref{P:PR:normalForm}).

Suppose that $T\Vdash\theta_H(\lambda tW_H^\theta(\tup{x},t);\tup{x})$.
Therefore $T\Vdash\phi_H(\lambda tW_H^\theta(\tup{x},t);\tup{x})$ and
$T\Vdash\psi_H(\lambda tW_H^\theta(\tup{x},t);\tup{x})$.
There are two cases.
In the first case we assume that $T\Vdash\phi(\tup{x})$.
In this case $T\Vdash U_\phi(\tup{x})=0$, so
$W^\theta_H(\tup{x},t)=W^\psi_H(\tup{x},t)$.
Therefore $T\Vdash\psi_H(\lambda tW_H^\psi(\tup{x},t);\tup{x})$,
and so by the induction hypothesis $T\Vdash\psi(\tup{x})$.
Thus $T\Vdash\theta(\tup{x})$.
In the other case we assume that $T\not\Vdash\phi(\tup{x})$.
Therefore there is some $T'\leq T$ such that $T'\Vdash\neg\phi(\tup{x})$,
and so $T'\Vdash U_\phi(\tup{x})\neq 0$.
Then $W^\theta_H(\tup{x},t)=W^\phi_H(\tup{x},t)$,
and $T'\Vdash\phi_H(\lambda tW_H^\phi(\tup{x},t);\tup{x})$.
By the induction hypothesis, $T'\Vdash\phi(\tup{x})$, a contradiction.

Now suppose that $T\Vdash\theta(\tup{x})$.
Therefore $T\Vdash\phi(\tup{x})$ and $T\Vdash\psi(\tup{x})$.
By the induction hypothesis, $T\Vdash\phi_S(\lambda tW_S^\phi(\tup{x},t);\tup{x})$
and $T\Vdash\psi_S(\lambda tW_S^\psi(\tup{x},t);\tup{x})$.
It follows that $T\Vdash\theta_S(\lambda tW_S^\theta(\tup{x},t);\tup{x})$.

\item Suppose that $\theta(\tup{v})=\forall w\leq F(\tup{v})\phi(\tup{v},w)$.

Let $W^\theta_S(\tup{v},t)$ be $W^\phi_S(\tup{v},\first{t},\lfloor(\second{t})/2\rfloor)$
if $\first{t}\leq F(\tup{v})$, and $0$ otherwise.
Let $W^\theta_H(\tup{v},0)=\sum_{w\leq F(\tup{v})}X(\tup{v},w)$,
where $X(\tup{v},w)=1\dotminus \sum_{u\leq w}U_\phi(\tup{v},u)$.
In other words, $W^\theta_H(\tup{v},0)$ is the first $w\leq F(\tup{v})$
such that $\phi(\tup{v},w)$ fails, if such a $w$ exists.
Otherwise $W^\theta_H(\tup{v},0)=F(\tup{v})+1$.
We then define $W^\theta_H(\tup{v},t)$ inductively
by letting $W^\theta_H(\tup{v},t+1)$ be $W^\phi_H(\tup{v},W^\theta_H(\tup{v},0),t)$
if $W^\theta_H(\tup{v},0)\leq F(\tup{v})$, and $0$ otherwise.

Suppose that
$$T\Vdash\theta_H(\lambda tW_H^\theta(\tup{x},t);\tup{x}).$$
Removing the shorthand,
$$\theta(\tup{v})=\forall w\neg[w\leq F(\tup{v})\land\neg\phi(\tup{v},w)].$$
Let $\psi(\tup{v},w)=\neg[w\leq F(\tup{v})\land\neg\phi(\tup{v},w)]$
so that $\theta(\tup{v})=\forall w\psi(\tup{v},w)$.
Then we have that $\theta_H(W;\tup{v})=\psi_H(\lambda tW(t+1);\tup{v},W(\tup{v},0))$
and so
$$\theta_H(\lambda tW_H^\theta(\tup{x},t);\tup{x})
	=\psi_H(\lambda tW_H^\psi(\tup{x},W^\theta_H(\tup{x},0),t);\tup{x},W^\theta_H(\tup{x},0)).$$
By the induction hypothesis $T\Vdash\psi(\tup{x},W^\theta_H(\tup{x},0))$.
Therefore there is no $T'\leq T$ such that
$T'\Vdash W^\theta_H(\tup{x},0)\leq F(\tup{x})$ and $T'\Vdash\neg\phi(\tup{x},W^\theta_H(\tup{x},0))$.
In other words, for all $T'\leq T$, if $T'\Vdash W^\theta_H(\tup{x},0)\leq F(\tup{x})$
then there is a $T''\leq T'$ such that $T''\Vdash\phi(\tup{x},W^\theta_H(\tup{x},0))$.
Therefore no such $T'$ exists, since otherwise we would have a condition $T''\leq T$
such that $T'\Vdash W^\theta_H(\tup{x},0)\leq F(\tup{x})$ and
$T''\Vdash\phi(\tup{x},W^\theta_H(\tup{x},0))$, contradicting how $W^\theta_H(\tup{x},0)$ was defined.
Therefore there is no $T'\leq T$ such that $T'\Vdash W^\theta_H(\tup{x},0)\leq F(\tup{x})$.
Notice that if there was a $w$ and a $T'\leq T$ such that
$T'\Vdash[w\leq F(\tup{x})\land\neg\phi(\tup{x},w)]$, then $T'\Vdash W^\theta_H(\tup{x},0)\leq F(\tup{x})$.
Therefore for all $w$ there is no $T'\leq T$ such that $T'\Vdash[w\leq F(\tup{x})\land\neg\phi(\tup{x},w)]$.
We conclude that $T\Vdash\theta(\tup{x})$.

Now suppose that $T\Vdash\theta(\tup{x})$.
Again we write $\theta(\tup{v})=\forall w\psi(\tup{v},w)=\neg[w\leq F(\tup{v})\land\neg\phi(\tup{v},w)]$.
In order to show that $T\Vdash\theta_S(W^\theta_S(\tup{x},t);\tup{x})$,
it suffices to show the following: for all $w$ and all $T'\leq T$,
if $T'\Vdash w\leq F(\tup{x})$ then there is a $T''\leq T'$ such that
$T''\Vdash\phi_S(\lambda tW^\theta_S(\tup{x},\langle w,t\rangle),\tup{x},w)$.
(This is just a matter of definition unpacking, albeit a good bit of it.)
Therefore we fix $w$ and $T'\leq T$ and find the appropriate $T''\leq T'$.
By assumption, if $T'\Vdash w\leq F(\tup{x})$,
then there is a $T''\leq T$ such that $T''\Vdash\phi(\tup{x},w)$.
By the induction hypothesis, $T''\Vdash\phi_S(\lambda tW^\phi_S(\tup{x},w,t);\tup{x},w)$.
It therefore remains to show that
$T''\Vdash\lambda tW^\phi_S(\tup{x},w,t)=\lambda tW^\theta_S(\tup{x},\langle w,2t+1\rangle)$
whenever $T''\Vdash w\leq F(\tup{x})$.
If $w\leq F(\tup{x})$ then
$$W^\theta_S(\tup{x},\langle w,2t+1\rangle)
	=W^\phi_S(\tup{x},w,\lfloor(2t+1)/2\rfloor)
	=W^\phi_S(\tup{x},w,t).$$
\end{itemize}
\end{proof}

We now prove the existence of $\BPi^0_1$ Skolem names.

\begin{prop}
Let $\Forc$ be an almost persistent notion of forcing.
Let $\theta(\tup{v})$ be a $\BPi^0_1$ formula of the forcing language.
There is a name $W^\theta_S(\tup{v},t)$
such that for all $T\in\Forc$ and all $\tup{x}$, if $\theta(\tup{x})$ is $T$-local,
then so is $W^\theta_S(\tup{v},t)$, and
\begin{align*}
T\Vdash\theta(\tup{x})
	\Leftrightarrow  T\Vdash\theta_S(\lambda tW_S^\theta(\tup{x},t);\tup{x}).
\end{align*}
\end{prop}
\begin{proof}
Note that the backwards direction follows from Proposition~\ref{P:easySkolemEquiv}.

Let $\theta(\tup{v})=\forall w\phi(\tup{v},w)$ where $\phi(\tup{v},w)$ is bounded.
We define $W^\theta_S(\tup{v},t)$ by $W^\theta_S(\tup{v},t)=W^\phi_S(\tup{v},\first{t},\second{t})$,
where $W^\phi_S$ is defined as in Proposition~\ref{P:witnessBnded}.

Assuming that $T\Vdash\theta(\tup{x})$, we show that $T\Vdash\theta_S(W^\theta_S(\tup{x},t);\tup{x})$.
In other words, we must show that $T\Vdash\phi_S(W^\theta_S(\tup{x},\langle w,t\rangle);\tup{x},w)$
for all $w$.
By assumption $T\Vdash\phi(\tup{x},w)$ for all $w$, so by
Proposition~\ref{P:witnessBnded} $T\Vdash\phi_S(\lambda tW^\phi_S(\tup{x},w,t);\tup{x},w)$ for all $w$.
Therefore the proposition follows from the fact that
$W^\theta_S(\tup{x},\langle w,t\rangle)=W^\phi_S(\tup{x},w,t)$.
\end{proof}

We now prove the existence of $\BSigma^0_1$ Skolem names.

\begin{prop}[\RCAo]\label{P:witnessSigma1}
Let $\Forc$ be an almost persistent notion of forcing.
Let $\theta(\tup{v})$ be a $\BSigma^0_1$ formula of the forcing language.
There is a name $W^\theta_S(\tup{v},t)$
such that for all $T\in\Forc$ and all $\tup{x}$, if $\theta(\tup{x})$ is $T$-local
and $T\Vdash\theta(\tup{x})$, then $W^\theta_S(\tup{v},t)$ is also $T$-local
and there is a $T'\leq T$ such that $T'\Vdash\theta_S(\lambda tW_S^\theta(\tup{x},t);\tup{x})$.

If $\Forc$ is persistent, then we may assume that $T'=T$.
\end{prop}

\begin{proof}
Let $\theta(\tup{v})=\exists w\varphi(\tup{v},w)$ where $\varphi(\tup{v},w)$ is bounded.
Define a name $X(\tup{v},w)=1\dotminus\sum_{u\leq w}U_{\neg\varphi}(\tup{v},u)$,
where $U_{\neg\varphi}$ is the normal form name, as in Proposition~\ref{P:PR:normalForm}.
Note that $X(\tup{v},w)$ can only switch from 1 to 0 once as $w$ increases.
We now let $W^\theta_S(\tup{v},0)=y$ if $y=\sum_{w\leq y}X(\tup{v},w)$.
In other words $W^\theta_S(\tup{v},0)$ is the least $w$ such that
$\varphi(\tup{v},w)$ holds, if such any such $w$ exists
(and is undefined otherwise).
Finally, we let $W^\theta_S(\tup{v},t+1)=W^\varphi_S(\tup{v},W^\theta_S(\tup{v},0),t)$.

Now we show that $W^\theta_S(\tup{x},t)$ is $T$-local.
Let $T'\leq T$.
There is a condition $S\leq T'$ and a $w$ such that $S\Vdash\varphi(\tup{x},w)$.
Since $\Forc$ is persistent, there is a $\sigma\in S$
such that $S_\sigma\in\Forc$ and
$(1\dotminus U_{\neg\varphi})^\sigma(\tup{x},w)=0$.
Therefore $W^\theta_S(\tup{x},0)$ is defined on everything in $[S_{\sigma}]$.
Since locality is closed under primitive recursion (Proposition~\ref{P:comp&pr}),
$W^\theta_S(\tup{x},t)$ is $S_\sigma$-local, so
$[S_\sigma]\cap\dom(W^\theta_S(\tup{x},t))\neq\emptyset$.
Therefore $W^\theta_S(\tup{x},t)$ is $T$-local.

Now assume that $T\Vdash\theta(\tup{x})$ and show that
$T\Vdash\theta_S(W^\theta_S(\tup{x},t);\tup{x})$.
Let $W_0=W^\theta_S(\tup{x},0)$.
Notice that
$$\begin{array}{lll}
\theta_S(W^\theta_S(\tup{x},t);\tup{x}) & = & \neg\neg\varphi_S(W^\theta_S(\tup{x},t+1);\tup{x},W_S^\theta(\tup{x},0))\\
& = & \neg\neg\varphi_S(W^\varphi_S(\tup{x},W_0,t);\tup{x},W_0).
\end{array}$$
By assumption, for all $T'\leq T$ there is a $w$ and a $T''\leq T'$
such that $T''\Vdash\varphi(\tup{x},w)$.
Therefore $T''\Vdash U_{\neg\varphi}(\tup{x},w)=y$ for some $y>0$,
and so $W^\theta_S(\tup{x},0)$ is defined on everything in $[T'']\cap\dom(U_\varphi)$.
Notice that $T''\Vdash\varphi(\tup{x},W_0)$.
By Proposition~\ref{P:witnessBnded} we have that
$T''\Vdash\varphi_S(W^\varphi_S(\tup{x},W_0,t);\tup{x},W_0)$.
This implies that $T''\Vdash\neg\neg\varphi_S(W^\varphi_S(\tup{x},W_0,t);\tup{x},W_0)$.
Note that $W^\theta_S(\tup{x},t+1)=W^\varphi_S(\tup{x},W_0,t)$.
Therefore we have shown that for all $T'\leq T$ there is a $T''\leq T$
such that $T''\Vdash\theta_S(W^\theta_S(\tup{x},t);\tup{x})$.
In other words, $T\Vdash\neg\neg\theta_S(W^\theta_S(\tup{x},t);\tup{x})$.
If $\Forc$ is persistent, then by Proposition~\ref{P:2neg} (the double negation rule),
$T\Vdash\theta_S(W^\theta_S(\tup{x},t);\tup{x})$.
Otherwise, by unraveling the definition of negation, there is a $T'\leq T$
such that $T'\Vdash\theta_S(W^\theta_S(\tup{x},t);\tup{x})$.
\end{proof}

We now improve Proposition~\ref{P:witnessSigma1}
by extending it to $\BPi^0_2$ formulas.

\begin{prop}\label{P:witnessPi2}
Let $\Forc$ be an almost persistent notion of forcing.
Let $\theta(\tup{v})$ be a $\BPi^0_2$ formula of the forcing language.
There is a name $W^\theta_S(\tup{v},t)$
such that for all $T\in\Forc$ and all $\tup{x}$, if $\theta(\tup{x})$ is $T$-local
and $T\Vdash\theta(\tup{x})$, then $W^\theta_S(\tup{v},t)$ is also $T$-local
and there is a $T'\leq T$ such that
$T'\Vdash\theta_S(\lambda tW_S^\theta(\tup{x},t);\tup{x})$.

If $\Forc$ is persistent, then we may assume that $T'=T$.
\end{prop}
\begin{proof}
Let $\theta(\tup{v})=\forall w\varphi(\tup{v},w)$ where $\varphi(\tup{v},w)$ is $\BSigma^0_1$.
We define $W^\theta_S(\tup{x},t)$ by
$W^\theta_S(\tup{x},t)=W^\varphi_S(\tup{v},\first{t},\second{t})$,
where $W^\varphi_S$ is defined as in Proposition~\ref{P:witnessSigma1}.
The proposition then holds for the $T'\leq T$ given by Proposition~\ref{P:witnessSigma1}.
\end{proof}

\begin{cor}\label{C:Pi1unif}
Let $\Forc$ be an almost persistent notion of forcing.
Let $\theta(\tup{v},w)$ be a $\BSigma^0_1$ formula of the forcing language
such that $T\Vdash\forall\tup{v}\exists w\theta(\tup{v},w)$.
There is a $T$-local name $W(\tup{v})$
and a $T'\leq T$ such that
$T'\Vdash\forall\tup{v}\theta(\tup{v},W(\tup{v}))$.

If $\Forc$ is persistent, then we may assume that $T'=T$.
\end{cor}

\section{Coloring Conditions \& Witnessing in \texorpdfstring{\ACAo}{ACA0}}\label{witACAo}
%\section{Coloring Conditions \& Witnessing in \ACAo}\label{witACAo}
	In this section we examine persistent notions of forcing which satisfy
a property we call \MCP.
The main result of this section is Proposition~\ref{P:PacaSigma2SkolemNames}.
Though somewhat technical, Proposition~\ref{P:PacaSigma2SkolemNames} is the key
to showing that if $\MM$ is a model of $\ACAo$ and $\Forc$ is a
persistent notion of forcing satisfying \MCP,
then the generic extension of $\MM$ is also a model of $\ACAo$.

Before we can define \MCP\ we must define what a set of layers for a condition $T$ is.
Two important examples to keep in mind are the levels of a tree
and the splitting levels of a tree.

\begin{definition}
Given a condition $T$, a \textit{set of layers} for $T$ is a
sequence $\seq{X_i:i\in\NN}$ of subsets of $T$ such that
\begin{itemize}
\item for each extension $T'\leq T$ and $i\in\NN$, $X_i\cap T'$ is a maximal antichain of $T'$,
\item if $\tau\in X_{i+1}$ then $\tau$ is a proper extension of some $\sigma\in X_i$.
\end{itemize}
Given a set of layers for $T$ and a $\tau\in T$,
we say that $\tau$ is above layer $n$ if $\tau\supseteq\sigma$
for some $\sigma\in X_n$.
\end{definition}

\begin{definition}
Given a condition $T$, we say that a coloring $c:T\to\{0,1\}$
is \textit{monotone} if $c(\tau)=1$ implies that $c(\sigma)=1$ for
all $\sigma\supseteq\tau$.
\end{definition}

\begin{definition}
\MCP\ (for Monotone coloring principle\index{Monotone coloring principle}\index{$\MCP$})
is the following statement:
For any condition $T$ and any infinite sequence $\langle c_i:i\in\NN\rangle$ of monotone
colorings $c_i:T\rightarrow \{0,1\}$,
there is a condition $T'\leq T$ and a set of layers for $T'$ such that
for every $i$ there is a $k$ such that if $\tau\in T'$ is above layer $k$
then $c_i$ is constant on $T'_\tau$.
\end{definition}

We now prove the existence of $\BSigma^0_2$ Skolem names.

\begin{prop}[\ACAo]\label{P:PacaSigma2SkolemNames}
Let $\Forc$ be a persistent notion of forcing which satisfies \MCP.
Then $\Forc$ has $\BSigma^0_2$ Skolem names.
In other words, for every $T$-local $\BSigma^0_2$ formula $\theta(\tup{v})$ of the
forcing relation there is an extension $T'\leq T$ and a name $W(\tup{v},t)$
such that for all extensions $S\leq T'$ and all $\tup{x}$,
if $S\Vdash\theta(\tup{x})$ then $\lambda tW(\tup{x},t)$
is $S$-local and $S\Vdash\theta_S(\lambda tW(\tup{x},t);\tup{x})$.
\end{prop}

\begin{proof}
Let $\theta(\tup{v})=\exists y\forall z\phi(\tup{v},y,z)$,
where $\phi$ is bounded.
Let $U=U_\phi$ be the normal form name from Proposition~\ref{P:PR:normalForm}.
Consider the sequence of monotone colorings
$c_{\seq{\tup{x},y}}:T\to\{0,1\}$ defined by
$$c_{\seq{\tup{x},y}}(\tau)=\begin{cases}
0&\text{if }(\forall z\leq|\tau|)U(\tup{x},y,z)\not>0,\\
1&\text{otherwise.}
\end{cases}$$
By \MCP\ there is a condition $T'\leq T$ and a set of layers
$X_0,X_1,\ldots$ for $T'$ such that
for every $\seq{\tup{x},y}$ there is a $k$ such that if $\tau\in T'$ is above layer $k$
then $c_{\seq{\tup{x},y}}$ is constant on $T'_\tau$.

We now define $W(\tup{x},t)$ in stages.
At stage $n=\langle \tup{x},y\rangle$ we consider the nodes in $X_n$.
If $W^\tau(\tup{x},0)$ is undefined for $\tau\in X_n$,
and $c_n(\sigma)=0$ for all $\sigma\in T'_\tau$,
then we let $W^\sigma(\tup{x},0)=y$ for every $\sigma\in T'_\tau$,
and we let $W(\tup{x},t+1)=W^\psi_S(\tup{x},W(\tup{x},0),t)$,
where $\psi(\tup{x},w)=\forall a\phi(\tup{x},w,a)$ and
$W^\psi_S$ is defined as in Proposition~\ref{P:witnessPi2}.

We now show that if $S\leq T'$ and $S\Vdash\theta(\tup{x})$,
then $\lambda tW(\tup{x},t)$ is $S$-local.
Since $\Forc$ is persistent, it suffices to show that if $S'\leq S$
then there is a $\sigma\in S'$ such that $W^\sigma(\tup{x},0)$ is defined.
Suppose that no such $\sigma$ exists.
For each $i\in\NN$ let $Y_i=X_i\cap S$.
Since $\seq{X_i:i\in\NN}$ is a set of layers for $T'$,
then $\seq{Y_i:i\in\NN}$ is a set of layers for $S$.
Therefore for every layer $n$ such that $n=\seq{\tup{x},y}$
for some $y$, we have that $c_n(\tau)=1$ for all $\tau\in Y_n$.
Then $S'\Vdash\forall y\exists z(1\dotminus U(\tup{x},y,z)=0)$,
contradicting that $S'\Vdash\theta(\tup{x})$.

Finally, we show that if $S\leq T'$ and $S\Vdash\theta(\tup{x})$,
then $S\Vdash\theta_S(\lambda tW(\tup{x},t);\tup{x})$.
We know that there is a $S'\leq S$ and a $y$ such that
$S'\Vdash\psi(\tup{x},y)$.
Therefore $W^\sigma(\tup{x},0)$ is defined for all $\sigma\in S'$
above layer $\seq{\tup{x},y}$.
Moreover $S'\Vdash\psi(\tup{x},W(\tup{x},0))$.
So by Proposition~\ref{P:witnessPi2},
$S'\Vdash\psi_S(\lambda tW^\psi_S(\tup{x},W(\tup{x},0),t);\tup{x},W(\tup{x},0))$.
Unraveling definitions, we see that $S\Vdash\theta_S(\lambda tW^\theta_S(\tup{x},t);\tup{x})$.
\end{proof}

We now improve Proposition~\ref{P:PacaSigma2SkolemNames}
by extending it to $\BPi^0_3$ formulas.

\begin{prop}[\ACAo]\label{P:PacaPi3SkolemNames}
Let $\Forc$ be a persistent notion of forcing which satisfies \MCP.
Then $\Forc$ has $\BPi^0_3$ Skolem names.
In other words, for every $T$-local $\BPi^0_3$ formula $\theta(v)$ of the
forcing relation there is an extension $T'\leq T$ and a name $W(v,t)$
such that for all extensions $S\leq T'$ and all $x$,
if $S\Vdash\theta(x)$ then $\lambda tW(x,t)$
is $S$-local and $S\Vdash\theta_S(\lambda tW(x,t);x)$.
\end{prop}
\begin{proof}
Let $\theta(\tup{v})=\forall w\phi(\tup{v},w)$ where $\phi(\tup{v},w)$ is $\BSigma^0_2$.
We define $W^\theta_S(\tup{x},t)$ by
$W^\theta_S(\tup{x},t)=W^\phi_S(\tup{v},\first{t},\second{t})$,
where $W^\phi_S$ is defined as in Proposition~\ref{P:PacaSigma2SkolemNames}.
\end{proof}

\begin{cor}\label{C:Pi2unif}
Let $\Forc$ be a persistent notion of forcing which satisfies \MCP.
Let $\theta(\tup{v},w)$ be a $\BSigma^0_2$ formula of the forcing relation
such that $T\Vdash\forall\tup{v}\exists w\theta(\tup{v},w)$.
There is a $T$-local name $W(\tup{v})$ such that
$T\Vdash\forall\tup{v}\theta(\tup{v},W(\tup{v}))$.
\end{cor}

\section{The Generic Extension}\label{generic}
	In this section we put the main results of the last two sections to use.
Theorem~\ref{T:presRCA} below says that for almost persistent notions of forcing,
the generic extension of a model of \RCAo\ is itself a model of \RCAo.
Similarly, Theorem~\ref{T:presACA} says that for persistent notions of forcing satisfying \MCP,
the generic extension of a model of \ACAo\ is itself a model of \ACAo.

The key to proving these two theorems is to prove special cases of the Truth Lemma.
Taken together, Proposition~\ref{P:witnessPi2} above and Proposition~\ref{P:presRCAkey} below
tell us that the Truth Lemma holds for $\BPi^0_2$ sentences.
In other words, and stated imprecisely, the $\BPi^0_2$ sentences which hold in
the generic extension are precisely those which are forced.
Similarly, Proposition~\ref{P:PacaPi3SkolemNames} above and Proposition~\ref{P:presACAkey} below
tell us that the Truth Lemma holds for $\BPi^0_3$ sentences.

Let $\MM$ be a model of second-order arithmetic and let $\Forc$ be a notion of forcing.
We say that a collection $\mathcal{D}\subseteq\Forc$ is \textit{dense}\index{dense} if
for every condition $T$ there is an extension $T'\leq T$ such that $T'\in\mathcal{D}$.
Note that $\mathcal{D}$, just like $\Forc$, is a third-order object and
therefore does not belong to $\MM$.
(The elements of $\mathcal{D}$ do, however, belong to $\MM$.)
We say that $\mathcal{D}$ is \textit{open}\index{open} if $T'\in\mathcal{D}$
whenever $T'\leq T\in\mathcal{D}$.
A collection $\mathcal{G}\subseteq\Forc$ is called a \textit{generic filter}\index{generic filter} for $\Forc$ if
\begin{itemize}
\item $\mathcal{G}$ is nonempty,
\item $\mathcal{G}$ is closed upward, meaning that if
		$T,S\in\Forc$, $T\leq S$, and $T\in\mathcal{G}$, then $S\in\mathcal{G}$,
\item any two elements of $\mathcal{G}$ are compatible in $\mathcal{G}$, meaning that if $T,T'\in\mathcal{G}$
		then there is an $S\in\mathcal{G}$ such that $S\leq T$ and $S\leq T'$,
\item and $\mathcal{G}$ meets every open dense collection, meaning that if $\mathcal{D}$ is open dense
		then $\mathcal{G}\cap\mathcal{D}\neq\emptyset$.
\end{itemize}
The first three requirements say that $\mathcal{G}$ is a \textit{filter}.
The last requirement says that $\mathcal{G}$ is \textit{generic}.
We will never actually use full genericity since only
some open dense collections need to be met.
Given a collection of formulas $\Gamma$,
we say that a filter $\mathcal{G}$ is \textit{$\Gamma$-generic}
if it meets every open dense collection which is defined by a formula in $\Gamma$.
In other words, $\mathcal{G}$ meets every open dense collection
$\mathcal{D}$ such that $T\in\mathcal{D}$ if and only if $T\in\Forc$ and $\phi(T)$
holds for some $\phi\in\Gamma$.

\begin{definition}
Let $\MM$ be a model and $\Forc$ be a notion of forcing.
Suppose that $\mathcal{G}$ is a generic filter for $\Forc$.
Let
$$G=\bigcup\Big\{\mathsf{stem}(T):T\in\mathcal{G}\Big\}.$$
We say that $G$ is a \textit{generic real}\index{generic real} for $\Forc$ if
$$T\in\mathcal{G}\ \ \Leftrightarrow\ \ G\in[T].$$
Note that in contrast with Definition~\ref{D:treestuff},
since we are now working outside of a model $\MM$,
we use $[T]$ to denote the set of all branches of $[T]$,
not just those branches in $\MM$.

If $G$ is a generic real for $\Forc$,
we say that a name $F$ is $G$\textit{-local}\index{$G$-local}\index{locality} if $F$ is $T$-local
for some $T\in\mathcal{G}$.
We say that a formula $\theta$ of the forcing language is $G$\textit{-local}
if every name occurring in $F$ is $G$-local.
\end{definition}

Later, in Section~\ref{forcingExamples} (the section on examples),
the following lemma will be useful for proving the existence of a generic real.

\begin{lem}\label{L:Gen:realForc}
Let $\Forc$ be an almost persistent notion of forcing such that
$$\mathcal{D}_T=\{S:S\cap T\text{ is finite }\lor\ S\leq T\}$$
is open dense for every $T\in\Forc$.
Let $\mathcal{G}$ be a $\BSigma^0_2$-generic filter for $\Forc$.
Then $G$ is a generic real for $\Forc$.
\end{lem}

\begin{proof}
For every $n\in\NN$, since $\Forc$ is almost persistent,
$$\mathcal{D}_n=\{S:|\mathsf{stem}(S)|\geq n\}$$
is open dense.

Let $T\in\mathcal{G}$.
For each $n$ there is a $S_n'\in\mathcal{G}\cap\mathcal{D}_n$.
Since $\mathcal{G}$ is a filter, there is a condition $S_n\leq S_n',T$.
Therefore $G\in[T]$.

Suppose now that $G\in[T]$.
Since $\mathcal{D}_T$ is open dense and $\BSigma^0_2$-definable,
there is a condition $S\in\mathcal{G}$ such that
either $S\cap T$ is finite or $S\leq T$.
But if $S\cap T$ is finite then $G\in[S]\setminus[T]$, a contradiction.
Therefore $S\leq T$ and $S\in\mathcal{G}$, so $T\in\mathcal{G}$ since
$\mathcal{G}$ is a filter.
\end{proof}

\begin{prop}\label{P:genericGivesTotFncs}
Let $\Forc$ be an almost persistent forcing.
Let $\mathcal{G}$ be a $\BSigma^0_1$-generic filter and
suppose that $G$ is a generic real for $\Forc$ corresponding to $\mathcal{G}$.
If $F$ is a $G$-local name then
$$F^G(\tup{x})=y\ \Leftrightarrow\ \exists nF^{G\uhr n}(\tup{x})=y$$
defines a total $k$-ary function.
\end{prop}
\begin{proof}
Let $F$ be $T$-local for some $T\in\mathcal{G}$.
Fix $\tup{x}\in\NN^k$ and $T'\leq T$.
Consider the collection $\mathcal{D}$ of all $S\in\Forc$
such that either $S\not\leq T$ or
$F^\tau(\tup{x})$ is defined by stage $|\tau|$,
where $\tau=\mathsf{stem}(S)$.
Note that $\mathcal{D}$ is $\BSigma^0_1$-definable.

Let $S\leq T$.
Since $F$ is $S$-local and $\Forc$ is almost persistent,
there is a $\tau\in S$ such that $F^\tau(\tup{x})$ is defined by stage $|\tau|$
and $S_\tau\in\Forc$.
Therefore $\mathcal{D}$ is open dense.
Therefore there is a $T'\in\mathcal{G}$ such that
$T'\leq T$ and $F^\tau(\tup{x})$ is defined by stage $|\tau|$,
where $\tau=\mathsf{stem}(S)$.
Therefore $F^G(\tup{x})$ defines a total $k$-ary function.
\end{proof}

We now use Proposition~\ref{P:genericGivesTotFncs} to define the generic extension.

\begin{definition}
Let $\MM$ be a model of \RCAo, $\Forc$ an almost persistent notion of forcing,
and $G$ a generic real for $\Forc$
corresponding to a $\BDelta^0_2$-generic filter $\mathcal{G}$.
We define the \textit{generic extension}\index{generic extension} of $\MM$,
denoted $\MM[G]$\index{$\MM[G]$}, to be the extension of $\MM$
whose sets are
$$\{\mathsf{zeros}(F^G):F\text{ is a }G\text{-local name}\},$$
where
$$\mathsf{zeros}(F^G)=\set{x}{F^G(x)=0},$$
and whose first-order part is the same as $\MM$.
\end{definition}

\begin{definition}
Let $\MM$ be a model  of \RCAo, $\Forc$ an almost persistent notion of forcing, and $G$ a generic real for $\Forc$
corresponding to a $\BDelta^0_2$-generic filter $\mathcal{G}$.
Given a formula $\phi$ of the forcing language,
we let $\phi^G$\index{$\phi^G$} be the formula obtained by replacing
all names $F$ in $\phi$ by $F^G$.

Note that $\phi^G$ is not a formula of the forcing language,
but rather a formula of the language of second-order arithmetic
augmented with constant symbols for elements of $\MM[G]$.
\end{definition}

\begin{prop}\label{P:Pi1ForcedIsTrue}
Let $\MM$ be a model of \RCAo, $\Forc$ an almost persistent notion of forcing,
and suppose that $G$ is a generic real for $\Forc$
corresponding to a $\BDelta^0_2$-generic filter $\mathcal{G}$.
Let $T\in\mathcal{G}$ and let $\phi$ be a $T$-local, $\BPi^0_1$-sentence
of the forcing language such that $T\Vdash\phi$.
Then $\MM[G]\vDash\phi^G$.
\end{prop}
\begin{proof}
Because of the existence of normal form names,
as given by Proposition~\ref{P:PR:normalForm},
we can assume that
$\phi=\forall \tup{v}(U(\tup{v})=0)$, where $U$ is $T$-local.
Since $T\Vdash\phi$, $U^G(\tup{v})=0$ for all $\tup{v}$.
Therefore $\MM[G]\vDash\forall\tup{v}(U^G(\tup{v})=0)$.
\end{proof}

\begin{cor}\label{C:Sforcedistrue}
Let $\MM$ be a model of \RCAo, $\Forc$ an almost persistent notion of forcing,
and suppose that $G$ is a generic real for $\Forc$
corresponding to a $\BDelta^0_2$-generic filter $\mathcal{G}$.
Let $\phi$ be a $T$-local sentence of the forcing language
for some $T\in\mathcal{G}$ ($T$ witnesses that $\phi$ is $G$-local).
If $T\Vdash_S\phi$ then $\MM[G]\vDash\phi^G$.
\end{cor}
\begin{proof}
Follows from Propositions \ref{P:skolemisPi1} and \ref{P:Pi1ForcedIsTrue}.
\end{proof}

We now begin to prove special cases of the Truth Lemma.

\begin{prop}\label{P:presRCAkey}
Let $\MM$ be a model of \RCAo, $\Forc$ an almost persistent notion of forcing,
and suppose that $G$ is a generic real for $\Forc$
corresponding to a generic filter $\mathcal{G}$.
Let $\phi$ be a $G$-local, $\BPi^0_2$-sentence of the forcing language.
Then $\MM[G]\vDash\phi^G$ if and only if there is a $T\in\mathcal{G}$
such that $T\Vdash_S\phi$.

Additionally, if $\Forc$ is either persistent or is almost persistent
and has an arithmetic definition, then $\mathcal{G}$ needs only be
$\BSigma^1_2$-generic.
\end{prop}

\begin{proof}
Let $T$ be a condition witnessing that $\phi$ is $G$-local
(so that $T\in\mathcal{G}$ and $F$ is $T$-local for every name $F$ in $\phi$).
Let $T'\leq T$ be any extension of $T'$.
By the definition of forcing the negation of a sentence,
there is a $T''\leq T'$ such that either $T''\Vdash\phi$ or $T''\Vdash\neg\phi$.

By Proposition~\ref{P:witnessPi2}, if $T''\Vdash\phi$ then
there is a $T'''\leq T''$ such that $T'''\Vdash_S\phi$.
On the other hand, suppose that $T''\Vdash\neg\phi$.
Write $\neg\phi=\exists v \theta(v,w)$, where $\theta(v)$ is $\BPi^0_1$.
Then there is an $S\leq T''$ and an $x$ such that $S\Vdash\theta(x)$.
By Proposition~\ref{P:witnessPi2} there is an $S'\leq S$ such that
$S'\Vdash_S\theta(x)$.

Therefore it is dense below $T$ that either
$T'\Vdash_S\phi$ or $T'\Vdash_S\theta(x)$ for some $x$.
Since $G$ is a generic filter there is an $S\in\mathcal{G}$ such that
$S\leq T$ and either $S\Vdash_S\phi$ or $S\Vdash_S\theta(x)$ for some $x$.
If $S\Vdash_S\phi$, then $\MM[G]\vDash\phi^G$ follows from Corollary~\ref{C:Sforcedistrue}.
If $S\Vdash_S\theta(x)$ for some $x$,
then $\MM[G]\vDash\theta(x)^G$ follows from Corollary~\ref{C:Sforcedistrue},
and so $\MM[G]\nvDash\phi^G$.

Finally, notice that we need only assume that
$\mathcal{G}$ is a $\BSigma^1_2$-generic filter by Corollary~\ref{C:skolemisPi11}.
\end{proof}

\begin{thm}\label{T:presRCA}
Let $\MM$ be a model of \RCAo, $\Forc$ an almost persistent notion of forcing,
and suppose that $G$ is a generic real for $\Forc$
corresponding to a generic filter $\mathcal{G}$.
Then $\MM[G]$ is also a model of $\RCAo$.

Additionally, if $\Forc$ is either persistent or is almost persistent
and has an arithmetic definition, then $\mathcal{G}$ needs only be
$\BSigma^1_2$-generic.
\end{thm}
\begin{proof}
It suffices to show that $\MM[G]$ satisfies
the following uniformization axiom:
\begin{quote}
For every $f:\NN^{k+1}\to\NN$ such that $\forall\tup{w}\exists xf(x,\tup{w})=0$
then there is a $g:\NN^k\to\NN$ such that $\forall\tup{w}f(g(\tup{w}),\tup{w})=0$.
\end{quote}
To see that it suffices to prove this uniformization axiom, see \cite{varMathias}.
Let $H$ be a $G$-local $(k+1)$-ary name such that
$\MM[G]\vDash\forall\tup{w}\exists xH^G(\tup{w},x)=0$.
By Proposition~\ref{P:presRCAkey} there is a $T\in\mathcal{G}$
such that $T\Vdash_S \forall\tup{w}\exists xH(\tup{w},x)=0$.
For any $T'\leq T$, by Proposition~\ref{P:easySkolemEquiv} and Corollary~\ref{C:Pi1unif}
there is a $T'$ local name $W$ and a $T''\leq T'$ and  such that
$T''\Vdash \forall\tup{w}H(\tup{w},W(\tup{w}))=0$.
Therefore it is dense below $T$ that
$S\Vdash \forall\tup{w}H(\tup{w},W(\tup{w}))=0$ for some $S$-local name $W$,
and so there is an $S\in\mathcal{G}$ and an $S$-local name $W$ such that
$S\Vdash \forall\tup{w}H(\tup{w},W(\tup{w}))=0$.
Therefore $\MM[G]\vDash \forall\tup{w}H^G(W^G(\tup{w}),\tup{w})=0$ by Proposition~\ref{P:Pi1ForcedIsTrue}.
\end{proof}

\begin{prop}\label{P:presACAkey}
Let $\MM$ be a model of \ACAo, $\Forc$ a persistent notion of forcing satisfying \MCP,
and suppose that $G$ is a generic real for $\Forc$
corresponding to a $\BSigma^1_2$-generic filter $\mathcal{G}$.
Let $\phi$ be a $G$-local, $\BPi^0_3$-formula of the forcing language.
Then $\MM[G]\vDash\phi^G$ if and only if there is a $T\in\mathcal{G}$
such that $T\Vdash_S\phi$.
\end{prop}
\begin{proof}
Let $T$ be a condition witnessing that $\phi$ is $G$-local
(so that $T\in\mathcal{G}$ and $F$ is $T$-local for every name $F$ in $\phi$).
Let $T'\leq T$ be any extension of $T'$.
By the definition of forcing the negation of a sentence,
there is a $T''\leq T'$ such that either $T''\Vdash\phi$ or $T''\Vdash\neg\phi$.

By Proposition~\ref{P:PacaPi3SkolemNames}, if $T''\Vdash\phi$ then $T''\Vdash_S\phi$.
On the other hand, suppose that $T''\Vdash\neg\phi$.
Write $\neg\phi=\exists v \theta(v,w)$, where $\theta(v)$ is $\BPi^0_1$.
Then there is an $S\leq T''$ and an $x$ such that $S\Vdash\theta(x)$.
By Proposition~\ref{P:witnessPi2} $S\Vdash_S\theta(x)$.

Therefore it is dense below $T$ that either
$S\Vdash_S\phi$ or $S\Vdash_S\theta(x)$ for some $x$.
Since $G$ is a generic filter there is an $S'\in\mathcal{G}$ such that
$S'\leq T$ and either $S'\Vdash_S\phi$ or $S'\Vdash_S\theta(x)$ for some $x$.
If $S'\Vdash_S\phi$, then $\MM[G]\vDash\phi^G$ follows from Corollary~\ref{C:Sforcedistrue}.
If $S'\Vdash_S\theta(x)$ for some $x$,
then $\MM[G]\vDash\theta(x)^G$ follows from Corollary~\ref{C:Sforcedistrue},
and so $\MM[G]\nvDash\phi^G$.

Finally, notice that we need only assume that
$\mathcal{G}$ is a $\BSigma^1_2$-generic filter by Corollary~\ref{C:skolemisPi11}.
\end{proof}

\begin{thm}\label{T:presACA}
Let $\MM$ be a model of \ACAo, $\Forc$ a persistent notion of forcing satisfying \MCP,
and suppose that $G$ is a generic real for $\Forc$
corresponding to a $\BSigma^1_2$-generic filter $\mathcal{G}$.
Then $\MM[G]$ is also a model of $\ACAo$.
\end{thm}
\begin{proof}
It suffices to show that $\MM[G]$ satisfies
the following minimization axiom:
\begin{quote}
For every $f:\NN^{k+1}\to\NN$ there is a $g:\NN^k\to\NN$ such that
$$\forall x\forall\tup{w}[f(x,\tup{w})\geq f(g(\tup{w}),\tup{w})].$$
\end{quote}
To see that it suffices to prove this minimization axiom, see \cite{varMathias}.
Let $H$ be a $G$-local $(k+1)$-ary name.
We use $H(\tup{w},y)\leq H(\tup{w},z)$ as a shorthand for $H(\tup{w},y)\dotminus H(\tup{w},z)=0$.
Since $\MM[G]\vDash \Ind{\BSigma^0_1}$,
$\MM[G]\vDash\forall\tup{w}\exists y\forall z[H^G(\tup{w},y)\leq H^G(\tup{w},z)]$.
By Proposition~\ref{P:presACAkey} there is a $T\in\mathcal{G}$
such that $T\Vdash_S\forall\tup{w}\exists y\forall z[H(\tup{w},y)\leq H(\tup{w},z)]$.
For any $T'\leq T$, by Proposition~\ref{P:easySkolemEquiv} and Corollary~\ref{C:Pi2unif}
there is a $T'$-local name $W$ such that $T'\Vdash\forall\tup{w}\forall z[H(\tup{w},W(\tup{w}))\leq H(\tup{w},z)]$.
Therefore it is dense below $T$ that
$S\Vdash\forall\tup{w}\forall z[H(\tup{w},W(\tup{w}))\leq H(\tup{w},z)]$ for some $S$-local name $W$,
and so there is an $S\in\mathcal{G}$ such that
$S\Vdash\forall\tup{w}\forall z[H(\tup{w},W(\tup{w}))\leq H(\tup{w},z)]$ for some $S$-local name $W$.
Therefore $\MM[G]\vDash\forall\tup{w}\forall z[H^G(\tup{w},W^G(\tup{w}))\leq H^G(\tup{w},z)]$
by Proposition~\ref{P:Pi1ForcedIsTrue}.
\end{proof}

\section{Examples}\label{forcingExamples}
	We now work through some examples.


	\subsection{Harrington Forcing}\label{harrington}
		The \textit{conditions of Harrington forcing} are infinite subtrees of $2^{<\omega}$.

This notion of forcing has been much studied in Reverse Mathematics \cite{SOSOA}.
We choose Harrington forcing as a first example, though nothing
proved here about Harrington forcing is really new.

Note that Harrington forcing is not persistent.
We now show that it is, however, almost persistent.

\begin{lem}[\RCAo]\label{L:H:locallem}
Let $T$ be an infinite subtree of $2^{<\omega}$,
$F$ be a $k$-ary $T$-local name, and $\tup{x}\in\NN^k$.
The tree
$$T_{F(\tup{x})}=\set{\tau\in T}{F^\tau(\tup{x})\text{ is undefined at stage }|\tau|}$$
is finite.
(Recall that we consider $F^\tau(\tup{x})$ undefined at stage $n$ if
there are no $w,y\leq n$ and $\sigma\subseteq\tau$
such that $w$ witnesses that $F^\sigma(\tup{x})=y$.)
\end{lem}
\begin{proof}
Suppose for the sake of contradiction that $T_{F(\tup{x})}$ is infinite.
Then $T_{F(\tup{x})}$ is an extension of $T$.
Therefore, since $F$ is $T$-local, there is a $\tau\in T_{F(\tup{x})}$
such that $F^\tau(\tup{x})$ is defined by stage $|\tau|$.
Moreover, there is an extension $\sigma\supseteq\tau$, such that
$\sigma\in T_{F(\tup{x})}$ and $F^\tau(\tup{x})$ is defined at stage $|\sigma|$,
contradicting the definition of $T_{F(\tup{x})}$.
\end{proof}

Recall that composition and primitive recursion for names
was defined in Definition~\ref{D:comp&pr}.

\begin{prop}[$\RCAo$]\label{P:H:comp&pr}
For any condition $T$,
the names defined by composition and primitive recursion
using $T$-local names are themselves $T$-local.
\end{prop}
\begin{proof}
First we consider composition.
Let $F_0$ be an $\ell$-ary $T$-local name, $F_1,\ldots,F_\ell$ be $k$-ary $T$-local names,
and $H=F_0\circ(F_1,\ldots,F_\ell)$.
We show that $H$ is $T$-local.
In other words, we let $T'\leq T$ and $\tup{x}\in\NN^k$
and show that there is a $\tau\in T'$ such that $H^\tau(\tup{x})$ is defined.

By Lemma~\ref{L:H:locallem}, given a $T$-local name $F$ we can define an increasing function
$B_F(\tup{x})$ such that $F^\tau(\tup{x})$ is defined for all $\tau$
on level $B_F(\tup{x})$ of $T'$.
Let $m_0=\max\{B_{F_1}(\tup{x}),\ldots,B_{F_\ell}(\tup{x})\}$, and
let $R$ be the set of all $\tup{y}\in\NN^\ell$ such that
for some $\tau$ on level $m_0$ of $T'$, $F_i^\tau(\tup{x})=y_i$ for each $1\leq i\leq \ell$.
Let $m_1=\max\set{B_{F_0}(\tup{y})}{\tup{y}\in R}$.
Then $H^\tau(\tup{x})$ is defined for every $\tau$ on level $m_1$ of $T'$.

Now we consider primitive recursion.
Let $F_0$ be an $(k-1)$-ary $T$-local name, $F_1$ be a $(k+1)$-ary $T$-local name,
and $H$ be the $k$-ary name defined by primitive recursion with $F_0$ and $F_1$.
We now show that $H$ is $T$-local.
In other words, we let $T'\leq T$, $\tup{x}\in\NN^{k-1}$, and $y\in\NN$
and we show that there is a $\tau\in T'$ such that $H^\tau(\tup{x},y)$ is defined.

Let $m_0=B_{F_0(\tup{x})}$ and let $R_0$ be the set of all
values $F_0^\tau(\tup{x})$ such that $\tau$ is on level $m_0$ of $T'$.
Let $m_1=\max\set{B_{F_1}(\tup{x},0,z_0)}{z_0\in R_0}$ and let $R_1$ be the set of all
values $F_1^\tau(\tup{x},0,z_0)$ such that $\tau$ is on level $m_0$ of $T'$ and $z_0\in R_0$.
Continuing in this way, we let $m_{i+1}=\max\set{B_{F_1}(\tup{x},i,z_i)}{z_i\in R_i}$
and let $R_{i+1}$ be the set of all
values $F_1^\tau(\tup{x},i,z_i)$ such that $\tau$ is on level $m_0$ of $T'$ and $z_i\in R_i$.
Then $H^\tau(\tup{x},y)$ is defined for all $\tau$ on level $m_y$ of $T'$.
\end{proof}

\begin{prop}[\RCAo]\label{P:H:almost persistent2}
For every $n\in\NN$ and condition $T$ there is a $\tau\in T$ such that
$|\tau|\geq n$ and $T_\tau$ is a condition.
\end{prop}

Note that this proposition is slightly stronger
than the second requirement for being
an almost persistent notion of forcing.

\begin{proof}
For each $m\geq n$ let $L_m$ be the set of elements on level $n$ of $T$
which have an extension on level $m$.
In other words,
$$L_m=\set{\tau\in T}{|\tau|=n\text{ and }(\exists\sigma\in T)(\tau\subseteq\sigma \land |\sigma|=m)}.$$
Notice that $\seq{L_m}_{m\in\NN}$ is a nonincreasing sequence of finite sets.
By $\Ind{\BSigma^0_1}$ there is an $m\geq n$
such that $L_{m'}=L_{m''}$ for all $m',m''\geq m$.
Therefore given any $\sigma\in L_m$,
$T_\sigma$ is infinite.
\end{proof}

\begin{prop}[\RCAo]\label{P:H:almost persistent3}
Let $T$ be an infinite subtree of $2^{<\omega}$,
$F$ be a $k$-ary $T$-local name, and $\tup{x}\in\NN^k$.
There is a $\tau\in T$ such that $F^\tau(x)$ is defined
and $T_\tau$ is infinite.
\end{prop}
\begin{proof}
By Lemma~\ref{L:H:locallem} there is an $n\in\NN$ such that
$F^\tau(\tup{x})$ is defined for all $\tau\in T$ such that $|\tau|=n$.
For each $m\geq n$ let $L_m$ be the set of elements on level $n$ of $T$
which have an extension on level $m$.
In other words,
$$L_m=\set{\tau\in T}{|\tau|=n\text{ and }(\exists\sigma\in T)(\tau\subseteq\sigma \land |\sigma|=m)}.$$
Notice that $\seq{L_m}_{m\in\NN}$ is a nonincreasing sequence of finite sets.
By $\Ind{\BSigma^0_1}$ there is an $m\geq n$
such that $L_{m'}=L_{m''}$ for all $m',m''\geq m$.
Therefore given any $\sigma\in L_m$,
$T_\sigma$ is infinite and $F^\sigma(\tup{x})$ is defined.
\end{proof}

\begin{cor}[\RCAo]\label{C:H:almost persistent}
Harrington forcing is almost persistent.
\end{cor}
\begin{proof}
Follows immediately from Propositions~\ref{P:H:comp&pr}, \ref{P:H:almost persistent2}, and \ref{P:H:almost persistent3}.
\end{proof}

\begin{lem}[\RCAo]\label{L:H:realForc}
For every $\BSigma^0_2$-generic filter over a model of \RCAo\
there exists a generic real for Harrington forcing.
\end{lem}
\begin{proof}
Since $\mathcal{D}_T=\{S:S\cap T\text{ is finite }\lor\ S\leq T\}$ is clearly
open dense for every condition $T$, the lemma follows from Lemma~\ref{L:Gen:realForc}.
\end{proof}

\begin{thm}
Let $\MM$ be a model of \RCAo\
and suppose that $G$ is a generic real for $\Forc$
corresponding to a $\BSigma^1_2$-generic filter $\mathcal{G}$ for Harrington forcing.

Then $\MM[G]$ is a model of $\RCAo$.
\end{thm}
\begin{proof}
Follows from Corollary~\ref{C:H:almost persistent}, Lemma~\ref{L:H:realForc}, and Proposition~\ref{T:presRCA}.
\end{proof}

We finish this example by showing that Harrington
forcing does not add unbounded reals.

\begin{thm}[\RCAo]\label{P:H:bndedNames}
If $F$ is a $T$-local name, then there is a function $B$
such that $$T \Vdash (\forall\tup{v})[F(\tup{v}) \leq \check{B}(\tup{v})].$$
\end{thm}
\begin{proof}
By Lemma~\ref{L:H:locallem} we can define an increasing function
$L(\tup{x})$ such that $F^\tau(\tup{x})$ is defined for all $\tau$
on level $L(\tup{x})$ of $T$.
Let $B(\tup{x})=\max\set{F^\tau(\tup{x})}{|\tau|=L(\tup{x})}$.
Then $B$ satisfies the conclusion of the lemma.
\end{proof}

\begin{cor}
Let $\MM$ be a model of \RCAo\
and suppose that $G$ is a generic real for $\Forc$
corresponding to a generic filter $\mathcal{G}$ for Harrington forcing.

For every function $f:\NN\to\NN$ in $\MM[G]$
there is a function $b$ in $\MM$ such that
$f(x)\leq b(x)$ for all $x$.
\end{cor}

	\subsection{Random Forcing}\label{random}
		The conditions of random forcing are the closed subsets of
Cantor space with positive Lebesgue measure.
We now make this precise.

Given a tree $T\subseteq\bin$, let
$$\mu_n(T) = |\{\tau\in T:|\tau|=n\}|\ /\ 2^n.$$
In other words, $\mu_n(T)$ is the number of nodes on level $n$
of $T$, divided by $2^n$.

We say that a tree $T\subseteq 2^{<\omega}$ has \textit{positive measure} if
$$\text{there is an }\epsilon>0\text{ such that } \mu_n(T)>\epsilon \text{ for all }n.$$
If $T$ does not have positive measure, we say that $T$ has \textit{measure zero}.
The \textit{conditions for random forcing} are the trees $T\subseteq 2^{<\omega}$ with positive measure.

Note that for any tree $T\subseteq 2^{<\omega}$,
$\mu_n(T)$ is monotonically decreasing.
From this it follows that if $T$ has measure zero then
$\displaystyle \lim_{n\to\infty} \mu_n(T)=0$.

The first thing to notice about random forcing is that
it fails to be persistent.
In other words, it could be the case that $T_\tau$ does not have positive measure
for some condition $T$ and $\tau\in T$.
We now show that it is, however, almost persistent.

\begin{lem}[\RCAo]\label{L:R:locallem}
Let $T$ be an subtree of $2^{<\omega}$ with positive measure,
$F$ be a $k$-ary $T$-local name, and $\tup{x}\in\NN^k$.
The tree
$$T_{F(\tup{x})}=\set{\tau\in T}{F^\tau(\tup{x})\text{ is undefined at stage }|\tau|}$$
has measure zero.
(Recall that we consider $F^\tau(\tup{x})$ undefined at stage $n$ if
there are no $w,y\leq n$ and $\sigma\subseteq\tau$
such that $w$ witnesses that $F^\sigma(\tup{x})=y$.)
\end{lem}
\begin{proof}
Suppose for the sake of contradiction that $T_{F(\tup{x})}$ has positive measure.
Then $T_{F(\tup{x})}$ is an extension of $T$.
Therefore, since $F$ is $T$-local, there is a $\tau\in T_{F(\tup{x})}$
such that $F^\tau(\tup{x})$ is defined by stage $|\tau|$,
contradicting the definition of $T_{F(\tup{x})}$.
\end{proof}

Recall that composition and primitive recursion for names
was defined in Definition~\ref{D:comp&pr}.

\begin{prop}[$\RCAo$]\label{P:R:comp&pr}
For any condition $T$,
the names defined by composition and primitive recursion
using $T$-local names are themselves $T$-local.
\end{prop}
\begin{proof}
First we consider composition.
Let $F_0$ be an $\ell$-ary $T$-local name, $F_1,\ldots,F_\ell$ be $k$-ary $T$-local names,
and $H=F_0\circ(F_1,\ldots,F_\ell)$.
We show that $H$ is $T$-local.
In other words, we let $T'\leq T$ and $\tup{x}\in\NN^k$
and show that there is a $\tau\in T'$ such that $H^\tau(\tup{x})$ is defined.

By Lemma~\ref{L:H:locallem}, given a $T$-local name $F$ we can define a function
$B_F(\tup{x};\epsilon)=b$ so that $F^\tau(\tup{x})$ is defined by stage $b$ for all
but $(\epsilon\cdot 2^b)$-many $\tau$ on level $b$ of $T'$.
Fix $\epsilon>0$.
Let $m_0=\max\{B_{F_1}(\tup{x};\epsilon/2\ell),\ldots,B_{F_\ell}(\tup{x};\epsilon/2\ell)\}$, and
let $R$ be the set of all $\tup{y}\in\NN^\ell$ such that
for some $\tau$ on level $m_0$ of $T'$, $F_i^\tau(\tup{x})=y_i\leq m_0$ for each $1\leq i\leq \ell$.
Notice that there are at most $(\frac{\epsilon}{2}\cdot 2^{m_0})$-many elements $\tau$ on level
$m_0$ of $T'$ such that $\seq{F_1^\tau(\tup{x}),\ldots,F_\ell^\tau(\tup{x})}$ is not defined.
Let $m_1=\max\set{B_{F_0}\left(\tup{y};\frac{\epsilon}{2|R|}\right)}{\tup{y}\in R}$.
Then $H^\tau(\tup{x})$ is defined for all
but $(\epsilon\cdot 2^{m_1})$-many $\tau$ on level $m_1$ of $T'$.

Now we consider primitive recursion.
Let $F_0$ be an $(k-1)$-ary $T$-local name, $F_1$ be a $(k+1)$-ary $T$-local name,
and $H$ be the $k$-ary name defined by primitive recursion with $F_0$ and $F_1$.
We now show that $H$ is $T$-local.
In other words, we let $T'\leq T$, $\tup{x}\in\NN^{k-1}$, and $y\in\NN$
and we show that there is a $\tau\in T'$ such that $H^\tau(\tup{x},y)$ is defined.

Fix $\epsilon>0$.
Let $m_0=B_{F_0}\left(\tup{x};\frac{\epsilon}{y+1}\right)$ and let $R_0$ be the set of all
$z_0$ such that $F_0^\tau(\tup{x})=z_0\leq m_0$ for some $\tau$ is on level $m_0$ of $T'$.
Notice that there are at most $\left(\frac{\epsilon}{y+1}\cdot 2^{m_0}\right)$-many elements $\tau$ on level
$m_0$ of $T'$ such that $F_0^\tau(\tup{x})$ is not defined.
Let $m_1=\max\set{B_{F_1}\left(\tup{x},0,z_0;\frac{\epsilon}{(y+1)|R_0|}\right)}{z_0\in R_0}$
and let $R_1$ be the set of all $z_1$ such that $F_1^\tau(\tup{x},0,z_0)=z_1\leq m_1$
for some $\tau$ is on level $m_1$ of $T'$ and $z_0\in R_0$.
Notice that there are at most $\left(\frac{2\epsilon}{y+1}\cdot 2^{m_1}\right)$-many elements $\tau$ on level
$m_1$ of $T'$ such that $H^\tau(\tup{x},1)=F_1(\tup{x},0,F_0^\tau(\tup{x}))$ is not defined.
Continuing in this way, we let
$m_{i+1}=\max\set{B_{F_1}\left(\tup{x},i,z_i;\frac{\epsilon}{(y+1)|R_i|}\right)}{z_i\in R_i}$
and let $R_{i+1}$ be the set of all $z_{i+1}$ such that $F_1^\tau(\tup{x},i,z_i)=z_{i+1}\leq m_{i+1}$
for some $\tau$ is on level $m_{i+1}$ of $T'$ and $z_i\in R_i$.
Notice that there are at most $\left(\frac{(i+2)\epsilon}{y+1}\cdot 2^{m_{i+1}}\right)$-many elements $\tau$ on level
$m_{i+1}$ of $T'$ such that $H^\tau(\tup{x},i+1)$ is not defined.
Then $H^\tau(\tup{x},y)$ is defined for all
but $(\epsilon\cdot 2^{m_y})$-many $\tau$ on level $m_y$ of $T'$.
\end{proof}

\begin{prop}[\RCAo]\label{P:R:almostPersistent2}
For every $n\in\NN$ and condition $T$ there is a $\tau\in T$ such that
$|\tau|\geq n$ and $T_\tau$ is a condition.
\end{prop}

Note that this proposition is slightly stronger
than the second requirement for being
an almost persistent notion of forcing.

\begin{proof}
Suppose, for the sake of contradiction, that $T_\tau$ has measure zero
for all $\tau\in T$ such that $|\tau|\geq n$.
Fix $\epsilon>0$.
Let $\tau_1,\ldots,\tau_k$ be the elements on level $n$ of $T$.
Since $T_{\tau_i}$ has measure zero for each $1\leq i\leq k$,
there is an $m\geq n$ such that the number of nodes on level $m$ of $T_{\tau_i}$,
divided by $2^m$, is less than $\epsilon/k$.
Therefore the total number of nodes of $T$ on level $m$, divided by $2^m$,
is less than $k(\epsilon/k)=\epsilon$,
contradicting that $T$ has positive measure.
\end{proof}

\begin{prop}[\RCAo]\label{P:R:almostPersistent3}
Let $T$ be an infinite subtree of $2^{<\omega}$,
$F$ be a $k$-ary $T$-local name, and $\tup{x}\in\NN^k$.
There is a $\tau\in T$ such that $F^\tau(\tup{x})$ is defined
and $T_\tau$ has positive measure.
\end{prop}

\begin{proof}
Suppose, for the sake of contradiction, that $T_\tau$ has measure zero
for all $\tau\in T$ such that $F^\tau(\tup{x})$ is defined.
Fix $\epsilon>0$.
By Lemma~\ref{L:R:locallem} $T_{F(\tup{x})}$ has measure zero,
so there is a level $N$ of $T$ such that the number of nodes $\tau$ on level
$N$ such that $F^\tau(\tup{x})$ is undefined at stage $|\tau|$,
divided by $2^N$, is less than $\epsilon/2$.

Let $\tau_1,\ldots,\tau_k$ be the elements on level $N$ of $T$
such that $F^{\tau_i}(\tup{x})$ is defined at stage $|\tau_i|$.
By assumption $T_{\tau_i}$ has measure zero for each $1\leq i\leq k$.
Therefore there is an $M\geq N$ such that for each $1\leq i\leq k$,
the number of nodes on level $M$ of $T_{\tau_i}$, divided by $2^M$,
is less than $\epsilon/2k$.

Therefore the total number of nodes of $T$ on level $M$, divided by $2^M$,
is less than $\epsilon/2+k(\epsilon/2k)=\epsilon$, contradicting that
$T$ has positive measure.
\end{proof}

\begin{cor}[\RCAo]\label{C:R:almost persistent}
Random forcing is almost persistent.
\end{cor}
\begin{proof}
Follows immediately from Propositions~\ref{P:R:comp&pr}, \ref{P:R:almostPersistent2}, and \ref{P:R:almostPersistent3}.
\end{proof}

\begin{lem}\label{L:R:realForc}
For every $\BSigma^0_2$-generic filter over a model of \RCAo\
there exists a generic real for random forcing.
\end{lem}

\begin{proof}
By Lemma~\ref{L:Gen:realForc} it
suffices to show that if $T$ is a condition, then
$$\mathcal{D}_T=\{S:S\cap T\text{ is finite }\lor\ S\leq T\}$$
is open dense.

Suppose that $T'$ is a condition and that
$T'\cap T$ has measure zero.
Since $T'$ has positive measure
there is a $\tau\in T'\setminus(T'\cap T)$ such that $T'_\tau$ also has positive measure.
Then $T'_\tau\leq T'$ and $T'_\tau\cap T$ is finite,
so $T'_\tau\in\mathcal{D}_T$.
\end{proof}

\begin{thm}
Let $\MM$ be a model of \RCAo\
and suppose that $G$ is a generic real for $\Forc$
corresponding to a $\BSigma^1_2$-generic filter $\mathcal{G}$ for random forcing.

Then $\MM[G]$ is a model of $\RCAo$.
\end{thm}
\begin{proof}
Follows from Corollary~\ref{C:R:almost persistent}, Lemma~\ref{L:R:realForc}, and Proposition~\ref{T:presRCA}.
\end{proof}

Over models of $\ACAo$ we can actually assume that
random forcing is persistent.
More precisely, the next propositions shows that for every condition $T$
there is an extension $T\leq T'$ with the same measure which obeys the persistent criterion.

\begin{prop}[\ACAo]\label{P:R:makePersistent}
Let $T$ have measure at least $\epsilon$.
Let $$T'=\set{\tau\in T}{T_\tau\text{ has positive measure}}.$$
Then $T'$ has measure at least $\epsilon$.
\end{prop}
\begin{proof}
Suppose that $\mu_n(T')<\epsilon$ for some $n$.
Let $\tau_0,\ldots,\tau_k$ be the nodes on level $n$ of $T\setminus T'$.
Since $T_{\tau_i}$ has measure zero for all $i\leq k$,
for any $\delta>0$ there is a level $m_\delta$ such that
$\mu_{m_\delta}(\bigcup_{i\leq k}T_{\tau_i})<\delta$.
Therefore there is a level $m$ such that
$$\mu_m(T)=\mu_m(T')+\mu_{m}\left(\bigcup_{i\leq k}T_{\tau_i}\right)<\epsilon.$$
But this contradicts that $T$ has measure at least $\epsilon$.
\end{proof}

We now proceed to show that random forcing satisfies $\MCP$.
We begin with some lemmas.

\begin{lem}[\RCAo]\label{L:R:subConditionPartition}
Let $T$ have measure at least $\epsilon>0$ and
let $S$ be a subtree of $T$, possibly with measure zero.
Let $$R(n)=\bigcup_{\tau\in X_n}T_\tau$$
where $X_n$ is the set of all $\tau\in (T\setminus S)$ such that $|\tau|=n$.
Then for all $\delta>0$ there is an $n$ such that the
the measure of $S\cup R(n)$ is at least $\epsilon-\delta$.
\end{lem}
\begin{proof}
Note that the elements of $T\setminus(S\cup R(n))$ are precisely
the elements $\tau\in T\setminus S$ such that $\tau\uhr n\in S$.

Suppose that $S$ has measure at least $\delta_1$ and has
measure no greater than $\delta_2$ for some
$0\leq \delta_1<\delta_2\leq \epsilon$.
We will show that there is an $n$ such that
$S\cup R(n)$ has measure at least $\delta_1+(\epsilon-\delta_2)$.
The lemma then follows by choosing $\delta_1$ and $\delta_2$
to be as close as is necessary.

Since $T$ has measure at least $\epsilon$,
and since $S$ has measure no greater than $\delta_2$,
there is a level $n$ such that for every $m\geq n$
there are at least $\lfloor(\epsilon-\delta_2)\cdot 2^m\rfloor$-many
elements of $(T\setminus S)$ on level $m$.
Moreover, since $S$ has measure at least $\delta_1$,
there are at least $\lfloor\delta_1\cdot 2^m\rfloor$-many
elements of $S$ on level $m$ for every $m\geq n$.
Therefore $S\cup R(n)$ has measure at least $\delta_1+(\epsilon-\delta_2)$.
\end{proof}

\begin{lem}[\RCAo]\label{L:R:preMCP}
Suppose that $c:T\to\{0,1\}$ is a monotone coloring
of a tree $T$ which has measure at least $\epsilon$.
For any $\delta>0$ there is a level $n$ and values
$y_0,y_1,\ldots,y_k\in\{0,1\}$ such that
$$\mu\left(\bigcup_{i=0}^k\set{\sigma\in T_{\tau_i}}{c(\sigma)=y_i}\right)>\epsilon-\delta,$$
where $\tau_0,\tau_1,\ldots,\tau_k$ are the nodes of $T$ on level $n$.
\end{lem}
\begin{proof}
Let $S=\set{\tau\in T}{c(\tau)=0}$.
Let $n$ be the level guaranteed by Lemma~\ref{L:R:subConditionPartition}.
Since $c$ is a monotone coloring, if $c(\tau)=1$ then
$\set{\sigma\in T_\tau}{c(\sigma)=1}=T_\tau$.
The lemma then follows by letting $y_i=0$ if and only $\tau_i\in S$.
\end{proof}


\begin{prop}[\ACAo]\label{P:R:MCP}
Random forcing satisfies $\MCP$.

In fact, for any condition $T$ and any infinite sequence $\langle c_i:i\in\NN\rangle$ of monotone
colorings $c_i:T\rightarrow \{0,1\}$, there is a condition $T'\leq T$
whose measure is arbitrarily close to the measure of $T$ such that
for every $i$ there is a $k$ such that if $\tau\in T'$ is above level $k$
then $c_i$ is constant on $T'_\tau$.
\end{prop}

\begin{proof}
Fix $k\in\NN$.
We will construct $T'\leq T$ so that if $T$ has measure
at least $\epsilon$, then $T'$ has measure at least $\epsilon-1/k$.
We define $T'$ and $H$ in stages.

At stage $0$ we find a level $m_0$ and values $y_0,y_1,\ldots,y_\ell\in\{0,1\}$ such that
$$\mu\left(\bigcup_{i=0}^\ell\set{\sigma\in T_{\tau_i}}{c_0(\sigma)=y_i}\right)>\epsilon-1/(k+1),$$
where $\tau_0,\tau_1,\ldots,\tau_\ell$ are the nodes of $T$ on level $m_0$.
Such nodes and numbers exist by Lemma~\ref{L:R:preMCP}.
We then let
$$T^0=\bigcup_{i=0}^\ell\set{\sigma\in T_{\tau_i}}{c_0(\sigma)=y_i}.$$

At stage $n+1$ we find a level $m_{n+1}>m_n$ and values $y_0,y_1,\ldots,y_\ell\in\{0,1\}$ such that
$$\mu\left(\bigcup_{i=0}^\ell\set{\sigma\in T^n_{\tau_i}}{c_0(\sigma)=y_i}\right)>\epsilon-1/(k+1)^{n+2},$$
where $\tau_0,\tau_1,\ldots,\tau_\ell$ are the nodes of $T$ on level $m_n$.
Such nodes and numbers exist by Lemma~\ref{L:R:preMCP}.
We then let
$$T^{n+1}=\bigcup_{i=0}^\ell\set{\sigma\in T^n_{\tau_i}}{c_0(\sigma)=y_i}.$$

Finally, we let $T'=\bigcap T^n$.
Notice that $T'\leq T$, $T'$ has measure at least
$\epsilon-\sum_{i=0}^\infty1/(k+1)^{i+1}=\epsilon-1/k$,
and $T'$ thus satisfies the conclusion of $\MCP$.
\end{proof}

\begin{thm}
Let $\MM$ be a model of \ACAo\
and suppose that $G$ is a generic real for $\Forc$
corresponding to a $\BSigma^1_2$-generic filter $\mathcal{G}$ for random forcing.

Then $\MM[G]$ is a model of $\ACAo$.
\end{thm}
\begin{proof}
Follows from Proposition~\ref{P:R:makePersistent}, Lemma~\ref{L:R:realForc}, and Proposition~\ref{T:presACA}.
\end{proof}

We finish this example by showing that random
forcing does not add unbounded reals.

\begin{thm}[\RCAo]\label{P:R:bndedNames}
Let $T$ have measure at least $\epsilon$,
$F$ be a $T$-local name, and $0<\delta<\epsilon$.
There is an extension $T'\leq T$ with measure at least $\epsilon-\delta$
and a function $B$
such that $$T' \Vdash (\forall\tup{v})[F(\tup{v}) \leq \check{B}(\tup{v})].$$
\end{thm}
\begin{proof}
The proof is very similar to the proof of Proposition~\ref{P:R:MCP}.
Fix $k\in\NN$.
We will construct $T'\leq T$ so that if $T$ has measure
at least $\epsilon$, then $T'$ has measure at least $\epsilon-1/k$.
We define $T'$ and $H$ in stages.

For ease of notation we assume that $F$ is a $1$-ary name.
By Lemma~\ref{L:R:locallem}, for each $x\in\NN$ the tree
$T_{F(x)}=\set{\tau\in T}{F^\tau(x)\text{ is undefined at stage }|\tau|}$
has measure zero.

We begin at stage 0.
Let $S_0=T_{F(0)}$.
By Lemma~\ref{L:R:subConditionPartition} there is a level $m_0$ such that
$$\mu\left(\bigcup_{\tau\in C_0}T_\tau\right)> \epsilon-1/(k+1),$$
where $C_0$ is the set of nodes on level $m_0$ of $T$ such that
$F^\tau(0)$ is defined at stage $|\tau|$.
Let $\displaystyle T^0=\left(\bigcup_{\tau\in C_0}T_\tau\right)$ and
$B(0)=\max\set{F^\tau(0)}{\tau\in C_0}$.

At stage $n+1$ we do the following.
Let
$$S_{n+1}=T^n_{F(n+1)}=\set{\tau\in T^n}{F^\tau(n+1)\text{ is undefined at stage }|\tau|}.$$
By Lemma~\ref{L:R:subConditionPartition} there is a level $m_{n+1}>m_n$ such that
$$\mu\left(\bigcup_{\tau\in C_{n+1}}T_\tau\right)> \epsilon-1/(k+1)^{n+2},$$
where $C_{n+1}$ is the set of nodes on level $m_{n+1}$ of $T^n$ such that
$F^\tau(n+1)$ is defined at stage $|\tau|$.
Let $\displaystyle T^{n+1}=\left(\bigcup_{\tau\in C_{n+1}}T_\tau\right)$ and
$B(n+1)=\max\set{F^\tau(n+1)}{\tau\in C_{n+1}}$.

Finally, we let $T'=\bigcap T^n$.
Note that the intersection $\bigcap T^n$ is well defined in \RCAo\
since $\displaystyle \tau\in \bigcap_{n\in\NN} T^n$ if and only if
$\displaystyle \tau\in \bigcap_{n\leq k} T^n$ for some $k$ such that $|\tau|\leq m_k$.
Notice also that $T'\leq T$, $T'$ has measure at least
$\displaystyle \epsilon-\sum_{i=0}^\infty1/(k+1)^{i+1}=\epsilon-1/k$,
and that $F^\tau(x)\leq B(x)$ for every $x\in\NN$ and
every $\tau\in T'$ such that $F^\tau(x)$ is defined.
\end{proof}

\begin{cor}
Let $\MM$ be a model of \RCAo\
and suppose that $G$ is a generic real for $\Forc$
corresponding to a generic filter $\mathcal{G}$ for random forcing.

For every function $f:\NN\to\NN$ in $\MM[G]$
there is a function $b$ in $\MM$ such that
$f(x)\leq b(x)$ for all $x$.
\end{cor}

	\subsection{Sacks \& Silver Forcing}\label{sacks}
		We now consider two closely related notions of forcing,
namely Sacks forcing and Silver forcing.
In addition to applying the results of Section~\ref{generic}
to these two examples, we show that generic reals
for these two notions of forcing can be made to be well-behaved in certain ways.

The \textit{conditions for Sacks forcing} are perfect subtrees of $2^{<\omega}$.
In other words, the conditions are subtrees $T\subseteq 2^{<\omega}$
such that for all $\tau\in T$ there is a $\sigma\supseteq\tau$ such that
$\sigma$ is a splitting node.
We say that $\sigma$ is a splitting node if
$\sigma\cat{0},\sigma\cat{1}\in T$.
Notice that Sacks forcing is persistent.

The \textit{conditions for Silver forcing} are usually thought of as
partial functions $f:\NN\to\{0,1\}$ whose domain is coinfinite.
Silver conditions can also be thought of as certain subtrees of $2^{<\omega}$.
Given a partial function $f:\NN\to\{0,1\}$ whose domain is coinfinite,
consider the tree $T$ defined by
\begin{itemize}
\item $\seq{}\in T$,
\item if $\tau\in T$ and $|\tau|\notin\dom(f)$ then $\tau$ splits in $T$,
\item if $\tau\in T$ and $|\tau|\in\dom(f)$ then $\tau$ has exactly one extension
	in $T$, namely $\tau\cat{f(|\tau|)}$.
\end{itemize}
Notice that the branches through $T$ are exactly the total extensions of $f$.
We can therefore define Silver conditions to be all trees $T\subseteq \bin$
with infinitely many nodes that split and such that for every $n\in\NN$,
one of the following holds:
\begin{itemize}
\item if $|\tau|=n$ then $\tau$ splits in $T$, or
\item if $|\tau|=n$ then $\tau\cat{0}\in T$ and $\tau\cat{1}\notin T$, or
\item if $|\tau|=n$ then $\tau\cat{0}\notin T$ and $\tau\cat{1}\in T$.
\end{itemize}
In other words, Silver conditions are the Sacks conditions such that
on every level, either every node splits, every node branches only to the left,
or every node branches only to the right.

Silver forcing and Sacks forcing are very similar.
In fact, everything that we will prove about Sacks
forcing will be proved about Silver forcing with a nearly identical proof.

\begin{prop}[\ACAo]\label{P:Sk:MCP}
Sacks forcing satisfies $\MCP$.
\end{prop}
\begin{proof}
Fix a sequence of monotone colorings $c_i:2^{<\omega}\rightarrow \{0,1\}$
of a condition $T$.
We will construct an extension $T'\leq T$ in stages,
so that the $n$-th splitting level of $T'$ is defined at stage $n$.
In fact, $T'$ will be defined so that for all $\sigma$ such that $|\sigma|\geq n$,
$c_n(\sigma)$ will be determined at the $n$-th splitting level of $T'$.
Therefore $T'$, along with its splitting levels as a set of layers,
will satisfy the conclusion of \MCP.

At stage $0$ we look for a $\sigma$ such that $c_0(\sigma)=1$.
If such a $\sigma$ exists, we place it (and every node below it) into $T'$.
Otherwise, we place $\langle\rangle$ into $T'$.

At stage $n+1$ we look at each of the $2^n$ nodes defined at stage $n$.
List these nodes as $\tau_1,\ldots,\tau_{2^n}$.
We look for a $\sigma\supseteq\tau_1\cat{0}$ such that $c_0(\sigma)=1$.
If such a $\sigma$ exists, we place it (and every node below it) into $T'$.
Otherwise, we place $\tau_1\cat{0}$ into $T'$.
We now repeat this process for $\tau_1\cat{1}$, and also for
$\tau_i\cat{j}$ for each $2\leq i\leq 2^n$ and $j<2$.
\end{proof}

\begin{lem}\label{L:Sk:realForc}
For every $\BSigma^0_2$-generic filter over a model of \RCAo\
there exists a generic real for Sacks forcing.
\end{lem}

\begin{proof}
By Lemma~\ref{L:Gen:realForc} it
suffices to show that if $T$ is a condition, then
$$\mathcal{D}_T=\{S:S\cap T\text{ is finite }\lor\ S\leq T\}$$
is open dense.

Suppose that $T'\cap T$ contains no perfect subtree.
If $T'\cap T$ is empty, then we're done.
Otherwise we can choose a $\sigma\in T'\cap T$ such that no
$\tau\supseteq\sigma$ splits in $T'\cap T$.
We can therefore find a $\tau\supseteq\sigma$ such that
$\tau\in T'\setminus T$.
Then $T'_\tau\leq T'$ and $T'_\tau\cap T$ is finite,
so $T'_\tau\in\mathcal{D}_T$.
\end{proof}

\begin{thm}
Let $\MM$ be a model of \RCAo\ (\ACAo)
and suppose that $G$ is a generic real for $\Forc$
corresponding to a $\BSigma^1_2$-generic filter $\mathcal{G}$ for Sacks forcing.

Then $\MM[G]$ is a model of \RCAo\ (\ACAo).
\end{thm}
\begin{proof}
Follows from Propositions \ref{T:presRCA} and Proposition~\ref{T:presACA}.
\end{proof}

\begin{prop}[\ACAo]\label{P:Sl:MCP}
Silver forcing satisfies $\MCP$.
\end{prop}
\begin{proof}
The proof is nearly identical to that of Proposition~\ref{P:Sk:MCP}.
We explain briefly how to modify the proof of Proposition~\ref{P:Sk:MCP}.

The proofs of of Proposition~\ref{P:Sk:MCP} used a stagewise
construction where at stage $n+1$ the $n$-th splitting
level was extended to the $(n+1)$-th splitting level, meeting some requirements.
Since Silver conditions are a bit more restrictive than Sacks conditions,
care is needed when extending nodes from one level to a higher level.

Suppose that $\tau_1,\tau_2,\ldots,\tau_k$ are the nodes of level $n$
of a Silver condition $T$, and that all these nodes split.
Just as was done in Sacks forcing,
we find an appropriate splitting node $\sigma_1\supseteq\tau_1$.
We then extend each of the other nodes $\tau_i$, for $2\leq i\leq k$
to match $\sigma_1$: let $\tau_i'(x)=\sigma_1(x)$ for each $|\tau_1|\leq k<|\sigma_1|$.
Note that $\sigma_i\in T$ for all $2\leq i\leq k$ since
$\sigma_1\in T$ and $T$ is a Silver condition.
It is necessary that we extend all the nodes $\sigma_i$
like this so that we end up with a Silver condition.
At this point, of the nodes $\sigma_1, \tau_2',\tau_3',\ldots,\tau_k'$,
only $\sigma_1$ meets whatever requirement we are concerned with.
We now find an extension $\sigma_2\supseteq\tau_2'$ that meets the relevant requirement.
We then extend each of $\sigma_1,\tau_3',\tau_4'$ so that the extensions match $\sigma_2$.
Continuing in this way, and eventually finding an extension $\rho_i$ of each $\tau_i$,
for $1\leq i\leq k$, such that $\rho_i$ meet the requirement and the set of all $\rho_i$
is a valid $(n+1)$-splitting level for a Silver condition.

With this adjustment, the proof of Proposition~\ref{P:Sk:MCP} suffices to prove
the same proposition for Silver forcing.
\end{proof}

\begin{lem}\label{L:Sl:realForc}
For every $\BSigma^0_2$-generic filter over a model of \RCAo\
there exists a generic real for Silver forcing.
\end{lem}
\begin{proof}
The proof is similar to that of Lemma~\ref{L:Sk:realForc}.
\end{proof}

\begin{thm}
Let $\MM$ be a model of \RCAo\ (\ACAo)
and suppose that $G$ is a generic real for $\Forc$
corresponding to a $\BSigma^1_2$-generic filter $\mathcal{G}$ for Sacks forcing.

Then $\MM[G]$ is a model of \RCAo\ (\ACAo).
\end{thm}
\begin{proof}
Follows from Propositions \ref{T:presRCA} and Proposition~\ref{T:presACA}.
\end{proof}

We now show that we can add a cone avoiding, or generalized low-2,
Sacks generic real.

\begin{prop}\label{P:Sk:CAgeneric}
Let $\MM$ be a model of $\ACAo$ and $A$ be a noncomputable set in $\MM$.
There is a Sacks condition $T$ such that no branch through $T$ computes $A$.
\end{prop}

\begin{proof}
We construct $T$ in stages so that $T$ is defined up
to the $n$-th splitting level at stage $n$.
Moreover, at stage $n$ we will guarantee
that $\Phi^f_n\neq\chi_A$ for all $f\in[T]$.

At stage 0 we look for a $\sigma\in\bin$
and an $x$ such that $\Phi_0^{\sigma}(x)\downarrow\neq\chi_{A}(x)$.
If such $\sigma$ and $x$ exist, we place $\sigma$ into $T$.
We have then ensured that $\Phi_0^f\neq\chi_A$ for all
$f\in[\bin]$ which extend $\sigma$.
Otherwise, if no such $\sigma$ and $x$ exist,
we place $\seq{}$ into $T$ and claim that
if $f\in [\bin]$ then $\Phi_0^f$ is \textit{not} total
(and hence distinct from $\chi_A$).
Suppose, for the sake of contradiction, that $\Phi_0^f$ is total.
We can then compute $\chi_A(x)$ by searching for
a $\sigma$ such that $\Phi_0^\sigma(x)\downarrow$.
Such a $\sigma$ exists since $\Phi_0^f$ is total,
and $\Phi_0^\sigma(x)=\Phi_0^f(x)=\chi_A(x)$ since,
by assumption, there is no $\sigma$ such that
$\Phi_0^\sigma(x)\neq\chi_A(x)$.
This is a contradiction since $A$ was assumed to be noncomputable.

At stage $(n+1)$ we begin by looking at $\rho\cat0$ and $\rho\cat1$
for each node $\rho$ defined during stage $n$.
List these nodes by $\tau_1,\tau_2,\ldots,\tau_{2^{n+1}}$.
We now look for a node $\sigma\supseteq\tau_1$ and an $x$
such that $\Phi_{n+1}^\sigma(x)\downarrow\neq\chi_A(x)$.
If such $\sigma$ and $x$ exist, we place $\sigma$ into $T$.
Otherwise we place $\tau_1$ into $T$.
By the same argument as in the previous paragraph,
$\Phi_{n+1}^f\neq\chi_A$ when ever $f$ is a path
through $2^{<\omega}$ which extends $\tau_1$.
We now repeat this procedure for $\tau_2,\ldots,\tau_{2^{n+1}}$.
\end{proof}

\begin{cor}
Let $\MM$ be a model of \ACAo.
Suppose that $A$ is a noncomputable set in $\MM$
and that $\mathcal{G}$ is a generic filter for Sacks forcing.
Then there is a generic real $G$ corresponding to $\mathcal{G}$
such that $A\not\leq_{\text{T}}G$.
\end{cor}

\begin{prop}\label{P:Sk:GL2generic}
Let $\MM$ be a model of $\ACAo$.
There is a Sacks condition $T$ such that for every branch $G\in[T]$,
$G''\leq G\oplus0''$.  In other words, $T$ only contains GL$_2$ generic reals.
\end{prop}

\begin{proof}
We construct $T$ in stages so that $T$ is defined up to the $n$-th splitting level at stage $n$.
At the beginning of stage $n+1$, each of the $2^n$ nodes defined in stage $n$
will be marked as either ``needing help with $m$" or ``not needing help with $m$" for each $m\leq n$.

At stage 0 we look for a node $\sigma$ and a $k$
such that for all $\tau\supseteq\sigma$, $\Phi_0^\tau(k)\uparrow$.
Note that we can ask such questions with a $0''$ oracle.
If such $\sigma$ and $k$ exist, we let $\sigma'$ and $\sigma''$
be any two incomparable nodes above $\sigma$.
We place $\sigma'$ and $\sigma''$
into $T$ and mark them as ``not needing help with 0".
Notice that if $G:\NN\to\{0,1\}$ extends $\sigma$
then $\Phi_0^f$ is not total.

If no such $\sigma$ and $k$ exist, then we know
that for every $\tau\supseteq\sigma$ and every $k$
there is a $\rho\supseteq\tau$ such that $\Phi_0^\rho(k)\downarrow$.
We place $\langle 0\rangle$ and $\langle 1\rangle$
into $T$ and mark them as ``needing help with 0".

At stage $n+1$ we begin by considering all the nodes
$\sigma_1',\ldots,\sigma_{2^{n}}'$ that were defined in the previous stage.
For each $\sigma_i'$ we find a $\sigma_i\supseteq\sigma_i'$ such
that for all $m\leq n$, if $\sigma_i'$ is marked as
``needing help with $m$", then $\Phi_m^{\sigma_i}(k)\downarrow$
for each $k\leq n$.
This is possible because $\sigma_i'$ could only have been
marked as ``needing help with $m$" if such an extension
is always possible.

For each $1\leq i\leq 2^n$ we do the following.
We look for a node $\sigma\supseteq\sigma_i$ and a $k$
such that for all $\tau\supseteq\sigma$, $\Phi_{n+1}^\tau(k)\uparrow$.
Note that we can ask such questions with a $0''$ oracle.
If such $\sigma$ and $k$ exist, we let $\sigma'$ and $\sigma''$
be any two incomparable nodes above $\sigma$.
We place $\sigma'$ and $\sigma''$
into $T$ and mark them as ``not needing help with $n+1$".
For $m\leq n$, we mark $\sigma'$ and $\sigma''$
as ``needing help with $m$" if and only if $\sigma_i$
was marked as ``needing help with $m$".
Notice that if $G:\NN\to\{0,1\}$ extends $\sigma$
then $\Phi_0^f$ is not total.

If no such $\sigma$ and $k$ exists, then we know
that for every $\tau\supseteq\sigma_i$ and every $k$
there is a $\rho\supseteq\tau$ such that $\Phi_0^\rho(k)\downarrow$.
We let $\sigma'$ and $\sigma''$
be any two incomparable nodes above $\sigma_i$, and we
place them into $T$ and mark them as ``needing help with $n+1$".
For $m\leq n$, we mark $\sigma'$ and $\sigma''$
as ``needing help with $m$" if and only if $\sigma_i$
was marked as ``needing help with $m$".

This ends the construction.
We now show that if $G\in[T]$, then $G''\leq G\oplus 0''$.
It suffices to show that we can use $G$ and $0''$
to compute the set of indices $e$ such that
$\Phi_e^G$ is total.
To determine if $\Phi^G_e$ is total, we do the following.
Suppose that $\sigma$ is the topmost node at
stage $e$ that meets $G$.
If $\sigma$ was marked as ``not needing help with $e$"
then $\Phi_e^G$ is not total.
In fact, if $k$ is the number that witnesses that
$\sigma$ was marked as ``not needing help with $e$",
then $\Phi_e^G(k)\uparrow$.
If, on the other hand, $\sigma$ was marked as
``needing help with $e$", then $\Phi_e^G(m)$
converges by the $m$-th splitting level of $T$.
In other words, if $\sigma$ is the topmost node at
stage $m$ that meets $G$, then $\Phi_e^G(m)\downarrow$.
\end{proof}

\begin{cor}
Let $\MM$ be a model of \ACAo.
Suppose $\mathcal{G}$ is a generic filter for Sacks forcing.
Then there is a generic real $G$ corresponding to $\mathcal{G}$
such that $G$ is GL$_2$.
\end{cor}

We now show that Sacks forcing does not add unbounded reals.

\begin{prop}[\RCAo]\label{P:Sk:StrBnding}
Let $T$ be a Sacks condition and $F$ be a $T$-local name.
There is a condition $T'\leq T$ such that for all $n$,
if $\tau\in T'$ is above the $n$-th splitting level of $T'$
then $F_\tau(x)$ is defined by stage $|\tau|$ for all $x\leq n$.
\end{prop}

\begin{proof}
We assume, for ease of notation,
that $F$ is a 1-ary name.
We will construct $T'$ in stages.

We begin with stage 0.
Since $F$ is $T$-local, there is a $\tau\in T$ such that $F^\tau(0)$ is defined.
Let $\sigma$ be a splitting node above $\tau$.
We place $\sigma\cat{0}$ and $\sigma\cat{1}$ into $T'$.

At stage $n+1$ we begin with the the $2^{n+1}$-many
nodes that were defined at stage $n$.
Let $\rho$ be one such node.
Since $F$ is $T_\rho$-local, there is a $\tau\in T_\rho$ such that $F^\tau(n+1)$ is defined.
Let $\sigma$ be a splitting node above $\tau$.
We place $\sigma\cat{0}$ and $\sigma\cat{1}$ into $T'$.

Finally, we close $T'$ downward so that it is in fact a tree.
This ends the construction, and it is easy to see that $T'$
satisfies the proposition.
\end{proof}

\begin{cor}[\RCAo]\label{P:Sk:bndedNames}
Let $T$ be a Sacks condition and $F$ be a $T$-local name.
There is an extension $T'\leq T$ and a function $B$
such that $$T' \Vdash (\forall\tup{v})[F(\tup{v}) \leq \check{B}(\tup{v})].$$
\end{cor}
\begin{proof}
Let $T'\leq T$ be as in the conclusion of Proposition~\ref{P:Sk:StrBnding}.
Let \newline $B(x)=\max\set{F^\tau(x)}{\tau\text{ is on the }n\text{-th splitting level of }T'}.$
\end{proof}

\begin{cor}
Let $\MM$ be a model of \RCAo\
and suppose that $G$ is a generic real for $\Forc$
corresponding to a generic filter $\mathcal{G}$ for Sacks forcing.

For every function $f:\NN\to\NN$ in $\MM[G]$
there is a function $b$ in $\MM$ such that
$f(x)\leq b(x)$ for all $x$.
\end{cor}

The results which we have now proved about Sacks forcing
also hold for Silver forcing.
In the same way that we modified the proof of Proposition~\ref{P:Sk:MCP}
to work for Silver forcing, namely by being careful about extending splitting levels,
we can modify the proofs of our recent results about Sacks forcing to hold for Silver forcing.
We now state, without proof, the corresponding results for Silver forcing.

\begin{prop}\label{P:Sl:CAgeneric}
Let $\MM$ be a model of $\ACAo$ and $A$ be a noncomputable set in $\MM$.
There is a Silver condition $T$ such that no branch through $T$ computes $A$.
\end{prop}

\begin{cor}
Let $\MM$ be a model of \ACAo.
Suppose that $A$ is a noncomputable set in $\MM$
and that $\mathcal{G}$ is a generic filter for Silver forcing.
Then there is a generic real $G$ corresponding to $\mathcal{G}$
such that $A\not\leq_{\text{T}}G$.
\end{cor}

\begin{prop}\label{P:Sl:GL2generic}
Let $\MM$ be a model of $\ACAo$.
There is a Silver condition $T$ such that for every branch $G\in[T]$,
$G''\leq G\oplus0''$.  In other words, $T$ only contains GL$_2$ generic reals.
\end{prop}

\begin{cor}
Let $\MM$ be a model of \ACAo.
Suppose $\mathcal{G}$ is a generic filter for Silver forcing.
Then there is a generic real $G$ corresponding to $\mathcal{G}$
such that $G$ is GL$_2$.
\end{cor}

\begin{prop}[\RCAo]\label{P:Sl:StrBnding}
Let $T$ be a Silver condition and $F$ be a $T$-local name.
There is a condition $T'\leq T$ such that for all $n$,
if $\tau\in T'$ is above the $n$-th splitting level of $T'$
then $F_\tau(x)$ is defined by stage $|\tau|$ for all $x\leq n$.
\end{prop}

\begin{cor}
Let $\MM$ be a model of \RCAo\
and suppose that $G$ is a generic real for $\Forc$
corresponding to a generic filter $\mathcal{G}$ for Silver forcing.

For every function $f:\NN\to\NN$ in $\MM[G]$
there is a function $b$ in $\MM$ such that
$f(x)\leq b(x)$ for all $x$.
\end{cor}

	\subsection{Miller Forcing}\label{miller}
		The \textit{conditions for Miller forcing} are superperfect subtrees of $\omega^{<\omega}$.
In other words, perfect subtrees of $\omega^{<\omega}$ such that
every splitting node is infinitely splitting.
Notice that Miller forcing is persistent.

\begin{prop}[\ACAo]\label{P:M:MCP}
Miller forcing satisfies \MCP.
\end{prop}
\begin{proof}
Given a Miller condition $S$ and a $\sigma\in S$,
we let $Sp(S,\sigma)$ be the set of all immediate successors of $\tau$,
where $\tau$ is the first splitting node above $\sigma$.

Let $T$ be a Miller condition and $\langle c_i:i\in\NN\rangle$ be an
infinite sequence of monotone colorings $c_i:T\rightarrow \{0,1\}$.
We will construct $T'\leq T$ and a set of layers $X_0,X_1,\ldots$ for $T'$ in stages.

At stage 0 we do the following.
For each $\tau\in Sp(T,\langle\rangle)$,
if there is a $\sigma\in T_\tau$ such that $c_0(\sigma)=1$,
then we place $\sigma$ into $X_0$.
Otherwise, if $c_0(\sigma)=0$ for all $\sigma\in T_\tau$,
then we place $\tau\in X_0$.
This completely defines $X_0$.
We now define a new condition $T^0$ to be the set of all
nodes comparable to some element of $X_0$.

At stage $(n+1)$ we consider all
$\tau\in \bigcup_{\rho\in X_n}Sp(T^n,\rho)$.
If there is a $\sigma\in T^n_\tau$ such that $c_0(\sigma)=1$,
then we place $\sigma$ into $X_{n+1}$.
Otherwise, if $c_0(\sigma)=0$ for all $\sigma\in T^n_\tau$,
then we place $\tau\in X_{n+1}$.
This completely defines $X_{n+1}$.
We now define a new condition $T^{n+1}$ to be the set of all
nodes comparable to some element of $X_{n+1}$.

Finally, we let $T'=\bigcap_n T^n$.
Notice that $T'$ is a Miller condition, $T'\leq T$,
and $X_0,X_1,\ldots$ are a set of layers for $T'$
that satisfy the conclusion of \MCP.
\end{proof}

\begin{lem}\label{L:M:realForc}
For every $\BSigma^0_2$-generic filter over a model of \RCAo\
there exists a generic real for Miller forcing.
\end{lem}

\begin{proof}
By Lemma~\ref{L:Gen:realForc} it
suffices to show that if $T$ is a condition, then
$$\mathcal{D}_T=\{S:S\cap T\text{ is finite }\lor\ S\leq T\}$$
is open dense.

Suppose that $T'\cap T$ contains no superperfect subtree.
If $T'\cap T$ is empty, then we're done.
Otherwise we can choose a $\sigma\in T'\cap T$ such that no
$\tau\supseteq\sigma$ splits infinitely in $T'\cap T$.
We can therefore find a $\tau\supseteq\sigma$ such that
$\tau\in T'\setminus T$.
Then $T'_\tau\leq T'$ and $T'_\tau\cap T$ is finite,
so $T'_\tau\in\mathcal{D}_T$.
\end{proof}

\begin{thm}
Let $\MM$ be a model of \RCAo\ (\ACAo)
and suppose that $G$ is a generic real for $\Forc$
corresponding to a $\BSigma^1_2$-generic filter $\mathcal{G}$ for Miller forcing.

Then $\MM[G]$ is a model of \RCAo\ (\ACAo).
\end{thm}
\begin{proof}
Follows from Propositions \ref{T:presRCA} and Proposition~\ref{T:presACA}.
\end{proof}



\backmatter

\bibliographystyle{amsplain}
\addcontentsline{toc}{chapter}{References}
\bibliography{Jrefs}

\printindex

\end{document}
