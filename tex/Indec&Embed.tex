This section is joint work with Dorais.

We now consider $RT^1(\omega^n)$ and the
more natural definition of `order type $\omega^n$'.
As in Section~\ref{ElemIndec}, we choose a particular
representation of the elements of $\omega^n$, namely $\NN^n$,
which we order lexicographically.
In other words, if $\tup{x},\tup{y}\in\NN^n$, then
$\tup{x}\lex\tup{y}$ if $x_i<y_i$, where $i=\mathsf{min}\set{j<k}{x_j\neq y_j}$.
The following definition is then a special case of Definition~\ref{D:RamseyPOsets}.

\begin{definition}
$RT_k^1(\omega^n)$ is the following statement:

For every finite coloring $c: \NN^n \to \{0,\dots,k-1\}$,
there is a lexicographic embedding\index{lexicographic embedding}
$h: \NN^n \to \NN^n$ such that $c \circ h$ is constant.
\end{definition}

We will use $RT^1(\omega^n)$ to denote $(\forall k)RT_k^1(\omega^n)$.
Notice that $RT^1(\omega^1)$, just like $\EIndec^1$,
is a rephrasing of the infinite pigeonhole principle,
which Hirst proved equivalent to $\Bnd{\BSigma^0_2}$ over \RCAo\ \cite{Hirst:thesis}.
The main result of this section is that $RT^1_2(\omega^3)$ is equivalent to $\ACAo$
over \RCAo, which we will prove using three lemmas.

Given $h:\NN^n\to\NN^n$ and $1\leq i\leq n$,
we let $h_i:\NN^n\to\NN$ be projection of $h$
onto the $i$-th coordinate.

\begin{lem}[Corduan-Dorais; $\RCAo$; $n\geq 1$]\label{L:lexfst}
If $h:\NN^n\to\NN^n$ is a lexicographic embedding then
$$x_1 \leq h_1(x_1,x_2,\dots,x_n) < h_1(x_1+1,0,\dots,0)$$
for all $x_1,\dots,x_n \in \NN$.
\end{lem}

\begin{proof}
By (external) induction on $n$.
The case $n=1$ is trivial.
Suppose then that the result is true for $n$.
Let $h:\NN^{n+1}\to\NN^{n+1}$ be a lexicographic embedding.

We show that
$$h_1(x_1,x_2,\dots,x_{n+1}) < h_1(x_1+1,0,\dots,0)$$
for all $x_1,x_2,\dots,x_{n+1} \in \NN$.
The fact that $x_1 \leq h_1(x_1,x_2,\dots,x_{n+1})$ then follows by induction.
Suppose, for the sake of contradiction, that
$h_1(x_1,x_2,\dots,x_{n+1})=h_1(x_1+1,0,\dots,0)$.
Let $\tilde{h}:\NN^n\to\NN^n$ be the lexicographic embedding defined by
$$\tilde{h}_i(z_1,\dots,z_n) = h_{i+1}(x_1,x_2+1+z_1,z_2,\dots,z_n)$$
for each $1\leq i\leq n$.
By the induction hypothesis,
$$z_1 \leq \tilde{h}_1(z_1,0,\dots,0) = h_2(x_1,x_2+1+z_1,0,\dots,0).$$
Since $h$ is a lexicographic embedding and
$h_1(x_1,x_2,\dots,x_{n+1})=h_1(x_1+1,0,\dots,0)$,
then $h_2(x_1,x_2+y,0,\ldots,0)\leq h_2(x_1+1,0,\dots,0)$ for all $y>0$.
Therefore, for all $z_1 \in \NN$,
$$z_1 \leq \tilde{h}_1(z_1,0,\dots,0) = h_2(x_1,x_2+1+z_1,0,\dots,0)\leq h_2(x_1+1,0,\dots,0),$$
which is clearly impossible.
\end{proof}

\begin{lem}[Corduan-Dorais; $\RCAo$; $n\geq 1$]\label{L:lexlim}
If $h:\NN^n\to\NN^n$ is a lexicographic embedding and
$1 \leq j < i \leq n$, then
$$\lim_{x_i\to\infty} h_j(x_1,\dots,x_{i-1},x_i,0,\dots,0)$$
exists and is bounded above by $h_j(x_1,\dots,x_{i-1}+1,0,\dots,0)$.
\end{lem}

\begin{proof}
We proceed by induction on $j < i$.
By the induction hypothesis, find $\tilde{x}_i$ such that
$$h_k(x_1,\dots,x_{i-1},x_i,0,\dots,0) = h_k(x_1,\dots,x_{i-1},\tilde{x}_i,0,\dots,0)$$
for all $x_i \geq \tilde{x}_i$ and $1 \leq k < j$.
Note that we must then have
\begin{multline*}
  h_j(x_1,\dots,x_{i-1},x_i,0,\dots,0) \\
  \leq h_j(x_1,\dots,x_{i-1},x'_i,0,\dots,0) \\
  \leq h_j(x_1,\dots,x_{i-1}+1,0,0,\dots,0)
\end{multline*}
for all $x'_i \geq x_i \geq \tilde{x}_i$.
It follows immediately that
$$\lim_{x_i\to\infty} h_j(x_1,\dots,x_{i-1},x_i,0,\dots,0)$$
exists and is bounded above by $h_j(x_1,\dots,x_{i-1}+1,0,0,\dots,0)$.
\end{proof}

\begin{lem}[Corduan-Dorais; $\RCAo$; $n\geq 1$]\label{L:lexinf}
If $h:\NN^n\to\NN^n$ is a lexicographic embedding and
$1 \leq i \leq n$, then
$$\lim_{x_i\to\infty} h_i(x_1,\dots,x_{i-1},x_i,0,\dots,0) = \infty$$
for all $x_1,\dots,x_{i-1} \in \NN$.
\end{lem}

\begin{proof}
By Lemma~\ref{L:lexlim}, we can find $\tilde{x}_i$ such that
$$h_j(x_1,\dots,x_{i-1},x_i,0,\dots,0) = h_j(x_1,\dots,x_{i-1},\tilde{x}_i,0,\dots,0)$$
for all $x_i \geq \tilde{x}_i$ and all $1 \leq j < i$.
Note that the map $\tilde{h}:\NN^{n-i+1}\to\NN^{n-i+1}$ defined by
$$\tilde{h}_k(y_1,\dots,y_{n-i+1}) = h_{i+k-1}(x_1,\dots,x_{i-1},\tilde{x}_i+y_1,y_2,\dots,y_{n-i+1})$$
is then a lexicographic embedding and the result follows immediately
by applying Lemma~\ref{L:lexfst} to $\tilde{h}$.
\end{proof}

\begin{thm}[Corduan-Dorais]
  $RT^1_2(\omega^3)$ is equivalent to $\ACAo$ over $\RCAo$.
\end{thm}

\begin{proof}
Proving $RT^1_2(\omega^3)$ in \ACAo\ is straightforward.
Since $\Ind{\BSigma^0_4}$ holds in \ACAo, by Theorem~\ref{T:EIndec&Ind}
we may assume $\EIndec^3$.
Therefore for every coloring $c:\NN^3\to\{0,1\}$, there is a $d<2$ such that
$(\exists^\infty x_1)(\exists^\infty x_2)(\exists^\infty x_3)[c(x_1,x_2,x_3)=d].$
We can then use arithmetic comprehension to define a lexicographic embedding
into the set $\set{(x_1,x_2,x_3)\in\NN^3}{c(x_1,x_2,x_3)=d}$.

We now show how to compute the range of a function $f:\NN\to\NN$ using $RT^1_2(\omega^3)$.
For each $z$, let $f[z] = \{f(0),\dots,f(z)\}$.
Consider the coloring $c:\NN^3\to\{0,1\}$ defined by
$$c(x,y,z) =
\begin{cases}
0 & \text{when $(\forall w \leq x)(w \in f[y] \leftrightarrow w \in f[z])$,} \\
1 & \text{otherwise.}
\end{cases}$$
Suppose $h:\NN^3\to\NN^3$ is a lexicographic embedding
such that $c\circ h$ is constant.
First, note that $c \circ h$ must have constant value $0$.

For suppose instead that $c \circ h$ has constant value $1$.
By Lemma~\ref{L:lexlim} there is a $y_0$ such that
$h_1(0,y,0) = h_1(0,y_0,0)$ for all $y \geq y_0$.
Again by Lemma~\ref{L:lexlim} there is a $z_0$ such that
$h_2(0,y_0,z) = h_2(0,y_0,z_0)$ for all $z \geq z_0$.
By Lemma~\ref{L:lexinf} there is a $y_1>y_0$ such that
$h_2(0,y_0,z_0)<h_2(0,y_1,0)$.
Putting this together we see that
$h_2(0,y_0,z_0) < h_2(0,y_1,0) \leq h_3(0,y_0,z_0)$.
Similarly, we can find can find pairs $(y_1,z_1),\dots,(y_k,z_k)$,
where $k=h_1(0,y_0,0)+1$,
such that $h_3(0,y_{i-1},z_{i-1}) \leq h_2(0,y_i,z_i) < h_3(0,y_i,z_i)$
for each $1 \leq i \leq k$.
Since $c(h(0,y_i,z_i)) = 1$, we can find
$w_i \in f[h_3(0,y_i,z_i)] - f[h_2(0,y_i,z_i)]$ where
$w_i \leq h_1(0,y_i,z_i) = h_1(0,y_0,0)$.
Our choice of $y_i,z_i$ guarantees that the $w_i$ must be distinct,
which is impossible since $k > h_1(0,y_0,0)$.
Therefore $c \circ h$ has constant value $0$.

We now show how to compute the range of $f$ from $h$ and $c$.
To determine whether $x$ is in the range of $f$, use the following procedure:
\begin{itemize}
\item[] First find $y$ such that $h_1(x,y,0) = h_1(x,y+1,0)$.
		Answer yes if $x \in f[h_2(x,y+1,0)]$, and answer no otherwise.
\end{itemize}
This procedure will never return false positive answers,
so we suppose that $x = f(s)$ and check that the algorithm answers yes on input $x$.
The existence of a $y$ such that $h_1(x,y,0) = h_1(x,y+1,0)$
is guaranteed by Lemma~\ref{L:lexlim}.
Given such a $y$ we can then use Lemma~\ref{L:lexinf} to find $z$
such that $s \leq h_3(x,y,z)$.
Since
$$h_1(x,y,0) = h_1(x,y,z) = h_1(x,y+1,0),$$
we then have
$$h_2(x,y,0) \leq h_2(x,y,z) \leq h_2(x,y+1,0).$$
Since $c(h(x,y,z)) = 0$ and $x \leq h_1(x,y,z)$ by Lemma~\ref{L:lexfst},
we know that
$$x \in f[h_2(x,y,z)] \leftrightarrow x \in f[h_3(x,y,z)].$$
Since $s \leq h_3(x,y,z)$ we know that $x \in f[h_3(x,y,z)]$,
and since $h_2(x,y,z) \leq h_2(x,y+1,0)$ we conclude that $x \in f[h_2(x,y+1,0)]$.
\end{proof}
